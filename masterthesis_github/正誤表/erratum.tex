\begin{center}
 \textbf{正誤表} 
\end{center}

\textbf{誤記}
\begin{enumerate}
  \item p.~5 誤)Alastair Brotchie,\emph{Alfred Jarry : ein pataphysisches Leben}, Bern, Piet Meyer, 2014, p. 72-3 -> 正)Alastair Brotchie,\emph{Alfred Jarry : ein pataphysisches Leben}, Bern, Piet Meyer, 2014, p. 72-3.
  \item p.~5 誤)\emph{Philosophes du Travail} -> 正)\emph{Philosophies du Travail}
  \item p.~7 誤)イドロジェーヌ -> 正)イドロジェンヌ
  \item p.~ 誤) -> 正)
  \item p.~ 誤) -> 正)
  \item p.~ 誤) -> 正)
  \item p.~ 誤) -> 正)
  \item p.~ 誤) -> 正)
  \item p.~ 誤) -> 正)
  \item p.~ 誤) -> 正)
  \item p.~ 誤) -> 正)
  \item p.~ 誤) -> 正)
  \item p.~ 誤) -> 正)
  \item p.~ 誤) -> 正)
  \item p.~ 誤) -> 正)
  \item p.~ 誤) -> 正)
  \item p.~ 誤) -> 正)
  \item p.~ 誤) -> 正)
  \item p.~ 誤) -> 正)
  \item p.~ 誤) -> 正)
  \item p.~ 誤) -> 正)
  \item p.~ 誤) -> 正)
  \item p.~ 誤) -> 正)
  \item p.~ 誤) -> 正)
  \item p.~ 誤) -> 正)
  \item p.~ 誤) -> 正)
  \item p.~ 誤) -> 正)
  \item p.~ 誤) -> 正)
  \item p.~ 誤) -> 正)
  \item p.~ 誤) -> 正)
  \item p.~ 誤) -> 正)
  \item p.~ 誤) -> 正)
  \item p.~ 誤) -> 正)
  \item p.~ 誤) -> 正)
  \item p.~ 誤) -> 正)
  \item p.~ 誤) -> 正)
  \item p.~ 誤) -> 正)
  \item p.~ 誤) -> 正)
  \item p.~ 誤) -> 正)
  \item p.~ 誤) -> 正)
  \item p.~ 誤) -> 正)
  \item p.~ 誤) -> 正)
  \item p.~ 誤) -> 正)
  \item p.~ 誤) -> 正)
  \item p.~ 誤) -> 正)
  \item p.~ 誤) -> 正)
  \item p.~ 誤) -> 正)
  \item p.~ 誤) -> 正)
  \item p.~ 誤) -> 正)
  \item p.~ 誤) -> 正)
  \item p.~ 誤) -> 正)
  \item p.~ 誤) -> 正)
  \item p.~ 誤) -> 正)
  \item p.~ 誤) -> 正)
  \item p.~ 誤) -> 正)
  \item p.~ 誤) -> 正)
  \item p.~ 誤) -> 正)
  \item p.~ 誤) -> 正)
  \item p.~ 誤) -> 正)
  \item p.~ 誤) -> 正)
  \item p.~ 誤) -> 正)
  \item p.~ 誤) -> 正)
  \item p.~ 誤) -> 正)
  \item p.~ 誤) -> 正)
  \item p.~ 誤) -> 正)
  \item p.~ 誤) -> 正)
  \item p.~ 誤) -> 正)
  \item p.~ 誤) -> 正)
  \item p.~ 誤) -> 正)
  \item p.~ 誤) -> 正)
  \item p.~ 誤) -> 正)
  \item p.~ 誤) -> 正)
  \item p.~31 誤)読者は自然の中には存在しない幾何学的対象をいくつか知っている -> 正)読者は自然の中には存在しない幾何学的対象を知っている
  \item p.~37 誤)自著改題 -> 正)自著解題
  \item p.~41 誤)特異であることが -> 正)特異な点であり、それは
  \item p.~41 誤)別の機会取り上げる -> 正)別の機会に取り上げる
  \item p.~43 誤)主に運動の観点 -> 正)運動の観点
  \item p.~43 誤)隣り合った原子の間の質の変化によって移動が生じる«~Un déplacement se fait donc par un échange de qualités entre atomes voisins~» (46/79) ->隣り合った原子の間の質の変化によって移動が生じる」«~Un déplacement se fait donc par un échange de qualités entre atomes voisins~» (46/79) 
  \item p.~46 誤)l'ombre des résultats réalisés, fixé -> 正)l'ombre des résultats réalisés, fixés
  \item p.~51 誤)遠心力によって違いが -> 正)遠心力によって互いが
  \item p.~53 誤)導かれなない -> 正)導かれない
  \item p.~56 誤)v.79, n. 870, 2001, p. 49. 32-50. -> 正)v.79, n. 870, 2001, p. 49.
  \item p.~57 誤)革命の四現象 -> 正)革命の四象限
  \item p.~57 誤)一致させている予めある予定調和 -> 正)一致させている予定調和
  \item p.~58 誤)このその力と偶然 -> 正)その力と偶然
  \item p.~58 誤)で示している。-> 正)で言及している。
  \item p.~61 誤)原子とは4次元の特有で完全な世界から作られていて -> 正)精神とは4次元の特有で完全な精神から作られていて
  \item p.~63 誤)水素 -> 正)イドロジェンヌ
  \item p.~65 誤)センモウゴケ -> 正)モウセンゴケ
  \item p.~ 誤) -> 正)
  \item p.~ 誤) -> 正)
  \item p.~ 誤) -> 正)
  \item p.~ 誤) -> 正)
  \item p.~ 誤) -> 正)
  \item p.~ 誤) -> 正)
  \item p.~ 誤) -> 正)
  \item p.~ 誤) -> 正)
  \item p.~ 誤) -> 正)
  \item p.~ 誤) -> 正)
  \item p.~ 誤) -> 正)
  \item p.~ 誤) -> 正)
  \item p.~ 誤) -> 正)
  \item p.~ 誤) -> 正)
  \item p.~ 誤) -> 正)
  \item p.~ 誤) -> 正)
  \item p.~ 誤) -> 正)
  \item p.~ 誤) -> 正)
  \item p.~ 誤) -> 正)
  \item p.~ 誤) -> 正)
  \item p.~ 誤) -> 正)

\end{enumerate}
\textbf{参考文献遺漏}
\begin{enumerate}
  \item p.~87 Régnier, Pilippe, «~Place, fonctions et formes de l'ecriture utopique chez Fourier~» pp.~385-401.
\end{enumerate}
\textbf{人名}
\begin{enumerate}
  \item 誤)エドモン・ハロクール(Edmond Haraucourt) -> 正)エドモン・アロクール
  \item 誤)ブライアン・ステーブルフォード(Brian Stablefold) -> 正)ブライアン・ステーブルフォールド
\end{enumerate}
\textbf{図表}
