\chapter{ガストン・ド・パヴロフスキーと『四次元郷への旅』について}
\section{ガストン・ド・パヴロフスキーの来歴}
ガストン・ウィリアム・アダム・ド・パヴロフスキー(Gaston William Adam de Pawlowski)は,ヨンヌ(Yonne)のジョワニィ(Joigny)に1874年6月14日に生まれ,パリで1933年2月2日に亡くなった.彼の父であるアルベール・ド・パヴロフスキー(Albert de Pawlowski)は鉄道会社「西武鉄道(Compagnie des chemins de fer de l'Ouest)」に勤める技師だった.母のヴァレリー・ド・トリオン=モンタランベール(Valérie de Tryon-Montalembert)は,家門の貴族(noble d'extraction)で14世紀頃のド・トリオン家まで遡ることができる貴族の血筋にある人物だった.パリ市内のリセ・コンドルセ(Lycée Condorcet)に通っていた頃に,彼の遊び仲間を通じて,『ユビュ王』や『超男』の作者として知られるアルフレッド・ジャリと交友関係を結んでいた.『超男性』は自転車乗りが主人公であるが,ジャリはパヴロフスキーらとツーリング仲間だったことが知られている\footnote{Alastair Brotchie, \emph{Alfred Jarry~: ein pataphysisches Leben}, Bern, Piet Meyer, 2014, p. 72-3}.その後,バカロレアでは文学と科学の二つをパスして進学し,エコール・ド・ルーヴル(école du Louvre)に通うかたわら,パリ政治学院(école des sciences morales et politique)の法学部(La Faculté de droit)に入学した.1901年,博士論文に『労働の哲学\emph{Philosophies du Travail}』を提出して,法学博士になり,同年に博論を出版した. 博士論文を提出する以前から,パヴロフスキーは雑誌編集に関わるようになる.

パヴロフスキーの編集者としての活動は大学生時代から始まっており,1894年から『自転車(\emph{Vélo})』と『自動車雑誌(\emph{Journal de l’Automobile})』などの編集や記事の執筆をしていた.現在のスポーツ誌の編集者と言える.その後,『オピニオン(\emph{l’Opinion})』では政治記事を執筆し,『ル・ジュルナル(\emph{Le Journal})』では芸術時評の担当をした.最終的には『ル・リール(\emph{Le Rir})』でユーモア風の記事を書くことに専心することになり,その嗜好がのちに創刊されることとなる雑誌の基本的な方針となっていった.

パヴロフスキーがその生涯のほとんどを通じて活動したのが,ラテン語の「喜劇(commediae)」を思わせるタイトルを持った日刊紙『コメディア(\emph{Comœdia})』が,ツール・ド・フランスを主催したアンリ・アントワーヌ・デグランジュ(Henri Antoine Desgrange)が出資して1907年10月1日に刊行された.パヴロフスキーは編集長となったが,1914年に第一次世界大戦が始まると徴兵され,自動車整備のエンジニアとなり,同年8月に廃刊した.1919年には軍役を終えると,同年10月1日に復刊し,編集長を再び務めた.そして,1923年まで『コメディア』に身を捧げた.二度目の編集長となった頃,1921年にはマルグリット・マンガン(Marguerite Mangin)と結婚した\footnote{1912年に刊行された『四次元郷』が独身者の文学の系譜に並ぶような作風であるのに比べて,1923年版には愛についてのエピソードや考察が増えているのはこの結婚と無関係ではないだろう.}.

『コメディア』は刊行から3年で28,000部の売り上げた\footnote{\emph{Histoire générale de la presse française}, dir. Claude Béllanger, et al. ,t. 3, Paris, Presses universitaires de France, p. 296.}.同紙は,演劇・文学・芸術を中心に取り上げ,政経営者たちを保守的なクラブやサロンの社交界から募集していたからか,政治的話題を直接扱うことはなかったという点でユニークな新聞で,「その時代にとってそれに相当するもののない,フランスにいながら外国の雑誌であった(\emph{Comœdia} était sans équivalent pour l'époque, en France comme à l'étranger)\footnote{Claude Béllanger, \emph{op. cit. }, p. 381.}」.当時では珍しかった作家インタビュー記事を記載するなど,同時代の雑誌に比べて先進的でもあった\footnote{以上は下記を参照のこと.Claude Béllanger, \emph{op. cit.} , pp. 381-2. ただし,マリー・エヴ・テランティ(Marie-Ève Thénty)が指摘しているように,19世紀の終わりになる頃には電報の発達により速報性が高くなったメディアは,ルポルタージュ・三面記事・インタビューの3つが日刊紙の中心となっていることが明らかとされているのでインタビュー記事を載せること自体はそれほど特別な判断ではなかったと考えられる.Marie-Ève Thénty, \emph{La Littérature au quotidien. Poétiques Journalstique au XX\textsuperscript{e} siècle}, Paris, Seuil, 2007.}.パヴロフスキーは,週刊連載で書評欄「今週の文学(Semaine Littérature)」を担当していたが,マルセル・プルーストの『失われた時を求めて スワン家の方へ』を書評した際に,プルースト本人から,パヴロフスキーが作品をベルクソンの理論を敷衍して批評したことに反論の手紙が寄せられたことは文学史的にも注目するに値する事件だろう\footnote{Marcel Proust, \emph{Correspondance de Marcel Proust}, éd. Philip Kolb, t. 3, Paris, Plon, 1976, pp. 54-5.}.当時のパヴロフスキーの様子を伝える貴重な証言として,『コメディア』で芸術部門を担当していたコラムニストのアンドレ・ヴァルノー(André Warnod)の言葉を紹介する.1955年の回想録に「ガストン・ド・パヴロフスキーは普通の物差しから外れた男だった.その精神と知性のように大男だった.彼はラブレーの主人公のように見えた.今という時間を超えていたのは,思考の方法のうちよりかは存在のあり方においてだった(Gaston de Pawlowski était un homme qui échappait à toute commune mesure. Il était d'une taille gigantesque, comme son esprit et son intelligence. Il faisait figure de héros de Rabelais. Il était hors du temps présent, aussi biens dans sa façon d'être que dans sa façons de penser)」.

編集長を辞めた後,公に発表された最後の小説は今までのところ1925年に確認されており,『レ・アナル(\emph{Les Annales})』にドラリュ=ヌーヴェリエール(Delarue-Nouvellière)の挿画が添えられた「どこに行くのか?(Où allons-nous)\footnote{Gaston de Pawlowski, «~Où allons-nous?~», \emph{Les Annales politiques et littéraires~: revue populaire paraissant le dimanche}, dir. Adolphe Brisson, 6. Déc. 1925, Paris, [s. n.], pp. 581-2.}」という短編だった.『四次元郷』の登場人物の一人である「イドロジェーヌ(Hydrogène)\footnote{水素を意味する単語.}」の20世紀に対する風刺が描かれている.それ以後も,散発的に評論文を書いていたが,この8年後,1933年2月2日に亡くなり,『パリ・ミディ(\emph{Paris-Midi})』が一報を出し,翌日の『コメディア』には,その時に編集長だったガブリエル・ボワシー(Gabriel Boissy)が追悼記事を寄せている\footnote{Gabriel Boissy, «~Gaston de Pawlowski premier rédacteur en chef de «~Comœdia~» est mort prèmaturément hier~», \emph{Comœdia}, 3 Fév 1933, pp. 1-2.}.とりわけ『四次元郷』の作者として著名だったパヴロフスキーを讃えて次のように述べている.

\begin{quote}
 結局,この4次元は,時間や例の相対性[=相対性理論]から予想されることとなり,パヴロフスキーは予知するような仕方でそれを研究しつつ,1912年から早くも,アインシュタインの先取りをもたらしていたのだろうか.
\end{quote}
\begin{quote}
 Enfin, cette quatrième dimension, qui ne faisait que présager du temps et de sa relativité, Pawlowski en l'étudiant de façon divinatoire, n'apportait-il pas, dès 1912, une anticipation d'Einstein?\footnote{\emph{Ibid. }, p. 1.}
\end{quote}

アイシュタインの1905年以降の特殊相対性理論に関する断続的な研究は,徐々に受け止められ,1915年に一般性相対性理論が発表されて以来,相対性理論は1つの流行語となっていた.パヴロフスキーに対するこのような評価も,そうした文脈に基づいていると考えられる.しかし,後で見るように,少なくともアインシュタインと4次元の2つのイメージが大衆的な教養として定着する以前から4次元についての研究や『タイム・マシン』のような小説が書かれていたので,「先取り」とは言うことはできない.ただし,パヴロフスキーは彼の哲学的テーマを背景とした独自の4次元の概念を構築した.

ボワシーの追悼記事の終わりにはパヴロスキーの生涯とその最期が簡潔にまとめられている.フェザンデリー通り(la rue de la Faisanderie)の自宅で2月2日の朝6時に心臓発作で死亡した\footnote{\emph{Ibid. }, p. 2.}」.翌日の記事の詳細によれば,パヴロスキーの葬式は,午後2時から開催され,パリ8区ロケピーヌ通り(rue Roquépine)の5番地にあるサン=エスプリ教会(l'église du Saint-Esprit)にて行われて,ペール・ラシェーズに埋葬されることとなった.

\section{『四次元郷への旅』の書誌情報とあらすじ}
\subsection{書誌情報}
パヴロフスキーが編集長を務めている間に『コメディア』で連載された『四次元郷』は,パヴロフスキーの自身説明によれば,博論を提出する前の1895年から書き始めていたという\footnote{以下の記述を参照のこと.「時間探索についての最初の物語を書き始めた1895年の初めから,この『四次元郷への旅』の初版が刊行された1912年まで(Depuis le début de 1895 où j'écrivais un premier conte sur l'exploration du temps, jusqu'en 1912, date à laquelle parut la première édition de ce volume [= \emph{Voyage au pays de la quatrième dimension}])(1923, 9)}.作品の前身となった「旅(Voyage)」シリーズは1908年11月8日に「奇妙な旅(L'Etrange Voyage)」と題されて始まり,11月22日に「未来の話(Contes futurs)」にシリーズ名が改められ,「幻視者(Visionnaire)」が掲載された.12月13日には再びシーリズ名が改められて「超人時間物語(Récits ds Temps Surhumains)」と題されて,「死んだ愛(l'Amour Mort)」が掲載された.その後,『四次元郷』の前半のシリーズである「リヴァイアサン(\emph{Le Léviathan})」は1909年12月24日から1910年10月31日にかけて掲載されたが,この時にはまだタイトルは書籍と違っていた.書籍名と同じタイトルになって連載が再開し,1912年2月24日から1912年12月9日の間に不定期に連載された.連載の終了した1912年にファスケル(Fasquelle)社から初版が2000部刊行され,翌年にはさらに1000部が増刷された.第一次世界大戦の後,1923年に決定版が出版された.オランダの象徴派画家であるレオナルド・サルリュス(Léonard Sarluis)によるイラストや,「批判的吟味(examen critique)」と題した序文が追加され,内容面でも1909年に出版された『長枕~---~動く景色,空想の景色(\emph{Polochon~: paysages animées, paysages chimérique})』の一部が21章に「死んだ愛(L'amour mort)」として新しく付け加えられるなど,他の章についても削除や書き直しなど全面的な変更がなされている.10年に渡る書籍の来歴を見ると,ボワシーによる述懐が示しているように,20世紀前半のフランスにおける4次元に関する言説の一角を占めていていたことがわかる.

\subsection{あらすじ}
『4次元郷』は邦訳がなく物語の全体があまり知られていないため,以下にあらすじを示す.

語り手(パヴロフスキー)はある日,不思議なインド風の箱(le coffret hindou)に手紙をしまって,紐で結んで封蝋をした.彼はしばらくして中を検めることにした.手紙は難なく確認することができたのだが,パヴロフスキーは結んでいた紐がいつのまにか解けて,さらには封蝋もなくなっていたということに気づいた.慌ててまた箱を確認すると先ほどまで開けることができた箱が元のように封印されていた.パヴロフスキーはこの経験から天啓を得ることとなった.彼はフェリックス・クラインのクラインの壺の結び目のことを思い出し,その壺の結び目もまた3次元では解けてしまうこともあるのを思い出す.その後,パリに存在しないはずの「中駅(Gare du Midi)」を見てしまったり,森の中で正面玄関しかない平面の家(maison plate)と遭遇することが重なり,4次元の実在を確信するようになる.あらゆる場所と場所が短絡してしまうという空間の「抽象化(abstraction)」から,時間移動の秘密を理解するようになった語り手はついに4次元を通して,未来への移動を可能とする.

未来を見た語り手によれば,1912年を境にして,世界は大きく変貌していく.そして,かつてホッブズがその存在を予言したリヴァイアサンという超巨大生命体によって人類は支配されてしまう.リヴァイアサンはただの象徴的な存在ではなく,個人の道徳や科学的な考え方や方法までも規定してしまう人類にとっての上位存在の怪物として描かれている.人間はリヴァイアサンの支配する社会の細胞に還元されることで個人としての意志を失い,社会から切り離されて生きることができなくなってしまっていた.

人々はリヴァイアサンの支配に気づかずにいたが,いくつかの事件を通じて,統一された観念に向かっていくものではなくて,個別の分析に個別の観念が備わっているものであるという考え方が主流になり,それがあらゆる分野で進んだ.その結果,統一された観念を要請するリヴァイアサンを拒絶することになった.そうして,科学の時代が訪れる.

科学の時代は大きく分けて2つの時期に分けられる.第一の科学時代では,科学技術の発達とともに,人知の及ばない事件が多発するようになる.その中でも物語のあらすじと関係して最も重要なものは,「機械の反乱」の章で描かれる機械の暴走である.工場の機械があたかも生物のように自律的に動き回って人間を翻弄させたことをきっかけにして,物質にも生命が宿っているのであることが知られるようになる.その一方で,極端な機械化の果てに,個人は機械の所有者の奴隷のような状態になっており,人間は愛といった感情を忘れてしまっていた. 第二の科学時代になると,最終的に世界で最も科学研究が進んだ「大中央研究所(Grand Laboratoire Centrale)」によって世界は支配されるようになる.この学者たちは不死の秘密を手に入れ長期間にわたってその支配の座を譲らなかったが,生殖という概念が改めて取り上げられ実験がなされ,科学者の支配を通じて人類が忘れてしまっていた愛を再発見する.そして,世界が隅々まで機械化されることで人間が自らを維持するためのエネルギーを労働することによって得る必要がなくなった.

科学の時代が進むにつれて,科学主義の唯物論的世界観から,観念論が再び世界の思想の中心となる.それは黄金鳥の時代(l'Oiseau d'or)または黄金鷲の時代(l'Aigle d'or)\footnote{黄金鷲に名前が1923年版になる時に追加されている.パヴロフスキーは,「\bou{黄金鷲の時代}と呼ばれていて,時により親しみやすく,\bou{黄金の鳥}の時代と呼ばれている(on l'appela l'\emph{Age de l'Aigle d'or} et parfois plus familièrement, celui de l'\emph{Oiseau d'or})」(1923, 231)と述べている.黄金鷲の時代という呼称が一般的ではないということの理由は明記されていない.ただし,鷲を\emph{Oiseau de Jupiter}と呼称することがあることとこの時代の理想的なモデルが古代ギリシャに由来していることを考えると,この語の持っている古代ギリシャ的なニュアンスを意識していると考えられる.}と呼ばれている.フランス人権宣言から2000年後の3798年に「物質と自然の権利(les Droits de la matière et de la nature)」が宣言され,世界が固有の「原子(atome)」によって様々な形を変えた表現であることが確認され,四次元の秘密が人類に明かされる.私たちの生きている世界はただ1つの原子によって織り成されている無数の差異の集合で,人類は精神によって4次元に至り,個人が誰かに隷属することもなく,労働もする必要がない自由を手に入れることができたのだった.

以上のあらすじはまとめると表~\ref{tab:age}のようになる.

\begin{mytab}[htbp]{物語中の時代}{tab:age} \begin{tabular}{c|c|c}
  連合 & リヴァイアサン &  \\
  ↓& ↓ &   \\
  分離 & 第一・第二の科学時代 & 唯物論の時代\\
  ↓ & ↓ &  \\
  統一体(unité) & 黄金鳥(黄金鷲)の時代 & 観念論の時代\\
\end{tabular}
\end{mytab}

\section{先行研究}
『四次元郷』は多くの文学者や芸術家に影響を与えてきたが,作品研究はあまり進んでいない.しかし,世に問われて以来,高次元や非ユークリッド幾何学を扱った小説を挙げる際にほとんど必ず言及される作品である.例えば,1948年に出版されたル・リヨネの編纂した『数学思想の流れ(\emph{Les grands courants de la pensée mathématique})』に所収されているアンドレ・サント・ラーゲ(André-Sainte Lagüe)の小論「四次元の旅」も,4次元を扱っている作品として『四次元郷』を挙げている.数学の専門知識と大衆におけるイメージの落差を語りながら,位相幾何学における思考実験とその検証をわかりやすく説明している.その中で『四次元郷』が例に挙げられ,物語の冒頭に登場する箱について,数学的には3次元の結び目を4次元で解くことが可能であることが説明されている\footnote{アンドレ・サント・ラーゲ「四次元の旅」『数学思想の流れ』, ル・リヨネ編,村田全監訳,東京図書,1974年,pp. 174-5.}.戦後まもなく4次元を題材にした小説であればパヴロフスキーの名前が挙がっていたことを示す貴重な記録となっている.しかし,大衆向けの科学小説に関する研究が始まるようになってからも,パヴロフスキーが研究として取り上げられることはほとんどなかった.原因はいくつか考えられる.主要なものとして,パヴロフスキーがジャーナリストとしての活動を中心に行っており,その時の見聞が『四次元郷』には大きく反映されているものの,作品中の表現の典拠を新聞記事に求めることや,1912年の初版を手に入れるといったことが困難であったと考えられる.しかし,電子アーカイブズの登場と漸進的な電子資料の登録数の増加によってこうした問題は解決されつつある.本論が以下に示すこれまでの研究と大きく方向性が異なっているのは,そうした資料が整ってきているからである.

パヴロフスキーに関する言及が増加してくるのは,1970年代に入ってからである.まずSF研究の文脈で取り上げられ,次にモダンアート研究で取り上げられるようになった.

SF研究におけるパヴロフスキーの位置付けの変遷をまず見ておこう.1971年にアンリ・ボダン(Henri Baudin)が,科学主義の様々な文学的な可能性を開いた初期の事例として,エドモン・ハロクール(Edmond Haraucourt)の『最初の人間~---~ダー(\emph{Daâh} \emph{le premier homme})』(1925\footnote{連載自体は1912年から1914年にかけてなされた.以下を参照のこと.Edmond Haraucourt, «~Daâh le premier homme~», \emph{Le Journal}, du 26 décembre 1912 au 3 avril 1913.}),シャルル・ドレンヌ(Charles Derennes)の『極地の人々(Le peuple du Pôle)』(1907)などと一緒に取り上げられている.パヴロフスキーは作品中でオカルト現象の典型が多用されているたためか,ボダンはパヴロフスキーを「心霊主義の小説家(spirituel romancier)」と特徴づけている\footnote{Henri Baudin, \emph{La science-fiction~: un univers en expansion}, Paris, Bordas, 1971, p. 238.}.ボダンの研究の後にジャック・ヴァン=エルプ(Jacques Van Herp)がフランスを中心にテーマごとにSF史をまとめ,その中でパブロフスキーを4次元を扱った小説の代表例として挙げている.また,ユーモア作家で知られるアルフォンス・アレー(Alphons Allais)の影響が大きいことが指摘されている\footnote{Jacques Van Herp, \emph{Panorama de la Science Fiction. Les Thème, Les Genres, Les écoles, Les Problèmes}, Verviers, Gérard \& Co, 1973, p. 79.}.加えて,フランスSF研究で最も重要な研究者であるピエール・ヴェルサン(Pierre Versins)の『驚異の旅とSFのユートピア事典\footnote{Pierre Versins, \emph{Encyclopédie de l'utopie des voyages extraordinaires et de la science fiction}, Lausanne, L'Age D'Homme, 1972.} 』が1972年に刊行された.1000頁に及ぶ事典にはパヴロフスキーの項目もあり,1972年以降にパヴロフスキーについて言及される場合にはほとんど必ず引用される記事となった\footnote{\emph{Ibid. }, pp. 658-9.}.パヴロフスキーの作品を複数取り上げ,それらが『四次元郷』に収斂していくという記述など,現在の標準的な『四次元郷』解釈の基礎を作っている.また,他の作家の項目の中にもパヴロフスキーは多くの顔を出しており,例えば,パヴロフスキーの短編作品対してシュルレアリストグループと関わりのあった作家ジャック・リゴーが応答する作品を発表しているという記述はパヴロフスキーの小説がどのような影響を周囲に与えていたかを理解するための貴重な資料となっている.

次にモダンアートの文脈を見てみよう. 1970年代の終わりから1980年の初めにかけてパヴロフスキーを有名にした2つの著作がある.1つがジャン・クレール(Jean Clair)の『マルセル・デュシャン,あるいは大いなる虚構~---~大ガラスの神話分析試論(\emph{Marcel Duchamp, ou, Le grand fictif : essai de mythanalyse du Grand verre})\footnote{Jean Clair, \emph{Marcel Duchamp, ou, Le grand fictif~: essai de mythanalyse du Grand verre}, Paris, Galilée, 1975.}』(1975 以下,『大いなる虚構』)であり,もう1つがリンダ・ダリンプル・ヘンダーソン(Linda Dalrymple Henderson)の『モダンアートにおける4次元と非ユークリッド幾何学』(Linda Dalrymple Henderson, \emph{The fourth dimension and non-Euclidean geometry in modern art})\footnote{Linda Dalrymple Henderson, \emph{The fourth dimension and non-Euclidean geometry in modern art}, London, Cambridge, 1983. ただし,引用に際しては全て以下の版に依拠した.Henderson, Linda Dalrymple, \emph{The fourth dimension and non-Euclidean geometry in modern art}, Massachusetts, MIT Press, 2013.}』(1983 以下,『4次元と非ユークリッド幾何学』)である.

まず,『大きなる虚構』から見てみよう.パヴロフスキーがデュシャンにどのような影響を与え,その後に『四次元郷』に関するクレールの見解を示し,パヴロフスキーが時間と空間をどのように解釈していたのかを,それぞれ順に示す.デュシャン研究でよく取り上げられるピエール・カバンヌ(Pierre Cabanne)とのインタビューで,デュシャンが「ポヴォロヴスキー(Povolowsky)」という人物を知っているか,と尋ねたことが記録されている\footnote{Marcel Duchamp, \emph{Entretiens avec Pierre Cabanne}, Paris, Belfond, 1967, p. 67.}.これはカバンヌがパヴロフスキーを知らなかったための聞き違いか誤記だと考えられている.クレールは,デュシャンのこの発言を取り上げて,デュシャンが残した「グリーンボックス」と呼ばれる箱に入っている大量のメモを参考に,1923年に発表した彼の代表作である「彼女の独身者たちによって裸にされた花嫁,さえも(\emph{La Mariée mise à nu par ses célibataires, même})」,通称「大ガラス」について『四次元郷』からの影響を考察した解釈を提示している.とくに,デュシャンの「大ガラス」制作時代にキュビズムや未来主義と距離を置いたことに着目し,クレールは,デュシャンの作品制作の背景にあったのは,主体がどのように世界を認識しているかという観察者(observateur)の位相をめぐるものだったという説を提示している\footnote{Clair, \emph{op. cit. }, pp. 45-47.}.この時,観察者というテーマは『四次元郷』の1923年版に追加された序文すなわち「批判的吟味」に由来しているというのだ.また,クレールはデュシャンの『大ガラス』制作の構想の中で,4次元に時間は存在せず全てが同時的であることや,運動も本来は存在せず4次元においてすべてが静止しているというアイディアが見られることについて,『四次元郷』の冒頭でなんども繰り返されているテーマであることを指摘している\footnote{詳しくは第4章で扱う}.また,「大ガラス」の制作背景にある思想だけでなく,作品の具体的な分析も『四次元郷』のモチーフが現れていると指摘している.モチーフに通底するテーマはエロティシズムである.パヴロフスキーは1923年の決定版で「工業恋愛(L’Amour industriel) 」の章を追加し,エロティックなエネルギーを工業的に利用するという話を描いている.このようなエロティシズムと機械が組み合わさった運動のイメージは「大ガラス」の花嫁と独身者たちのイメージに反映されていると考えられる\footnote{\emph{Ibid. }, p. 128. }.そして,同じく決定版で言及されるエロスのメタモルフォーゼによって4次元に至るというパヴロフスキーの考えでは,プラトンの『饗宴』でアリストファネスが紹介する,男女はもともと1つの存在であり,現世ではそれが2つに分かれているという神話が触れられている.この2つに分かれた男女とエロス,そして機械のモチーフこそ,「大ガラス」の構成要素に他ならないとクレールは主張している\footnote{\emph{Ibid. }, p. 141.}.

これらの研究は「大ガラス」の解釈のためにパヴロフスキーの作品を読解しているが,クレールはのちにパヴロフスキーに焦点を当てた論考を発表する.それが,『四次元郷』が2004年にイマージュ・モデルヌ(Images Modernes)社から再刊された際に寄せられた『四次元郷』の「序文」である\footnote{Jean Clair, «~Introduction~», \emph{Voyage au pays de la quatrième dimension}, Gaston de Pawlowski, Images modernes, 2004, pp. 11-27.}.クレールはこの「序文」でまずパヴロフスキーの世代に注目している.パヴロフスキーは,象徴主義の世代と言えるものの,初期の未来派の芸術家たちの世代とも重なる転換期の作家だった.『四次元郷』の章のうち,同時代の作家と似たようなテーマがある章をそれぞれ指摘して,20世紀初めのロマン主義的スピリチュアリズムと科学主義の交差地点に『四次元郷』が位置していると述べている.さらに,『大いなる虚構』でも指摘しているように,一読してもコラージュされているようにしか見えないそれぞれの章は,読み進めていくと統一的に把握することが可能となり,異常な状態になってしまった社会の復興と人類の完全なる状態への進化という再生(palingénésie)を描いている物語であるということを改めて示している.また,最後に『四次元郷』の第二の科学時代で支配者として描かれている大中央研究所の学者たちが最終的に不死になることで,死ぬこととは別の形で3次元的な存在を終わらせようとすることについて,クローン技術などが発達しつつある2000年代からすると,不死の描写自体が極めて現実的であるように思われるとして「序文」を締めくくっている.

最後に,クレールの『四次元郷』の解釈について,「序文」で示されているパヴロフスキーの時間観を見てみよう.すなわち,パヴロフスキーにとって時間は実在している所与のものとして考えられており,倫理的,道徳的,政治的なものなのだ\footnote{\emph{Ibid. }, p. 23.}.第4章で見ていくように,パヴロフスキーは時間は存在せず全てが同時的であるということを明確に述べているので,誤った指摘のようにも思われるが,クレールは,4次元において時間が空間化されていると捉えている,と考えられる.

しかし,クレールの研究には問題点があることも指摘しておく.まず,全体の議論が1923年の決定版に依拠しているために,最初の時点ではほとんど注目されていなかった概念が焦点化されている.例えば,パヴロフスキーは初版の時点ではあまり観察(observation)という語彙を用いていないうえ,観察者(observateur)という語に至っては1912年版には1度も登場しない.別の問題として,3次元における時間の実在に基づいた解釈を示すが,パヴロフスキーが描く4次元や物語の挿話が他の同時代の作品とどう異なっているのかを示すことができてない.

ところで,パヴロフスキーの4次元の特徴はどこにあるのかという問いが明快に説明されているのは,ヘンダーソンの著作である.実際,クレールは「序文」で,4次元や非ユークリッド幾何学についての歴史的説明に関してヘンダーソンの議論を下敷きにしている.ヘンダーソンによるパヴロフスキーの4次元の特徴については第4章で詳しく見ていくのでここでは一旦おくとして,どのようにパヴロフスキーを扱っているのかを見ておこう.

パヴロフスキーが『四次元郷』を連載し始めた頃のキュビストのうち,ジャン・メザンジェ(Jean Metzinger)とアルベール・グレーズ(Albert Gleizes)は,『四次元郷』を着想源として,絵画空間の中で時間と空間のモデルを革新しようとしていた.その時に参照したと考えられるのは,時間と空間の抽象化という考えと時間の存在しない4次元空間における同時性というキーワードであった.後者について補足しておくと,ベルクソンがカントの時間概念を批判して持続という概念を提唱して広く知られていたのに対して,2人のキュビストにとってはパヴロフスキーの同時性の4次元は,そのカウンター概念として現れたように考えられたのだった.いずれにせよ,クレールが注目するデュシャン以外のモダンアートの作家たちが作品から霊感を受けてそれぞれの絵画空間を構築していったことはパヴロフスキーの作品の重要性を示す.ヘンダーソンの非ユークリッド幾何学の文化史的な調査の影響は非常に大きく,ロシアで近年出版された数学の一般書で4次元を取り上げている本で,4次元の文化史に関する記述ではほとんどがヘンダーソンに依拠しており,その中で『四次元郷』を取り上げている\footnote{\foreignlanguage{russian}{Рауль Ибаньес, \emph{Четвертое измерение~: Является ли наш мир тенью другой Вселенной?~}, Москва, Де Агостини}, 2014, p. 96.}.

また,ヘンダーソンのパヴロフスキー論は彼の芸術論にも焦点を当てている点でも重要なものとなっている.ヘンダーソンはメザンジェとグレーズがパヴロフスキーに言及している一方で,パヴロフスキーの芸術観について保守的であったと指摘している.『コメディア』ではパヴロフスキーの采配で2人の評論家が雇われていたが,キュビズムなど新しいムーヴメントに対して批判的だったアルセーヌ・アレクサンドル(Arsène Alexandre)に記事で肯定的に言及していることや,メザンジェとグレーズの著作『キュビズムについて(\emph{Du Cubisme})』とアポリネールの『キュビズム画家(\emph{Les Penitres Cubistes})』の書評でもキュビズムの絵画に最後まであまり肯定的でなかったことをヘンダーソンは示している.パヴロフスキーのこうした絵画批評の他に,別の研究者によって演劇批評が注目されている.ナンシー・トロイ(Nancy J. Troy)は20世紀初頭のフランスにおけるオートクチュールとファッションショーの関係についての論文で,パヴロフスキーの記事を引用している\footnote{Nancy J. Troy, "Staging Haute Couture in Early 20th-Century France", \emph{Theatre Journal}, v. 53, n. 1, 2001, pp. 1-32.}.トロイによると,演劇における登場人物の服装がコマーシャリズムと結びついたのが20世紀初頭で,とくに「平和通り(\emph{La Rue de la Paix})」は様々な洋服店,いわゆるメゾンが軒を並べ,マネキンの陳列による服の展示がパレードの様相を呈していたという.1912年に公開された風刺演劇『平和通り』は,伝統的なクチュールと新しいスタイルを取り入れたクチュールの対立を戯曲の主題としており,実際のメゾンがその衣装を作っていた.この戯曲はそうした事情からファッションショーのように受容され,演劇がどうあるべきかという論争に発展した.トロイは論争から距離を取って冷静な分析を加えつつも,保守的な態度を見せる論者としてパヴロフスキーを紹介している\footnote{Troy, \emph{op. cit. }, p. 30.}.パヴロフスキーによれば,演劇がどうあるべきかという論争は商業的衝突が道徳的衝突と入れ替わった現象に過ぎず,文学においても同様のことが生じていると評している現代的な風潮であると判断している.しかし,商業的性格があからさまなのを嘆いてもいる.『四次元郷』でも1912年第13章1923年第13章「リヴァイアサンの劇場(Le Théâtre du Léviathan)」で戯曲には社会における個人道徳が表現されており,戯曲の内容の規制によってその社会のことを知ることができると論じられている(78/101).パヴロフスキーは自身の芸術論をまとめることがなかったが,こうした先行研究での文脈を足がかりに芸術論の全体像を描く必要があるだろう.

クレールとヘンダーソンのモダンアートの文脈におけるパヴロフスキーによってパヴロフスキーはモダンアート研究で知られる人物となり,文学研究の対象となることはほとんどなかった.書誌情報と活動歴などのパヴロフスキーの詳細が明らかになったのは1998年のことで,エリック・ヴァルベック(Eric Walbecq)の研究成果による\footnote{Eric Walbecq, «~Gaston de Pawlowski~», \emph{Le Visage Vert}, n. 4, Paris, Joëlle Losfeld, 1998, pp. 148-52.}.ヴァルベックの調査によって,パヴロフスキーのジャーナリストとしての活動は非常に旺盛なもので,調査された記事の量だけでも60巻程度になるような分量であるということが明らかになった.また,ヴァルヴェクは,2003年にはパヴフスキーの『四次元郷』以外の作品の復刊も行なっている\footnote{Gaston de Pawlowski,\emph{Paysages animées}, dir. Eric Walbecq et Jacques Damade, Paris, La Bibliothèque, 2003.}.

ヴァルベックの次にパヴロフスキーが研究で取り上げられてたのは,2001年には研究雑誌『ヨーロッパ(\emph{Europe})』でSF特集があった時に寄せられたフィリップ・キュルヴァル(Philippe Curval)の論考「シュルレアリスムとサイエンスフィクション\footnote{Philippe Curval, «~Surréalisme et Science-Fiction~», \emph{Europe}, v. 79, n. 870, octobre, 2001, pp. 32-50.}」である.キュルヴァルはクレールがパヴロフスキーの4次元の概念について『大いなる虚構』で述べている,観察者とその対象の関係が修正されうる可能性があるという考えを引用することで,最初の量子物理学的な作品であると独自の見解を示している.量子物理学者のエヴェレットが提起した観測問題などを背景とした記述であると考えられるが,パヴロフスキーは第5章で見ていくように原子は実在するものではないと考えていたことは明らかであり,量子力学に関する知識もなく,それが大衆的な科学知識として広まっていくのと同時期に亡くなったので,この解釈は牽強付会のきらいがある.

キュヴァル以降はあまり取り上げられることがないパヴロフスキーであったが,ブライアン・ステーブルフォード(Brian Stablefold)が『四次元郷』を自ら英語に翻訳して2009年に出版し,そのために執筆した「序文」は非常に重要な論考にもなっている\footnote{Brian Stableford, "Introduction", \emph{Journey to the Land of the Fourth Dimension},  London, Black Coat, 2011, Electronic.}.ステーブルフォードは,ヴェルサンやヴァルベックの提供するパヴロフスキーの活動歴を網羅し,その評価と独自の解釈を記している.ステーブルフォードがまず評価しているのは,『四次元郷』が他の未来予想小説とは異なって,その未来の描写が全く陳腐化していない点である.パヴロフスキーの未来史には他の小説に見られるような典型的な描写が見られないのである.その大きな要因として,パヴロフスキーがアルフォンス・アレーと引き合いによく出されているように,機知に富んだ表現を多用しているからであろう.また,ウェルズの『タイム・マシン』の影響があるとされながらも『四次元郷』ではウェルズの名前が挙がっていないことについて,未来の労働者の一人の名前が「H・G・28」であるのに着目して,それがウェルズのイニシャルを暗示しているという説を提示している\footnote{私はこの説に一定の留保を持ちつつも賛成している.HG28が登場するのは『四次元郷』においても極めて重要な「機械の反乱」という章であり,本論でも詳しく検討することになるが,パヴロフスキーが明示しない引用を多用する作家であることを考えると28という数字にも由来があると考えられる.これは,同じ4次元を取り扱った例として頻繁に取り上げられる2作品の関係を明らかにするためにも重要な今後の研究課題と言える.}.ステーブルフォードはクレールと同様に,挿話の1つ1つを取り上げて同時代の作品や思想家との影響関係を詳しく見ている.

ステーブルフォードは,パヴロフスキーの作品に見られる同時代の作品や思想家への参照が見られるのはパヴロフスキーがそれらを乗り越えようとしているからであるという.パヴロフスキーの交友関係を見ていくとガブリエル・タルドと知り合いであったと考えられており,『四次元郷』は人類を3次元から4次元に解放していると見ることができる点は,タルドの太陽が冷えた後の地球で暮らす人類の姿を描いたエッセイのような終末的な物語を乗り越えようとしていると考えられる\footnote{Gabriel de Tarde, \emph{Fragment d'Histoire Future}, Paris, V. Giard \& E. Brière, 1896.}.他にも,時間改変SFの初期の代表作であるシャルル・ルニヴィエの『ウクロニー(\emph{Uchronie})』やサン=シモンの未来予想なども乗り越えようとしていたと考えられている.さらに,ステーブルフォードは,クレールの「序文」と同様に各章についての解釈も簡単に示している.それぞれ簡単に見ておこう.本論第5章で私が考察することになる「引き裂かれた犬」の章では最初に人類は火星に知的生命体が住んでいると考えて通信を幾度か試みる.その最初に試みようとした火星人との通信方法の描写はギュスターヴ・ル・ルージュ(Gustave Le Rouge)に由来しているという.本論第6章で私が扱う「巨大バクテリア」の章では,バクテリアを巨大化させることによって退治するが,これはアンドレ・クヴルール(André Couvreur)の『バクリテリアの侵略(\emph{Une Invasion de Macrobes})』(1908)を踏まえていると考えられる.

また,ラーゲが数学的に解説を施している箱をめぐる挿話も参照先があるという.天文学者で心霊主義者のカール・フリードリッヒ・ツェルナー(Karl Friedrich Zöllner)の『超越論的物理学(\emph{Transcendental Physik})』(1878)では,封印された箱が手を触れずに開いてしまうという挿話があり,それに基づいたエピソードであると推測されている.パヴロフスキーが直接ツェルナーの著作を読んだ証拠はないものの,心霊主義が流行していた世紀末のフランスでは,ウィリアム・クルックスの4次元と心霊研究が紹介されていた.パヴロフスキー自身もスピリチュアリストたちと関係があった.文通相手だったジュール・ボワ(Jules Bois)が執筆した『永劫回帰(\emph{L'Eternel Retour})』(1914)の内容を出版前に知っていたこと,それに加えてボワが所属していた薔薇十字団の先輩だったジョセフ・ペラダン(Joseph Péladan)は『コメディア』に記事を寄せたこともある.心霊現象の描写については,この他にも,心霊現象がやや皮肉を込めて描写されているものの,4次元の理論によってそれらは全て実際に起こっていることだとパヴロフスキーは説明している.ステーブルフォードによれば,パヴロフスキーはネオプラトニズム的な観念論に依拠しているために,心霊現象の描写をすることで,実体としての精神を描くことになってしまったと考えられている.

ステーブルフォードは「序文」の末尾でパヴロフスキーの4次元はウェルズやその他の多次元空間を扱う他の作家ととは違って,量的な高次元を拒否し,時空間から離脱しており,4次元の質に依拠することで外側から時間を見つめて未来史をまとめていると結論づけている.しかし,ステーブルフォードの「序文」の結論から,異なった解釈を引き出している論者がいる.その論者について,以下で見ていこう.

エレーナ・ゴメル(Elana Gomel)は大衆小説と見なされ文学史に組み込まれていないSF作品と文学史に登録されている作品を統一的に把握する視座を提案するために,非ユークリッド幾何学以後の時空間認識の詩学の構築を試みる著作『物語の空間と時間(\emph{Narrative Space and Time})\footnote{Elana Gomel, \emph{Narrative Space and Time~: representing impossible topologies in literature}, New York, Routledge, 2014.}』の中でパヴロフスキーに言及している.ゴメルは非ユークリッド幾何学の影響を受けた英文学作品の代表的な2つの作品であるウェルズの『タイム・マシン』とジョージ・マクドナルド(George MacDonald)の『リリス(\emph{Lilith})』から始まるジャンルの物語群があるとしている.このジャンルには『四次元郷』も分類されているので,ここでは『タイム・マシン』を例にとることでこのジャンルについて詳しく説明する.

『タイム・マシン』のあらすじを簡単に見ておこう.時間を移動する機械「タイム・マシン」を発明した主人公は紀元~802,701年に移動して,階級社会によって2分化された人類がイーロイとモーロックという2つの異なる人種としてそれぞれ進化を遂げている世界に出会う.その後,その世界の見聞を広めているうちに,あるイーロイ人を守るためにモーロック人を殺害してしまう.その世界から帰還した主人公は周囲に未来の世界のことを話し,その後時間の中に再び旅立っていく.

ゴメルがこのあらすじの中で着目しているのは,主人公が運命論者としてこの未来を避けられてないものと述べている箇所が散見されているのに対して,決定されているはずの未来に介入する行動ばかりしており~---~モーロック人の殺害,未来から花を持ち帰るなど~---~,タイムパラドックスに陥っている点である.ゴメルは,非ユークリッド幾何学の大衆化によって空間を移動するようにして時間を移動するという考えが一般化したため,こうしたパラドックスを避けることができたと考えている.同じことが『リリス』でも同様のことが指摘できるので,こうした作品群をゴメルは「回避(Sidestepping)」と名付けている.「回避」の作品は,20世紀に入ると非ユークリッド幾何学の大衆化によって多次元空間を扱う小説と結びつき,ユートピアやディストピアといった現実には不可能な場所と現実の空間をつなぐための方法論となっていく.ところで,ゴメルは『四次元郷』を「回避」と多次元空間の結びつきの優れた例だと考えている.なぜなら,多次元空間である以上はそこにはユートピアとディストピアが互いを回避しあい,別の時空間として物語の中で展開されているはずだからである.その点,リヴァイアサンによる支配とその解放,科学者たちの支配から4次元への人類の解放といったユートピアとディストピアを物語の展開で示している『四次元郷』は「回避」の好例となっているのだ.しかし,「回避」の範例として完全であるために,かえって物語的な叙述を失っている\footnote{Gomel, \emph{op. cit. }, p. 157.}.また,ゴメルはパヴロフスキーの4次元の描写や心霊主義的なエピソードが多用されていることから,パヴロフスキーは結局のところ,反科学的であると述べている\footnote{\emph{Ibid. }, p. 158.}.反科学的な思想によって4次元を数学で扱う量的な空間ではなくてスピリチュアルな場所と扱われているのだとゴメルは考えている.

最初に,ゴメルはステーブルフォードの結論から別の解釈を引き出していると述べた.ここで改めて確認しておくと,ステーブルフォードは,パヴロフスキーは量的な高次元の拒絶し,時空間から離脱して外側から時間を見つめることによって未来史をまとめていると述べていた.この結論をゴメルは「回避」の文脈に置き直すことで,量的な次元とは空間と時間を同じようにみなすことなので,運命論を時間の空間的な移動によって別の物語の軸に接続するウェルズに対して,運命論的な未来の歴史をすべて放棄するのがパヴロフスキーだと述べる.ステーブルフォードとゴメルの結論の微妙な差異は歴史を物語る行為の問題に発展するもの,ここで一旦措くことしよう.

以上で中心となる『四次元郷』の研究をみてきた.近年でもパヴロフスキーや『四次元郷』に関する研究が定期的に発表されているが\footnote{カンファレンスでは,『コメディア』が取り上げられたり,幻想文学とSF文学の境目が曖昧であった頃の作品を扱うカンファレスなどで定期的に言及されている.以下の2つを参照のこと.Sophie Lucet, «~Gaston de Pawlowski et le journalisme ou l’art de l’échappée~», Comœdia (1907-1937). un continent inexploré dans l'histoire du théâtre, Bibliothèque Seebacher, Université Paris-Diderot, 21 juin 2014. Patrizia D’Andrea, «~Voyage au pays du silence~: rêve, philosophie, utopie entre symbolisme et anticipation (Han Ryner, Gaston de Pawlowski)~», Mobilités dans les récits de fantastique \& de science­fiction (XIX~-~XXIe siècles)~: quête \& enquête(s), l'IUT Sénart-Fontainebleau, Université Paris Est Créteil, Jeudi 20 Novembre.},ここではその詳細を扱わない.最後に,先行研究を受けての本論の位置を示す.

先行研究は論者と合わせて以下の点にまとめることができる.
\begin{itemize}
  \item SF研究において
    \begin{itemize}
      \item パヴロフスキーは,フランスで4次元をテーマに扱った代表的な作家であり,作風はユーモアを基調としていた.
    \end{itemize}
  \item モダンアート研究において
    \begin{itemize}
      \item 非ユークリッド幾何学を独自に理解し,4次元の同時性と不動性を強調し,観察者とその対象の関係を相対的なものとした.
    \end{itemize}
  \item 物語と歴史をめぐる論点
    \begin{itemize}
      \item 同時代の未来を描く物語に対して別のアプローチを行い,量的な次元を拒否しすることで,未来史の記述を可能としている,あるいは,運命論的な未来の歴史を拒絶している.
    \end{itemize}
  \item 参照関係
    \begin{itemize}
      \item 各章のテーマや登場するモチーフが同時代に参照点を持っている.
    \end{itemize}
\end{itemize}

以上の先行研究を踏まえて,本論のアウトラインを示すことにしたい.私は,ヘンダーソン,ステーブルフォードのパヴロフスキーの4次元の分析のうち,とりわけヘンダーソンが提示するいわば「4次元の文化史」に基づいてパヴロフスキーの4次元の特徴について第4章で見ていきたい.次に,4次元や同時代に同じように非ユークリッド幾何学の影響を受けている別の作品をめぐる影響関係ではなくて,ヴァルベックがパヴロフスキーのジャーナリストとしての盛んな活動に注目していたのをふまえて,出版史における科学とジャーナリズムの関係に基づいて『四次元郷』の背景を分析してみたい.そこで,第5章では,19世紀末のフランスのジャーナリズム,とりわけ『四次元郷』が新聞に登場した時の形式である「連載」の手法に注目しつつ,それがどのように科学記事に影響を与え,『四次元郷』が生まれる土壌を作っていったのかを見ていきたい.

そして,作品の成立背景のほかにも,私は新しい論点を追加したいと考えている.それは,4次元の文化史とは異なったテーマについての,科学の大衆化に注目した読解である.というのも,『四次元郷』はそのタイトルから4次元の世界に入り込んで冒険するようなニュアンスが与えられているにもかかわらず,描かれているのは未来の人類の姿なのであって,別の次元での別の知的生命体の物語を描いてはいないのだ.3次元の世界の未来を描いているのである.すると,4次元をどのように描いているかよりも,『四次元郷』の3次元の世界はどのように説明されているのかという問いも非常に重要となる.そして,私は,本論にて,3次元を理解する上で重要だと考えられるテーマをエネルギー・進化論・遺伝という19世紀の科学の大衆化によって,4次元と同じように人々に知られるようになった知識を手がかりに『四次元郷』を解釈する.以下で,その概括をする.

私は,『四次元郷』の中で頻繁に用いられる物質(matière)というキーワードを手がかりに2つの論点を提示する.まず,3次元におけるエネルギー(énergie)との関係について扱う.物質は分離(dissociation)によってエネルギーが生じるとパヴロフスキーは述べており,このエネルギーと4次元との関わりについての記述から『四次元郷』において3次元がどのように描かれているのかを,第5章で私は考察する.次の論点として,物質から生命へ,というテーマを扱う.『四次元郷』ではある章のエピソードで物質から生命が誕生するという説が紹介されている.第6章では,1912年版と1923年版の異同に注目しながら,進化論と遺伝という潜在するテーマがあることを示し,そこから「遺伝的類似」という4次元と遺伝のアナロジカルな関係を示す概念が取り出せることを明らかにする.
