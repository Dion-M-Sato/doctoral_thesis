\documentclass[uplatex,a4j,11pt]{jsreport}
\renewcommand{\abstractname}{要旨}
%\renewcommand{\baselinestretch}{1.1}
\usepackage[vdivide={3cm,,3cm}]{geometry}
\usepackage{fixltx2e}
\usepackage{setspace}
\usepackage{wrapfig}
\usepackage{setspace}
\usepackage{indent}
\usepackage{fancybox}
\usepackage{type1ec} %T2Aなどを使う.
\usepackage{textcomp} % T1のシンボルを使う
\usepackage[T2A,T1]{fontenc}%フォントでT2AとT1を使う
\usepackage[utf8]{inputenc}% ファイルがUTF8であること
\usepackage{zi4}%等幅フォントをInconsolataで
\usepackage[multi,deluxe,uplatex,jis2004]{otf}%繁簡ハングル多ウェイトOpenTypeなし
\usepackage[prefernoncjk]{pxcjkcat} % なるべく「半角」扱いで.
\cjkcategory{sym18}{cjk} % sym18 (U+25A0 - U+25FF Geometric Shapes) を和文文字あつかい
\usepackage[russian]{babel} %babelで多言語対応
\usepackage[main=japanese]{pxbabel} %基底言語は日本語
\usepackage{pxjahyper}
\usepackage{plext} %傍点をふる
\usepackage{pxrubrica} %ルビをつける
\usepackage{comment} %コメント環境の用意
\setcounter{tocdepth}{3}
\renewcommand{\thefootnote}{\arabic{footnote}}
\usepackage[dvipdfmx]{graphicx}
\newcommand*{\tabref}[1]{表~\ref{tab:#1}} %表のカウント
\newcommand*{\figref}[1]{図~\ref{fig:#1}} %図のカウント
\newenvironment{mytab}[3][htbp] %表を中央揃え
 {\begin{table}[#1]\begin{center}\caption{#2}\label{#3}}
 {\end{center}\end{table}}
\newcommand{\myfig}[4][width=.8\linewidth]{ %図を中央揃え
\begin{figure}[htbp]
   \centering\includegraphics[#1]{#2}
   \caption{#3}\label{fig:#4}
\end{figure}}

\title{修士学位論文\\{\Large ガストン・ド・パヴロフスキー『四次元郷への旅』について}\vspace{7cm}}
\author{平成29年度\\言語情報科学専攻\\31-166004\\佐藤正尚}
\date{\empty}

\begin{document}
\maketitle
%\thispagestyle{empty}
\newpage
\tableofcontents
\begin{center}
  {\Large 凡例\\}
\end{center}
\vspace{2cm}
\begin{itemize}
  \item 本文中引用での原文イタリック体または和文傍点はすべて原著ママ.
  \item 本文中にとくに断りのなく(数字/数字)が挿入されている場合,斜線左側には\emph{Voyage au pays de la quatrième dimension}の1912年版の引用頁数を,左側には1923年版の再版であるDenoël版の引用頁数を示している.例えば,(8/59)とあれば,1912年版8頁と同一の文章はDenoël版の59頁にあることを示している.
  \item 1912年版と193年版の章題は常に併記する.ただし,版ごとに章題の異同がある場合,それも併記する.例えば,1912年版第31章「自然の力の彼方へ(Au delà des forces naturelles)」1923年版第34章「自然の形態の彼方へ(Au delà des formes naturelles)」,表記した場合は,1912年版においては第31章である「自然の力の彼方へ」は,1923年版において第34章に変更され,章題が「自然の形態の彼方へ」となっていることを示している.
  \item 本文中にとくに断りなく(1912, 10)や(1923, 10)と挿入されている場合,それぞれその版にのみある文章の頁数を示す.例の場合,前者は1912年版10頁,後者は1923年版10頁からの引用を示す.
  \item 引用箇所で和文で【 | 】という記号がある時,縦棒左側は1912年版を右側は1923年版を示す.また,フランス語の原文で\{ | \}とある場合,縦棒左側は1912年版を右側は1923年版を示す.
  \item 引用した原文中に[= ]という記号がある時,原文の指示語の補足である.ただし,和文中に同様の記号が用いられている場合のみ,それは文脈の補足であり,必ずしも原文に対応する文章があることを意味しない.
  \item \emph{Philosophie du Travail}からの引用は例えば(PT10)と略記する.この場合,10頁からの引用を示す.
  \item \emph{Sociologie Nationale. Une définition de l'état}からの引用は例えば(SN10)と略記する.この場合,10頁からの引用を示す.
  \item (中略)ないし(...)と示した場合,それは引用箇所の一部を省略していることを示している.
  \item 『』の鉤括弧は書名を示し,『』(数字)とある場合,原著年の刊行年を()内の数字で示す.例えば,『科学と仮説』(1912)とあれば,『科学と仮説』の原著は1912年に刊行されたことを意味する.
\end{itemize}

\chapter{序文}
ガストン・ド・パヴロフスキー.フランス人,ジャーナリスト,小説家,自転車愛好家.彼は1912年に『四次元郷への旅(\emph{Voyage au pays de la quatrième dimension})\footnote{日本では,一般的に『四次元国への旅』と訳されることが多い.パヴロフスキーは国家論に関する著作を残していることから,「国」を意味するのであれば\emph{État}という語を用いると考えられる.さらに,パヴロフスキーは作品の冒頭でその題名における「\bou{4次元}とは,それ自身が,総合的な状態に他ならない(\emph{quatrième dimension} n'est, elle-même, que la manifestation d'un état synthétique)」(8/59)と述べているように,それは私たちがイメージするような国のことでないことが示されている.このことから,郷土や黄金郷といったようなどこかの場所を示す\emph{pays}だと考えるのが適切である.以上の理由から,「四次元郷」という訳語を用いた.}』(以下,『四次元郷』)を世に送り出し,世紀末から知識人たちの間で流行していた4次元思想をフランスで普及する第一人者となり,芸術家たちに4次元を題材にした作品を制作させるほどの影響力を持っていた.その後,アインシュタインの一般相対性理論が登場し,ミンコフスキ空間では4次元が幾何学的に計算可能な対象に過ぎないことが明らかになってなお,パヴロフスキーは自身の4次元の概念を捨てることはなかった.アインシュタインの理論が人口に膾炙するようになる頃に彼は心臓発作で亡くなり,そのうちに人々は4次元の崇高さを完全に忘れてしまった.

パヴロフスキーは研究の主題となることがほとんどないので,日本でもあまり知られていない.20世紀の最後の年に英訳された他に,私家版をおくとして,翻訳は管見の限り確認されていない.それでも日本でこの名前を何度か言及した人物は存在して,その最初の例として中沢新一を挙げることができるかもしれない\footnote{中沢新一「四次元の花嫁」,『東方的』,講談社,2012年,48-120頁.}.彼のエッセイは,南方熊楠の書庫の探索から始まり,高次元空間に神秘主義的な期待を寄せた人々の思想が紹介され,最後に,現代の宇宙物理学から提示された超弦理論に触れて,この宇宙には高次元が現実に存在している可能性があることを示している.

確かに,現実に高次元空間があるかどうかという宇宙論は興味深い話題であり,いずれ天体観測のデータから宇宙の曲率を割り出して,私たちの現実がいくつの次元で構成されているのか知ることになるだろう.しかし,そもそも高次元空間の理解なしに多くの学問領域や私たちの生活はもはや立ちゆかなくなっていることを理解しておく必要がある.例えば,線形代数という数学の理論を取り上げてみよう.線形代数は統計学から計算機科学まであらゆる領域で基礎となっている一連の数学理論である.線形代数はベクトルを代数的に扱うことができて,ベクトル空間という幾何学的対象を計量することができる.ベクトル空間は実数や複素数の集合を扱うため,ベクトルが含んでいる集合の要素の数だけ次元が高くなる.4個の要素をとれば4次元となり,1000個の要素でとれば1000次元となる.ベクトル空間は,要素の数だけ無限に増えていくn次元空間なのだ.このn次元空間を扱う線形代数は,21世紀になって再び流行語となった人工知能の基礎となっている機械学習でも前提となっている.人間にとって認識できない高次元空間は数学的な概念道具としてなくなてはならない存在だ.

ところで,パヴロフスキーの『四次元郷』はこうした高次元空間を利用する未来を予想していたわけでは全くない.ハーバード・ジョージ・ウェルズ(Herbert George Wells)の『タイム・マシン(\emph{Time Machine})』(1895)を代表として,様々な論者が4次元に熱狂していたこの時代に,彼もまた4次元に魅せられ,時間と空間をめぐる理論を構築していくなかでこの小説を書いた.しかし,彼が残したこの小説は,4次元とは関わりのないような,科学技術に対する風刺のきいたグロテスクなコントや,哲学的なエッセイによって構成されている.そのため,先に紹介した中沢は,『四次元郷』を人類の愛が4次元への鍵となっているといったように,ごく一部分の要素だけを切り取って物語を解釈することはできない.『四次元郷』は複雑なテーマを有した物語なのだ.

中沢の他にも,これまで多くの論点が与えられてきた『四次元郷』について,本論はそれらをまとめた上で,新しい論点を提示する初めてのモノグラフとなるだろう.本論は,この著作を構成するエピソードの中に隠されたイメージを繋ぎ合わせることで,19世紀フランスにおける科学の大衆化によって人口に膾炙した4次元と遺伝というテーマが伏在する『四次元郷』の姿を明らかにする.

\chapter{ガストン・ド・パヴロフスキーと『四次元郷への旅』について}
\section{ガストン・ド・パヴロフスキーの来歴}
ガストン・ウィリアム・アダム・ド・パヴロフスキー(Gaston William Adam de Pawlowski)は,ヨンヌ(Yonne)のジョワニィ(Joigny)に1874年6月14日に生まれ,パリで1933年2月2日に亡くなった.彼の父であるアルベール・ド・パヴロフスキー(Albert de Pawlowski)は鉄道会社「西武鉄道(Compagnie des chemins de fer de l'Ouest)」に勤める技師だった.母のヴァレリー・ド・トリオン=モンタランベール(Valérie de Tryon-Montalembert)は,家門の貴族(noble d'extraction)で14世紀頃のド・トリオン家まで遡ることができる貴族の血筋にある人物だった.パリ市内のリセ・コンドルセ(Lycée Condorcet)に通っていた頃に,彼の遊び仲間を通じて,『ユビュ王』や『超男』の作者として知られるアルフレッド・ジャリと交友関係を結んでいた.『超男性』は自転車乗りが主人公であるが,ジャリはパヴロフスキーらとツーリング仲間だったことが知られている\footnote{Alastair Brotchie, \emph{Alfred Jarry~: ein pataphysisches Leben}, Bern, Piet Meyer, 2014, p. 72-3}.その後,バカロレアでは文学と科学の二つをパスして進学し,エコール・ド・ルーヴル(école du Louvre)に通うかたわら,パリ政治学院(école des sciences morales et politique)の法学部(La Faculté de droit)に入学した.1901年,博士論文に『労働の哲学\emph{Philosophies du Travail}』を提出して,法学博士になり,同年に博論を出版した. 博士論文を提出する以前から,パヴロフスキーは雑誌編集に関わるようになる.

パヴロフスキーの編集者としての活動は大学生時代から始まっており,1894年から『自転車(\emph{Vélo})』と『自動車雑誌(\emph{Journal de l’Automobile})』などの編集や記事の執筆をしていた.現在のスポーツ誌の編集者と言える.その後,『オピニオン(\emph{l’Opinion})』では政治記事を執筆し,『ル・ジュルナル(\emph{Le Journal})』では芸術時評の担当をした.最終的には『ル・リール(\emph{Le Rir})』でユーモア風の記事を書くことに専心することになり,その嗜好がのちに創刊されることとなる雑誌の基本的な方針となっていった.

パヴロフスキーがその生涯のほとんどを通じて活動したのが,ラテン語の「喜劇(commediae)」を思わせるタイトルを持った日刊紙『コメディア(\emph{Comœdia})』が,ツール・ド・フランスを主催したアンリ・アントワーヌ・デグランジュ(Henri Antoine Desgrange)が出資して1907年10月1日に刊行された.パヴロフスキーは編集長となったが,1914年に第一次世界大戦が始まると徴兵され,自動車整備のエンジニアとなり,同年8月に廃刊した.1919年には軍役を終えると,同年10月1日に復刊し,編集長を再び務めた.そして,1923年まで『コメディア』に身を捧げた.二度目の編集長となった頃,1921年にはマルグリット・マンガン(Marguerite Mangin)と結婚した\footnote{1912年に刊行された『四次元郷』が独身者の文学の系譜に並ぶような作風であるのに比べて,1923年版には愛についてのエピソードや考察が増えているのはこの結婚と無関係ではないだろう.}.

『コメディア』は刊行から3年で28,000部の売り上げた\footnote{\emph{Histoire générale de la presse française}, dir. Claude Béllanger, et al. ,t. 3, Paris, Presses universitaires de France, p. 296.}.同紙は,演劇・文学・芸術を中心に取り上げ,政経営者たちを保守的なクラブやサロンの社交界から募集していたからか,政治的話題を直接扱うことはなかったという点でユニークな新聞で,「その時代にとってそれに相当するもののない,フランスにいながら外国の雑誌であった(\emph{Comœdia} était sans équivalent pour l'époque, en France comme à l'étranger)\footnote{Claude Béllanger, \emph{op. cit. }, p. 381.}」.当時では珍しかった作家インタビュー記事を記載するなど,同時代の雑誌に比べて先進的でもあった\footnote{以上は下記を参照のこと.Claude Béllanger, \emph{op. cit.} , pp. 381-2. ただし,マリー・エヴ・テランティ(Marie-Ève Thénty)が指摘しているように,19世紀の終わりになる頃には電報の発達により速報性が高くなったメディアは,ルポルタージュ・三面記事・インタビューの3つが日刊紙の中心となっていることが明らかとされているのでインタビュー記事を載せること自体はそれほど特別な判断ではなかったと考えられる.Marie-Ève Thénty, \emph{La Littérature au quotidien. Poétiques Journalstique au XX\textsuperscript{e} siècle}, Paris, Seuil, 2007.}.パヴロフスキーは,週刊連載で書評欄「今週の文学(Semaine Littérature)」を担当していたが,マルセル・プルーストの『失われた時を求めて スワン家の方へ』を書評した際に,プルースト本人から,パヴロフスキーが作品をベルクソンの理論を敷衍して批評したことに反論の手紙が寄せられたことは文学史的にも注目するに値する事件だろう\footnote{Marcel Proust, \emph{Correspondance de Marcel Proust}, éd. Philip Kolb, t. 3, Paris, Plon, 1976, pp. 54-5.}.当時のパヴロフスキーの様子を伝える貴重な証言として,『コメディア』で芸術部門を担当していたコラムニストのアンドレ・ヴァルノー(André Warnod)の言葉を紹介する.1955年の回想録に「ガストン・ド・パヴロフスキーは普通の物差しから外れた男だった.その精神と知性のように大男だった.彼はラブレーの主人公のように見えた.今という時間を超えていたのは,思考の方法のうちよりかは存在のあり方においてだった(Gaston de Pawlowski était un homme qui échappait à toute commune mesure. Il était d'une taille gigantesque, comme son esprit et son intelligence. Il faisait figure de héros de Rabelais. Il était hors du temps présent, aussi biens dans sa façon d'être que dans sa façons de penser)」.

編集長を辞めた後,公に発表された最後の小説は今までのところ1925年に確認されており,『レ・アナル(\emph{Les Annales})』にドラリュ=ヌーヴェリエール(Delarue-Nouvellière)の挿画が添えられた「どこに行くのか?(Où allons-nous)\footnote{Gaston de Pawlowski, «~Où allons-nous?~», \emph{Les Annales politiques et littéraires~: revue populaire paraissant le dimanche}, dir. Adolphe Brisson, 6. Déc. 1925, Paris, [s. n.], pp. 581-2.}」という短編だった.『四次元郷』の登場人物の一人である「イドロジェーヌ(Hydrogène)\footnote{水素を意味する単語.}」の20世紀に対する風刺が描かれている.それ以後も,散発的に評論文を書いていたが,この8年後,1933年2月2日に亡くなり,『パリ・ミディ(\emph{Paris-Midi})』が一報を出し,翌日の『コメディア』には,その時に編集長だったガブリエル・ボワシー(Gabriel Boissy)が追悼記事を寄せている\footnote{Gabriel Boissy, «~Gaston de Pawlowski premier rédacteur en chef de «~Comœdia~» est mort prèmaturément hier~», \emph{Comœdia}, 3 Fév 1933, pp. 1-2.}.とりわけ『四次元郷』の作者として著名だったパヴロフスキーを讃えて次のように述べている.

\begin{quote}
 結局,この4次元は,時間や例の相対性[=相対性理論]から予想されることとなり,パヴロフスキーは予知するような仕方でそれを研究しつつ,1912年から早くも,アインシュタインの先取りをもたらしていたのだろうか.
\end{quote}
\begin{quote}
 Enfin, cette quatrième dimension, qui ne faisait que présager du temps et de sa relativité, Pawlowski en l'étudiant de façon divinatoire, n'apportait-il pas, dès 1912, une anticipation d'Einstein?\footnote{\emph{Ibid. }, p. 1.}
\end{quote}

アイシュタインの1905年以降の特殊相対性理論に関する断続的な研究は,徐々に受け止められ,1915年に一般性相対性理論が発表されて以来,相対性理論は1つの流行語となっていた.パヴロフスキーに対するこのような評価も,そうした文脈に基づいていると考えられる.しかし,後で見るように,少なくともアインシュタインと4次元の2つのイメージが大衆的な教養として定着する以前から4次元についての研究や『タイム・マシン』のような小説が書かれていたので,「先取り」とは言うことはできない.ただし,パヴロフスキーは彼の哲学的テーマを背景とした独自の4次元の概念を構築した.

ボワシーの追悼記事の終わりにはパヴロスキーの生涯とその最期が簡潔にまとめられている.フェザンデリー通り(la rue de la Faisanderie)の自宅で2月2日の朝6時に心臓発作で死亡した\footnote{\emph{Ibid. }, p. 2.}」.翌日の記事の詳細によれば,パヴロスキーの葬式は,午後2時から開催され,パリ8区ロケピーヌ通り(rue Roquépine)の5番地にあるサン=エスプリ教会(l'église du Saint-Esprit)にて行われて,ペール・ラシェーズに埋葬されることとなった.

\section{『四次元郷への旅』の書誌情報とあらすじ}
\subsection{書誌情報}
パヴロフスキーが編集長を務めている間に『コメディア』で連載された『四次元郷』は,パヴロフスキーの自身説明によれば,博論を提出する前の1895年から書き始めていたという\footnote{以下の記述を参照のこと.「時間探索についての最初の物語を書き始めた1895年の初めから,この『四次元郷への旅』の初版が刊行された1912年まで(Depuis le début de 1895 où j'écrivais un premier conte sur l'exploration du temps, jusqu'en 1912, date à laquelle parut la première édition de ce volume [= \emph{Voyage au pays de la quatrième dimension}])(1923, 9)}.作品の前身となった「旅(Voyage)」シリーズは1908年11月8日に「奇妙な旅(L'Etrange Voyage)」と題されて始まり,11月22日に「未来の話(Contes futurs)」にシリーズ名が改められ,「幻視者(Visionnaire)」が掲載された.12月13日には再びシーリズ名が改められて「超人時間物語(Récits ds Temps Surhumains)」と題されて,「死んだ愛(l'Amour Mort)」が掲載された.その後,『四次元郷』の前半のシリーズである「リヴァイアサン(\emph{Le Léviathan})」は1909年12月24日から1910年10月31日にかけて掲載されたが,この時にはまだタイトルは書籍と違っていた.書籍名と同じタイトルになって連載が再開し,1912年2月24日から1912年12月9日の間に不定期に連載された.連載の終了した1912年にファスケル(Fasquelle)社から初版が2000部刊行され,翌年にはさらに1000部が増刷された.第一次世界大戦の後,1923年に決定版が出版された.オランダの象徴派画家であるレオナルド・サルリュス(Léonard Sarluis)によるイラストや,「批判的吟味(examen critique)」と題した序文が追加され,内容面でも1909年に出版された『長枕~---~動く景色,空想の景色(\emph{Polochon~: paysages animées, paysages chimérique})』の一部が21章に「死んだ愛(L'amour mort)」として新しく付け加えられるなど,他の章についても削除や書き直しなど全面的な変更がなされている.10年に渡る書籍の来歴を見ると,ボワシーによる述懐が示しているように,20世紀前半のフランスにおける4次元に関する言説の一角を占めていていたことがわかる.

\subsection{あらすじ}
『4次元郷』は邦訳がなく物語の全体があまり知られていないため,以下にあらすじを示す.

語り手(パヴロフスキー)はある日,不思議なインド風の箱(le coffret hindou)に手紙をしまって,紐で結んで封蝋をした.彼はしばらくして中を検めることにした.手紙は難なく確認することができたのだが,パヴロフスキーは結んでいた紐がいつのまにか解けて,さらには封蝋もなくなっていたということに気づいた.慌ててまた箱を確認すると先ほどまで開けることができた箱が元のように封印されていた.パヴロフスキーはこの経験から天啓を得ることとなった.彼はフェリックス・クラインのクラインの壺の結び目のことを思い出し,その壺の結び目もまた3次元では解けてしまうこともあるのを思い出す.その後,パリに存在しないはずの「中駅(Gare du Midi)」を見てしまったり,森の中で正面玄関しかない平面の家(maison plate)と遭遇することが重なり,4次元の実在を確信するようになる.あらゆる場所と場所が短絡してしまうという空間の「抽象化(abstraction)」から,時間移動の秘密を理解するようになった語り手はついに4次元を通して,未来への移動を可能とする.

未来を見た語り手によれば,1912年を境にして,世界は大きく変貌していく.そして,かつてホッブズがその存在を予言したリヴァイアサンという超巨大生命体によって人類は支配されてしまう.リヴァイアサンはただの象徴的な存在ではなく,個人の道徳や科学的な考え方や方法までも規定してしまう人類にとっての上位存在の怪物として描かれている.人間はリヴァイアサンの支配する社会の細胞に還元されることで個人としての意志を失い,社会から切り離されて生きることができなくなってしまっていた.

人々はリヴァイアサンの支配に気づかずにいたが,いくつかの事件を通じて,統一された観念に向かっていくものではなくて,個別の分析に個別の観念が備わっているものであるという考え方が主流になり,それがあらゆる分野で進んだ.その結果,統一された観念を要請するリヴァイアサンを拒絶することになった.そうして,科学の時代が訪れる.

科学の時代は大きく分けて2つの時期に分けられる.第一の科学時代では,科学技術の発達とともに,人知の及ばない事件が多発するようになる.その中でも物語のあらすじと関係して最も重要なものは,「機械の反乱」の章で描かれる機械の暴走である.工場の機械があたかも生物のように自律的に動き回って人間を翻弄させたことをきっかけにして,物質にも生命が宿っているのであることが知られるようになる.その一方で,極端な機械化の果てに,個人は機械の所有者の奴隷のような状態になっており,人間は愛といった感情を忘れてしまっていた. 第二の科学時代になると,最終的に世界で最も科学研究が進んだ「大中央研究所(Grand Laboratoire Centrale)」によって世界は支配されるようになる.この学者たちは不死の秘密を手に入れ長期間にわたってその支配の座を譲らなかったが,生殖という概念が改めて取り上げられ実験がなされ,科学者の支配を通じて人類が忘れてしまっていた愛を再発見する.そして,世界が隅々まで機械化されることで人間が自らを維持するためのエネルギーを労働することによって得る必要がなくなった.

科学の時代が進むにつれて,科学主義の唯物論的世界観から,観念論が再び世界の思想の中心となる.それは黄金鳥の時代(l'Oiseau d'or)または黄金鷲の時代(l'Aigle d'or)\footnote{黄金鷲に名前が1923年版になる時に追加されている.パヴロフスキーは,「\bou{黄金鷲の時代}と呼ばれていて,時により親しみやすく,\bou{黄金の鳥}の時代と呼ばれている(on l'appela l'\emph{Age de l'Aigle d'or} et parfois plus familièrement, celui de l'\emph{Oiseau d'or})」(1923, 231)と述べている.黄金鷲の時代という呼称が一般的ではないということの理由は明記されていない.ただし,鷲を\emph{Oiseau de Jupiter}と呼称することがあることとこの時代の理想的なモデルが古代ギリシャに由来していることを考えると,この語の持っている古代ギリシャ的なニュアンスを意識していると考えられる.}と呼ばれている.フランス人権宣言から2000年後の3798年に「物質と自然の権利(les Droits de la matière et de la nature)」が宣言され,世界が固有の「原子(atome)」によって様々な形を変えた表現であることが確認され,四次元の秘密が人類に明かされる.私たちの生きている世界はただ1つの原子によって織り成されている無数の差異の集合で,人類は精神によって4次元に至り,個人が誰かに隷属することもなく,労働もする必要がない自由を手に入れることができたのだった.

以上のあらすじはまとめると表~\ref{tab:age}のようになる.

\begin{mytab}[htbp]{物語中の時代}{tab:age} \begin{tabular}{c|c|c}
  連合 & リヴァイアサン &  \\
  ↓& ↓ &   \\
  分離 & 第一・第二の科学時代 & 唯物論の時代\\
  ↓ & ↓ &  \\
  統一体(unité) & 黄金鳥(黄金鷲)の時代 & 観念論の時代\\
\end{tabular}
\end{mytab}

\section{先行研究}
『四次元郷』は多くの文学者や芸術家に影響を与えてきたが,作品研究はあまり進んでいない.しかし,世に問われて以来,高次元や非ユークリッド幾何学を扱った小説を挙げる際にほとんど必ず言及される作品である.例えば,1948年に出版されたル・リヨネの編纂した『数学思想の流れ(\emph{Les grands courants de la pensée mathématique})』に所収されているアンドレ・サント・ラーゲ(André-Sainte Lagüe)の小論「四次元の旅」も,4次元を扱っている作品として『四次元郷』を挙げている.数学の専門知識と大衆におけるイメージの落差を語りながら,位相幾何学における思考実験とその検証をわかりやすく説明している.その中で『四次元郷』が例に挙げられ,物語の冒頭に登場する箱について,数学的には3次元の結び目を4次元で解くことが可能であることが説明されている\footnote{アンドレ・サント・ラーゲ「四次元の旅」『数学思想の流れ』, ル・リヨネ編,村田全監訳,東京図書,1974年,pp. 174-5.}.戦後まもなく4次元を題材にした小説であればパヴロフスキーの名前が挙がっていたことを示す貴重な記録となっている.しかし,大衆向けの科学小説に関する研究が始まるようになってからも,パヴロフスキーが研究として取り上げられることはほとんどなかった.原因はいくつか考えられる.主要なものとして,パヴロフスキーがジャーナリストとしての活動を中心に行っており,その時の見聞が『四次元郷』には大きく反映されているものの,作品中の表現の典拠を新聞記事に求めることや,1912年の初版を手に入れるといったことが困難であったと考えられる.しかし,電子アーカイブズの登場と漸進的な電子資料の登録数の増加によってこうした問題は解決されつつある.本論が以下に示すこれまでの研究と大きく方向性が異なっているのは,そうした資料が整ってきているからである.

パヴロフスキーに関する言及が増加してくるのは,1970年代に入ってからである.まずSF研究の文脈で取り上げられ,次にモダンアート研究で取り上げられるようになった.

SF研究におけるパヴロフスキーの位置付けの変遷をまず見ておこう.1971年にアンリ・ボダン(Henri Baudin)が,科学主義の様々な文学的な可能性を開いた初期の事例として,エドモン・ハロクール(Edmond Haraucourt)の『最初の人間~---~ダー(\emph{Daâh} \emph{le premier homme})』(1925\footnote{連載自体は1912年から1914年にかけてなされた.以下を参照のこと.Edmond Haraucourt, «~Daâh le premier homme~», \emph{Le Journal}, du 26 décembre 1912 au 3 avril 1913.}),シャルル・ドレンヌ(Charles Derennes)の『極地の人々(Le peuple du Pôle)』(1907)などと一緒に取り上げられている.パヴロフスキーは作品中でオカルト現象の典型が多用されているたためか,ボダンはパヴロフスキーを「心霊主義の小説家(spirituel romancier)」と特徴づけている\footnote{Henri Baudin, \emph{La science-fiction~: un univers en expansion}, Paris, Bordas, 1971, p. 238.}.ボダンの研究の後にジャック・ヴァン=エルプ(Jacques Van Herp)がフランスを中心にテーマごとにSF史をまとめ,その中でパブロフスキーを4次元を扱った小説の代表例として挙げている.また,ユーモア作家で知られるアルフォンス・アレー(Alphons Allais)の影響が大きいことが指摘されている\footnote{Jacques Van Herp, \emph{Panorama de la Science Fiction. Les Thème, Les Genres, Les écoles, Les Problèmes}, Verviers, Gérard \& Co, 1973, p. 79.}.加えて,フランスSF研究で最も重要な研究者であるピエール・ヴェルサン(Pierre Versins)の『驚異の旅とSFのユートピア事典\footnote{Pierre Versins, \emph{Encyclopédie de l'utopie des voyages extraordinaires et de la science fiction}, Lausanne, L'Age D'Homme, 1972.} 』が1972年に刊行された.1000頁に及ぶ事典にはパヴロフスキーの項目もあり,1972年以降にパヴロフスキーについて言及される場合にはほとんど必ず引用される記事となった\footnote{\emph{Ibid. }, pp. 658-9.}.パヴロフスキーの作品を複数取り上げ,それらが『四次元郷』に収斂していくという記述など,現在の標準的な『四次元郷』解釈の基礎を作っている.また,他の作家の項目の中にもパヴロフスキーは多くの顔を出しており,例えば,パヴロフスキーの短編作品対してシュルレアリストグループと関わりのあった作家ジャック・リゴーが応答する作品を発表しているという記述はパヴロフスキーの小説がどのような影響を周囲に与えていたかを理解するための貴重な資料となっている.

次にモダンアートの文脈を見てみよう. 1970年代の終わりから1980年の初めにかけてパヴロフスキーを有名にした2つの著作がある.1つがジャン・クレール(Jean Clair)の『マルセル・デュシャン,あるいは大いなる虚構~---~大ガラスの神話分析試論(\emph{Marcel Duchamp, ou, Le grand fictif : essai de mythanalyse du Grand verre})\footnote{Jean Clair, \emph{Marcel Duchamp, ou, Le grand fictif~: essai de mythanalyse du Grand verre}, Paris, Galilée, 1975.}』(1975 以下,『大いなる虚構』)であり,もう1つがリンダ・ダリンプル・ヘンダーソン(Linda Dalrymple Henderson)の『モダンアートにおける4次元と非ユークリッド幾何学』(Linda Dalrymple Henderson, \emph{The fourth dimension and non-Euclidean geometry in modern art})\footnote{Linda Dalrymple Henderson, \emph{The fourth dimension and non-Euclidean geometry in modern art}, London, Cambridge, 1983. ただし,引用に際しては全て以下の版に依拠した.Henderson, Linda Dalrymple, \emph{The fourth dimension and non-Euclidean geometry in modern art}, Massachusetts, MIT Press, 2013.}』(1983 以下,『4次元と非ユークリッド幾何学』)である.

まず,『大きなる虚構』から見てみよう.パヴロフスキーがデュシャンにどのような影響を与え,その後に『四次元郷』に関するクレールの見解を示し,パヴロフスキーが時間と空間をどのように解釈していたのかを,それぞれ順に示す.デュシャン研究でよく取り上げられるピエール・カバンヌ(Pierre Cabanne)とのインタビューで,デュシャンが「ポヴォロヴスキー(Povolowsky)」という人物を知っているか,と尋ねたことが記録されている\footnote{Marcel Duchamp, \emph{Entretiens avec Pierre Cabanne}, Paris, Belfond, 1967, p. 67.}.これはカバンヌがパヴロフスキーを知らなかったための聞き違いか誤記だと考えられている.クレールは,デュシャンのこの発言を取り上げて,デュシャンが残した「グリーンボックス」と呼ばれる箱に入っている大量のメモを参考に,1923年に発表した彼の代表作である「彼女の独身者たちによって裸にされた花嫁,さえも(\emph{La Mariée mise à nu par ses célibataires, même})」,通称「大ガラス」について『四次元郷』からの影響を考察した解釈を提示している.とくに,デュシャンの「大ガラス」制作時代にキュビズムや未来主義と距離を置いたことに着目し,クレールは,デュシャンの作品制作の背景にあったのは,主体がどのように世界を認識しているかという観察者(observateur)の位相をめぐるものだったという説を提示している\footnote{Clair, \emph{op. cit. }, pp. 45-47.}.この時,観察者というテーマは『四次元郷』の1923年版に追加された序文すなわち「批判的吟味」に由来しているというのだ.また,クレールはデュシャンの『大ガラス』制作の構想の中で,4次元に時間は存在せず全てが同時的であることや,運動も本来は存在せず4次元においてすべてが静止しているというアイディアが見られることについて,『四次元郷』の冒頭でなんども繰り返されているテーマであることを指摘している\footnote{詳しくは第4章で扱う}.また,「大ガラス」の制作背景にある思想だけでなく,作品の具体的な分析も『四次元郷』のモチーフが現れていると指摘している.モチーフに通底するテーマはエロティシズムである.パヴロフスキーは1923年の決定版で「工業恋愛(L’Amour industriel) 」の章を追加し,エロティックなエネルギーを工業的に利用するという話を描いている.このようなエロティシズムと機械が組み合わさった運動のイメージは「大ガラス」の花嫁と独身者たちのイメージに反映されていると考えられる\footnote{\emph{Ibid. }, p. 128. }.そして,同じく決定版で言及されるエロスのメタモルフォーゼによって4次元に至るというパヴロフスキーの考えでは,プラトンの『饗宴』でアリストファネスが紹介する,男女はもともと1つの存在であり,現世ではそれが2つに分かれているという神話が触れられている.この2つに分かれた男女とエロス,そして機械のモチーフこそ,「大ガラス」の構成要素に他ならないとクレールは主張している\footnote{\emph{Ibid. }, p. 141.}.

これらの研究は「大ガラス」の解釈のためにパヴロフスキーの作品を読解しているが,クレールはのちにパヴロフスキーに焦点を当てた論考を発表する.それが,『四次元郷』が2004年にイマージュ・モデルヌ(Images Modernes)社から再刊された際に寄せられた『四次元郷』の「序文」である\footnote{Jean Clair, «~Introduction~», \emph{Voyage au pays de la quatrième dimension}, Gaston de Pawlowski, Images modernes, 2004, pp. 11-27.}.クレールはこの「序文」でまずパヴロフスキーの世代に注目している.パヴロフスキーは,象徴主義の世代と言えるものの,初期の未来派の芸術家たちの世代とも重なる転換期の作家だった.『四次元郷』の章のうち,同時代の作家と似たようなテーマがある章をそれぞれ指摘して,20世紀初めのロマン主義的スピリチュアリズムと科学主義の交差地点に『四次元郷』が位置していると述べている.さらに,『大いなる虚構』でも指摘しているように,一読してもコラージュされているようにしか見えないそれぞれの章は,読み進めていくと統一的に把握することが可能となり,異常な状態になってしまった社会の復興と人類の完全なる状態への進化という再生(palingénésie)を描いている物語であるということを改めて示している.また,最後に『四次元郷』の第二の科学時代で支配者として描かれている大中央研究所の学者たちが最終的に不死になることで,死ぬこととは別の形で3次元的な存在を終わらせようとすることについて,クローン技術などが発達しつつある2000年代からすると,不死の描写自体が極めて現実的であるように思われるとして「序文」を締めくくっている.

最後に,クレールの『四次元郷』の解釈について,「序文」で示されているパヴロフスキーの時間観を見てみよう.すなわち,パヴロフスキーにとって時間は実在している所与のものとして考えられており,倫理的,道徳的,政治的なものなのだ\footnote{\emph{Ibid. }, p. 23.}.第4章で見ていくように,パヴロフスキーは時間は存在せず全てが同時的であるということを明確に述べているので,誤った指摘のようにも思われるが,クレールは,4次元において時間が空間化されていると捉えている,と考えられる.

しかし,クレールの研究には問題点があることも指摘しておく.まず,全体の議論が1923年の決定版に依拠しているために,最初の時点ではほとんど注目されていなかった概念が焦点化されている.例えば,パヴロフスキーは初版の時点ではあまり観察(observation)という語彙を用いていないうえ,観察者(observateur)という語に至っては1912年版には1度も登場しない.別の問題として,3次元における時間の実在に基づいた解釈を示すが,パヴロフスキーが描く4次元や物語の挿話が他の同時代の作品とどう異なっているのかを示すことができてない.

ところで,パヴロフスキーの4次元の特徴はどこにあるのかという問いが明快に説明されているのは,ヘンダーソンの著作である.実際,クレールは「序文」で,4次元や非ユークリッド幾何学についての歴史的説明に関してヘンダーソンの議論を下敷きにしている.ヘンダーソンによるパヴロフスキーの4次元の特徴については第4章で詳しく見ていくのでここでは一旦おくとして,どのようにパヴロフスキーを扱っているのかを見ておこう.

パヴロフスキーが『四次元郷』を連載し始めた頃のキュビストのうち,ジャン・メザンジェ(Jean Metzinger)とアルベール・グレーズ(Albert Gleizes)は,『四次元郷』を着想源として,絵画空間の中で時間と空間のモデルを革新しようとしていた.その時に参照したと考えられるのは,時間と空間の抽象化という考えと時間の存在しない4次元空間における同時性というキーワードであった.後者について補足しておくと,ベルクソンがカントの時間概念を批判して持続という概念を提唱して広く知られていたのに対して,2人のキュビストにとってはパヴロフスキーの同時性の4次元は,そのカウンター概念として現れたように考えられたのだった.いずれにせよ,クレールが注目するデュシャン以外のモダンアートの作家たちが作品から霊感を受けてそれぞれの絵画空間を構築していったことはパヴロフスキーの作品の重要性を示す.ヘンダーソンの非ユークリッド幾何学の文化史的な調査の影響は非常に大きく,ロシアで近年出版された数学の一般書で4次元を取り上げている本で,4次元の文化史に関する記述ではほとんどがヘンダーソンに依拠しており,その中で『四次元郷』を取り上げている\footnote{\foreignlanguage{russian}{Рауль Ибаньес, \emph{Четвертое измерение~: Является ли наш мир тенью другой Вселенной?~}, Москва, Де Агостини}, 2014, p. 96.}.

また,ヘンダーソンのパヴロフスキー論は彼の芸術論にも焦点を当てている点でも重要なものとなっている.ヘンダーソンはメザンジェとグレーズがパヴロフスキーに言及している一方で,パヴロフスキーの芸術観について保守的であったと指摘している.『コメディア』ではパヴロフスキーの采配で2人の評論家が雇われていたが,キュビズムなど新しいムーヴメントに対して批判的だったアルセーヌ・アレクサンドル(Arsène Alexandre)に記事で肯定的に言及していることや,メザンジェとグレーズの著作『キュビズムについて(\emph{Du Cubisme})』とアポリネールの『キュビズム画家(\emph{Les Penitres Cubistes})』の書評でもキュビズムの絵画に最後まであまり肯定的でなかったことをヘンダーソンは示している.パヴロフスキーのこうした絵画批評の他に,別の研究者によって演劇批評が注目されている.ナンシー・トロイ(Nancy J. Troy)は20世紀初頭のフランスにおけるオートクチュールとファッションショーの関係についての論文で,パヴロフスキーの記事を引用している\footnote{Nancy J. Troy, "Staging Haute Couture in Early 20th-Century France", \emph{Theatre Journal}, v. 53, n. 1, 2001, pp. 1-32.}.トロイによると,演劇における登場人物の服装がコマーシャリズムと結びついたのが20世紀初頭で,とくに「平和通り(\emph{La Rue de la Paix})」は様々な洋服店,いわゆるメゾンが軒を並べ,マネキンの陳列による服の展示がパレードの様相を呈していたという.1912年に公開された風刺演劇『平和通り』は,伝統的なクチュールと新しいスタイルを取り入れたクチュールの対立を戯曲の主題としており,実際のメゾンがその衣装を作っていた.この戯曲はそうした事情からファッションショーのように受容され,演劇がどうあるべきかという論争に発展した.トロイは論争から距離を取って冷静な分析を加えつつも,保守的な態度を見せる論者としてパヴロフスキーを紹介している\footnote{Troy, \emph{op. cit. }, p. 30.}.パヴロフスキーによれば,演劇がどうあるべきかという論争は商業的衝突が道徳的衝突と入れ替わった現象に過ぎず,文学においても同様のことが生じていると評している現代的な風潮であると判断している.しかし,商業的性格があからさまなのを嘆いてもいる.『四次元郷』でも1912年第13章1923年第13章「リヴァイアサンの劇場(Le Théâtre du Léviathan)」で戯曲には社会における個人道徳が表現されており,戯曲の内容の規制によってその社会のことを知ることができると論じられている(78/101).パヴロフスキーは自身の芸術論をまとめることがなかったが,こうした先行研究での文脈を足がかりに芸術論の全体像を描く必要があるだろう.

クレールとヘンダーソンのモダンアートの文脈におけるパヴロフスキーによってパヴロフスキーはモダンアート研究で知られる人物となり,文学研究の対象となることはほとんどなかった.書誌情報と活動歴などのパヴロフスキーの詳細が明らかになったのは1998年のことで,エリック・ヴァルベック(Eric Walbecq)の研究成果による\footnote{Eric Walbecq, «~Gaston de Pawlowski~», \emph{Le Visage Vert}, n. 4, Paris, Joëlle Losfeld, 1998, pp. 148-52.}.ヴァルベックの調査によって,パヴロフスキーのジャーナリストとしての活動は非常に旺盛なもので,調査された記事の量だけでも60巻程度になるような分量であるということが明らかになった.また,ヴァルヴェクは,2003年にはパヴフスキーの『四次元郷』以外の作品の復刊も行なっている\footnote{Gaston de Pawlowski,\emph{Paysages animées}, dir. Eric Walbecq et Jacques Damade, Paris, La Bibliothèque, 2003.}.

ヴァルベックの次にパヴロフスキーが研究で取り上げられてたのは,2001年には研究雑誌『ヨーロッパ(\emph{Europe})』でSF特集があった時に寄せられたフィリップ・キュルヴァル(Philippe Curval)の論考「シュルレアリスムとサイエンスフィクション\footnote{Philippe Curval, «~Surréalisme et Science-Fiction~», \emph{Europe}, v. 79, n. 870, octobre, 2001, pp. 32-50.}」である.キュルヴァルはクレールがパヴロフスキーの4次元の概念について『大いなる虚構』で述べている,観察者とその対象の関係が修正されうる可能性があるという考えを引用することで,最初の量子物理学的な作品であると独自の見解を示している.量子物理学者のエヴェレットが提起した観測問題などを背景とした記述であると考えられるが,パヴロフスキーは第5章で見ていくように原子は実在するものではないと考えていたことは明らかであり,量子力学に関する知識もなく,それが大衆的な科学知識として広まっていくのと同時期に亡くなったので,この解釈は牽強付会のきらいがある.

キュヴァル以降はあまり取り上げられることがないパヴロフスキーであったが,ブライアン・ステーブルフォード(Brian Stablefold)が『四次元郷』を自ら英語に翻訳して2009年に出版し,そのために執筆した「序文」は非常に重要な論考にもなっている\footnote{Brian Stableford, "Introduction", \emph{Journey to the Land of the Fourth Dimension},  London, Black Coat, 2011, Electronic.}.ステーブルフォードは,ヴェルサンやヴァルベックの提供するパヴロフスキーの活動歴を網羅し,その評価と独自の解釈を記している.ステーブルフォードがまず評価しているのは,『四次元郷』が他の未来予想小説とは異なって,その未来の描写が全く陳腐化していない点である.パヴロフスキーの未来史には他の小説に見られるような典型的な描写が見られないのである.その大きな要因として,パヴロフスキーがアルフォンス・アレーと引き合いによく出されているように,機知に富んだ表現を多用しているからであろう.また,ウェルズの『タイム・マシン』の影響があるとされながらも『四次元郷』ではウェルズの名前が挙がっていないことについて,未来の労働者の一人の名前が「H・G・28」であるのに着目して,それがウェルズのイニシャルを暗示しているという説を提示している\footnote{私はこの説に一定の留保を持ちつつも賛成している.HG28が登場するのは『四次元郷』においても極めて重要な「機械の反乱」という章であり,本論でも詳しく検討することになるが,パヴロフスキーが明示しない引用を多用する作家であることを考えると28という数字にも由来があると考えられる.これは,同じ4次元を取り扱った例として頻繁に取り上げられる2作品の関係を明らかにするためにも重要な今後の研究課題と言える.}.ステーブルフォードはクレールと同様に,挿話の1つ1つを取り上げて同時代の作品や思想家との影響関係を詳しく見ている.

ステーブルフォードは,パヴロフスキーの作品に見られる同時代の作品や思想家への参照が見られるのはパヴロフスキーがそれらを乗り越えようとしているからであるという.パヴロフスキーの交友関係を見ていくとガブリエル・タルドと知り合いであったと考えられており,『四次元郷』は人類を3次元から4次元に解放していると見ることができる点は,タルドの太陽が冷えた後の地球で暮らす人類の姿を描いたエッセイのような終末的な物語を乗り越えようとしていると考えられる\footnote{Gabriel de Tarde, \emph{Fragment d'Histoire Future}, Paris, V. Giard \& E. Brière, 1896.}.他にも,時間改変SFの初期の代表作であるシャルル・ルニヴィエの『ウクロニー(\emph{Uchronie})』やサン=シモンの未来予想なども乗り越えようとしていたと考えられている.さらに,ステーブルフォードは,クレールの「序文」と同様に各章についての解釈も簡単に示している.それぞれ簡単に見ておこう.本論第5章で私が考察することになる「引き裂かれた犬」の章では最初に人類は火星に知的生命体が住んでいると考えて通信を幾度か試みる.その最初に試みようとした火星人との通信方法の描写はギュスターヴ・ル・ルージュ(Gustave Le Rouge)に由来しているという.本論第6章で私が扱う「巨大バクテリア」の章では,バクテリアを巨大化させることによって退治するが,これはアンドレ・クヴルール(André Couvreur)の『バクリテリアの侵略(\emph{Une Invasion de Macrobes})』(1908)を踏まえていると考えられる.

また,ラーゲが数学的に解説を施している箱をめぐる挿話も参照先があるという.天文学者で心霊主義者のカール・フリードリッヒ・ツェルナー(Karl Friedrich Zöllner)の『超越論的物理学(\emph{Transcendental Physik})』(1878)では,封印された箱が手を触れずに開いてしまうという挿話があり,それに基づいたエピソードであると推測されている.パヴロフスキーが直接ツェルナーの著作を読んだ証拠はないものの,心霊主義が流行していた世紀末のフランスでは,ウィリアム・クルックスの4次元と心霊研究が紹介されていた.パヴロフスキー自身もスピリチュアリストたちと関係があった.文通相手だったジュール・ボワ(Jules Bois)が執筆した『永劫回帰(\emph{L'Eternel Retour})』(1914)の内容を出版前に知っていたこと,それに加えてボワが所属していた薔薇十字団の先輩だったジョセフ・ペラダン(Joseph Péladan)は『コメディア』に記事を寄せたこともある.心霊現象の描写については,この他にも,心霊現象がやや皮肉を込めて描写されているものの,4次元の理論によってそれらは全て実際に起こっていることだとパヴロフスキーは説明している.ステーブルフォードによれば,パヴロフスキーはネオプラトニズム的な観念論に依拠しているために,心霊現象の描写をすることで,実体としての精神を描くことになってしまったと考えられている.

ステーブルフォードは「序文」の末尾でパヴロフスキーの4次元はウェルズやその他の多次元空間を扱う他の作家ととは違って,量的な高次元を拒否し,時空間から離脱しており,4次元の質に依拠することで外側から時間を見つめて未来史をまとめていると結論づけている.しかし,ステーブルフォードの「序文」の結論から,異なった解釈を引き出している論者がいる.その論者について,以下で見ていこう.

エレーナ・ゴメル(Elana Gomel)は大衆小説と見なされ文学史に組み込まれていないSF作品と文学史に登録されている作品を統一的に把握する視座を提案するために,非ユークリッド幾何学以後の時空間認識の詩学の構築を試みる著作『物語の空間と時間(\emph{Narrative Space and Time})\footnote{Elana Gomel, \emph{Narrative Space and Time~: representing impossible topologies in literature}, New York, Routledge, 2014.}』の中でパヴロフスキーに言及している.ゴメルは非ユークリッド幾何学の影響を受けた英文学作品の代表的な2つの作品であるウェルズの『タイム・マシン』とジョージ・マクドナルド(George MacDonald)の『リリス(\emph{Lilith})』から始まるジャンルの物語群があるとしている.このジャンルには『四次元郷』も分類されているので,ここでは『タイム・マシン』を例にとることでこのジャンルについて詳しく説明する.

『タイム・マシン』のあらすじを簡単に見ておこう.時間を移動する機械「タイム・マシン」を発明した主人公は紀元~802,701年に移動して,階級社会によって2分化された人類がイーロイとモーロックという2つの異なる人種としてそれぞれ進化を遂げている世界に出会う.その後,その世界の見聞を広めているうちに,あるイーロイ人を守るためにモーロック人を殺害してしまう.その世界から帰還した主人公は周囲に未来の世界のことを話し,その後時間の中に再び旅立っていく.

ゴメルがこのあらすじの中で着目しているのは,主人公が運命論者としてこの未来を避けられてないものと述べている箇所が散見されているのに対して,決定されているはずの未来に介入する行動ばかりしており~---~モーロック人の殺害,未来から花を持ち帰るなど~---~,タイムパラドックスに陥っている点である.ゴメルは,非ユークリッド幾何学の大衆化によって空間を移動するようにして時間を移動するという考えが一般化したため,こうしたパラドックスを避けることができたと考えている.同じことが『リリス』でも同様のことが指摘できるので,こうした作品群をゴメルは「回避(Sidestepping)」と名付けている.「回避」の作品は,20世紀に入ると非ユークリッド幾何学の大衆化によって多次元空間を扱う小説と結びつき,ユートピアやディストピアといった現実には不可能な場所と現実の空間をつなぐための方法論となっていく.ところで,ゴメルは『四次元郷』を「回避」と多次元空間の結びつきの優れた例だと考えている.なぜなら,多次元空間である以上はそこにはユートピアとディストピアが互いを回避しあい,別の時空間として物語の中で展開されているはずだからである.その点,リヴァイアサンによる支配とその解放,科学者たちの支配から4次元への人類の解放といったユートピアとディストピアを物語の展開で示している『四次元郷』は「回避」の好例となっているのだ.しかし,「回避」の範例として完全であるために,かえって物語的な叙述を失っている\footnote{Gomel, \emph{op. cit. }, p. 157.}.また,ゴメルはパヴロフスキーの4次元の描写や心霊主義的なエピソードが多用されていることから,パヴロフスキーは結局のところ,反科学的であると述べている\footnote{\emph{Ibid. }, p. 158.}.反科学的な思想によって4次元を数学で扱う量的な空間ではなくてスピリチュアルな場所と扱われているのだとゴメルは考えている.

最初に,ゴメルはステーブルフォードの結論から別の解釈を引き出していると述べた.ここで改めて確認しておくと,ステーブルフォードは,パヴロフスキーは量的な高次元の拒絶し,時空間から離脱して外側から時間を見つめることによって未来史をまとめていると述べていた.この結論をゴメルは「回避」の文脈に置き直すことで,量的な次元とは空間と時間を同じようにみなすことなので,運命論を時間の空間的な移動によって別の物語の軸に接続するウェルズに対して,運命論的な未来の歴史をすべて放棄するのがパヴロフスキーだと述べる.ステーブルフォードとゴメルの結論の微妙な差異は歴史を物語る行為の問題に発展するもの,ここで一旦措くことしよう.

以上で中心となる『四次元郷』の研究をみてきた.近年でもパヴロフスキーや『四次元郷』に関する研究が定期的に発表されているが\footnote{カンファレンスでは,『コメディア』が取り上げられたり,幻想文学とSF文学の境目が曖昧であった頃の作品を扱うカンファレスなどで定期的に言及されている.以下の2つを参照のこと.Sophie Lucet, «~Gaston de Pawlowski et le journalisme ou l’art de l’échappée~», Comœdia (1907-1937). un continent inexploré dans l'histoire du théâtre, Bibliothèque Seebacher, Université Paris-Diderot, 21 juin 2014. Patrizia D’Andrea, «~Voyage au pays du silence~: rêve, philosophie, utopie entre symbolisme et anticipation (Han Ryner, Gaston de Pawlowski)~», Mobilités dans les récits de fantastique \& de science­fiction (XIX~-~XXIe siècles)~: quête \& enquête(s), l'IUT Sénart-Fontainebleau, Université Paris Est Créteil, Jeudi 20 Novembre.},ここではその詳細を扱わない.最後に,先行研究を受けての本論の位置を示す.

先行研究は論者と合わせて以下の点にまとめることができる.
\begin{itemize}
  \item SF研究において
    \begin{itemize}
      \item パヴロフスキーは,フランスで4次元をテーマに扱った代表的な作家であり,作風はユーモアを基調としていた.
    \end{itemize}
  \item モダンアート研究において
    \begin{itemize}
      \item 非ユークリッド幾何学を独自に理解し,4次元の同時性と不動性を強調し,観察者とその対象の関係を相対的なものとした.
    \end{itemize}
  \item 物語と歴史をめぐる論点
    \begin{itemize}
      \item 同時代の未来を描く物語に対して別のアプローチを行い,量的な次元を拒否しすることで,未来史の記述を可能としている,あるいは,運命論的な未来の歴史を拒絶している.
    \end{itemize}
  \item 参照関係
    \begin{itemize}
      \item 各章のテーマや登場するモチーフが同時代に参照点を持っている.
    \end{itemize}
\end{itemize}

以上の先行研究を踏まえて,本論のアウトラインを示すことにしたい.私は,ヘンダーソン,ステーブルフォードのパヴロフスキーの4次元の分析のうち,とりわけヘンダーソンが提示するいわば「4次元の文化史」に基づいてパヴロフスキーの4次元の特徴について第4章で見ていきたい.次に,4次元や同時代に同じように非ユークリッド幾何学の影響を受けている別の作品をめぐる影響関係ではなくて,ヴァルベックがパヴロフスキーのジャーナリストとしての盛んな活動に注目していたのをふまえて,出版史における科学とジャーナリズムの関係に基づいて『四次元郷』の背景を分析してみたい.そこで,第5章では,19世紀末のフランスのジャーナリズム,とりわけ『四次元郷』が新聞に登場した時の形式である「連載」の手法に注目しつつ,それがどのように科学記事に影響を与え,『四次元郷』が生まれる土壌を作っていったのかを見ていきたい.

そして,作品の成立背景のほかにも,私は新しい論点を追加したいと考えている.それは,4次元の文化史とは異なったテーマについての,科学の大衆化に注目した読解である.というのも,『四次元郷』はそのタイトルから4次元の世界に入り込んで冒険するようなニュアンスが与えられているにもかかわらず,描かれているのは未来の人類の姿なのであって,別の次元での別の知的生命体の物語を描いてはいないのだ.3次元の世界の未来を描いているのである.すると,4次元をどのように描いているかよりも,『四次元郷』の3次元の世界はどのように説明されているのかという問いも非常に重要となる.そして,私は,本論にて,3次元を理解する上で重要だと考えられるテーマをエネルギー・進化論・遺伝という19世紀の科学の大衆化によって,4次元と同じように人々に知られるようになった知識を手がかりに『四次元郷』を解釈する.以下で,その概括をする.

私は,『四次元郷』の中で頻繁に用いられる物質(matière)というキーワードを手がかりに2つの論点を提示する.まず,3次元におけるエネルギー(énergie)との関係について扱う.物質は分離(dissociation)によってエネルギーが生じるとパヴロフスキーは述べており,このエネルギーと4次元との関わりについての記述から『四次元郷』において3次元がどのように描かれているのかを,第5章で私は考察する.次の論点として,物質から生命へ,というテーマを扱う.『四次元郷』ではある章のエピソードで物質から生命が誕生するという説が紹介されている.第6章では,1912年版と1923年版の異同に注目しながら,進化論と遺伝という潜在するテーマがあることを示し,そこから「遺伝的類似」という4次元と遺伝のアナロジカルな関係を示す概念が取り出せることを明らかにする.

\chapter{出版史とジャンルの観点からみる『四次元郷への旅』}
\section{出版史における『四次元郷への旅』}
1980年代以降,ジャン=イヴ・モリエ(Jean-Yves Mollier)の研究以降\footnote{Jean-Yves Mollier, \emph{L'argent et les lettres, histoire du capitalisme d'édition, 1880-1920}, Fayard, Paris, 1988.},19世紀出版史は文学研究においてとみにその重要性を増している.その理由は,出版社ないし編集者による要請は作品のスタイルや内容にも大きな影響を与えていることが明らかになってきたからである\footnote{編集者と作家の関係ついては以下を参照のこと.Jean-Yves Mollier, «~écrivaint-édituer~: un face-à-face déroutant~», \emph{Travaux de littérature}, v. XV, t. 2, Paris, L'Adirel, 2002, pp. 17-39. 石橋正孝『〈驚異の旅〉または出版をめぐる冒険:ジュール・ヴェルヌとピエール=ジュール・エッツェル』,左右社,2013年,第1章.}.ガストン・ド・パヴロフスキーの作品を扱う場合も,それは例外ではない.なぜなら,彼は編集者でなおかつ作家であり,19世紀以降のフランスの出版文化を背景に活動していたからだ.パヴロフスキーの活動において出版文化の影響が特に表れているのは連載小説という形式とそれがもたらす内容である.そこで,新聞や雑誌で見られた連載という形式の誕生からまずは見ていきたい.

小説が雑誌に連載されることが連載小説であるとするなら,18世紀の時点から小説が分載されることは珍しくはなかった\footnote{ダニエル・コンペール『大衆小説』,宮川朗子訳,国文社,2014年,35頁.}.しかし,19世紀以降のフランスの出版市場の拡大はそれまでの連載小説の性質を大きく変えることとなった.出版市場の拡大は,行商や貸本屋が増えていったことや,カトリック系の教会の派閥による教化運動の一貫としての印刷物の普及,科学技術の発達に伴う技術書の需要の高まり,特に1933年のギゾー法によって各県に1つの小学校が設置されたことによる識字率の向上といった複合的な原因によってもたらされた.

この当時,小説は,詩や演劇と消費者が異なり,主に下層中産階級や職人・奉公人といった人々の娯楽の1つとなっていた.小説は,主に貸本屋や行商によって消費者に供給されていた.この受容の形態が大きく変化するのは,「連載小説(roman-feuilleton)」の普及によるものであった.新聞の最下欄にある連載欄(feuilleton)は科学,文芸,小説などが掲載されている.このシステムを最初に生み出したのは1800年代の『ジュルナル・デ・デバ(\emph{Journal des Débats})』だった\footnote{Lise Dumasy-Queffélec, «~Le feuilleton~», \emph{La Civilisation du Jounal, Histoire Culturelle et Littérature de la Présse Française au XIX\textsuperscript{e} Siècle}, éd. Dominique Kalifa et. al. , Paris, Nouveau édition, 2011, p. 925.}.ナポレオン帝政期に登場した連載欄は,当初政治的話題以外に触れる文化通信欄として用いられていた.七月王政期(1830-1848)に入ると,『両世界評論(\emph{Reuve des deux mondes})』や『ルヴュ・ド・パリ(\emph{Revue de Paris})』に定期的な発表に合わせて小説が分載されるようになり,若い世代の小説家たちをとり込んでいった\footnote{Marie-ève Thérenty, \emph{Mosaïques. être écrivain entre presse et roman (1829-1836)}, Paris, Honoré Champion, 2003, Paris, Honoré Champion, 2003.}.

1836年に商業広告を掲載することで値段を引き下げることに成功した『プレス(\emph{Press})』は,連載欄に小説を掲載することで売り上げ部数を伸ばしていった.しかし,現在の新聞でイメージされる連載小説は日毎に掲載されるというのが連載小説の形式と思われているが,それとは異なり,『プレス』では曜日ごとに掲載されるジャンルが異なり,月曜日は連載記事はなかった.以下に,どのような連載がなされていたのかを示しておこう.日曜日はアレクサンドル・デュマの歴史の名場面集,火曜日はフェデリック・スリエの短編の悲劇小説(feuilleton dramatique),水曜日はランベル(Lambert)博士の科学アカデミーに関する記事,木曜日は書籍・音楽・演劇の紹介,金曜日は農業や産業に関する記事,土曜日は海外の珍しい習俗などを伝える記事が載っていた.これらの連載のうち,日曜日と火曜日の小説の連載が新聞の売り上げに繋がり,多くの小説家がそちらの方に流れることとなった\footnote{Dumasy-Queffélec, \emph{op. cit. }, p. 926.}.こうして生まれたのが「連載小説」という形式だった\footnote{ただし,これを現代の連載小説の起源であるとは言い難いことも付記しておきたい.デュマジー=クフェレク(Dumasy-Queffélec)の研究によると,当時の代表的な小説家ウージェーヌ・シュー(Eugène Sue)が1837年から『アルチュール(\emph{Arthur})』の連載を『プレス』で始めた時,それは雑報欄(Variété)で始められており,連載が終了した1839年に連載欄だったことから中短編(courtes nouvelles)中心の連載欄と長編中心の雑報欄という使い分けが少なくとも1839年まではあったのではないかと考えられるという.以下を参照のこと.\emph{Ibid. }, p. 925.}.

連載小説が黄金時代を迎えたのはウージェーヌ・シュー『パリの秘密(\emph{Les Mystères de Paris})』が『ジュルナル・デ・デバ』で1842年6月19日から1843年10月15日まで連載されて時である.この成功をきっかけにして,1845年までには全ての日刊紙が連載小説の形式を受け入れた\footnote{コンペール,前掲書,39頁.}.連載小説の流行によって,多くの大衆小説家が生まれて,オノレ・ド・バルザックやジョルジュ・サンドが活躍し始めたのもこの頃だった.連載小説のブームは逐次刊行物の出版の誕生などに繋がり,文学そのものにも大きな影響を与えていくこととなる.

連載小説が同時代のロマン主義文学に与えた最も大きな影響とは,三面記事(fait divers)で話題となるような犯罪を取り上げる傾向が強まったことである.その具体例としてよく知られているのは,19世紀最も有名な犯罪者のフランソワ・ヴィドック(François Vidocq)が犯罪者から改心して正規の刑事にまで上り詰めて,『回想録』(1829)を執筆し,先の『パリの秘密』やロマン主義文学の参照点ともなった.このように,連載小説の流行と三面記事の流行は密接に関係していた.この流行は1863年,『プチ・ジュルナル(\emph{Petit Journal})』の刊行によって頂点を極める.『プチ・ジュルナル』はそれまでの新聞の3分の1程度の値段で販売され,犯罪,事故,痴情のもつれといったまさに三面記事的な話題を中心に取り上げていたこともあり広く読まれた\footnote{具体的な数字を述べておこう.初年38,000部程度だったこの日刊紙は,翌年には150,000部と約3倍に膨れ上がり,1867年には250,000部に到達する.パヴロフスキーの『コメディア』が刊行していた1910年にはその数は835,000部になっており,刊行されていた83種の日刊紙の総発行部数の1.5割程度が『プチ・ジュルナル』1紙によるものだった.下記を参照のこと.ルイ・シヴァリエ『三面記事の栄光と悲惨 近代フランスの犯罪・文学・ジャーナリズム』,小倉孝誠・岑村傑訳,白水社,2005年,199頁.\emph{Histoire générale de la presse française}, \emph{op. cit.} , p. 296.}.『プチ・ジュルナル』にも小説が連載され,ポンソン・デュ・テラーユの『謎の遺産(\emph{L'Héritage mystérieuse})』(1857)の連載の反響は大きく,「ロカンボル」という主人公が活躍するシリーズは多くの支持を得た.テラーユのロカンボルは大衆向けのシリーズ小説で1つの定型となる,蘇る主人公の造形の初出とも言われている\footnote{コンペール,前掲書,52-57頁.}.

19世紀後半を通じて犯罪を題材にとる小説は,連載小説という形式が一般的になっていくにつれてその傾向が強まっていくことは他にも多くの研究が示しているが\footnote{一例として以下挙げる.Christine Marcandier-Colard, \emph{Crimes de sang et scènes capitales : essai sur l'esthétique romantique de la violence}, Paris, PUF, 1998. },ロマン主義以外の文学作品にあったもう1つの局面を同じような構造において示すことができる.すなわち,科学の大衆化に伴ってその数が増えていくに従ってその数が増えていった科学記事と科学雑誌,そして現在はSFと呼ばれている一連の文学作品はちょうど三面記事とロマン主義と同じ枠組みにおいて語ることができると考えられる.こうした論点は今までほとんど示されてこなかった.

最初に,科学の専門的な内容が新聞でどのように取り上げられるようになったのかの歴史を見ていこう.1836年に登場した『プレス』の翌年,科学アカデミーの会議がジャーナリストに対して公開され,新聞でもその専門的な内容を解説するような新しいタイプの記事が生まれることとなった\footnote{下記の記述は注釈がない限り,以下に依拠している.Clauire Barel-Moisan, «~écrire pour instruire~», \emph{La Civilisation du Jounal, Histoire Culturelle et Littérature de la Présse Française au XIX\textsuperscript{e} Siècle}, éd. Dominique Kalifa et. al. , Paris, Nouveau édition, 2011, pp. 752-65.}.それが,連載科学記事(feuilleton scientifique)となっていった.『プレス』ではランベル博士による科学アカデミーの記事が週ごとに連載されていたことはすでに見た通りだが,この連載科学記事は19世紀後半の科学の大衆化を促した.その最初の契機を作ったのが,科学アカデミーでの会議の内容を題材にし,読者の関心を煽るような記事を書いたヴィクトル・ムニエ(Victor Meunier)などの存在である.彼は1853年3月22日の『プレス』にて,同紙の連載科学記事をいくつか取り上げ,その引用を組み合わせて自分の意見を加えて,「ハチの知性(Sur l'inteligence des abeilles)」や「月の住民(Les habitants)」という題名の記事を書き上げた.こうした大衆向けの科学記事は1880年代に頂点を極めて,その後どの新聞にも何らかの形で科学記事が連載されるのは,一般的なこととなっていく.その理由として,第二帝政以降,国家の産業化が促進され,科学技術に関係する媒体を支援していたので科学教育のための雑誌が多く求められていたことが挙げられる.さらに,普仏戦争のフランスの敗北によって愛国心から教育意識が高まったため,ドイツの科学力に対抗するための科学精神を養うことを謳ったに『ラ・ナチュール\emph{La Nature}』(1873),国際関係を意識して地理を取り上げた『ジュルナル・デ・ヴォイヤージュ\emph{Journal des voyages}』(1877)などが刊行される.この頃,新聞・雑誌メディア全般でジャーナリズム的言説のあり方が変化し,結果的に,ルポルタージュ・三面記事・インタビューの3つの領域で形成されるようになっていく\footnote{Marie-Yve Thérenty, \emph{La Littérature au quotidien. Poétique journalistiques au XIX\textsuperscript{e} siècle}, Paris, Seuil, 2007.}.その領域編成は連載科学記事の内容をも変化させた.異常気象や地震といった耳目を引くニュースが選ばれるようになり,『ジュ・セ・トゥ(\emph{Je sais tout})』が1905年に創刊され,同紙では,センセーショナルな内容の科学記事が増えていった.その一方で,1900年代に入る頃には,1860年代から続く科学精神を支えてきた実証主義派の巨魁たちが相次いで亡くなり,大衆に科学を広めようとするモチベーションが出版業界から去っていくにつれて,専門的な科学雑誌は次第に姿を消していった.
そうした中でも 1880年代から90年代を通じて人気を獲得していった科学雑誌は,レクリエーションを教育手段に用いた.『イリュストラシオン(\emph{L'Illustration})』はその中でも大きな成功を収めたが,その時に用いたフレーズが「楽しい(amusant)」であった.彼が編成したシリーズである「楽しい科学(La science amusante)」は1889年から1893年まで続き,それをまとめた書籍は40,000部を売り上げ,7ヶ国語に翻訳された.

この娯楽を重視した方針の変化によって,経済的に大きな成功をしたのが,小説による科学教育の実戦だった\footnote{本論では出版史の研究から科学小説による科学の大衆化について確認しているが,SF研究では以前よりこの教育的性格が指摘されている.Jean-Jacques Bridenne, \emph{La Littérature Français d'Iimagination Scientifique}, Paris, Gustave Arthur Dassonville, 1950, chapitre 1.}.ボーラン書店の経営者であり編集者でありエッツェルは『教育娯楽雑誌(\emph{magasin d'éducation et de récréation})』を1864年に刊行し,ジュール・ヴェルヌが「驚異の旅(voyage extraordinaire)」シリーズを連載すると,大成功を収めた.ヴェルヌのようなスタイルの科学小説(roman scientifique)は19世紀を通じて多く書かれたが,作者たちは教育的側面を意識して執筆していた.こうした小説のようにある特定の主題を意識して書かれているいわゆる問題小説(le roman thèse)と呼称されているが,問題小説の研究者シュザンヌ・シュレマン(Susan Suleiman)の指摘によれば,曖昧さを排して,記述が細部にわたる冗長さがあり,語の一義性に配慮しているという点を問題小説の特徴としてあげており\footnote{Susan Rubin Suleiman, \emph{Le Roman à Thèse ou l'Autorité Fictive}, Paris, PUF, 1983. },当時の科学小説がそうした側面を待っている.ヴェルヌの影響は大きく,たいていの科学小説は「冒険(aventure)」という言葉が題名につけられていた.ジャン・ド・ラ・イール(Jean de La Hire)が執筆した『火花散る歯車(\emph{La Roue fulguratnte})』は,当初「冒険の科学小説(roman scientifique d'aventures)」という題名だったことや,ファイヤール社が冒険叢書を出版したことなどを挙げられるだろうが,枚挙にいとまがない.大衆文学史家のジャック・ボドゥは,こうした科学小説におけるヴェルヌの影響の大きさとは別に,ウェルズの影響がもう1つの傾向を生み出していると指摘している\footnote{ジャック・ボドゥ『SF文学』,新島進訳,白水社,2011年,62-3頁.}.ウェルズが初めて著した科学小説は『タイム・マシン』(1895)で,1898年12月と1899年1月に分けて大衆向けではない文学性の高い雑誌『メルキュール・ド・フランス\emph{Mercure de France}』に翻訳され,アルフレッド・ジャリからプルーストまで幅広く読まれた.『タイム・マシン』は未来予想の小説の典型をこの時点で早くも生み出していたが,自然主義運動に参加していたJ・H・ロニー兄は,全く逆に,はるか遠い昔の時代を描いた『火の戦争\emph{La Guerre du Feu}』(1909)などを書き,先史時代の物語というジャンルを作り上げた.ボドゥは,これらいずれもヴェルヌを代表とする大衆小説とは違った受容がなされていたと考えている.

ここまで,科学小説のほとんどは新聞か雑誌に連載されたものがまとめられることで出版されており,三面記事とロマン主義の相互関係のように,科学小説はメディアのニーズすなわち教育を意識した内容がそのほとんどを占めていくことを示してきた.しかし,フランスの出版史における三面記事の影響を完全に免れたわけではない.コンペールは,19世紀後半に著名だったサイエンスライターで編集者であるルイ・フィギュエ(Louis Figuier)が1887年に創刊した『ラ・シアンス・イリュストレ(\emph{La Science Illustré})』の科学記事に自然現象や事故をドラマチックに報じたり,科学的観点からの分析ではなくて別の科学記事を引用し意見を加えていく手法が見られることを具体的に検討して「科学的知識を\bou{揺らがせて}いたものかのような(comme ce qui vient \emph{étonner} le savoir scientifique)\footnote{Daniel Compère, «~Fait divers et vulgarisation scientifique~», \emph{romantisme}, n. 97, Paris, Armand Colind, 1997, p. 76.}」状況になっていたものの,それは同時に科学記事のスタイルを変化させていただけで,科学知識をないがしろにしていたということでは決してないと指摘している\footnote{\emph{Ibid}.}.SF研究では,一般的に1870年代は普仏戦争や植民地戦争の激化の時代に合わせた戦争を題材にした近未来小説が増え,政治を題材にした小説がSFを要因として増えた時期だとされている点も,こうした三面記事と科学記事の関係を例示していると言える.ダルコ・スーヴィンは「物語は,超兵器の力に頼るだけで,(略)心理的側面の真実を想像できない無能さだけをぶざまにさらけだしている\footnote{ダルコ・スーヴィン『SFの変容』,大橋洋一訳,国文社,1991年,264頁.}」と評しているが,少なくとも出版史の観点からすると,それは科学アカデミーの会議が公に開かれて以来始まった科学記事の歴史の中で,三面記事化していった科学小説の姿であると言える.しかし,科学小説は科学専門誌が1900年代を超えるとほとんど生き残れなかったのと同様に,1920年代を頂点にして,それ以後は顧みられなくなっていく\footnote{ジャック・ボドゥ,前掲書,66頁.}.

\section{ジャンルを逸脱する『四次元郷への旅』}

私たちは,フランスにおける出版史の,とりわけ三面記事との文学の関係を見てきた.三面記事はロマン主義の多くの文学作品に影響を与え,教育を意識して書かれた科学記事もまたその影響を受け,同時代の合わせ鏡のように三面記事的な科学小説も増えていったのである
.『四次元郷』もまたその例外ではなかった.

『四次元郷』が執筆された時代を振り返っておくと,同作品は1900年代に連載が始められ,最終的に1912年に出版された.すでに,科学記事が連載の形で新聞に載るのは一般的だった.また,1860年代以降のヴェルヌの活躍により,1870年代生まれのパヴロフスキーらにとってヴェルヌを代表とする科学小説を読むことも極めて一般的なことだったと考えられる.すると,自身が編集する『コメディア』が演劇・文芸が中心の日刊紙でありながら,パヴロフスキーが投稿する科学的な知識を盛り込んだ小説が掲載されることや,のちに検討するように,パヴロフスキーが『四次元郷』に科学記事から再引用しつつ自分の文脈と繋げていく三面記事的な手法を用いていることは,自然なことだったのだ.しかし,同時代の科学小説に比べて,『四次元郷』はどのジャンルにも属し難い異なる特徴を持っていた.

『四次元郷』は未来を旅することで将来の社会を知るという点で,いわゆる未来空想物語(le roman d'anticipation)やウェルズの『タイム・マシン』的な筋書きを持っている.しかし,一読すればわかるように,ジャンルやあらすじの形式だけに注目することはこの小説の理解を困難なものとする.それを如実に示しているのは,ヴァン=エルプによる同作品の位置付けである.ヴァン=エルプは\emph{Panorama de la science fiction}の第2章「時空間の回廊で(Dans les corridors de l'espace-temps)」で,時空間の移動を題材にした小説を7種類に分類している\footnote{以下の通り.時間の中の旅(Le voyage dans le temps) ,パラドックスの時間(Le temps des paradoxes),操作された時間(Le temps manipulé),歴史改変(Les uchronies),ポリティカルフィクション(Politique-fiction),並行宇宙(Les univers parallèles),4次元(Quatrième dimension).}.そのほとんどは時間の移動に注目した分類項目やそこから発展したフィクションの形態に基づく分類である.最後に「4次元」という分類を設けている.そこではパヴロフスキーがまず紹介され,その淵源にアボット・アボットの『フラットランド』(1884)のような幾何学を題材にした小説群の系譜があるとされ,1930年代以降の系譜が辿られる.しかし,『四次元郷』で幾何学的な題材が扱われるのはほとんど冒頭の章だけで,あとは原子論的な宇宙観が提示され,リヴァイアサンの支配と解放が描かれる.『フラットランド』は2次元人と1次元人の対決を描いており,ヴァン=エルプが示す後継作品も異なる次元どうしの接触がテーマとなっていることからすると,『四次元郷』はそうしたジャンルには全く当てはまらない.こうした見解は他の論者も共有しており,ジャック・ボドゥは『SF文学』の「四次元およびその他の次元」というジャンルの作品として,20世紀のSF史上最も有名な「四次元」を名に冠したパヴロフスキーの同作品を挙げていない\footnote{ジャック・ボドゥ,前掲書,120-1頁. }.

異次元空間を舞台にしている作品ではないとしても,やはりウェルズや同時代の異次元ブームとの関係性を論じている研究者がいる.非ユークリッド幾何学の文化的流行が芸術に与えた影響を多角的に論じたヘンダーソンはパヴロフスキーをウェルズの「フランスの門弟(French disciple)\footnote{Henderson, \emph{op. cit. }, p. 134.}」と見做している.パヴォロフスキーはウェルズのような社会的批判をしながら\footnote{パヴロフスキーは第一次世界大戦後に手を加えた1923年版の『四次元郷』で,「その時代の科学の圧制に対抗する抗議(une protestation révolté contre tyrannie scientifique du moment)」(1923, 10)の試みであると述べているように,同時代の科学技術に対して楽観的な意見を持ち合わせていなかった.この態度は,科学の進展に対する興味について,オーギュスト・ブランキやシャルル・クロと比すべきものがあるというのが本論における私の主張でもある.},高次元空間の哲学を展開した数学者チャールズ・ハワード・ヒントン(Charles Howard Hinton)から続く文脈との「特有のブレンド(unique blend)\footnote{\emph{Ibid. }, p. 151.}」であると述べている.ところで,この「特有のブレンド」という言葉こそ『四次元郷』を特徴づけるものなのだ.確かに,SF研究では後続の作品と合わせて先駆的作品を決定していくために,注目する点は物語で採用されている舞台や展開,あるいは登場する機械などに絞られている.しかし,『四次元郷』で展開される世界観や,断片的なエピソードだけの報告といった三面記事的構成,非ユークリッド幾何学や原子論などの科学的知識の紹介といった科学記事的な文体の特徴は後続の作品で取り上げられることはなかった.また,同時に,それはヴェルヌに影響を受けた冒険譚とも違うし,未来社会の予想図が統一的に描かれていない.本書では触れなかったが,アルベール・ロビタ(Albert Robita)は,『20世紀(\emph{Le Vingtième Siècle})』(1883)に代表される未来の社会の様子を断片的に描く作品を残しているが,ヴェルヌらが常に同時代の科学知識から予見されるうる出来事を中心に描いていると評されるのと同様に\footnote{ジャン・ガッテニョ『SF小説』,小林茂訳,1971年,14頁.},ロビタの描く世界は物理的な現実に根ざしている.その一方で,パヴロフスキーは「心霊主義の小説家」とボダンに言われるように,魂が実在する世界を描き,現実の世界を突き放している.『四次元郷』という作品は非ユークリッド幾何学といった数学的知見や,化学や生物学,時には精神医学の知見が活かされているのにもかかわらず,この現実を描く意志を欠いてしまっているのだ.それは,カイヨワの言葉をもじるのであれば,まさに「SFから妖精へ」と言える幻想小説への回帰なのかもしれない.しかし,パヴロフスキーは「四次元は私たちの世界の理解を完全なものにする(Elle[=~la vision de la quatrième dimension] complète notre compréhension du monde)」(1/55)と述べているように,あくまでもこの世界の真実について論じようとしていたのであり,幻想小説と単純に分類することはできない.

以上のように『四次元郷』は出版史の観点からはその位置付けを明確にできる.しかし,SF研究からは曖昧な定義しかできない.さらに,文化史的な観点を援用することで小説と哲学の両方から作品の特徴を定めることが可能である一方,その試みは同時代に比べて評価が定めにくいことが明らかになった.そこで,次章では,4次元を一般に広めたパヴロフスキーの活動に注目することで,『四次元郷』がフランスの4次元の流行にどのように関わったのたかを知ることで,文化史的な位置づけを補足したい.


\chapter{4次元とベルクソン}
\section{4次元について}
\subsection{フランスにおける4次元の大衆化}
第3章では,『四次元郷』を出版史的な観点からその姿を位置付けることができるということを示した.実際は,パヴロフスキーが追悼記事で4次元の大衆化の立役者であったと評せられているように,『四次元郷』は4次元に関する科学記事のような役割を持っていた.以下では,第3章の最後で示した,ヘンダーソンが提示したパヴロフスキーの評価,すなわちウェルズとヒントンの「特有のブレンド」が具体的に何を意味するのかを,4次元の文化史を示すことで確認したい.

4次元の文化史的な起源を線・平面・立体の3つの次元に加えて第4の次元がある,という考えがあったことだとすれば,それは18世紀に求められるだろう.1754年にはダランベールが『百科全書』の\emph{Dimension}の項目の中で,もう1つの次元としての第4の次元は時間であることを指摘していることが知られている\footnote{次がその1節にあたる.「私は3次元以上は認識できないとはっきりと述べた.その一方で,私の知人の才気ある男は,持続は4次元だみなしうると信じている(J’ai dit plus haut qu’il n’étoit pas possible de concevoir plus de trois dimensions. Un homme d’esprit de ma connoissance croit qu’on pourroit cependant regarder la durée comme une quatrieme dimension)」(引用箇所は以下を参照のこと.\emph{Encyclopédie ou Dictionnaire raisonné des sciences, des arts et des métiers}, v. IV, 1754, pp. 1009-10. また,ENCCREが電子版で公開しているウェブページは次の通り.http:\slash\slash enccre.academie-sciences.fr\slash encyclopedie\slash article\slash v4-2546-0\slash )持続(durée)とは,ある運動などがある一定期間続く意味での時間のことである.}.結論から言えば,この18世紀的な4次元の定義が一般に流布するようになったのは,1895年のウェルズの『タイム・マシン』が登場してからのことだと考えられる.しかし,数学の研究における4次元は,ギリシャ以来のユークリッド幾何学の公理を超える新しい数学的概念として発明されたものであり,時間とは全く関係のないものであった.そこで,数学史における4次元の歴史を以下で概観する.

19世紀になると,数学者たちは,ユークリッド幾何学での平行線の公理や内角の和が180度になる公理などに従わない空間の存在を考えていた.いわゆる非ユークリッド幾何学である.この新しい幾何学の研究者として,カール・ガウス(Carl Gauss),ニコライ・ロバチェフスキー(Nikolai Lobachevsky), ヤノス・ボヤイ(Jànos Bolyai),エウジェニオ・ベルトラミ(Eugenio Beltrami),ベルンハルト・リーマン(Bernhard Riemman),ヘルマン・フォン・ヘルムホルツ(Hermann von Helmholtz),フェリックス・クライン(Félix Klein)が挙げられる.このうち,パヴロフスキーはロバチェフスキー,ベルトラミ,リーマンの名前を挙げている(11/60).比較的早い時期からガウス,ロバチェフスキー,ボヤイらは独自に非ユークリッド幾何学の可能性を模索していった.特に,ベルトラミの擬球が有名である~図~\ref{fig:pseudo}\myfig[height=6.4cm, width=9.1cm]{pseudo}{ベルトラミの擬球}{pseudo}\footnote{CC BY-SA 3.0で公開されている以下の画像を用いた.https:\slash\slash commons.wikimedia.org\slash wiki\slash File\%3APseudoSphere.jpg}.3人の成果以降,ベルトラミやヘルムホルツは,さらに発展的な研究をした.それはリーマンの研究が基礎となっていた.

リーマンが1854年にゲッティンゲン大学での教授資格審査(Habilitation)で発表した「幾何学の基礎にある仮説について(über die Hypothesen, welche der Geometrie zu Grunde liegen)」が1868年にゲッティンゲンの王立科学学会紀要にて出版され\footnote{\emph{Abhandlungen der Königlichen Gesellschaft der Wissenschaften zu Göttingen}, t. 13, 1868, pp. 133-150.},ギョーム=ジュール・オウエル(Guillaume-Jules Hoüel)が著名な数学誌『純粋・応用数学年報(\emph{Annali di matematica pura ed applicata})』にフランス語で訳出した後であった\footnote{Georg Friedrich Bernhard Riemann, «~Sur les hypothèses sur lesquelles est fondée la géométrie~», \emph{Annali di matematica pura ed applicata}, t. III, n. 2, trad. Jules Hoül, Heidelberg, Springer, 1870, pp. 309-26.}.リーマンがそこで主張していたのは,幾何学的対象となっているユークリッド幾何学は実は経験に基づいたものであり,そもそも幾何学の対象となる空間は全て仮説にほかならないので,一切の経験的対象から独立し,それ自体で自立した幾何学的な対象が考察しうるという主張だった.リーマンはさらに,その幾何学的対象は計測できるのだと考えた.これは,当時においては画期的な発想だった.ユークリッド幾何学における図形は経験的な対象で可視的な存在であり,外部との関係で位置を特定することができるからこそ計量ができるが,それ自体で自立している幾何学的対象を計測する方法をその時代の人々は知らなかった.リーマンはその問題を解決するために,それ自身によって存在する幾何学的対象を多様体(Mannigfaltigkeitslehre)と定義した.ユークリッド幾何学の図形の計量は他の図形との比較によって可能となっていたが,多様体では,ものさしとなる線素\footnote{ベクトル解析を行う場合の基本単位.空間の距離を意味する.}の無限小の運動可能性を仮定することで,条件によって変化し続けても計量できると考えた.この考えはユークリッド幾何学で扱われていた対象がいかなる運動においても変形しない対象(いわゆる剛体)であったのに対して,対象の変形(可変曲率)をも計測できるようになったことによって,あらゆる曲面を計測するための道を開いたのだった\footnote{以上のリーマンの解釈は以下を参照のこと.加藤文元『リーマンの数学と思想』,共立出版,2017年,第4章.}.しかし,リーマンの多様体の概念が論文の形になった後でも,一部の専門家にしか知られていなかった.一連の非ユークリッド幾何学が一般に知られていくようになるのは,数学に造詣のあるサイエンスライターや数学者たちによる一般に向けての解説によるものだった.フランスではその仕事の役割を担ったのは,とりわけアンリ・ポアンカレ(Henri Poincaré)であった.

では,ポアンカレが担った非ユークリッド幾何学の普及は,ポアンカレ以前のフランスではどのように進んだのかをその歴史的文脈から見ていこう.19世紀フランスで非ユークリッド幾何学を数学の専門家以外が扱った例として有名なものは哲学者イポリット・テーヌ(Hippolyte Taine)だろう.彼は1870年の著書『知性について(\emph{De l'Intelligence})』において,「読者は自然の中には存在しない幾何学的対象を知っている(le lecteur sait que les objets géométriques n'existent pas dans la nature)\footnote{Hippolyte Taine, \emph{De l'Intelligence}, Paris, Hachette, 1870, p. 57.}」例として,曲面や球体を挙げている.また,19世紀後半を通じてフランスでは,実証主義に由来した経験主義的な雰囲気が支配的だったためか,テーヌは非ユークリッド幾何学は現実で経験することはできないが,計算で扱うことができるものであるので感覚によって捉える必要がないと結論づけている\footnote{Taine, \emph{op. cit. }, pp. 58-9.}.こうした議論に関係し,ヘルムホルツの高次元空間に関する考察で,雑誌『哲学(\emph{Revue Philosophique})』や『形而上学と道徳(\emph{Revue  de Métaphysique et de Morale})』は1880年代から1890年代を通じて定期的にその翻訳が掲載された.ヘルムホルツはユークリッド幾何学の信奉者たちがカントの幾何学的時空間は直観によって把握されているものであり,非ユークリッド幾何学で想定される高次元空間はその直観では正当化できないとしていたのに対して,高次元空間もまた直観によって把握できるうえ,それ自体独立して表象もできると考えていた.その後,高次元空間の表象については,エスプリ・パスカル・ジュフレ(Esprit Pascal Jouffret)によって通俗化されたイメージがもたらされた.彼は1903年に『4次元幾何学序説(\emph{Traité de géométrie à quatre dimension})\footnote{Esprit Pascal Jouffret, \emph{Traité de géométrie à quatre dimension}, Paris, Gauthier-Villars, 1903.}』を発表して,3次元において4次元がどのように表現しうるかというを,エドウィン・アボット・アボット『フラットランド(\emph{Flatland})\footnote{Edwin Abbott Abbott, \emph{Flatland~: A Romance of Many Dimensions by Square}, London, Seeley \& Co. , 1884.}』の2次元人と3次元人の接触の場面を引き合いに出している\footnote{Jouffret, \emph{op. cit. }, p. 187.}.『フラットランド』はフランス語に翻訳されていなかったものの,非ユークリッド幾何学の発明より後に書かれるようになった異なる次元の空間どうしの接触というテーマはすでに英語圏ではよく知られていた.ジュフレの著作によってこのテーマはフランスに持ち込まれ,何人かのキュビストやマルセル・デュシャンなどに影響を与えたと考えられている\footnote{Henderson, \emph{op. cit. }, Chpater 2 and 3.}.

ジュフレーによる通俗化された高次元空間だったが,ポアンカレはヘルムホルツが行なっていた議論を再び取り上げることで,イメージとしてではなくて人間の近くにおいて高次元空間は把握できるのかどうかを解説した.ポアンカレはヘルムホルツの高次元空間をめぐる議論を大衆向けの著作『科学と仮説\footnote{Henri Poincaré, \emph{La Science et l'Hypothèse}, Paris, Ernst Flammarion, 1902.}』(1902)の中で,高次元空間の表象可能性は,数学の対象である幾何学空間におけるものなのか,それとも知覚によって把握される表象空間(espace représentatif)におけるものなのかを区別しなければならないと主張した.幾何学空間は連続(continu),無限(inifini),3次元空間を持っている,等質である(homogène)\footnote{空間上のあらゆる点はどの点についても点どうしは同じであるということ.あらゆる点の性質が同じであることを意味する.},等方的である(isotrope)\footnote{ある同一の点を通る空間上の全ての線についても線どうしは同じであるということ.あらゆる直線の性質が同じであることを意味する.},という5点によって定義される\footnote{Poincaré, \emph{op. cit.} , p. 69.}.一方で,表象空間は視覚的・触覚的・運動的(motrice)の3つによって定義される.この3点は,等質的でも等方的でもないのは自明であるので,「誰一人として3次元があるということさえ言うことはできない(on ne peut même pas dire qu'il ait trois dimension)\footnote{\emph{Ibid.} , p. 74.}」ということになる.ポアンカレはこれらの帰結として,空間をどのように扱うかはその時の「規約(convention)」に従うものであるとした.ヘンダーソンは,ヘルムホルツが1つの空間の中で高次元空間の実在を是非を思考したのに対して,ポアンカレは規約(convention)という概念を導入することで複数の空間があることを認めて,これらの問題を解決したが,それによっていわゆる「4次元世界の可能性を開いたままにしている(leaves open the possibility of a four-dimensional world)\footnote{Henderson, \emph{op. cit. }, p. 137.}」と指摘している.ポアンカレはその後の著作\footnote{ポアンカレは1904年に『科学の価値』を,1908年に『科学と方法』を出版した.Henri Poincaré, \emph{La Valeur de la science}, Paris, Ernest Flammarion, 1904. Henri Poincaré, \emph{Science et méthode}, Paris, Ernest Flammarion, 1908. }でも数学と物理学で議論となっていることやその理論的な背景を大衆に向けて発表しつづけ,それに伴って非ユークリッド幾何学の存在が広く知られていくこととなった.

『四次元郷』が出版された1912年にポアンカレは逝去した.1908年から連載されている『四次元郷』が取り扱っているテーマが読まれる下地を作ったのはポアンカレだったが,パヴロフスキーは『四次元郷』の中でポアンカレについて次のように述べている.
\begin{quote}
球面の表面を移動しているかもしれない平面の人間たちは,常に180度より大きい三角形の内角の和における幾何学を自然と考えているだろう.同じように,立体がない世界では,私たちの幾何学はわずかばかり苦しめられるのが明らかとなるだろう.アンリ・ポアンカレは優れた洞察力のある一節でこのテーマについて書いている.
\end{quote}
\begin{quote}
Des êtres plats, qui se déplaceraient sur une surface sphérique, concevraient tout naturellement une géométrie dans laquelle la somme des angles d'un triangle serait toujours supérieure à deux droits. De même aussi dans un monde dépourvu de solides, notre géométrie pourrait éprouver quelque peine à se faire jour. H. Poincaré a écrit sur ce sujet des pages fort clairvoyantes.(28/70)
\end{quote}
ここで述べられているポアンカレの一節とは切断(découper)の手法についてである.切断は高次元の対象を部分的に低次元に置き換えて考える手法である.例えば,3次元の立体は切断するとそこに2次元の平面が現れ,その平面もまた切断することで1次元の線が現れる.高次元の対象をそのまま扱うのは困難なので,例えば4次元の対象を切断してその断面を調べることで元の対象の全体を復元することができる.パヴロフスキーは,切断によって「私たちのユークリッド的な科学は欠けたものとなり消え失せる(notre science euclidienne fait défaut et s'évanouit)」(28/70)と考えている.このことから,パヴロフスキーは独自の論点を提出する.すなわち,切断は高次元空間を低次元空間に翻訳するという行為であり,それは科学の用いている慣用的な(conventionnel)言語が,物理的な現象を基礎付けている「質的世界(le monode des qualités)」を扱うことができないことを示している.この「質的世界」とはパヴロフスキーによれば4次元のことである.パヴロフスキーは『科学と仮説』や『科学の価値』で知られていたポアンカレの著作を引き合いに出すことによって自らをフランスの4次元の文化史的文脈に置きつつ,独自の4次元観を示したのだった.

\begin{comment}
Nous pouvons découper des volumes au moyen de surfaces. Nous pouvons découper des surfaces au moyen de lignes, nous pouvons déterminer des lignes au moyen de points. Mais, lorsqu'il s'agit pour nous de définir le point, notre science euclidienne fait défaut et s'évanouit. Lorsqu'il nous faut rendre compte du continu physique, notre impuissance est extrême. Nous comprenons bien que la science n'est autre chose qu'un langage conventionnel qui nous permet de cataloguer et de classifier certaines fractions de phénomènes que nous détachons artificiellement l'une de l'autre, d'après leurs qualités, mais nous sentons bien que cette science, de même que le langage, est incapable de traduire cette continuité qui appartient au monde des qualités et que l'on ne saurait définir par des chiffres.
4次元小説の代表例
ただし,これはEncyclopédie de l'utopie des voyages extraordinaires et de la science fictionのDimensionの項目に依拠している.

1844, Edwin. A. Abbott, Flatland, une aventure à plusieurs dimensions
1886, C. H. Hinton, A Plane World, Scientific Romances
1893, Ambrose Bierce, Charles Ashmore’s Trail, Can such things be?
1895, H. G. Wells, Un étrange phénomène
1896, H. G. Wells, L’histoire de Plattner
1921, Austin Hall, The Blind Spot
1922, Gabriel de Lautrec, Dan le monde voisin…, La vengeance du portrait ovale
1923, Claude Farrere, LA-BAS, Où?
1925, Jean Ray, Les étranges études du docteur Paukenschlager, Contes du whisky
1932, Homer Eon Flint, The Spot of life
1944, Léon Groc, La planète de cristal
1944, René Barjavel, Le Voyageur imprudent
1957, David Ducan, La raiser d’Occam
1969, Harlen Ellison, A Boy and his Dog
\end{comment}
\subsection{ヒントンの超空間哲学とパヴロフスキー}

パヴロフスキーがポアンカレの著作を踏まえたことで自らを4次元の文化史に位置付けていることはすでに示した通りであるが,それだけでは『四次元郷』の4次元の特徴は明らかにはならない.なぜなら,前章で述べたように,『四次元郷』における4次元はパヴロフスキーの独自の哲学を背景にしているために,他の4次元ないし異次元を扱った小説とは一線を画しているからである.ヘンダーソンは,それに対して,ウェルズとヒントンの「特有のブレンド」としてパヴロフスキーを4次元の文化史の中で位置付けようとした.そこで,4次元空間の哲学を展開したヒントンの哲学を概観することで「特有のブレンド」の意味を具体的に検討していきたい.

リーマンの多様体の影響が徐々に広まっていく中,1870年代になると,リーマンとは別のアプローチでn次元幾何学の一般化が試みられるようになっていく.1880年,アーヴィング・ストリングハム(Irving Stringham)が現在は\emph{Polytope}と呼ばれる図形のうち,4次元の場合の図形である\emph{Polychoron}(\emph{Polyhedroid})を図示した\footnote{Washington Irving Stringham, "Regular Figuers in n-Dimension Space", \emph{American Journal of Mechanics}, III, 1880, pp. 1-12.}.これは各次元の最小構成の図形を次元が増えるごとに構成し直すというアイディアに基づいている.例えば,2次元空間で線や面の数が最小の構成は三角形である.3次元空間では立体である.このことから,4次元空間では5つの四面体から構成された図形が最小の構成であると言える.ストリングハムの\emph{Polychoron}はそうして構成される図形のパターンを描いたものである.このように高次元を図形で表現した数学者としてチャールズ・ハワード・ヒントン(Charles Howard Hinton)の名前を挙げることができる.ヘンダーソンはヒントンを高次元空間に関する哲学を生み出した「真の超空間哲学者(true hyperspace philosopher)\footnote{Henderson, \emph{op. cit.} , p. 127.}」であると述べている.超空間哲学の核にあるのは,人間の空間感覚(space sence)の拡張によって3次元を超えた空間を知覚できるようになるという信念だった.1880年代にヒントンは精力的にこの思想を練り上げていき,ついに『思考の新時代(\emph{A New Era of Thought})』(1888)を,1904年には『四次元(\emph{The Fourth Dimension})』を刊行した.この2つの著作で展開されているのは,経験に基づいたユークリッド幾何学を直観として考えているカントを批判しているような,リーマンを代表とする非ユークリッド幾何学的な考えとは逆に,カントが空間を直観によって把握できるとすることは正しく,その直観を4次元にまで適応させればよいという経験主義的な理論であった.そして,その直観を手に入れるために考案された道具が\emph{tesseract}だった.その姿は『四次元』の口絵で非常によく知られている~図~\ref{fig:tes}\footnote{下記口絵を参照のこと.Charles Howard Hinton, \emph{The Fourth Dimension}, London, S. Sonnenschein, 1904.}.
\myfig[height=18.2cm, width=12.8cm]{tes}{\emph{tesseract}}{tes}
\emph{tesseract}が重要なのは,訓練方法によってヒントンが4次元の特徴をどのように示しているか明確になるからである.\emph{tesseract}による訓練の意義は,2次元と3次元の関係に置き換えると理解しやすい.もしも2次元に住んでいる人がいたとすれば,3次元の球体がそこをとおりぬける時,円が徐々に広がっていき,しばらくするとそれが閉じていくように見える.すわち,3次元の一部分しか知覚しえないのである.そうであるならば,4次元の図形が3次元を通り過ぎるときも同様でないかとヒントンは考え,4次元の図形の一部分はカラキューブのバリエーションとして表現されうると考えた.この時,ヒントンは4次元を運動において捉えようとしており,運動の持続する時間の中でしか捉えられないと考えていた.彼によれば,「あらゆる4次元を視覚化する試みは無駄である.それは3次元の中の時間的経験と関係しているに違いないからである(All attemps to visualize a fourth dimension are furtile. It must be connected with a time experience in three space)\footnote{Hinton, \emph{op. cit. }, p. 207.}」.また,ヒントンの超空間哲学のもう1つの特徴は自らをプラトンのイデアやカントの物自体と結びつけていることにある.『四次元』の第4章と第5章は4次元空間の歴史に当てられていて,プラトンのイデアの定義はより高い次元のアナロジーとして考えられるし,カントの物自体はヒントンにとって啓示されうる3次元を超えたものだった.そして,超空間を私たちが知覚しうるのは,3次元は4次元に延長しうるものであるからである.

ヘンダーソンは以上のようにヒントンの超空間哲学をまとめているが,確かに,パヴロフスキーの描く4次元と似ているように思われる.『四次元郷』では4次元の存在を確信するのは,あらすじで見たように,身辺で生じた異常な現象である.それは知覚によって4次元と接しているということである.また,本論では第6章で詳しく述べるように,パヴロフスキーは3次元は4次元という入れ物の中身であるかのような表現をしているので,3次元は4次元の延長であるかのように描かれている.しかし,次の一節はヒントンとも僅かに異なる.
\begin{quote}
3つの次元に加えられた第4の尺度のような,というよりもむしろ宇宙を理解するプラトン主義的方法のようであり,プラトン主義的方法のためにアリストテレスと仲違いする必要もなしに,永遠かつ不動の4次元の様相において物事を理解し,行為の質よりほかに到達しはしないための,量における運動からの解放を可能とする脱走方法のような,4次元.
\end{quote}
\begin{quote}
la quatrième dimension comme une quatrième mesure ajoutée aux trois autres, mais plutôt comme une façon platonicienne d'entendre l'univers, sans qu'il soit besoin pour cela de se brouiller avec Aristote, comme une méthode d'évasion permettant de comprendre les choses sous leur aspect éternel et immuable et de se libérer du mouvement en quantité pour ne plus atteindre que la seule qualité des faits.
\end{quote}
ここでプラトン主義に親和的である4次元の存在はやはりヒントンと関係していると言えるが,一方で,パヴロフスキーの4次元には「量における運動」は存在しないので時間もない.よって,「永遠かつ不動」なのである.ヘンダーソンが『四次元郷』を「特有のブレンド」と称している意味は,この差異がヒントンとパヴロフスキーを分けているということなのだ.

\section{ベルクソンとパヴロフスキーの4次元}
4次元の文化史的な観点からは,ヒントンとパヴロフスキーの相違点が明らかになった.ヘンダーソンはヒントンとの比較で時間と運動に注目したが,ヒントンとは別の人物とも比較している.それは4次元についての小説や哲学的テーゼを打ち出しているわけではないが,パヴロフスキーや彼と同時代の思想家・芸術家にとって重要な参照点だった.その人物はアンリ・ベルクソンという.

パヴロフスキーにとってアンリ・ベルクソンが大きな参照点であったことは,彼の博士論文『労働の哲学』の執筆時に『意識に直接与えられたものについての試論(\emph{Essai sur les données immédiates de la conscience})』(以下,『直接与えられたもの』)の議論を検討していることからもよくわかる(PT166).以下で,その議論について見てみよう.パヴロフスキーは『労働の哲学』で,社会理論(théories sociales)を構築するために科学を用いることはできないと述べている.というのも,社会とは個人で構成されているものであるから,社会理論とは,個人の行為を理論的に説明するものであり,個人の行為とは,意志と自由の問題であり,科学の対象とはならない.なぜなら,科学は私たちの外側にある世界について数量的に扱っているのに対して,私たちが生きている状態である質を扱うことができないからである.この根拠として用いられているのが『直接与えられたもの』であった.

パヴロフスキーは1923年の自著改題である「批判的吟味」や,1912年版の冒頭でも4次元について多くの言葉を費やしているが,それの多くは,質(qualité)と量(quantité),あるいは持続(durée)と同時性(simultané)という2つの言葉を鍵概念にして語られている.例えば,パヴロフスキーは精神が日々の複雑な活動を把握する方法は連続しているものをそれぞれ理解するのではなくて同時的に理解するほかないと考えている.パヴロフスキーは次のようにまとめている.
\begin{quote}
4次元についての全体的な理解の知性的なこれらのきらめき,私たちは時間の中のある持続を自然とそのきらめきだとしてしまうし,あまりにもきらめきがほのかなので私たちはそれを少なくとも何らかのつかの間の持続だと思っている.しかし,その持続は存在さえしない.というのは,4次元世界の中で持続がありえないようだし,したがって,要するに大理石の像の異ったあらゆる部分のような同時的な活動における必然的な連続などありえないだろうからだ.
\end{quote}
\begin{quote}
Ces lueurs intellectuelles de compréhension totale à quatre dimensions, nous leur attribuons nécessairement une durée dans le temps et, si fugitives qu'elles soient, nous leur supposons tout au moins une durée de quelques secondes. Or, cette durée n'existe même pas, car il ne saurait y 'avoir de durée dans le monde à quatre dimensions et par conséquent aucune succession nécessaire dans des actes qui sont, en somme, simultanés comme toutes les parties distinctes d'une statue de marbre.(37/74) 
\end{quote}
パヴロフスキーが大理石の像を持ち出しているのは,4次元には運動も時間も存在していないことを示すためだと考えられる.静止した運動しないものの全体を把握する方法は部分を取り上げることではなくて全体を同時的に把握するしかないのである.

ベルクソンとパヴロフスキーの差異は奇妙なねじれを含んでいる.カント以来の数学的な等質時空間モデルを否定して持続という概念を生み出し,それは根本的に質と量の混同に由来しているベルクソンと同じように,数学的な世界の限界と質への志向を主張しているのにもかかわらず,パヴロフスキーは持続を認めていないのである.ヘンダーソンもこの点に注目しており,「パヴロフスキーは,一方で,ベルクソンの優先順位を覆して,存在の真の同時性の啓示を4次元の美点の1つだと考えた(Pawlowski, on the other hand, reversed the preferences of Bergson and considered as one of the virtues of the fourth dimension its revalation of the true simultaneity of existence, in contrast to the appearance of succession in three dimensions.)\footnote{Henderson, \emph{op. cit. }, p. 204.}」とまとめている.それでは,この違いはどこに由来しているのだろうか.

この違いを説明するのは,パヴロフスキーとベルクソンの過去と未来への態度の差である.ベルクソンにとってこれらの時間の様相の問題は主体と記憶の問題に他ならなかった.1896年の著作『物質と記憶』が1911年に第7版の刊行に際して寄せた序文で,ベルクソンは「この書物は精神の実在と物質の実在を肯定し,両者の関係を特定の例,すなわち記憶の例によって規定しようとする(Ce livre affirme la réalité de l'esprit, la réalité de la matière, et essaie de déterminer le rapport de l'un à l'autre sur un example précis, celui de la matière)\footnote{アンリ・ベルクソン『物質と記憶』,田島節夫訳,白水社,1965年,5頁.Henri Bergson, \emph{Matière et Mémoire}, dir. Frédéric Worms, Paris,PUF, 2010, p. 1.}」と書き出しているように,記憶を考察することで実在論を組み立てることがベルクソンのプロジェクトの1つだった.

このプロジェクトでは,『直接与えられたもの』のテーマであった質としての時間すなわち持続によって物質はイマージュの総体であると説明される.ところで,記憶は「精神と物質の交錯点(le point d'intersection entre l'esprit et la matière)\footnote{同書,9頁.\emph{Ibid. }, p. 5.}」であり,さまざまな心理状態における記憶や記憶力が分析の対象となる.ところで,記憶は一般的に過去の出来事であり,記憶力によって現在において想起されるが,「記憶力の本質はけっして現在から過去への遡行にあるのではなく,反対に過去から現在への前進にあるのだ(La vérité est que la mémoire ne consiste pas du tout dans une régression du présent au passé, mais au contraire dans un progrès du passé au présent)\footnote{同書,266頁.\emph{Ibid. }, p. 269.}」から,持続とは現在へと流れていくものである.

この考えを踏まえてベルクソンの未来と過去についての考えを簡潔にまとめる.『物質と記憶』では生成される現在というテーマが中心であり,記憶の諸相の分析によって,過去は現在へ前進していくものであると考えられていた.『精神のエネルギー(\emph{L'énergie Spirituelle})』(1919)に収めらている1908年初出の「現在の回想と誤った認知(Le Souvenir du présent et la fausse reconnaissance)」の章で述べられているように,過去には決して現在にはならない純粋過去というものの存在を認めている.すなわち,流れゆく持続とは別に現在と共存する過去があると述べている\footnote{以下を参照のこと.Élie During, «Le souvenir du présent et la fausse reconnaissance», \emph{L'énergie Spirituelle}, dir. Frédéric Worms, Paris, PUF, pp. 307-13.}.一方で,人は一般的に未来の記憶を持っていることはないので,未来についてはこのようには語られていない.精神が記憶力として作用する場合,それは「未来をめざしての過去と現在の総合(synthèse du passé et du présent en vue de l'avenir)\footnote{ベルクソン,前掲書,246頁.Bergson, \emph{op. cit. }, p. 248.}」と表現されたりしているように,未来は持続の流れていく先のことを示している.よって,ベルクソンにとって未来は持続の流れていく方向のことを示している.実際に,ベルクソンが未来は独立して存在しているかのように語ることを錯覚だと示唆する逸話も存在する\footnote{Henri Bergson, «~Le réel et le possible~», \emph{La pensée et le mouvant}, Paris, PUF, 1955, p. 110.}.

ベルクソンの過去と未来の態度は以上に見たとおりである.次にパヴロフスキーについて見てみよう.パヴロフスキーが『物質と記憶』を『四次元郷』やそれ以前の著作で引用していることはないものの,明らかに参照していると思われる.例えば,プルーストが『失われた時を求めて スワン家の方へ』を出版した1914年に,パヴロフスキーが書いた同作についての書評では,プルーストがベルクソン的な方法によって執筆しているという指摘をしている.ベルクソン的な方法をとることによって,「著者は,私たちの印象が精神に生じるにつれて,脳内で相次いで起きている\bou{常に現実的な},無意志的想起,印象,感覚を書き記すことだけができる(Il[=~l'auteur] ne peut que noter, au fur et à mesure qu'elles[=nos impressions] se présentent à son esprit, les réminiscences, les impressions, les sensations \emph{toujours actuelles} qui se succèdent dans son cerveau. )\footnote{Gaston de Pawlowski, «~La Semaine Littéraire~», \emph{Comœdia}, 11 janvier, 1914, p. 3.}」と述べている.記憶の現実的な感覚が記されているという表現は,『物質と記憶』で「観念が言語的イマージュという特殊なイマージュの中で肉体を獲得するにいたる(l'idée arrive à prendre corps dans cette image particulière qui est l'image verbale)\footnote{ベルクソン,前掲書,148頁.Bergson, \emph{op. cit. }, p. 145.}」,あるいは「純粋記憶は,現実化するにつれて,対応するすべての感覚を身体の中に生ずる傾向をもつ(Le souvenir pur, à mesure qu'il s'actualise, tend à provoquer dans le corps toutes les sensations correspondantes)\footnote{同書,149頁.\emph{Ibid. }, p. 146.}」という一節を思わせるし,パヴロフスキーが「脳内で」と限定していることは記憶の現実化について論じている第2章の副題「記憶力と脳」が暗示されていると考えられる.では,ベルクソンの議論をパヴロフスキーが知っていたとして,パヴロフスキーは過去と未来についてどのように語っているのだろうか.

パヴロフスキーの過去と未来に対する態度は,極めて明確なテーゼにまとめることができる.それは,\bou{過去は存在しない},\bou{未来しか存在しない}というものである.以下を見てみよう.
\begin{quote}
過去の中の旅には関しては,この物語の途中で見つかることがないのに驚くこともないだろう,というのもその手の旅は不可能なのだから.四次元郷ではこの一瞬に未来だけが存在している.過去はもはや存在しない,それは全体的に現在の中に含まれているからであり,私たちに起きているあらゆることを知るための力ある意志とともに私たちの記憶を内側に喚起するのには十分である.
\end{quote}
\begin{quote}
Quant aux voyages dans le passé, on ne s'étonnera pas de n'en point trouver au cours de ce récit, car ces sortes de voyages sont impossibles. L'avenir seul existe en ce moment dans le pays de la quatrième dimension. Le passé n'existe plus, puisqu'il est entièrement contenu dans le présent, et il suffit d'évoquer intérieurement nos souvenirs avec une volonté puissante pour connaître tout ce qui s'est passé jusqu'à nous.(56/84)
\end{quote}
パヴロフスキーが過去は現在に含まれているために存在していないという考えは文脈は大きく異なるものの,ベルクソンの過去についての考えと近い部分がある.過去から現在への持続の流れは,過去が現在に含まれているという見方を可能にするものであるからだ.しかし,現在と共存する過去が存在することもベルクソンが認めているので,この点においてやはり決定的に異なっている.とりわけ,「未来だけが存在している」というフレーズは,生成し,展開していく現在というベルクソンのモデルと大きな隔たりがある.しかし,これらの差異はパヴロフスキーが4次元を対象に述べているのだから必然的に生じるものであると言える.

パウロフスキーにとってベルクソンの議論は3次元でしか成立しないのだ.パヴロフスキーが「過去はもはや存在しない,それは全体的に現在の中に含まれている」と述べているのは3次元であり,ベルクソンのとりわけ『物質と記憶』に基づいていることをうかがわせるのに対して,未来だけが存在する4次元というテーゼこそ,パヴロフスキーの独自性なのである.すでに本論ではヘンダーソンによるヒントンとパヴロフスキーの相違点の確認した.そこでは,単に高次元空間に時間と運動を認めるか否かが議論されていた.4次元は時間と運動がない一方で,未来だけが存在しているというヴィジョンこそパヴロフスキーの4次元が特異であることがベルクソンとの比較によって示すことができるのである.

ところで,近年,ベルクソン研究者のエリー・デュリングによって提示された「レトロ未来」という概念がある.これはベルクソンの「現在の回想と誤った認知」の研究から導かれた芸術論の1つである.現在と共存するが現在とは独立した過去の存在があるのであれば,現在と独立した未来も存在するのではないか,というのがレトロ未来の着想源である.私は,この概念はパヴロフスキーの4次元における未来だけが存在しているというテーゼと極めて近いものがあると考えている.というのも,パヴロフスキーは4次元が3次元に発現する契機に,芸術作品をあげているからだ.例えば,「着手された芸術作品,永遠の〈観念〉と物質の間の特有の接触点(l'œuvre d'art entreprise, la création personnelle qui est l'unique point de contact entre l'Idée éternelle)」(1912, 320)と述べている箇所では,4次元が「永遠の〈観念〉」と言い換えられて,まさに作られようとしている芸術作品は4次元が3次元に表現される契機であると語られている.レトロ未来では,現在から見た未来は存在せず,過去からみた未来しか存在しないというテーゼが掲げられている\footnote{エリー・デュリング「レトロ未来」,新村一宏訳,早稲田表象・メディア論学会,表象・メディア研究,第5号,2015年,12-3頁.}.つまり,未来が存在するのであれば,ベルクソン的にはそれは現在のただ中であり,そこから導かれるのは,現在と共存する,確かに存在している潜在的な未来なのである. パヴロフスキーの4次元は完成された状態のものなので潜在的なものなど存在していない.しかし,芸術作品の存在が示しているように,3次元では常に,未来のみが存在する4次元がその姿を間接的に見せているのである.パヴロフスキーの4次元の議論はこのようにレトロ未来というベルクソニズムのパターンに当てはめることによってまた1つ別の独自性を持っていることが明らかになった.すなわち,未来だけが存在しているのではなく,4次元においてはあらゆる未来が存在しており,それは芸術作品において現実化されるのである.パヴロフスキーにおいて潜在性を4次元の3次元的な表現と見做しうるかについてはまた別の機会取り上げる.



\begin{comment}
  
パヴロフスキーの4次元に関する理解が極めて特殊なのは,過去は存在せず未来だけが存在しているというその考えにある.

パヴロフスキーはクラインの壺のことを引用している.2次元多様体で,ユークリッド空間に埋め込むために4次元で曲率0とする5次元が必要となる..また,結び目だけに注目しているのも興味深い.しかもdurerと言う言葉を使っているので完全に自分の解けてしまったリボンの文脈に繋げてしまっている.
\end{comment}


\begin{comment}

杉山本で補足しておく.


竹内訳『意識に直接与えられているものについての試論』2010年
p. 140
「さらに言えば,原子の存在そのものが,何よりも疑わしいのである.原子に次から次に新しい属性が付与されてきた歴史を思い起こせば,原子というのは,どうやら実在する物ではなく,数々の力学的説明の物質化された残滓ではないかと思われるほどだ.」   

p. 203
「しかし,未来の予備的形成にはもう一つの別の形,意識が直接的にそのイメージを提供してくれるがゆえに,人間精神にはさらにいっそう身近な形がある.われわれはさまざまな意識状態のなかを通り過ぎてゆく.われわれはさまざまな意識状態のなかを通り過ぎてゆく.そして,後に続く状態がそれに先立つ状態のなかに含まれていなくとも,とにかく漠然とではあっても,そのような内在観念をわれわれが表象していたことは間違いない.この観念が現実化するものとして自覚されていたわけではないが,ありうるかもしれない可能性として感じられてはいたのだ.いずれにしろ,観念と行動のあいだには,ほとんど感じられない程度の中間項が入り込んでいて,その中間項の全体が,われわれにとってはある種の独特の形をとって現れる.それをわれわれは努力感情と呼んでいる.観念から努力へ,努力から行動への進展はきわめて連続的なので,どこで観念が終わり,どこで努力が終わったのか,どこで行動が始まったのか,それを言うことはできない.そこで人はこう考える.ある意味では,この場合も,未来が現在のなかであらかじめ形作られている,と言ってもよいのではないか,と.しかし同時に,ここで予備的に形成される未来はきわめて不完全なものである,と付け加えておく必要はあるだろう.」 
原因と結果は類比的な関係に過ぎない
因果律に関する2つの仮説
1 すべての現象は内的持続と同じように持続する=>現在から未来への移行は目的への努力,となる. 「自然現象にまで偶然性の概念を適用」(207)
2 持続を意識状態に固有のものとみなす=>現在のうちに未来が数理的な状態で存在している.「物理現象の必然的決定性を外的事物の非持続性によるものとする.」(207)
これら2つがベルクソンの批判する考え


P. 210
「例えば,科学者たちが,ファラデーのように,延長をもつ原子を力学的な点に置き換えても,彼らは力点とか力線を数理的に扱うのであって,活力とか努力とみなされる力そのものを問題にすることはない.したがって,外的な因果関係は純粋数理的なものであって,心的な力とそこから発現してくる行為とのあいだにある関係とは,どのような類似性もありえないということはすでに了解事項となっている.」

言語という枠組みにおいては自由を語ることができない,という考えは「3次元においては真理を語ることができない」というパヴロフスキーのテーゼと似ている.というか,内的持続の連続性こそ4次元のことではないか.
つまり,そもそも4次元はau-dessusではなくて,au-dessousではないか.

P. 219
「外的延長と内的持続,この相異なる二つの要素を,外的事物の研究を深化させるために,科学は切り離す.先に示したと思うが,持続からは同時性だけが,運動からは動くものの位置つまりは不動性だけが選びとられるのである.この切断は,科学においては,断乎として,しかも空間優位の形で遂行されるのである.
両者の分離はなお遂行されるべきであろうが,内的現象を研究する場合には,持続を優先させなければならない.内的現象といっても,おそらくその完了形でもなく,論証的知性がそれを説明するために,等質空間のなかにおいて分離し,展開した後の内的現象でもない.そうではなく,形成途上に」220「ある内的現象,相互浸透によって自由な一個の人格の持続的発展を構成するものとしての内的現象である.その原初の純粋状態に戻された内的持続は,まったき質として存在する多様性であり,相互に融合しあう内的諸要素の絶対的な相互異質性なのである.
そう考えてみれば,この当然なすべき両概念の分離を怠ったがゆえに,ある者は自由を否定することになり,またある者は自由を定義することになり,まさに定義することによって,意図に反して,これもまた自由を否定することになったのである.」
ベルクソンの話が科学の限界になる理由がまったくわからん・・・・

P. 221
「そうなると,異なったふたつの自我が存在することになる.一方の自我は,本来の自我が外界に投影された影絵のようなもの,その空間的表象であり,こう言ってよければ,社会的表象である.本来の自我に到達するためには,反省的思索をさらに深めることによって,われわれの内的諸状態を,絶えず生成途上にある,生きた存在として捉えなければならない.」


新訳 ベルクソン全集      月報1                                                                                                                                                                                                                                                                                               2       010年10月
加賀野井秀一 「解題」
「いわく,私たちは絶えず思考を言語によって表現し,しばしば空間的に考える.おかけで言語は,観念相互の間に,物質間に見られ(6)るような区別や不連続を設定する.こうした設定は実生活にも学問にも必要ではあるが,ひょっとすると,ある種の哲学的難問は,空間を占めていない諸現象を空間的に捉えることから来るのではないか.つまり,非延長的なものを延長に,質を量に,不当に翻訳したがために生じているのではないか.(7)」

本田裕志 ベルクソン哲学における空間・延長・物質 (一部引用記号を省略)
P. 9
『試論』における純粋持続に関する特性の1つ
「(5)意識への直接所与性 数学的計算の対象とならない純粋持続は
注意深い意識によって直接的に知覚され到達されるという仕方でのみ認識・把捉される.言いかえれば,それは私たちの直接的意識に現前するがままの持続,生きられる持続である.」

p. 16
『試論』における空間と延長の特徴
「(2)空間と延長の同義性 『試論』の多くの箇所では「広がりのある(extensif)」ものは計測可能であること,延長(étendue)は分割可能であり,延長をつうじて見られたものは量となること,延長を持つものは直接に数で表されること,などが語られ,また「持続=継起=非延長的なもの=質」と「延長(広がり)=同時性=量」とを対比した言い方もしばしば見出される.これらの箇所を見るかぎりでは,同書においてベルクソンは「延長」「広がり」を「空間」と区別せず,等質的環境としての空間そのものの外延的性質を言い表すのに用いている,と解される.
(3)等質的空間の実在性 次元を異にする二つの実在を私たちは知っている.その一つは異質的な,感覚的質の実在,すなわち実在的持続である純粋持続であり,いま一つは等質的な実在,すなわち空間である.空間は,意識への直接所与性によって実在性の確認される純粋持続=心的諸状態とは別次元に属するとはいえ,同様に堅固な実在である.」

p. 17 『試論』における物質的対象
「16物質的対象と空間性 『試論』において物質的対象(objet matériel)・物体(corps)・外的対象(objet extérieur)・17外的事物(chose extérieur)などと呼ばれるものは,空間性をその本質としている.私たちは物質的対象について語るとき,その位置を空間中に見出し,空間中に分散している諸物体の性質が知覚されるのは,空間とともにである.」
「無持続性 外的対象としての物質は,流れた時間のいかなるしるしも形跡も留めず,持続するとは言えない.私たちの外部すなわち外的世界には,空間と,現在すなわち同時性(simultanéité)のみが見出され,持続・継起は存在しない.」

p. 45 同書=『試論』
ベルクソンが感覚などの心的事象の実在性とは次元を異にする空間の実在性について同書で語る場合,それはあくまで感覚に付け加わる観念ないし直観としての空間についてのみ言われているのであって,意識の外なる空間の実在性が主張されているのではなく,したがって彼の言う空間の実在性は,カント流の用語によって」46「言えば,超越論的実在性ではなく経験的実在性にすぎない,と解することもできるように思われる.このように考えれば,空間中に広がる外的・物質的世界と持続する非延長的な心的事象という二実在の区別は,後者は意識の内にあるのに対して前者は意識の外なる実在である,という区別ではなくて,意識の内なる実在同士の区別であることになり,実在性ということの意味は2次元的ではなくて,意識への直接所与性ということに一元化されることになる.」 





ゼノンのパラドックスに関する論争
J. Milet, Bergson et la calcul infinitésimal, PUF, 1974, pp. 44-50. 
F. Heidesieck, Henri Bergson et la nation d'espace, Le cercle du livre, 1957, pp. 19-27. 

杉山本
72エヴェランに対するベルクソンの反論 
「「したがって我々は,現代における一人の思想家の繊細で深遠な分析の後においても,二つの運動体の遭遇が含意するのが,実在的運動と想像的運動との相違,即自的空間と無限分割可能な空間との相違,そして具体的時間と抽象的時間との相違であるという,そうした見解に同意しなければならないとは考えないのである」(DI 84-85/76 翻訳111-112).ここで言及される「思想家」エヴェランの考察は,実際には,「具体的時間/抽象的時間」といったベルクソン的な概念構成というよりも,時間や空間の分割可能性をめぐってのより伝統的なものであった.」二つの運動体とはアキレスと亀のこと


 P. 72エヴェランの議論について
「彼の議論のトポスはあくまでアリストテレス的なものであって,具体的な時間や空間,実在的運動は,何ら実無限を含意せず,無限分割を容れないものであるが,そうした単に潜在的なものにとどまる無限を実在としてしまうところに,ゼノンの逆説が生じるというのである.数学的無限とは実在なのか虚構なのかという問題は,当時においては「有限主義(finitisme)」と「無限主義(infinitisme)」との対立という形を採っていたのだが,エヴェランはそこにおいて明確な有限主義者として論を立てている.数学においてはともかく,実在する時間・空間は無限に分割されていないし,分割され得ない.それらは離散的な単位から成るのであり,そこから彼はゼノンの逆接を覆そうとするわけである.」
エヴェラン−アリストテレス的解決
時空は現勢的に無限の点から成っていないので,無限の数え上げという作業はそもそもない.有限主義.
極限値の非運動論的解釈(微分積分)
アキレウスと亀の間にある数列は有限値において収束することが数学から示せるので,追いつけないのは見かけの問題に過ぎない.



該当付近の翻訳
112
「具体的に存在する空間の分割可能性に一定の制限を設ける必要などどこにもない.二つの運動体が同時点で占める位置,これは確かに空間のなかにあるが,そのことと,空間内に位置付けることのできない運動,外的延長ではなく内的持続であり,質であって量ではない運動とを,明確に区別しておきさえすれば,空間はどこまでも分割できるものとして何の不都合もない.ある運動の速さを計測するのは,後にも述べるように,単に一つの同時性を確認しているのに過ぎない.(略)だから,ある時点においてアキレウスと亀の位置を同時に決定したり,この両者がある地点Xにおいて出会うこと,この出会いそれ自体が一つの同時性なのであるが,それをア・プリオリに予測したりするときには,数学はその役割の範囲にとどまっていると言ってよい.しかし,この二つの同時性の中間で起きていることを再構成できると主張するとすれば,数学はその分限(杉山訳 役割)を踏み越えることになる.あるいは少なくとも,そうしようとしても,相変わらず新規の同時性を考慮しなければならず,そのようにして無限に増大する同時性を前にして,不動のものをいくら集めてもそれから運動を作り出すことはできず,空間で時間を作り出すことはできないことを,数学は思い知ることになるだろう.要するに,内的持続のなかに等質なものが」113「あるとしても,それは持続しないもの,つまり空間でしかなく,そこに多数の同時性が並んでいるのだが,それと同じように,運動のなかに等質的要素があるとすれば,それは運動とは最も疎遠な要素,つまり運動体の軌跡空間であろうし,その軌跡空間とはすなわち不動のものなのである.
ところで,まさしくこれと同じ理由によって,科学は時間や運動というものを扱うときに,まずもってそれからその本質的な要素,つまりその質的な要素を──時間からは内的持続を,運動からは動性を,抹消せざるをえないことになる.それをわかりやすく説明するために,天文学と力学において,時間,運動,速度などがどのように扱われているかを検討してみたい.」
116
「力学は,時間に関しては同時性しか,運動に関しては不動性しか取り扱っていない,ということがこれで根拠づけられたと言って良いであろう.
このことは,力学が必然的に方程式の操作に基づいていること,代数的方程式は常に完了した自体を表現するものであることを,考えれば予見できた結果である.それに対して,われわれの意識に直接現前する持続や運動の本質は,それらが絶えず生成途上にあるという点にこそある.」
117
「空間はただ一つの等質的存在であり,空間内に置かれた事物は個別の多数体であり,これら個別の多数体なるものは空間内に展開されることで認識される,ということである.同様に,空間内には,意識が理解している意味での,持続も継起すらも存在しない,ということである.外的世界において継起すると言われる状態は,その一つ一つが単独で存在するものであって,それらの多数性は,それらを保持し,次いでそれらを相互に外的な関係において並置することができる意識にとってのみ存在しうるものである.意識がそれらを保持するのことができるのは,これら外的世界の様々な意識的事象を生み出し,これらの意識的事象が相互浸透し,それと認識されることなく有機的に統合され,この有機的統合の連帯によって」118「過去を現在に結びつけているからである」.
=>つまり,ゼノンのパラドックスは無限分割に由来するものでない.そもそも,空間中の位置移動としての運動に関してばかりでなく,幼年・成年・青年・老年といったように,生成変化一般に成り立つ(『創造的進化』).杉山73-74


杉山本
P. 141
「ビラン以降のいわゆる「スピリチュアリスム」が世界の客観的存在を主張する際には,主にビランが持ち出され,私の側の意志的努力に対する抵抗こそが非我の存在を告げるといった立論がむしろ普通だったのである(クーザン,ジャネ).」

同ページ
「観念論に対する彼の批判は,基本的に,観念論には説明できないことが多すぎる,という形を採る.何と言っても観念論にとって不利なのは,「科学が存在する(MM22/177-178)」という事実,つまり科学と呼ばれる活動に事実上の「成功(réussite)」(MM23/179)が見られるという事実である.別の言い方をするなら,現象には一定の秩序があり,「科学」という営みはそれを顕在化し,かつ精緻に突き詰めていくことに成功しつつあるのであって,観念論はその秩序を説明できないというのである.」
P. 143
「彼に言わせれば,観念論は,秩序だった外界を,あるいはより正確に言えば外界の「外在性」の意味であるところの「秩序」の先住を,どこかで前提としないで済ませられるものではない.」

パヴロフスキーがイマージュという場合,やはりベルクソン『物質と記憶』で述べられているイマージュの集合体としての世界という考えをうけているのだろうか.

p. 182
「すでに「記号(symbole)」に対する批判的観点は明らかである.『試論』が連合主義的な心理記述を「記号的表象(représentation symbolique)の価値しか持たない」(DI29/113)といった言い方で批判していることはよく知られていよう.「記号」とは,「直接与件」に代えて別のものをその「等価物(équivalent)」とする規約によって成立しながら,その恣意性によって実在を覆い隠し,私たちを実在から遠ざけるものなのである.」

p. 205
「しかしベルクソンは,「知性性」も「物質性」も,それぞれ発生を有していると述べている.知性は永遠の昔から今の人間の有するような知性として存在していたわけでもなければ,物質もまた今日人間知性が捉えるような姿を以前からずっと保持してきたわけではない──ベルクソンはそう言おうとしているのではないか.
不可能な解釈ではないが,相当に高いコストがかかると思う.高過ぎる,と私たちは見積もる.なぜなら,以上のように考えると,まず適応の「相互」性が,実際のところ理解しがたいものになる.頻繁なキャッチボールのやり取りを通じて,大きく見れば知性と物質は互いに互いを合わせてきた(単に知性が試行錯誤を繰り返して物質の実相に近づいてきた,というのではない.物質の方もまるで交渉の対話者のごとく知性に合わせて意見を次第に拵えてきた,というのである),と述べているのは,困難を小さくすれば見逃してもらえようといった詐術でしかあるまい.」
p. 208-9
「しかしベルクソンは物質を生命的傾向の欠如,「中断・逆転」として定義する.人間知性の機能を,生命や持続とその創造性を排除する諸表象に基づいた思考であると見定める.それによって,両者の一致と再会は,反対に,「全く自然に」生じるものと言い得るようになる.それを裏切る予見不可能性,創造性がここでは定義上」「排除されているのだから.」
«banqueroute de la science»
F.  Brunetière, «Après  une visite au Vatican», Revue des deux mondes, janvier 1895. 
科学の成功を認めなかった論者.



竹内訳『物質と記憶』
要約と結論
P. 329
「現実には粒子群に分割されている延長世界を一方に置き,空間内に投影されてはいるが,それ自身としては非外延的な諸感覚を伴う意識を他方に置いて見れば,このような物質とこのような意識とのあいだには,すなわち身体と精神とのあいだには,どんな共通点も見出せないのは自明のことであろう.しかし,この知覚と物質世界との対立というものは,自らの習慣や法則に沿うようにものごとを分解し,再構成する人間悟性の人為的構成物なのであって,無媒介の直観に直接与えられているものではない.直接的直観に与えられているのは,非外延的な諸感覚ではない.」
物質の分割された延長世界と精神の純粋な非延長世界が対応しているのは,その中間にある「外延的なもの」による.

P. 334
ベルクソンは生命体のことを,「生命ある物質」(matière vivante, PUF280)と述べている.ただし,当時の進化論を色濃く反映している.グーグルブックスの年代別検索でもっと深掘りする必要があるか.
「生命ある物質の進展=進化は,生命諸機能の分化に存するのであって,外的刺激を誘導し,行動を組織することを可能にする神経組織の形成と,それに続く段階的複雑化の歩みが,その分化によって開かれているのである.」
\end{comment}




\chapter{物質とエネルギー}
\section{物質について}

先行研究の終わりで示したように,本章では『四次元郷』で描かれる物質に関する描写を分析することによってパヴロフスキーが3次元の世界をどのように描いていたのかを分析する子.そのことによって,相対的にパヴロフスキーの4次元の特徴を明らかにすることができる.それでは,パヴロフスキーがどのように3次元について語っているか見てみよう.

彼は,第1章の始めで「他にもっとふさわしい言い方がないので私たちが4次元と呼んでいる量によっては測りしれない違いを,入れ物とその中身の間の,観念と物質の間の,芸術と科学の間の違いを,3次元で構築された数字や語では説明することができない(Et de cette différence non mesurable par des quantités, que faute de mieux nous appelons quatrième dimension, de cette différence entre le contenant et le contenu, entre l'idée et la matière, entre l'art et la science, ni les chiffres, ni les mots construits à trois dimensions ne peuvent rendre compte. )\footnote{1923年版では«quatrième dimension»は斜体.}」(2/2-3).パヴロフスキーによれば4次元は本来,3次元では表現できない翻訳不可能な世界とされていると述べている.その中でのこうした説明は,『四次元郷』における4次元の範例となっている.すなわち,私たちが生きている3次元の世界は,4次元が「入れ物」だとすると,「中身」にあたる.その入れ物は,「観念」と「芸術」とも言い換えることができる.そして,「観念」の中身には「物質」が,「芸術」の中身には「科学」が入っている.

これは『四次元郷』で語られる人類史と大きくかかわっている.2つの科学の時代を通じて,科学による物質の操作の技術が発展していき,最後に人類は4次元の世界に至る.4次元に至ることで知られるようになったのは,物質が観念という入れ物の中身であるので「物質は観念の指示に従ってその部分を変化させられ(la matière se modifiait d'après les indications de l'Idée)」(307/242)てきたことだった.この時,観念の入れ物にある物質は精神の働きかけをうける.ここで示されているのは,物質に関するもう1つの考えである.すなわち,観念は物質に直接働きかけるのではなくて,人間の精神を媒介にするのである.入れ物とその中身の関係は直接的ではなくて,その媒介が存在するのである.

精神と物質が関わるというこの考えは,1912年版第8章1923年版第8章「時間の原子の変異(La transmutation des atomes de temps)」で詳しく述べられている.「時間の原子の変異」は,未来をどのようにして知り,それをどのようにして書いたのかという時間の移動とその記述の方法が描かれている.つまり,4次元での移動とその経験をどのようにして3次元の言葉で語ることができたのかという4次元と3次元の関係が詳しく述べられている.その関係は主に,3次元と4次元の差異が主に運動の観点から説明されている.パヴロフスキーによれば,3次元における運動は,時間と場所の移動を伴うものであるが,4次元の世界はすべてが連続的につながっていて,連続的(continu)であるのでこうした移動は起きない.なぜなら,物質を構成しているとされている原子の「変異」が4次元における移動を表現しているからである.原子論において3次元における移動は原子が他の原子の位置と入れ替わっていくと説明できる.一方で,4次元においてはすべてが連続状態にあるので,「変異」しかしない.つまり,隣り合った原子の性質が変化することによって3次元における移動が表現されるのである\footnote{「隣り合った原子の間の質の変化によって移動が生じる«~Un déplacement se fait donc par un échange de qualités entre atomes voisins~» (46/79)}  .パヴロフスキーはこの説明をするために,海水面を進む船を例に持ち出す.3次元の観点からすれば,船の原子が水の原子を押して出して元の原子があった場所に収まることの繰り返しによって船が進んでいくように見える.一方で,これを4次元の観点からみると,船の原子は水の原子に変異していると表現できる.移動といった3次元で生じるあらゆる物質の運動は,4次元の観点からすると,原子の変異の連続の過程として説明することができる.ただし,原子は仮説にすぎないという点をパヴロフスキーは強調し,「原子は現実には存在していない(les atomes n'existent pas en réalité. 」(1912-48)と明言する.そして,物質が原子から構成されていると解釈する必要があるのは,「物質の特性や性質から物質を切り離す精神(l'esprit qui isole la matière avec tous ses attributs, avec toutes ses qualités.) 」(48/80) の作用によって4次元へと至ることができるからである. つまり,4次元に至ることができるために持ち出される原子論は,『四次元郷』の物語の帰結である人類の4次元への移行するうえで非常に重要だと考えられる.ところが,パヴロフスキーにとって原子論は仮説でしかない.精神が物質に介入するという先ほどの一節は,原子論は仮説であるものの,精神を媒介として観念と物質が関係を結んでいる場合には,原子は存在しているという奇妙な論理が立てられているのだ.これはどういうことなのだろうか.

原子論を仮説とみなしながらも,その存在を認めるという奇妙な思考法の萌芽は彼の博士論文である『労働の哲学』に見られる.社会と個人の関係を個人の主観的意識に主眼を向けることで隷属的な労働からの解放がいかにして可能なのかを考察したこの著作で,原子論が重要な概念となっている理由を,ヨーロッパ諸国の歴史に求めることができる.国民国家の出現し始めた近代において,神による人間の救済が個人との直接的な関係によるものであると信じられるにあたって個人の価値の称揚が促され,国家を構成する個人は原子のようであるという考えが広まっていった.17世紀にはガッサンディーがエピクロス派の復権と原子的宇宙論を語り,ライプニッツはモナドロジーの理論を構築することで個人と原子をアナロジカルにとらえた.こうした歴史認識は一定の説得力を持っている\footnote{中谷猛・足立幸男編『概説西洋政治思想史』,ミネルヴァ書房,1994年,82-86頁.}.さらに,パヴロフスキーが,個人によって構成される社会という図式が原子によって構成される物質と相似しているのに注目しているのはこの歴史的文脈の他にも哲学的な文脈がある\footnote{本論では触れないが,パヴロフスキーの原子論の需要と冒険譚には大きな関わりがある.彼は1924年に「偉大なる革命 シラノ復活!(Un grand révolté. Ressuscitons Cyrano!)」という文を執筆している(Gaston de Pawlowski, «Un grand révolté. Ressuscitons Cyrano!», \emph{Cyrano}, Paris, S.I. , 1924.).そして,シラノ・ド・ベルジュラックが扱っていたのはガッサンディ派の原子論であった.ガッサンディからライプニッツ,そしてベルクソンに至るまでの原子論の概略は下記を参照のこと.Marie Cariou, \emph{L'atomisme. trois essais : Gassendi, Leibniz, Bergson et Lucrèce}, Paris, A. Montaigne, 1978.}.当時,科学的にその実在が疑われていた原子は,世界に対する認識の問題としてとらえられていた.そして,私たちが存在している世界についての認識を左右しているのは科学を代表とする唯物論と形而上学を代表とする観念論の2つで,認識においてそれらは拮抗していると考えられていた.この拮抗は,『労働の哲学』の第4章「科学の限界」では,「私たちを占めている主体において,つまりは,社会の科学と個人の自由(l'étendue et la durée ou dans le sujet qui nous occupe : la science sociale et la liberté individuelle)」(PT184)の問題であると簡潔に表現されている.また,原子論は仮説であるものの,原子の存在が人格(personnalité)の存在様式を反映しているのではないか,というアルチュール・アヌカン(Arthur Hannequin)の学説がここでは同時に紹介される.パヴロフスキーが原子論でアヌカンの学説に立脚して人格と原子の相似形を語ることは,『四次元郷』における物質を考えるうえで極めて重要である.作品の中では,ブランキの宇宙を形成する原子の無限の反復といった考えが示されている.一方でフランスでは熱の運動をめぐる議論の中で,熱を伝えているものの正体として原子が再び注目されるようになる.しかし,ラプラースやアンペールらの力学における研究成果への評価が揺るぎなかった伝統的な科学者たちは原子の存在を基本的を認めようとはしなかった.それは以下のような科学アカデミーの記録にみることができる.

アウグスト・クント(August Kundt)とエミル・ヴァルブルク(Emil Warburg)は水銀蒸気の特定の熱量について,イギリスのクラジウスなどが新しく提唱していた気体運動理論に基づいた計算を行ったところ,計測結果と理論的数値が一致したという報告をした.それについてアカデミー会員のイヴォン・ヴィラルソー(Yvon Villarceau)が実験結果の重要性を指摘したところ,当時高名だったマルスラン・ベルトロ(Marcellin Berthelot)が「不可分である一方で広がったり連続したりしている原子という考え自体,\bou{また},\bou{固まりが与えられている一方で物質の点として還元される原子という考えも},それ自身において矛盾しているように思われる(La notion même d'atome indivisible, et cependant étendu et continu, \emph{aussi bien que celle d'un atome doué de masse et cependant réduit à un point matériel, }semble contradictoire en soi)\footnote{Marcelin Berthelot, «~Remarques sur l'existence réelle d'une matière monoatomique, à la suite d'une communication de M. Villarceau~», \emph{Académie des Sciences. Compte rendus hebdomadaires des séances}, t. 82, 1876, p. 1129. }」と述べた.ヴィラルソーはこの報告を受けて,ベルトロの観察不可能なものに対する実在を認めない実証主義的な態度を批判している\footnote{下記を参照のこと.João Príncipe,«~La physique laplacienne dans la seconde moitié du XIX\textsuperscript{e} siècle: Joseph Boussinesq---~la pratique et la réflexion autour de l’atomisme en France vers 1875~»,\emph{Kairos Journal of Philosophy and Science}, vol. 13, 2015, pp. 183-189.}.ベルトロを代表とするアカデミーの年長者たちの考えの影響は非常に大きく,アカデミーでの記事の後,1886年に文部大臣となったベルトロは,原子論は仮説に過ぎないので学校教育の場では教えるべきではないという法令を作成した\footnote{W・H・ブロック『化学の歴史 II』,大野誠他訳,朝倉書店,2006年,284頁.}.こうして,フランスでは原子論を科学的な概念として認めないような風潮が作られていった一方で,そのほかの国々では放射能現象や光電気をめぐる実験データの定性的な説明を可能とするジョセフ・ジョン・トムソンの原子モデルが支持を集めるなど,とりわけ化学者らを中心に原子の実在を前提とした学説が推し進められ,アーネスト・ラザフォードやニールス・ボーアなどが1900年代に実験を進めていき,最終的に電子の存在が次第に実証的に確認された\footnote{以下を参照のこと.ヘリガ・カーオ『20世紀物理学史:理論・実験・社会 上』,有賀暢廸・稲葉肇他訳,名古屋大学出版会,2015年,第4章.}.科学教育史家ニコル・ユラン(Nicole Hulin)の言葉を借りれば,1870年代から始まったフランスにおける「化学の活気(la vitalité de la chimie)\footnote{Nicole Hulin, «~Les doctorats dans les disciplines scientifiques au XIX\textsuperscript{e} siècle~», \emph{Revue d'histoire des sciences}, v. 43, n. 4, 1990, p. 425.} 」はこうして陰りを見せることとなった.

以上をふまえると,パヴロフスキーがアヌカンの『原子仮説についての批判的試論(\emph{Essai critique sur l'hypothèse des atomes})\footnote{Arthur Hannequin, \emph{Essai critique sur l'hypothèse des atomes}, Paris, Allan, 1899, p. 20.}』(1895)を参照していたのは(PT168),フランス特有の事情が原因だったと考えられる.ベルトロによる1886年の法令の制定された時にパヴロフスキーは12歳であり,アヌカンの著作に触れていた20代前半にフランスにおいて原子論を科学的に実証しうると考えることの方が困難だった.『労働の哲学』を執筆する過程で原子論を検討したパヴロフスキーにとって,ルクレティウスやエピキュロスといった古代の哲学からライプニッツのモナドロジーに至る存在論的な概念が原子だった\footnote{パヴロフスキーが1923年の時点でラザフォードを知っていたうえで原子の実在を認めなかったことがさらにこのことを確証している.(1923, 126)}.これらの時代背景に加えて,アヌカンはカントの批判哲学に基づいて原子論は仮説的存在でありながらも人間の認識の構成的原理となっているという学説を展開し,ある種の科学哲学を提案していた.パヴロフスキーはこのアヌカンの学説に依拠しており,仮説的にしか存在してないはずの原子がやはり存在しているという論理はここからそこに淵源があると考えられる.そこで,アヌカンの学説について以下で詳しく見てみよう.

アヌカンによれば,原子論は古代からなされている議論で科学者の間でも議論になっているのにもかかわらずその実在については確証が得られていないが,原子論が繰り返し提示されるのは,カントが『純粋理性批判』で述べているような意味で,人間の認識の構成的原理に要請されているからである.パヴロフスキーが『労働の哲学』で引用しているアヌカンは「原子というこの数学的な概念は,場所と時間において,ある相対的な存在だけを持っており,私たちを,おそらく,空間そして時間の上位で,絶えず生み出され自身を完成させていく存在の統一へと私たちを導き,決定論的な活動と創造するものの影をその持続に投じ,実現され,動けなくなり,すでに過ぎ去り,すでに死者として,結果の影をその延長に投じている.さらに,活動と,連続と原子はただの反射した姿でしかない真の統一体へと人が高められるのに従って,連続と連続とともにある原子は,モナドと精神に向かって,消え失せる(l'atome, ce concept mathématique, qui n'avait dans l'Espace et le Temps, qu'une existence relative nous, conduira peut-être, au-dessus de l'Espace et au-dessus du Temps, à l'unité d'un être qui sans cesse se fait et s'achève soi-même, en projetant dans la durée l'ombre de son action déterminante et créatrice, et dans l'étendue l'ombre des résultats réalisés, fixé, déjà passés et comme déjà morts. Ainsi s'évanouissent le continu et, avec lui, l'atome, à mesure qu'on s'élève vers les activités et les unités véritables, dont ils ne sont que le reflet, vers les monades et les esprits. )\footnote{Hannequin, \emph{op. cit. }, p. 21. }」と述べている.パヴロフスキーが注目しているのは,原子とは数学的な概念でしかなく実在してはいないという主張である.アヌカンは私たちが認識を構成するうえで必然的に要求されるのが原子論という仮説だと考えている\footnote{Hannequin,\emph{op. cit.} ,  p. 12.  アヌカンはカントの研究を1890年代の前半に行っている.カントは認識のあり方を「統制的」と「構成的」に分けているが,アヌカンは原子を認識そのものを形作る構成的原理として考えている.}.これを受けてパヴロフスキーは「時間の原子の変異」で,「精神は原子が自身のイメージでしかなく,原子から4次元の完全でただ1つの世界が作り上げられていて,それは,複数の鏡の中であるかのように,3次元の不完全な世界は,様々な様相の下でそのただ1つの原子が無限に反射しているという感覚の幻想なのである(L'esprit conçoit l'atome à son image, il en fait donc un monde complet et unique à quatre dimensions et c'est une illusion des sens qui reflète à l'infini comme dans des glaces multiples cet atome unique sous les aspects divers du monde incomplet à trois dimensions.)」(48/80)と言い換えている.原子とは精神自身の姿であり,その原子はただ1つで世界はその反射によって構成されている.これは世界は精神を通じてしか認識できないという考えがあってこそ初めて述べることができる.物語の冒頭に掲げられたこの世界観を最終的に人類は科学技術の発展によって手に入れるのだが,そのためには,精神による物質の操作を3次元において行う必要がある.つまり,物質を構成する原子自体を操作できることが求められている.もちろん,科学の時代において人類は原子の操作を最終的にものとするが,実はここで重要となるのが物質に干渉することで生じるエネルギーなのである.1912年第31章1923年第33章「悪魔祓い(La Conjuration des larves)」でのパヴロフスキーはエネルギーと物質がついにあることを示している.その具体的箇所を検討する前に,この章のあらすじを見ておこう.

人類は科学の時代にエネルギーを物質を分離することによって取り出そうとし失敗してしまった.そこで,物質の分離ではなくて物質を構成しているとされる原子を操作すれば,自由に物質やエネルギーを生み出すことができると考えた.そこで注目されたのは,すでに人類がその力や可能性を知っていて,組み合わせが容易な基本的原子だった.基本的原子は「幼生\footnote{ここで「幼生」と訳出した単語\emph{larve}はラテン語の\emph{larva},すなわち亡霊や仮面を意味する単語に由来している.「悪魔祓い」の章は,後者の意味で翻訳したが,作中では\emph{larve}を培養する描写があり,パヴロフスキーは地口を用いている.}」を呼ばれるようになった.科学者たちは,「幼生」を大中央研究所で培養してその数を増やそうとしたのだが,「幼生」は幽霊のように壁をすり抜けて研究所の外に出てしまう.その結果,記念碑などが生物のように動きはじめるなどといった様々な混乱が生じた.騒動が収まった後,人類は物質にも生命があるのではないか,という考えを抱くことになる.以上が章のあらすじである.

このエピソードで重要なのはエネルギーと物質が対になっているという考えがパヴロフスキーによって示されるところである.以下にその箇所を示す.
\begin{quote}
この基本的原子は,あらゆる単純な物質とあらゆる既知のエネルギーの父なのである.科学時代の続きの中で幼生と原子は名付けられ,最終的には原子を解放し,【エーテルの|星雲の】単純な粒子だけがあった世界の始まりに存在したような原始的な状態において合成することで原子を再構成するようになった.
\end{quote}
\begin{quote}
  Cet atome élémentaire, père de tous les corps simples et de toutes les énergies connues\{;|---\} cette \emph{larve}, comme on le surnomma dans la suite[=les chercheurs de la période scientifique]\{,|---\} on finit par le dégager, par le reconstituer par synthèse dans son état primitif, tel qu'il existe au début des mondes, lorsqu'il n'est encore qu'une simple particule de \{l'éther|la nébuleuse\}. (200/178-9)
\end{quote}
基本的原子を再構成することで人類は自由に物質を構成することができるようになったが,基本的原子は「エネルギーの父」でもあるということが示しているのは,物質とエネルギーは別の形でありながらも同じ根源的なものによって構成されているということである.では,『四次元郷』においてエネルギーはどのように描かれてきたのだろうか.

\section{エネルギーについて}

1912年版第20章(1923年版第19章)「引き裂かれた犬(Le Chien Dissocié)」は,パヴロフスキーがエネルギーをどのように考えていたかを知るうえで非常に重要な章である.あらすじを簡単に見ていこう.

科学の時代の重要な物語として「火星商業開発社会(Société d'Exploitation Commerciale de la Planète Mars)」(124/125)と後世の人々が呼ぶことになる事件があった.この章では,その顛末が語られている.火星人との交流を目指す人類が様々な試みを行っていく中で,ある日火星人からのメッセージを受信することに成功する.同じ頃,物質を分離することで得られるエネルギーによって商業的利益を得ようとしていたので,火星人にその方法を尋ねると,通信者の食べようとしてた仔牛の肉がいきなり焦げてしまうという火星人側からの実演がなされる.ところが,その後も物質の分離が続き,ついに通信所の守衛の犬にもその被害が及ぶ.狂ったように吠え出した犬は守衛のもとを走り去り,川辺で引き裂かれた状態で発見される.これが物語のあらましである.

まず注目したいのは火星人との交流が,黒色光(lumière noire)という当時存在すると考えられていた放射線の1種を通じてなされていたことである.この通信は極めて安定的であった.パヴロフスキーはこうした黒色光やエネルギーに関する知識をギュスターヴ・ル・ボンから得ていたことが以下の記述から窺える.

\begin{quote}
日毎に火星人との関係が発展していき,重要な質問が私たちの隣人になされた.それは,物質の分離によって安価にエネルギーを手に入れる方法についてである.現実に,長い間,ギュスターヴ・ル・ボン博士の予言的な仕事とラジウムの発見以来,この問題は地上のあらゆる学者の心を占めていた.事実,よく知られていたのは,物質は,かつて不活であり,それにあらかじめ与えてしまったエネルギーを回復することはできないということであったが,全く反対に,エネルギーの巨大な貯蔵庫であった.それに加えて,ル・ボン博士によれば,うまく分離されるかもしれないのである.
\end{quote}
\begin{quote}
Les relations se développant chaque jour davantage, d'importantes questions furent posées à nos voisins sur la façon dont on pouvait obtenir l'énergie à bon marché par la dissociation de la matière. Depuis longtemps en effet, depuis les travaux prophétiques du docteur Gustave Le Bon et la découverte du radium, cette question préoccupait vivement sur terre tous les savants. On comprenait bien, en effet, que la matière, jadis considérée comme inerte et ne pouvant restituer que l'énergie qu'on lui avait d'abord fournie, était au contraire un colossal réservoir d'énergie. C'est ainsi, d'après le docteur Le Bon, que si l'on arrivait à dissocier.
\end{quote}

ギュスターヴ・ル・ボンは,社会学者として『群集心理(\emph{La psychologie des foules})」(1895)を著す一方で,『物質の進化(\emph{L'évolution de la matière})』(1905)や『力の進化(\emph{L'évolution des force})』(1907)など科学研究をまとめた著作も執筆していた.ステーブルフォードが指摘していたように,パヴロフスキーはル・ボンの影響を強く受けていたことが知られている.パヴロフスキーのエネルギー観はル・ボンの著作に基づいていたと考えられる.それについて以下で見ていこう.

『物質の進化』では,物質は不安定なものであるとされる.物質は放射線を放射することでエーテルへと変化するため,安定的な存在ではない.そして,エーテルは物質の源であり,物質はエーテルから生まれる.さらに,物質からエーテルへの変化の過程で放出される放射線がある.それが,ル・ボンがその存在を確証したと考えていた黒色光の正体であった.『力の進化』第2巻第4部「黒色光」によれば,レントゲンのX線に関する研究を知ってすぐに,「私[=ル・ボン]は物質を通過することができる特別な放射線を黒色光と命名したのは,全く目に見えない光のように時折振る舞う特性を考慮してのことだった(Je les [=des radiations particulières, capables de traverser les corps] désignai sous le nom de Lumière noire en raison de leurs propriétés d'agir quelquefois comme la lumière tout en ètant invisibles.)」.そして,その性質は3つある.まず,「陰極線のグループの放射性粒子である(Paricules radio-actives de la famille des rayons cathodiques.).次に,「波長が長大な放射線(Radiations de grande longueur d'onde)」.最後に,「不可視のリン光に起因する放射線(Radiations dues à la phosphorescence invisible.)\footnote{Gustave Le bon, \emph{L'évolution des Forces}, Flammarion, Paris, 1907, p. 277.}」.また,『力の進化』では燃焼を例に物質がエネルギーを貯蔵し,また原子が分離することで生じるエネルギーについての説明がある\footnote{\emph{Ibid. }, p. 196.}.「引き裂かれた犬」では,物質を分離することでエネルギーを取り出す人類の姿が描かれているが,ル・ボンのこの記述が参照されているのは明らかである.

現在,ル・ボンが発見したと考えた黒色光はもちろん存在していない.彼が述べているような「不可視のリン光」なるものは存在しない\footnote{リン光とは,芳香族化合物などが光を吸収した場合に,エネルギーの高いとそれを放出する時の発光現象なのであって,「不可視のリン光」など存在しないし,ましてそれが放射線であることもありえない.}.しかし,量子論史研究者の森川亮の述べているように,「ル・ボンのアイデア(彼自身はそれが単なるアイデアなのではなく本当に存在すると確信していたのだが)は,いわば,時代の変わり目にあって,それまでの物質観,さらに広く述べれば世界観が変貌しつつあることを傍証する」ようであり,「物質の非物質性,言い換えれば物質と非物質の連続性とでも述べるべき物質観であった\footnote{森川亮,「量子論の歴史 --- 未知なる放射線,その発見ラッシュの裏面史」,生駒経済論叢,第13巻,第2号,2015年11月,298頁.}」.「物質と非物質の連続性」は,パヴロフスキーにおいては観念と物質の連続性と言い換えられる.物質が黒色光を放ちながらエーテルに還るようにして,3次元の物質は精神を介して4次元の観念へと還るのである.3次元における物質の変化は,精神の働きかけによって生じ,パヴロフスキーはそれを最終的に「脱物質化(dématérialisation)」(314)と呼んでいる.つまり,精神は物質からその特性を引き離し,そこからエネルギーが生じる.それによって初めて物質は4次元における本来の姿を取り戻すのである.物質とエネルギーが対になっている「悪魔祓い」の章での記述は,精神の作用を介した物質とエネルギーの関係を示している.ただし,エネルギーという言葉は『四次元郷』において経済活動と結びついた語彙でしかなく,精神の作用と関わっているような直接的な記述は見られない.エネルギーと精神が結びつくためには,さらに上位の宇宙の法則を人類は理解する必要があった.1912年版第21章1923年版第23章「万有浮揚力(Lévitation Universelle)」ではその上位の法則が明らかにされる.まずはあらすじを見てみよう.

火星人との接触での失敗の後,物質の分離によるエネルギーの抽出に成功することができた.しかし,それはより上位の法則を知らないために世界的な危機を招いてしまう.物質を分離することでエネルギーを得ていく一方で地球を軽くしてしまい,地球自体に働く力もまた弱くなってしまったのだ.ここで弱まる力というのは遠心力(force centrifuge)である.人類がそのことに気付いた頃に地球に接近していたのが,1910年に地球を通過する際に人々を驚かせたハレー彗星だった.そこで,人類はハレー彗星の持っているエネルギーを奪うことで失われたエネルギーを補填しようとした.その試みは成功し,人類はそのエネルギーを転用して地球の回転速度まで操れるようになった.これがのあらすじである.

この章で明らかになるのは,章題にもある通りの万有浮揚力である.この「浮揚」はオカルト用語で用いられる幽体によって物体が浮き上がる現象や,超伝導磁石によって金属が落下せず空中に固定されているような状態を指す言葉である.万有引力(Gravitation Universelle)の地口で名付けられたと考えられる万有浮揚力は,万有引力を補完するものだと述べられている.そして,「それは,結局,【物質の力の完全な説明という,世界の根源についての|引力と斥力,連合と分離という拮抗する2つの力,つまり,世界,いわゆる物質の出現と消失を左右する反対の2つのエネルギーという】決定的な啓示だった(Ce fut, en somme, la révélation définitive des \{origines des mondes, l'explication complète des forces matérielles ;|deux forces antagonistes d'atttraction et de répulsion, d'association et de dissociation, des deux énergies contraires dont dépendent l'apparition et la disparition des mondes, c'est-à-dire de la matiére\})」(132/144).この「決定的な啓示」は,まとめると,万有引力は全てを引き付ける力,万有浮揚力は全てを引き離す力としてそれぞれ働き,2つの組み合わせによって宇宙は成り立っている.「引き裂かれた犬」の章では,分離によるエネルギーが重要であったように,ここでも万有浮揚力という引き離す力が重要となる.「万有浮遊力」の章では,万有浮遊力は現実では「遠心力(force centrifuge)」として様々な場所で働いていると考えられている.これは慣性の法則で働く遠心力とは違い,パヴロフスキーは物質が持っているエネルギーとして捉えている.パヴロフスキーはこの遠心力の重要性を,ラプラースが唱えた星雲説におけるあるミッシングリンクを埋めることができると考え,強調している.なぜ,「遠心力」は重要であり,星雲説という宇宙の起源に関する議論と関わっているのだろうか.

星雲説は,宇宙の塵が凝集していくことで,星雲になり,次第に恒星や惑星を形成するというものである.これが私たちの宇宙の始まりであったと星雲説は教えている.パヴロフスキーは,ではなぜ全ての塵が凝集して1つだけの塊とならないか説明できない,と星雲説に反論している.パヴロフスキーによれば,物質が持っている遠心力によって違いが万有浮揚し合うことによって1つの塊になることは決してないのだという.パヴロフスキーがこうした仮説を持ち出すのは,物質のエネルギーとしての側面を見落としている科学の時代の人類の姿を強調するためである.あらすじであったように,人類が物質を分離することでエネルギーを得ていくと同時に「遠心力」が弱まっていったのは,物質が持っている「遠心力」を失っていたからなのだ.

ところで,「万有浮揚力」の章は物質のもつエネルギーの正体を教え,その時に生じた危機を回避する出来事を物語っているが,危機の回避が彗星によってもたらされるというのはいささか唐突な顛末である.1910年のハレー彗星が連載時に記憶に新しかったために時事的な話題を挿入したとも考えられるが,戦争を挟んだ1923年版においてもその表現を変更していないので,時事的な逸話であるという説明には説得力があまりない.すると,何か彗星でなければならない理由があると考えられる. その理由は,彗星について言及する箇所に隠されている.

\begin{quote}
最も完成された空気ポンプによって作られる相対的な真空は彗星の実体よりもなおずっと密度が高い
\end{quote}
\begin{quote}
le vide relatif produit par la machine pneumatique la plus parfaite est encore beaucoup plus dense que la substance cométaire(1912, 134)
\end{quote}

1912年版のこのフレーズは,ある著作を一部改変したものである.原文では以下のようになっている.

\begin{quote}
空気ポンプによって作られた完全な真空は彗星の実体よりもなおずっと密度が高い
\end{quote}
\begin{quote}
  le vide le plus parfait d'une machine pneumatique est encore beaucoup plus dense que la substance cométaire\footnote{Auguste Blanqui, \emph{L'éternité par les astres}, G. Baillière, Paris, 1872, p. 18.}
\end{quote}

その著作とは19世紀の革命家オーギュスト・ブランキによって執筆された『天体による永遠』(1872)である.『天体による永遠』は,パヴロフスキーが「遠心力」を説明した時に批判の対象として挙げていたラプラスの星雲説の批判から始まる.ラプラスが彗星を太陽と同等のものとみなしているのに対して,ブランキは地球を通過しても影響を与えることはない空虚な存在であり,惑星や恒星とは別の存在であると断言する.彗星は全くの未知の物体なのである.これが意味するのは,彗星は全くの未知であるためにあらゆる秩序から逃れ去るということだ.星雲説で,ラプラスが必然性をもった宇宙の秩序という必然的な宇宙観を提示したとするのであれば,ブランキは彗星が決して秩序の中には回収されない偶然に満ちた宇宙観を提示した.この考えの背景にあるのは,1864年にハギンズが太陽と他の恒星は同じ元素で構成されていることの発見である.ブランキはこの観測事実から,宇宙がもしも有限の元素によって構成されているのであれば,元素の組み合わせも有限に違いないのであり,宇宙のどこかで同じ元素の組み合わせが反復されているに違いないと考えた.すなわち,地球で起きるあらゆることもまたどこかで反復されているのに違いないと考えた.しかし,発表された当時,カミーユ・フラマリオンが論駁していたように,元素の数がたとえ1つしかなかったとしてもそこから作られるものがたかだか有限個しかないということは論理的な帰結として導くことはできない\footnote{Camille Flammarion, «~L'Eternité par les astres par A. Blanqui~», \emph{L'Opinion Nationale}, 25 mars 1872, p. 3.}.鈴木雅雄の的確な要約を加えておけば,「要素の有限性とそこから作られるものの有限性とはまったく別の問題である\footnote{鈴木雅雄「星々は夢を見ない --- オーギュスト・ブランキに関する覚え書き」,早稲田大学大学院文学研究科紀要,第2分冊,53号,2007年,6頁.}」.しかし,ブランキにとって重要だったのは,この無限の反復は,少しずつ差異を含んでいるということだった.なぜなら,未知の彗星の介入によって少しずつ反復される歴史の結果が異なるからである.ナポレオン戦争おけるいくつかの戦いの結果が変わっている地球の話をブランキは例に挙げている\footnote{オーギュスト・ブランキ『天体による永遠』浜本正文訳,雁思社,1985年,95-6頁.}.こうしたブランキの考えは,当時のフランスの人文学者の間での宇宙論においても極めて特殊なものであった.マイケル・J・クロウが語っているように,フランスにおいては基本的に宇宙には生命体がおり,異なる世界が繰り広げられているという多世界論に関する言説が中心的だったからだ\footnote{マイケル・J・クロウ『地球外生命論争1750-1900』鼓澄治・山本啓二・吉田修訳,工作者,2001年}.では,この論争が起きていた頃に生まれたパヴロフスキーはブランキをどのように評価していたのだろうか.『四次元郷』にはそれがわかる一節がある.

\begin{quote}
ただ一人の人間が,かつての唯物論において,自分の意見を持つ勇気をもち,限界までそれを追求した.その男はブランキだった.トーロー要塞に収容されていた時,孤立と監獄での内省の中で,『天体による永遠』と題された興味深い冊子を書き上げた.厳密な論理はあらゆる同時代人に衝撃を与えたはずだったろう.
\end{quote}
\begin{quote}
Un seul homme, dans le matérialisme ancien, eut le courage de son opinion et la poursuivit jusqu'à ses extrêmes limites ; cet homme fut Blanqui. Dans la solitude et le recueillement de son cachot, lorsqu'il fut enfermé au fort du Taureau, il écrivit une curieuse brochure intitulée, l'\emph{Eternité par les Astres}, dont la logique rigoureuse aurait dû frapper tous les contemporains.
\end{quote}

ブランキの『天体による永遠』はパヴロフスキーにとって重要であったことがここから窺える.では,パヴロフスキーはブランキのどの論理を取り上げているのだろうか.

ブランキの宇宙観の基礎をなしていたのは基本的な元素が有限しかなく,それが反復することで宇宙が形成されているというものであった.パヴロフスキーは世界を構成する物質の基礎である原子をそれに重ね,世界が原子の反復によって多様な形を織りなしていると考えていた,と推論できる.ところで,この推論は正しい.さらに,パヴロフスキーはフラマリオンが批判していた極端な例を採用してすらいる.原子はただ1つしかないのだ.パヴロフスキーがブランキの影響を受けているとしたら,このような極端な結論は導かれなないはずである.これは何を意味しているのであろうか.

\section{観念的な偶然性}

パヴロフスキーの宇宙観について考えるためには,宇宙の歴史という時間に注目する必要がある.本章では,物質をめぐってパヴロフスキーの3次元について考察してきたが,時間もまた3次元を構成する重要な要素である.よく知られているように,19世紀は自然が永遠のものから歴史的な時間軸へと移行することになる転換期であった.パヴロフスキーがブランキの宇宙観をどのように引き受けたのかを知るために,この転換期からまずは見ていきたい.

永遠で循環するものだと考えられていた自然にも歴史があることが説明されるようになったのは,星雲説が嚆矢となった.例えば,ラプラースと同時代に古生物学者だったラマルクは,形態変化という,生物の種の形態が漸進的に変化していく説を唱えていたが,大局的には生命は円環的(cyclique)な時間を循環していると考えていた.それが19世紀の半ばを境にして,進化論の登場によって一方向への矢のように(sagittal)進んでいく時間が自然に導入されるようになる\footnote{Stéphane Tirard, «~L’histoire du commencement de la vie à la fin du XIX\textsuperscript{e} siècle~», \emph{Cahiers François Viète}, dir. Gabriel Gohau et Stéphane Tirard, n. 9, pp. 106-9. }.このように自然は19世紀に歴史化されていく過程にあったと考えられる.自然観に変容を与えた様々な科学の様々な発展の中で,ブランキが注目したのは分光学を代表する天文学的知識であり,これは19世紀後半のフランスにおいては居住世界の複数性というテーマにつながっていた.これを代表する論者がカミーユ・フラマリオンやルイ・フィギエらであり,カトリックの一部論者たちも生命体が他の世界に存在していることを認める議論をしていた.この中でも反教権派のブランキにとって主要な論敵は神学者のアルフォンス・グラトリーだった.ブランキはグラトリーに対して極めて批判的であった.ブランキのグラトリーに対する批判内容は,ブランキの宇宙観を浮かび上がらせている.

グラトリーが天体の運行においては目的地が存在していると考えていたのに対して,ブランキは「私としては,この惑星がいつどのようにして,どこかへ\bou{到着する}気になったのか,一向にわからない(Pour moi, j'ignore où, quand et comment, notre planète se propose d'\emph{arriver})」\footnote{Suzamel (Blanqui), «~Le Père Gratry. Science et foi. (3e article)~», \emph{Candide. Journal à Cinq centimes}, 1ème année, n. 8, 27 mai 1865, p. 1.}と述べており,『天体による永遠』の偶発的な進展を見せる宇宙論のモデルが定時されている\footnote{鈴木雅雄,前掲書,15頁.}.こうした宇宙論はブランキの革命のイデオロギーをも支えていた.宇宙でさえ偶然によってその運命が左右される.それと同時に,元素の組み合わせに限りがあるためにあらゆる場所で私たち自身の歴史が反復されている.もしかすると,私たちが生きている世界では革命が成功するかもしれない.こうして,革命は宇宙論的に基礎付けられ正当化される.こうした正当化のプロセスは,ベンヤミンによれば,「希望のない諦観(resignation sans espoir)」であり,「世紀は,技術的な新しい潜在性に対して新たな社会秩序をもって応ずることができ(Le siècle n'a pas su répondre aux nouvelles virtualités techniques par un ordre social nouveau)」ず\footnote{ヴァルター・ベンヤミン『パサージュ論I』今村仁・三島憲一他訳,1993年,57頁.Walter Benjamin, Gesammelte Schriften, v. I, éd. Rolf Tiedemann, Frankfurt, Suhrkamp, 1991, p. 76.},革命によって来るべき未来の社会像を持ち出していない.鈴木はベンヤミンのこの記述を受けて,「未来について積極的に語れると思いこんでいるものたちはみな「狂人」,つまりは白昼夢を見ながら夢見ていることを知らないものたちであろう\footnote{鈴木雅雄,前掲書,11頁.}」と指摘している.19世紀のこうした「狂人」たちは,鈴木の見立てに従うと次のような革命思想の四象限図~\ref{fig:fourtypes}を描くことになる.\myfig[height=18.2cm, width=12.8cm]{fourtypes}{革命思想の四象限}{fourtypes}第一象限では,未来の社会はこうなるべきであるという必然性に由来するサン=シモンを代表とするユートピア思想である.そうしたユートピア思想を批判したマルクスを代表とする共産主義は現在的な問題から導かれる必然的な世界の展開に基づいている(第四象限).一方で,これら必然性の革命思想に対して,フーリエは異なった視点を持っていた.フーリエはファランジュと呼ばれる来たるべき未来の共同体を構想した.フィリップ・レニエ(Philippe Régnier)によれば,こうした共同体の構想は,未来の社会から現在を見た時に,いつの地点からでも偶然的理想的な社会が出現するという思想であるという
\footnote{以下の解釈に基づく.Pilippe Régnier, « Place, fonctions et formes de l'ecriture utopique chez Fourier », \emph{Pamphlet, utopie, manifeste, XIX\textsuperscript{e}-XX\textsuperscript{e} siècles}, textes réunis par Lise Dumasy et Chantal Massol, L'Harmattan, 2001, pp. 385-401.}.この説を踏まえれば,フーリエは未来の社会から現在を見て,偶然性に依拠した理論を立てていると言える(第二象限).そして,これらのいずれも当てはまらないのが,現在への偶然の介入によって理想的な社会が生まれうるという革命思想を持っているブランキなのである(第三象限).

ところで,これらの思想には,いずれも特徴的なある種の科学的進歩に対する肯定感がある.実際,ブランキ以外は未来の社会の理想像を持っている.このように,科学の進歩が理想社会の実現をしうるという考えは19世紀後半に広まったものである.セルジュ・レーマン(Serge Lehman)が指摘しているように,科学が実現する未来の肯定的なヴィジョンが文学的な想像力として表現されるのは,19世紀後半のフランスにおいては極めて例外的なことであった\footnote{Serge Lehman, «~La Physique des metaphores~», \emph{Europe}, octobre, v.79, n. 870, 2001, p. 49. 32-50.}.

パヴロフスキーが評価したブランキとその時代の思想を偶然と必然の宇宙観,そして未来と現在のどちらの地点から来るべき社会,あるいは単なる未来を予想しているかという点から以上のように分類した.では,パヴロフスキーはこの四象限のどこに入るのであろうか.パヴロフスキーが『四次元郷』で描いている未来はリヴァイアサンによる支配,科学者たちの支配,4次元への到達である.しかし,パヴロフスキーは博士論文『労働の哲学』の中で,そもそも科学の進歩そのものに限界を感じていたのだった.

その時代においてパヴロフスキーは社会と個人を労働の観点から形而上学的に考察する自身のプログラムにおいて科学に限界を見ていた.科学は確かに人間の労働を軽減してきたが,未だに余暇を持つことができない悲惨な労働者を生み出してきた.科学がこの問題を解決できるかできないかの悲観的ないし楽観的な見方はいずれにせよ,「鏡の一種によって未来に過去のイメージを投影しており,そうした味方はは私たちに常に根本的に悲観的と思われるのは,これらの見方が,未来について,私たちがすでに離れたいと欲しているより劣った状況を強固にさせ,統制させるのみであるからで(s’établissent toujours sur l’étude du passé, c’est a tort qu’elles[= théories basées] prétendent avec ce passé constituer le devenir qui, par définition même, en demeure différent.)」(PT19),「より劣った状況」に陥っている社会問題を解決するためには行為の分析といった道徳に関する考察を扱うほかない.この時,形而上学が再び有用となるのである.しかし,パヴロフスキーによれば,形而上学には修正すべき問題があるという.パヴロフスキーはベーコン的帰納法とコントの社会学的演繹法の2つが形而上学で支配的な思考法であり,どちらの立場も「私たちが自分たちの経験の円環から少しでも出るはずがない(nous ne devons pas sortir du petit cercle de notre expérience)」(PT8)し,「行為において内的な必然性などはない」(PT8)コント的な立場が観察による客観的な経験主義が行為の分析の妨げなっていることを示している(PT10).そこで,ベルクソンが『直接与えられたもの』で述べているように,持続と連続性に基づく主観性による形而上学によって行為を論じることが可能であると考えた.

しかし,客観的な経験主義に基づく科学をパヴロフスキーは却けていたわけではない.そもそも,『四次元郷』での様々なエピソードは,科学技術の進展なしにはありえない.すると,パヴロフスキーは主観性に基づく形而上学を用いつつ,科学と共存し,4次元においてそうした対立が解消されるという奇妙な論理を立てている.よって,パヴロフスキーは,先の革命の四象限のどこに位置するのか不明確なままである.では,ブランキらの宇宙観と社会論にどのようにしてパヴロフスキーを比較すればよいのだろうか.

革命の四現象との比較のためには,パヴロフスキーが革命思想や社会問題に対してどのように考察していたのかを具体的に知る必要がある.それは,『労働の哲学』で取り上げられている.パヴロフスキーは労働問題の解決を目指して道徳と行為が労働においてどのように現れるかを研究していた.その結果,機械化による労働の削減を道徳によってコントールしつつ進めていくことで,全ての人類が余暇を手に入れることができると述べている.そして,『四次元郷』でも,4次元に到達する人類が「魂の静けさ(l'âme silencieuse)」を手に入れて現実の煩わしさから解放さえるということが第1章から述べられている.これらは共通して人類の未来を語っているのだが,驚くべきことに,ブランキと同じく,具体的な来るべき理想的社会像は全く描かれていない.しかし,パヴロフスキーをブランキと同じ象限に収めると,ある問題が生じる.まず,パヴロフスキーは,この世界に複数の同じような世界があるとは全く考えていない.次に,ベルクソンとの影響関係を調べた際に引用したように,未来しか存在しないというテーゼを立てているため,4次元に到達した途端に,3次元のあらゆる偶然によって生じる出来事は,少なくとも偶然とはみなせなくなってしまうのである.以上のことから,私は,次のように考えている.パヴロフスキーはブランキの影響を強く受けいるので,先行世代の革命思想とブランキの関係の図式に収めることができると考えていたが,そもそも4次元という宇宙観による大きな違いを乗り越えることはできないのではないか.そして,4次元が3次元の入れ物であるという本章で最初に検討したことに従えば,入れ物とその中身として語られていた,観念と物質を新しい四象限の軸の基準とすべきではないか.

問題となっていた未来と現在の縦軸を観念と物質に書き換えるのであれば,革命思想の四象限の人々は全て物質の側に配置される.なぜなら,社会を思考の出発的にするということは,3次元的な物質の世界を対象としているからである.一方で,パヴロフスキーは科学の進展の結果としての4次元の到達を描いているため,観念における必然性の宇宙を描いていると考えられる.しかし,これは2つの理由から誤っている.

第一の理由は,原子という,3次元においては観念的であるものの運動を必然性によって説明しているのがライプニッツである,とパヴロフスキーは考えているからである.それと同時に,パヴロフスキーによれば,「ライプニッツのモナドは,\bou{ただ一つの多様性}の中で,そして,言葉と思考といったものを一致させている予めある予定調和の奇跡の中で無残にも頓挫した(la monade de Leibnitz échouait lamentablement dans la \emph{multiplicité de l'uniuqe} et dans le miracle de l'harmonie préétablie qui faisait coïncider par exemple la parole et la pensée)」(1923, 247)のであり,観念的世界の必然性は4次元においては頓挫してしまっているというのである.この理由から,4次元の四象限は次のように描かれる~図~\ref{fig:pawfourtypes}.\myfig[height=18.2cm, width=12.8cm]{pawfourtypes}{4次元の四象限}{pawfourtypes}

図を示したところで,もう1つの理由を示す.パヴロフスキーは『四次元郷』の中でほとんど偶然という言葉を用いないが,3次元における物質と偶然の関係について,ブランキを高く評価した直後で偶然について述べている.パヴロフスキーは,ブランキの著作名を暗示しつつ,ブランキの宇宙論が示すように,全てが過去や未来に繰り返されるのであれば,意志や進歩といったものは全て意味がないということになり,それは科学が基盤としている唯物論的思考そのものの限界を示していると考える(270/219).パヴロフスキーはこの限界は,人類が4次元に到達することで乗り越えることができると考えている.この乗り越えは,ライプニッツの予定調和ではない偶然と,物質ではない観念を繋いでいる.そのことを詳しく示しているのが,1912年第44章1923年第45章の「観念による不死(L'immotalité par les idée)」だ.この章では,肉体を持ちながらにして精神によって不死に至ることができるということがテーマになっている.そして,この精神の持っている力の1つが偶然に関わっている.パヴロフスキーはこのその力と偶然について以下のように述べている.

\begin{quote}
それこそ,古代の宗教や哲学が,いつもさらなる高みにある地上で連続的に進んでいく魂の漸進的な浄化によって象徴化したものなのだ. 人間の研究は,この観点において,貴重な情報を与えた.3次元空間を自由にできるだけの身体が,生きている間と死んだ瞬間での,いわゆる部分と全体の鋳直しの連続であるような,分離の運命に結びつけられているのに対して,人間の精神はすでに4次元に到達していたし,そのために不死にも接近していた.つまり,精神は同じ瞬間に過ぎ去った,あるいは来るべき現象をそこから思い浮かべることができる.精神は抽象によって,物質的な偶然性を超えて,高みに登り,何らかの,事物の普遍的で変化しない実体をそのうちに持つことができる.
\end{quote}
\begin{quote}
C'est ce que les religions et les philosophies anciennes symbolisaient fort justement par l'épuration progressive de l'âme passant successivement dans des sphères toujours plus élevées. L'étude de l'homme donnait, à ce point de vue, de précieux enseignements. Tandis que le corps, ne disposant que de l'espace à trois dimensions, est voué fatalement à la désagrégation, c'est-àdire à une suite de refontes partielles ou totales, durant sa vie ou au moment de sa mort, l'esprit humain atteint déjà la quatrième dimension et se rapproche par là de l'immortalité ; il peut envisager, dans le même instant, des phénomènes passés ou à venir ; il peut s'élever, par l'abstraction, au-dessus des contingences matérielles, et participer, en quelque sorte, de la substance universelle et immuable des choses. (271/220) 
\end{quote}

精神は4次元へと向かっていくのだが,その際に「物質的な偶然性を超え」るのだという.偶然性を乗り越えるという記述は対立概念である必然性の肯定にも取りうるが,その一方で,精神が4次元に至ることが必然的な宇宙の宿命なのであるという記述は存在しない.では,偶然を超えるということは何を意味しているのだろうか.

ブランキは宇宙論に偶然を持ち込むことで,必然性の革命思想と袂を分かっていた.従って,パヴロフスキーがブランキについて言及しながら偶然を乗り越えるというのは,ブランキの偶然を乗り越えるということであると考えられる.ブランキは物質の偶然的な変化を3次元の宇宙の本質に見出し,一方でパヴロフスキーは4次元の観念においてそれを示そうとした.それは,4次元の四象限に従えば,\bou{観念的な偶然性}と言うことができる.この観念的な偶然性について,パヴロフスキーは,1912年版第3章1923年版第3章の「数え切れないほどの乗合馬車(La Diligence Innombrable)」で示している.

「数え切れないほどの乗合馬車」は4次元による移動とはどのようなものなのかを示している章である.パヴロフスキーはまず次のような伝説を紹介する.なんと,アジアやアラブの世界では信じらないほど遠くの距離でも電報なしでコミュニケーションができるというのだ.それは空間の抽象化(abstraction),すなわち2点間の距離を取り払ってしまうということができることを示しているとパヴロフスキーは指摘する.こうしたことが私たちに不可能であるのは,かつての乗合馬車による輸送が車にとって代わられたように,ひたすら量的な速度が追い求められている時代に生きているからである.ところで,精神は4次元に至ることができるので実は私たちは普段から4次元に通じているのだが,3次元の物質的な身体に囚われているためにそれを理解できない.パヴロフスキーが4次元による空間の抽象化に気づくことができなかったのは,かつての乗合馬車のおかげであった.ある日,田舎に滞在している時に心が望むと,「4次元の中を動き回る私の意志の気まぐれに従って(suivant le caprice de ma volonté agissant dans l'espace à quatre dimensions)」(20/65)必ず乗合馬車がやってきてそれに乗ることができた.パヴロフスキーはこの日の体験について,「その現象は私にとって合理的な説明のつかない自然発生的に起きたのだった(le phénomène produisit pour moi spontanément sans explication raisonnable)」(20/65)と述べている.精神が4次元に繋がっているために,物質的な偶然性という3次元の出来事ではなくて,精神の作用による観念的な偶然が起きているのである.

ブランキとパヴロフスキーの関わりは,パヴロフスキーが結局,新しい社会を示すことはできなかった理由を説明していると思われる.ベンヤミンも指摘していたように,ブランキもまた,来るべき社会の理想像など抱いていなかったからである.科学技術の発展の先で人類が見るものは「\bou{私たちの知的生が宇宙の生でさえあり,最も高度な宇宙の表出である}(\emph{notre vie intellectuelle est la vie même de l'univers et son expression la plus haute})(1923,249).私たちは宇宙論的に基礎付けられ,それは私たちの知性と一致するのである.こうした宇宙論はブランキのそれと非常に似通っている.しかし,決定的に違うのは,ブランキは有限の元素の組み合わせが反復されるとしたのであって,決して原子をただ1つであるとしなかったのに対して,パヴロフスキーはただ1つの原子によって3次元的宇宙が織り成されていると考えた点である.パヴロフスキーは以下のように,原子はただ1つしかないと考えている.

\begin{quote}
  精神は原子が精神自身のイメージであることを知っており,精神は4次元の特有で完全な原子から作られていて,そしてそれは,3次元の不完全な世界の多様な様態のもとで,合わせ鏡のようにただ1つの原子が無限に像を映しているというある感覚の幻想なのである.
\end{quote}
\begin{quote}
  L'esprit conçoit l'atome à son image, il en fait donc un monde complet et unique à quatre dimensions et c'est une illusion des sens qui reflète à l'infini comme dans des glaces multiples cet atome unique sous les aspects divers du monde incomplet à trois dimensions.(48/80)
\end{quote}

私たちの世界はただ1つの原子によって構成されており,それは本当の意味では存在していない.これをパヴロフスキーは最終的に「統一体はただ1つだった(\emph{l'unité était unique})」(1912, 315)と言い表している.この表現は,原子によって構成されていると考えられていた異なる様々な物質の集合はたった1つの原子によって成り立っていることを簡潔に表現しており,これは物質とエネルギーをめぐるテーマと物語が大きく関わっていることを示してもいる.

3次元における物質は原子によって構成されている.これは連合(association)と分離(dissociation)の関係にある.すなわち,リヴァイアサンの時代に1つの巨大な連合体であった人類が科学の時代の移行に際して,分離したという構造とパラレルである.人類が4次元に到達する大いなる観念論の復興の時代において,人類は再び4次元において連合するが,このとき人々の精神の働きであるところの知性は宇宙論的に基礎付けられているのであり,物質・原子・人間の集団が相似した図式のうちにある.原子の分離と集合によって物質は変化し,人類は同じ作用によってリヴァイアサンを生み出し,そこから脱出して個性を回復したパヴロフスキーは未来の社会像を打ち出すことなく,観念的な偶然性によって結ばれる人々の姿をブランキの偶然性を超えて描いている.

\section{文学と科学}

鈴木雅雄はブランキの特徴として,科学的真実から自らの文学的想像力を切り離していると述べている\footnote{鈴木,前掲書,14頁.}.つまり,ブランキは科学的な事実に依拠しながらも,決定的に科学とはずれた結論を引き出しているということである.科学の進歩と文学や思想がその歩調を合わせていた時代に,ブランキが科学的事実に依拠しながらもそれを世界を語るための真理とみなさなかったことは確かに他と比して注目に値するだろう.

ところで,この科学的真実と文学的想像力の関係は,パヴロフスキーの場合,揺らいでいる.それは,「文学的なてらいはなしで(sans la prétention littéraire)」(3/56)という作品の冒頭の表現によく表れている.「批判的吟味」においても,パヴロフスキーは自分は「作品(travail)」を書いたのであり,その結果出版されたものはあくまで「本(livre)」であって,それを「小説(roman)」であると規定することは決してない.あくまでも「時間の探求(exploration du temps)」(1923, 9)なのだ.芸術作品には3次元と4次元の接触が現れているという考えを抱いているのにもかかわらず,自身の4次元を題材にした作品はそういった身分にはふさわしくないかのような主張は,『四次元郷』という明らかな文学的想像力を前にした読者に奇妙な印象を与える.また,科学の進展がもたらした人類の災禍であった第一次世界大戦の前からすでに,パヴロフスキーはこの現実の世界における進歩のもたらす福音を前世代に比べていささかも信じていない点も奇妙である.リヴァイアサンという国家有機体説の最悪のヴィジョンを見せ,その後個人の価値が復権するものの,再び人類は失敗し,最後は4次元がすべてを解決するが,そこには人間が理想的に暮らす社会などない.魂の解放と福音というキリスト教的終末論の1つのバージョンにも思われる展開だが,結局のところ,パヴロフスキーは来るべき社会は存在しないうえ,個人の意志によって人々が結びつくことができないと考えていたのだ.ブランキの偶然性を超えて人々は4次元で統一されることになる.しかし,それは全てが1つの原子に由来しているという真理によって可能となっていて,その事実に至るのは人類の科学の進歩の結果にほからない.科学は真理を語らないことを何度も示すことによってパヴロフスキーが4次元の本質を語るときに,逆説的に科学なしでは4次元について何も語れないという論理の捻れを示している.文学的想像力を科学的真実から引き離せば引き離すほど,その繋がりを強調してしまっているのだ.

パヴロフスキーにおける文学的想像力と科学的真実の関係は以上のように,ブランキと異なっている.パヴロフスキーは科学に反対するための想像力を科学の進歩を描き続けることでしか真実を示すことができないという困難な立場を取っている.これこそ,彼がユーモアという概念を『四次元郷』の中で何度も示している態度と関係していると考えられるが,それについては別の機会に改めて論じる.

\chapter{遺伝と4次元}
\section{テツノミ}
物質を構成する原子のうち,すでに知られている力を有しているものは「幼生」と呼ばれていたのは前章で見たとおりである.そこでは取り上げなかったが,『四次元郷』は4次元やそれに関わるモチーフよりも「幼生」といった語が示すように,生物のモチーフが頻出している.具体的に,本章で取り上げるもののみを挙げていくと,1912年版第26章1923年版第27章「テツノミ(ferropuceron)」では鉄でできたノミ,1912年版第23章「生を捕らえて(La Capture De La Vie)」1923年版第30章「産業植物(Les Plantes Industrielles)」での工場生産のために機械化された植物,1912年版第33章1923年版第37章「巨大バクテリア(Les Bactéries Géantes)」では,章題の通り巨大なバクテリアが登場する.『四次元郷』で重要なテーマとなっているこれらの生物は,前章で焦点を当てた物質とも密接な関わりを持っている.パヴロフスキーは科学の時代に物質から生命に至るという自然発生説を取り上げ,それは先の章で見たいように,全てはただ1つの原子からできているという4次元の原理によって保証されている.しかし,今までの研究では4次元と生物のモチーフが関係しているという指摘はなされてこなかった.パヴロフスキーは4次元と生物のモチーフをどのように関係付けているのか,先に挙げた生物のモチーフのある章を取り上げて検討していこう.

「l2人の不死老人(des Douze Vieillards immortels)」の1人である「水素(Hydrogène)」は4次元を通じて2000年代に起きた昔話を語り手に聞かせる.その話によると,この時代には人造鳥に人が乗って狩猟などをしていた.ある時,パイロット「671-98」が乗り込んでいたハヤブサを模倣した人造鳥が高度3000mから墜落し,671-98は病院で36時間かけて子牛,猿,犬の臓器を移植され,命をつなぎとめた.その後,人造鳥墜落事故の調査行ったものの,機械に異常が見られなかった.調査を一旦終わらせて飛行テストをすると,右脚に相当する部位を横側に自動的に動かそうとしてか平衡を失って墜落してしまうという挙動が繰り返されたのでハンガーで精密検査を行うこととなった.その結果,なんと鉄でできたノミが人造鳥の表面から発見された.鉄でできたノミは「テツノミ(ferropuceron)」と呼ばれ,人造鳥のある特徴のために事故を引き起こしてしまったということが判明した.その特性とは,人造鳥に与えられていた,渡り鳥が磁気に従って飛行するなどといった鳥の性質や,加えて,感覚(sensibilité)の機能も与えられていたために,テツノミに由来する痒みを感じて,前脚で掻こうとしてしまったことが事故の原因だったのである.環境に適応してヤスリ粉を栄養とするノミが生まれたことは,のちに起きることとなる「機械の反乱」という工場の機械が生命のように暴走することになる事件を前日譚であるとパヴロフスキーは別の章への関係があることを示唆している.

テツノミの出現に際して,「かつてのノミが,どうやってやすり粉を栄養にし,飛行機の翼の上で生き,その環境に適応しながら変化したのかをできる限り説明しようと努力がなされた(On s'ingénia donc, du mieux que l'on put, à expliquer comment d'anciens pucerons, s'alimentant avec de la limaille de fer et vivant sur les ailes des aéroplanes, avaient pu se transformer en s'adaptant au milieu.)」(171/159-60)と述べている.また,テツノミは,ノミが環境に適応した結果生まれたと考えられて,それは擬態(mimétisme)の一例であるというある種の自然選択の理論が人類によって展開されたという(171/160).しかし,パヴロフスキーはそれに対して,「この時代,未だに進化論者の馬鹿げた教義が考えに染み込んでいて,自然発生は馬鹿げたことの1つだと思われていた(on était encore, à cette époque, entièrement imbu des absurdes doctrines évolutionnistes, et la génération spontanée paraissait une simple absurdité.)」(171/159)と述べられている.つまり,進化論だけでテツノミの出現を理解しようとした人類が批判されているのである.パヴロフスキーの自然発生説の支持は,前章で取り上げた章である「悪魔祓い」で,あらゆる物質にも生命が宿っているという考えが示されていることから科学の時代に入って人類は自然発生説を完全に肯定するようになる.以上のように,パヴロフスキーは人類が4次元に到達する過程で自然発生説を採用していることから,自然発生説は『四次元郷』において非常に重要な思想であると言える.パヴロフスキー自身が自然発生説について語っているところは,『四次元郷』の「批判的吟味」に見つけることができる.そこでは,ステファン・ルデュク(Stéphane Leduc)の浸透(osomotique)という現象に依拠することで打ち立てられた自然発生説が取り上げられている(1923, 40).ルデュクの浸透の実験の多くは問題が多いのものであったためにアカデミーから彼の実験結果は否定されることになる.パヴロフスキーはルデュクの仮説のうち,物質から自然に生命が生まれるというアイディアの重要性を4次元と3次元の接触によって生じる芸術を引き合いに出して,「芸術の最も優れた表れとは,私たちに物質の最も優れた集合とともに現れる(les premières manifestations de l'art nous apparaissent avec les premiers groupements de la matière)」(1923, 40)と述べている.パヴロフスキーが自然発生説を芸術論の根拠にもしていたことがここからわかる.

そもそも自然発生説とは,1つの種が長い期間を経てその特徴を変化させていく進化論とアリストテレスが『動物誌』の中で種子に依らない植物の発生を説明する理論がその嚆矢だった.自然発生説は生命起源論争と合流し,19世紀のフランスの科学の諸分野では,その正当性をめぐって意見が二分された. パヴロフスキーはルデュクの他にも生命起源論争や,テツノミで否定はされていた進化論に関心があったことが『四次元郷』の多くの場所から読み取ることができる.具体例を挙げれば,『四次元郷』の中でダーウィンに2度言及しており,1度目は『種の起源』と『人間の由来』の作者で自然選択説を提唱した学者として(94/115),2度目はモウセンゴケ(drosera rotondifolia)の報告者としてである(183/169).のちに触れるように,パヴロフスキーは『労働の哲学』や『国家社会学』の中で有機体国家説を取り上げる際にハーバート・スペンサー(Herbert Spencer)へ言及することが多く,必ずしもダーウィンの進化論に直接依拠して物語を書いているとは限定できない.しかし,下記で述べるように,例えばパヴロフスキーがセンモウゴケの報告者としてダーウィンを挙げて,センモウゴケが登場する章は植物のサイボーグ化が遺伝現象によって失敗するという科学的管理が生命の抵抗力に及ばないというエピソードであり,当時のフランスの進化論の議論の一部がが園芸や農業における人工交配と深く結びついていたことから,スペンサー的な社会進化論よりも,ダーウィンの進化論をめぐるフランスにおける議論にパヴロフスキーは影響を受けていた.そこで,ダーウィンの進化論がフランスでどのように受容されたのか,そして生命起源論争として自然発生説はどのように取り上げられたのかを最初に見ていき,次に,パヴロフスキーが誰のテキストからその影響を受けてどのような考えを示しているのかを見ていこう.

\section{フランスにおける進化論の導入とパヴロフスキー}

フランスでは,解剖学や生理学を背景とした遺伝をめぐる言説,人類学・古生物学・生物学に影響を与えたダーウィニズムをめぐる言説が,19世紀後半に反目し合いながら共存していた.パヴロフスキーが言及する哲学者や思想家もその例に漏れない.例えば,ドイツの唯物論の議論を俯瞰したポール・ジャネ\footnote{Paul Janet, \emph{Matérialisme Contemporain en Allemangne}, Paris, G. Baillière, 1864},観念力(idée-force)という用語を生み出したルフレッド・フイエ\footnote{Alfred Fouillée, «~Le plaisir et la douleur du point de vue de la sélection naturelle~», \emph{Revue des Deux Mondes}, 1er avril 1886, pp. 658-682.},『知られざるもの\footnote{Camille Flammarion, \emph{L'Inconnu}, Paris, E. Flammarion, 1900.}』で人間の可視光帯域や可聴帯域の外側を感覚することで未知の世界へ至るという議論を展開しているカミーユ・フラマリオン\footnote{Camille Flammarion, \emph{Le monde avant la création de l'homme}, Paris,  C. Marpon et E. Flammarion, 1886.}もダーウィニズムの影響が色濃く見られている\footnote{この具体例はすべてイヴェット・コンリーの下記の著作を参考にした.Yvette Conry, \emph{Introduction du Darwinisme en France au XIX\textsuperscript{e}siècle}, Paris, J. Vrin, 1974.}.また先行研究で言及したエレーナ・ゴメルも,4次元での時間旅行のアイディアを最初に小説にしたウェルズも,4次元が3次元の存在の進化だと考えていたと指摘している\footnote{Gomel, \emph{op. cit. }, p. 155.}.また,スティーヴン・マクリーン(Steven McLean)のような論者は,『タイムマシン』を「ポスト・ダーウィン的な宇宙における不運な出来事(Misadventures in a Post-Darwinian Universe)\footnote{Steven McLean, \emph{The Early Fiction of H.G. Wells: Fantasies of Science}, London, Palgrave Macmillan, 2009, p. 6.}」という枠組みに位置付けている.さらには,ウェルズが生物学者T・H・ハクスリー(T. H. Huxley)とともに,生物学者になるための勉強していたことから,進化論に通じていたと考えられているし,マクリーンが『タイムマシン』で描かれている退化した人類の姿は,動物学者のレイ・ランカスター(Ray Lankester)のダーウィニズムの影響下にあった退化論が関係している指摘もあり\footnote{\emph{Ibid.} , p. 25-6.},当時の4次元を扱った小説は進化論と深い関係を結んでいた.
しかし,パヴロフスキーのいたフランスでは,イヴェット・コンリー(Yvette Conry)が明らかにしたように,進化論は,その提唱者ダーウィンの名を冠したダーウィニズムの名前で普及し,ダーウィン自身とフランスのアカデミアンたちの専門的な論争,それを伝える大衆雑誌や新聞の言説,それに反応する知識人たちといったようにダーウィニズム論争の言説全体を複雑なものにしている.とりわけ,フランスではラマルクが19世紀の始めに提唱した生物変移(transformism)の仮説が1871年に終わった普仏戦争以降のナショナリティックな雰囲気の中で自国の進化論的モデルとして再評価され,ネオラマルキストを生むこととなった.ネオラマルキストは生物変移説をより科学的なものにしようとして,唯物論的宇宙における機械論的因果論という基礎的理論を求めた\footnote{Laurent Loison, «~Le projet du néolamarckisme français (1880-1910)~», \emph{Revue d'histoire des sciences}, t. 65, janv. 2012, pp. 61-79.}.一方で,ダーウィン的な進化論を独自に展開させたネオダーウィニトは,進化が先天的に要因にあるものだ考え,目的論的進化観を展開した\footnote{Paola Marrati, «~Le nouveau en train de se faire~», \emph{Revue internationale de philosophie}, n. 241, mars 2007, pp. 267-8.}.パヴロフスキーもまた,『労働の哲学』や『四次元郷』で目的論(finalime)や機械論(machinisme)を,個人の自由や意志について論じるためによく取り上げている.個人の行動が目的論的であれば,あらかじめ決められたプロセス以上のことをなしえないために自由はない.一方で,機械論の立場をとる場合は,個人を包摂する社会との関係によって自動的に行動が決まってしまうので,やはり自由はない.では,個人の自由を論じるためにはどうすれば良いのか,というのがパヴロフスキーの問いの中心であった.こうした立論自体が,ネオラマルキスムとネオダーウィニスムという進化論をめぐる2つの立場を背景としているものだったのだ. 

以上のように,哲学的な議論に影響を与えた進化論だが,『四次元郷』に登場するテツノミや人造鳥,さらには巨大なバクテリア,産業植物といった未来の生物を解釈するためには,さらに広いスコープで進化論の議論を振り返る必要がある.「変異の発見,遺伝の科学,有機体の粒子的仕組みは,生態学的な実験と衝突したのは,ダーウィニズムから真理,多産性,必然性を導くためだった.(Une détection des mutations, une science de l'hérédité, un schéma particulaire de l'organisation interféreront avec une expérimentaion écologique pour induie la vérité, la fécondité et la nécessité du darwinisme)\footnote{Conry, \emph{op. cit. }, p. 424.}」とコンリーが述べているように,当時のフランスにおける進化論は生物学に止まらない広範囲で取り上げられた.それはダーウィンが『人間の由来(\emph{The Descent of Man, and selection in Relation to Sex})』(1871)を発表したことで,人類学における進化論の検討がなされたことからもよくわかる.加えて,コンリーは,当時のフランスでは,人類学は医学の一部と見なされていたので,進化論の議論が病跡学と結びついたと指摘している\footnote{\emph{Ibid.} , p. 81-2.}.そして,医学の中でも精神疾患を取り扱う精神医学(psychiatrie)では,奇形学(tératologie)と呼ばれていた遺伝疾患の事例を先祖返り的(atavique)な現象とみなされ,ダーウィンはそれを後退とみなし,人間と動物の連続性を示す事例であると考えていたために,進化論が注目されることがあった\footnote{\emph{Ibid. }, p. 82.}.『四次元郷』1912年版では,人類がリヴァイアサンから解放される契機がドイツにおける類人猿(grands singes)の実験であること(1912, 76)や,「テツノミ」で飛行士の手術で動物の臓器を人間に移植する話が取り上げられるのは,動物が人間の先祖であり,類縁性を持っているという議論を背景として描かれていると考えられる.また,人間の精神を動物に移植するストーリーが登場する1912年版第31章「自然の力の彼方へ(Au delà des forces naturelles)」1923年版第34章「自然の形態の彼方へ(Au delà des formes naturelles)」では,動物たちが人間のような精神的疾患を引き起こす様子が描かれるといったように,フランスにおけるダーウィニズムの導入によって引き起こされた議論がパヴロフスキーの知的背景を形作っていたことは想像に難くないのである.

進化論のフランスにおける影響に話を戻すと,こうした人間と動物の類縁性をめぐる議論は,遺伝(hérédité)と深く関わることとなる.19世紀フランスでは遺伝の議論には雑種状態(hybridité)という問題系があった.メンデルに先んじて人工的な配合から遺伝について考えていた人物として知られているシャルル・ノダン(Charles Naudin)は雑種状態に関して代表的な論者だった.1852年,ノダンは園芸雑誌にて「種と多様性についての哲学的考察」\footnote{Charles Naudin, «~Considérations philosophiques sur l'espèce et la variété~», \emph{Revue horticole}, v. 4, n. 1, 1852, p. 104.})を公表した.この論文で彼は,種の形成が自然の場合も人工の場合も同じであるということを指摘している.しかし,人工交配などの個体の選択が自然界で起きていることはノダンは考えなかった.ノダンは後年,自然選択の反証を示しているとされる,アレクシス・ジョルダン(Alexis Jordan)の固定主義(fixisme)の立場を採用して,進化論における自然選択には反対している.その一方で,ダーウィンはノダンの研究が自分の仮説を裏付けるものであると考えたことが知られている\footnote{Conry, \emph{op. cit. }, pp.110-11}.それというのも,遺伝を決定する因子に関してノダンとダーウィンは,意見を異にしていたが,あらゆる生物の種は同じ方法で進化するという概念は共通していた.ただし,ノダンにとって「進化は,原理的な種の分割という過程であり,「形態の増殖の方法」の1つであると定義される(l'évolution, dont le processus consiste en une subdivision des types primordiaux, se définit comme un «~procédé de multiplication des formes~»\footnote{\emph{Ibid. }, p. 117.}」.Conryはそれを「増殖とは,ダーウィニズムにおける条件であり,Naudinにとって進化そのものだった (La multiplication, qui est condition dans le darnisme, devient pour Naudin l'évolution même)\footnote{\emph{Ibid. }, p. 118}」と評しているように,Naudinは進化の本質は分裂と増殖にあると考えていた.パヴロフスキーが進化論における形態の変化に興味があったことは,1912年版第33章1923年版第37章「巨大バクテリア(Les Bactéries Géantes)」から知ることができる.

この章では,大中央研究所から19世紀から20世紀にかけての古いバクテリアが流出し,感染病が人々の間で広がってしまう.ある日本人科学者はバクテリアを目に見えるように巨大化させることに成功し,体内に侵入できないようにするという方法によってこの事件は終息した.このエピソードで日本人科学者はバクテリアの「巨大化」という秋からに不条理な解決手段をとっている.一見して,パヴロフスキーの時代の医学の知識の限界であるようにも思われるが,その可能性は低い.パヴロフスキーが『四次元郷』を執筆していた1900年代には,バクテリアないし細菌への対処は現代に比べれば難しいとはいえ,全くの未知というわけではなかった.実際,作中で復活した病は「【チフス,コレラ,ペスト,黄熱病,狂犬病,破傷風|梅毒疹,黄熱病,ヘルペス脳炎,狂犬病,破傷風】(\{le typhus, le choléra, le pester, la fiévre jaune, la rage, le tetanos | la syphilèpre, la fièvre jaune, l'herpès encéphalique, la rage, le tétanos\}」(1912-213/1923-193)であり,いずれもワクチン接種によって治癒可能である.なので,科学者によるバクテリア自体の「巨大化」というのは非常に不条理な処置のように思われる.このエピソードの不条理さはパヴロフスキーの単なるユーモアとして片付けることはできない意味を持っている.なぜなら,日本人科学者の処置は形態を変化させることによって事件を解決しているからである.日本人科学者のとった手段である「巨大化」は,当時の医学用語でもあったからだ.

「巨大化」はフランス語で\emph{gigantisme}とされており,これは19世紀の頃から,医学用語では「巨人症」という遺伝疾患の1つであり,植物学でも植物体の巨大化現象を示す際の用語でとして用いられている.奇形学の症例の1つとしてどちらの分野でも扱われていた\footnote{医学では以下を参照のこと.Charles Bouchard, \emph{Traité de pathologie générale}, t. 1, Masson, 1895, p. 317..植物学では以下を参照のこと.Auguste Bellynck, \emph{Cours élémentaire de botanique}, G. Mayolez, 1876, p. 158.}.よって,日本人科学者のとった手段である「巨大化」は遺伝の文脈に読み替えると,「巨人症」として解釈することを可能にしているし,精神医学や遺伝学に関心があれば,これは\emph{gigantisme}の二重性にかけた地口であると言える.現代的に言い換えると,日本人科学者の行なっているのは遺伝子の組み替えによって変異を作り出すことなのだ.巨大なバクテリアをめぐるエピソードは形態の操作であり,遺伝学的な文脈が背景にあった.そこで,進化における遺伝現象がフランスでどのように受容されたか見ていこう.

当時,ダーウィンとノダンは雑種状態と遺伝2つの間で対立している点があるかのように論じられていたが,種の親子は類似する一方で変化し続け続けてくことは避けられないということを背景にしているため,理論の基礎的な部分で共通している点が多い\footnote{Conry, \emph{op. cit. }, pp. 129-30.}.ただし,ノダンが形態の哲学から出発しているのに対して,ダーウィンがパンゲネシス粒子の伝播効果として遺伝が生じると考えている点が多く違っている.結果的に19世紀後半のフランスでダーウィンのパンゲネシス説が遺伝の説明で採用されなかったのは,代謝の研究から生化学(biochimique)の理論や原子の力のメカニズムに関する研究が発展していたために,形態学のような形の類似性に基礎をおいた研究はあまり顧みられなくなっていた.しかし,遺伝の考え自体は19世紀後半のフランスにおいて非常に強い影響力を持っていた\footnote{\emph{Ibid. }, pp. 327-8. また,以下の資料で詳細を知ることができる.Prosper Lucas, \emph{Traité philosophique et physiologique de l'hérédité naturelle dans les états de santé et de maladie du système nerveux}, t. 1 et 2, Paris, J. -B. Baillière, 1847-50.}.パヴロフスキーの次のエピソードも,生化学と遺伝が背景にあるよと考えられる.
それは,1912年版第23章「生を捕える(La Capture de la Vie)」1923年版第30章「産業植物(Les Plantes Industrielles)」である.以下で簡単に物語のあらすじを見ていこう.

「機械の反乱」があった後,物質を操作するのが人間だけではなく,植物もまたその能力を有しているということに気づいた.植物はそもそも巨大な工場や複雑な仕掛けもなくそこから多様な要素からなる物質を取り出せるのではないかと人類は考えた.穀物の種子はある土地に蒔くことで,根と茎を伸ばし,豊かな色素を抽出でき,芳香を放ち,人間に滋養のある果実をつける.さらに,植物は,化学的にも興味深い性質を持っている.土壌が炭酸成分を含んでいてもそこから炭素を固定することができたし,不活性(inerte)な単体ないし化合物を与えても,有機物を生み出すことができた.「つまり,植物だけで,見えない過程による,おそらくは困惑するような単純な方法で,植物は単体を他の単体への信じられないような変異を実現しているのだった(A elles[=~plantes] seules, en un mot, au moyen de procédés invisibles et sans doute d'une déroutante simplicité, les plantes réalisaient ces invraisemblables transmutations d'un corps simple en un autre)」(180/168) .そこで,人類は新しい植物を作ることで自由に物質を生産しようと考えた.人間の求める化合物を生み出す新しい植物は生産者が育成をしやすいように動物のように栄養摂取するようにした.また,その植物の根と茎は土壌と機械の間でのみ生育するようにもした.その改良した植物のおかげで化学物質の生産が数年間飛躍的に向上した.人々は更なる生産を求めたために,植物のための工場が次々と建てられていった.しかし,「産業化された植物は生殖の喜びを奪われて,刺激のある状態が永遠に保たれ,下等動物のやり方で,気味が悪く陰険でむごたらしくなった(Les plantes industrialisées, privées des joies de la reproduction, maintenues dans un perpétuel état d'excitation, devinrent méchantes, sournoises, cruelles, à la manière des animaux inférieurs.)」(182-83/169).その結果,産業植物はモウセンゴケのような触手を伸ばすようになり,犬や猫,兎といった小型動物を捕食するようになってしまった.さらに時が経つと,ついに産業植物は根から養分を吸収しなくなり,人間の脳のような機能をもち,感覚器官としての役割を持った.その後,根の先端は小さな感受性菌(champignon sensible)へと変化して,大地を自由に移動できるになった.この頃になると,産業植物は嫌悪を催す(infâme)物質を放出するようになっていた.人々はこれを遺伝における劣化現象の1つである変質(dégénérescence)だと考えた.産業植物はすでに自然にあるような姿をしていなかった.人々は工場の壁に沿って生えている産業植物を慰めようと,櫛でといたり,植物が繋がっている機械を彩色したりした.しかし,効果は見られなかったため,産業植物の花がむしられ,それらは死滅した.

この産業植物のエピソードは農業ないし園芸における人工交配を背景にしていると考えれる.そして,さらに興味深いことに,一般的に,農業で人工交配が行われる理由は生産高の増加や害虫への耐性を上げるためになされるのに対して,このエピソードはどの植物でも行なっている生体機能自体を,人間が求めている化合物を取り出すために改造する点にある.つまり,「産業植物」での植物は,進化論において言い換えると,その種全体として扱われているのである.それぞれの種の緩やかな変化が進化論という概念が発明される契機となっていたことを踏まえると,このエピソードは次のように総括することができる.産業植物は根や茎が機械と繋がれたサイボーグ状態にあり,本来の生体機能を超えた能力を植物に与えたが,結局はそれは変質という遺伝現象が原因で死ぬことになってしまうのである.変質は,19世紀後半,精神医学と進化論が遺伝という現象を鎹にしてその関係が論じられていた.以下では,進化論の中に変質が登場する文脈を見ていこう.

%そして,分裂と増殖が必要としているのは,エネルギーであると考えた. Piquemalが指摘しているように(Canguilhem G., Lapassade G., Piquemal J., Ulmann J. (1962). Du développement à l’évolution au XIX ème siècle, PUF, p.21),Naudinはキュヴィエから借りたtourbillon vitalによるエネルギーによる形態変化のイメージ125注釈6を持っており,また,遺伝と先祖返りを同一視していたと考えられる(128).スペンサーが1862年に発表し,1871年にフランス語に訳されたFirst Principalから,Naudinは形態とは運動のリズムであるということを述べている(ちなみに,本では126,1863年に初版であるかのような記述がなされていて誤解を招くが,発表されたのは1862年.The Synthetic Philosophyの1冊として刊行された).

ダーウィンの進化論は,「科学的経験主義(empirisme scientifique)」の名の下で精神医学における変質の理論に組み込まれた.精神医学者たちは,病を分類できても治すことができない病気が変質であると考えていたので,そうした傾向が促された.結果的に,精神疾患患者のデータを調べることで,家族内に精神疾患がいる場合の発症率や患者の死亡率が高いことから,自然選択(selection naturelle)が疾患の発症において働いているのではないかと考えられた\footnote{変質説を唱えたベネディクト・モレル(Bénédict Morel)は,優生学的な記述もしており,結婚の制限などが推奨されるなどとしていた.Conry, \emph{op. cit. }, p. 328.}.このように,精神医学での進化論はフランス独自の遺伝の概念に適合する形で受容された.実験による証明を重視する実証主義科学の陣営からは,そもそも実験で確かめようのない進化論は否定された.よって,先ほど取り上げた産業植物のエピソードが重要なのは,生化学的な技術の応用が遺伝現象によって失敗する物語であると解釈できるからである.物語の中では,進化論の理論の一部を構成していた遺伝が実証主義科学を代表とする生化学に勝利しているのである.そこで,生化学陣営のからの進化論の批判を取り上げつつ,パヴロフスキーがそれに対して『四次元郷』でどのように応じているのか検討していこう.

当時の代表的な生化学者だったクロード・ベルナール(Claude Bernard)は,進化論に対して否定的な意見を寄せている.1879年には「ダーウィニズムは,生命機構が他のものからあるものを生じさせるという進化を備えることができることを認めつつ,何も説明していないし,私たちにとって全く理解できないままにとどまっているその最初の力について比較的には何も述べていない(la darwinisme, en admettant que les \emph{mécanismes} vitaux peuvent avoir une évolution qui les fasse tout procéder les uns des atures, n'explique rien et ne dit rien relativement à cette force première qui reste tout aussi incompréhensible pour nous)\footnote{Claude Bernard, \emph{Leçons de Physiologie Opératoire}, Paris, M. Duval, p. XIV.}」と述べている.また,他の著作では,「もし生体が進化論の法則に従うなら,再生と癒合の能力は,その法則がより盛んに現出する生体におけるどんなことであれ,生体にとって独占的でもないことなる(Si les corps vivants ne sont pas seuls soumis à la loi d’évolution, la faculté de se régénérer, de se cicatriser, ne leur est pas non plus exclusive, quoique ce soit sur eux qu’elle se manifeste plus activement)\footnote{Claude Bernard, \emph{La Science Epérimentale}, Paris, J. -B. Bailliere, 1878, p. 172}」とあるように,進化論は科学にとってあまり有益な見解を引き出せないと考えていた.さらに,『動植物に共通する生命現象についての講義\footnote{Claude Bernard, \emph{Leçons sur les phénomènes de la vie communs aux animaux et aux végétaus}, t. 1, Paris, J. -B. Baillière, 1879.}』の中で,「実のところ,遺伝は実験の条件として考えられている(A la vérité, on peut considérer l'hérédité comme une condition expérimentale)\footnote{\emph{Ibid. }, p. 342.} 」と述べ,ベルナールにとって遺伝は生命の根本原理というよりも,実験のさいの条件に過ぎなかった.すなわち,遺伝はベルナールにとって実在するものとしては考えられなかった.

生化学と進化論の対立,遺伝の実在の是非を考えると,ベルナールに依拠した論をパヴロフスキーがそのまま展開することはありえないはずだ.では,パヴロフスキーはベルナールの何に注目していたのだろうか.『四次元郷』で引用されているベルナールの言葉は次のようなものである.
\begin{quote}
生命の固有性は,現実的には,生きている細胞の中にのみにある.その他の全て[=~細胞以外に生命を構成しているもの]は適切に配置されて機械的である
\end{quote}
\begin{quote}
  Les propriétés vitales, avait-il dit, ne sont en réalité que dans les cellules vivantes~; tout le reste est arrangement et mécanique(100/111)\footnote{Claude Bernard, \emph{La Science epérimentale}, p. 167.}
\end{quote}
ベルナールが引用されている箇所では,リヴァイアサンが「社会的な有機体の発展(un développement de l'organisme social)」(100/111)の結果として生じ,支配された個人が細胞の状態にあるということについて説明している.ベルナールが指摘する生命の固有性が細胞にあるということがこの文脈の中で引かれている.このことから,パヴロフスキーは,ここで社会有機体説を踏まえつつ,社会にとっての個人を生命にとっての細胞であると述べていると考えられる.これは,スペンサーの社会と有機体のアナロジーを踏まえた,パヴロフスキーの次の引用によく表れている.
\begin{quote}
スペンサーは,同じようなものであれ[=~有機体であれ]有機体でないものであれ,全体が部分のために生きており,人間の体でもそうであるように,部分が全体のために現れてくるということを理解させた.
\end{quote}
\begin{quote}
Spencer avait fait entendre que dans un pareil et non organisme, le tout vivait pour les parties, point, comme dans le corps humain, les parties pour le tout.(100/111)
\end{quote}
ここでパヴロフスキーは『生物学の原理』でスペンサーが生物は群体からは個体を切り離して存在することは不可能であると論じているのを踏まえていると考えられる\footnote{Herbert Spencer, \emph{The Principles of Biology, The Works of Herbert Spencer}, v. II, Osnabrück, Otto Zelleri, 1966, p. 244.}.これは,社会を構成する個人はどのような存在なのか,というパヴロフスキーのテーマと関係している.前章では,原子と個人のアナロジカルな関係に焦点を当ててこのテーマを取り扱ったが,本章で見てきたように,パヴロフスキーはもう1つの別のアナロジーを用いていたのは,社会の発展が進化論に重ねられるという19世紀に強い影響力を持っていたアナロジーである.

フランスでは,進化論の流行と時期を同じくして,元来コントの用語であった社会学が独自の発展をし,デュルケームなどの近代的な社会学の嚆矢となる思想家がいたように,社会と個人がどのように関係を取り結んでいるかという議論に,進化論が反映させられていた.例えば,1906年にはピョートル・クロポトキンが執筆した『相互扶助論\footnote{Pierre Kropotkine, \emph{L'Entr'aide, un facteur d'évolution}, Paris, Hachette, 1906.}』の仏訳が刊行され,植物・動物・人間の社会性こそが生存に不可欠であること,そして,個体の能力獲得は生存するために絶対に必要なわけではないことが述べられている.こうした考えはとりわけ機械論的世界観を重視したネオラマルキストが好んで取り上げ,社会と個人が自然選択の中でいかなる変化をするのかという点が重要となった\footnote{Berdoulay Vincent et Soubeyran Olivier, «~Lamarck, Darwin et Vidal~: aux fondements naturalistes de la géographie~», \emph{Annales de Géographie} , t. 100, n. 561-562, 1991, pp. 617-634.}.パヴロフスキーは『労働の哲学』(PT20)や『国家社会学』(SN12)で社会有機体説に基づいた議論を展開している.同じことが,『四次元郷』においてリヴァイサンという怪物の名によって語られている.すなわち,『四次元郷』の最初の支配者であるリヴァイアサンは国家の進化の結果だったのである.

パヴロフスキーが進化論の遺伝と国家有機体説を扱っていたことは以上に述べた通りである.本章の冒頭で述べたように,パヴロフスキーは進化論とは別に,生命起源論争に関心を持ち,自然発生説を支持しているのであった.無機物から有機物への変化というテーマは,進化論でも扱われていた.ダーウィンは生命の起源について明確な言及を避けていたものの,ダーウィンの書簡や暗示的な言及箇所から,ダーウィンが無生物から生物へ移行していたことを認めていると考えられている\footnote{以下の文献を参照のこと.Michel Morange, «~Darwin dans L'histoire de la Pensée~», \emph{Transversalités}, n. 114, fév. 2010, p. 112. あるいは以下の文献を参照のこと.Stéphane Tirard, «~Origin of Life and Definition of Life, from Buffon to Oparin~», \emph{Origins of Life and Evolution of Biospheres}, n. 40, 2010, p. 217.}.しかし,パヴロフスキーは書簡を参照することはもちろんできないし,遺伝や生命の起源に関してダーウィンへ言及がなされているわけではないので,ダーウィンから自然発生説の着想をえたとは考えられない.これを踏まえて『四次元郷』を読み直すと,物質から生命が生まれるエピソードが語られる「機械の反乱(La Révolte des machines)」という章において,パヴロフスキーが自然発生説をどのように理解していたかを知ることができる.「機械の反乱」を読み直してみると,生命がどのような物質で構成されているのかを述べる際に1923年版では具体的なそれらの具体的な名前が挙げられており,「生命はもっぱら炭素,酸素,水素,硫黄,リン,貴金属といったある物質の物理化学的な属性に由来しているとして(la vie comme émanant uniquement des propriétés physico-chimiques de certains corps~: carbone, oxygène, hydrogène, soufre, phosphore et métaux catalyseurs)」(1923, 164)とある.この1節は,『ジュルナル・デバ(\emph{Journal Débat})』でサイエンスライターとして活躍したアンリ・クロスニエ・ド・ヴァリニィ(Henry Crosnier de Varigny)が執筆した1923年2月8日付のジュルナル・デバの科学記事「生命はどこから始まったのか? 脾臓抽出と結核(Où commence le vivant? L'extrait de rate et la tuberculose)」のある1節から引用されていると考えられる.その1節とは,「頑固にも,機械論者であるジャン・ナジョット氏は生命は「もっぱら,炭素,酸素,水素,硫黄,リン,とりわけ貴金属といった他の何らかものといったある物質の物理化学的な属性に基づいている(Inflexiblement mécaniste, M. J. Nageotte considère la vie comme reposant «~uniquement~» sur les propriétés physicochimiques de certqins corps~: carbone, oxygène, azote, hydrogéne, soufre, phosphore et quelques autres~: des métaux catalyseurs en particulier)
\footnote{Henry de Varigny, \emph{Journal Débat}, 8 fév 1923, p. 4.}」である.これらの文面はほとんど同一である.また,1923年版でこの1節がある段落では,細胞に寄生するウィルスであるバクテリオファージ(bactériphage)が最も原始的な細胞(la cellule la plus primitive)として紹介されている.ヴァリニィは記事で2つのトピックを紹介しており,その最初が単純な化合物で構成されているバクテリオファージに注目して,最小限の構成要素から生命は生まれえるかどうかを実験を紹介し,実験の示す意義は哲学的な議論に任せるというものだった\footnote{もう一方が,脾臓抽出による結核治療の話題である.}.この点からいっても,パヴロフスキーがこの記事を読むことで小説を加筆していたことは間違いないと断定できる.この事実から,2つのことが推論できる.まず,1923年版が加筆されていた時期である.少なくとも,1923年2月8日の時点までは加筆されていたことがこのことから明らかとなった.次に,パヴロフスキーが定期的にこの連載を読んでいたと考えられ,ヴァリニィの記事の影響を強く受けていたと考えられることである.とりわけ後者の可能性から初めて私たちはパヴロフスキーがどのように自然発生説に触れていたのかを理解することができる.

アンリ・ド・ヴァリニィがいかなる人物であったかはイヴ・カルトン(Yves Carton)による研究に詳しい\footnote{Yves Carton, \emph{Darwinien convaincu: médecin, chercheur et journaliste, 1855-1934}, Hermann, 2008.}.ハワイ領事だったシャルル・クロスニエ・ド・ヴァリニィ(Charles Crosnier de Varigny)の息子に生まれ,科学書の翻訳や大衆向けの科学記事を多く執筆した.その活動は主に1893年から記者となった『ジュルナル・デバ』であった.ある時期からダーウィンに関する著作を多く翻訳するようになり,『チャールズ・ダーウィン(\emph{Charles Darwin})』(1889)という概説書も執筆するほどのダーウィニストだった.ダーウィンに関係する論者のうち,とりわけ注目したいのは,T・H・ハクスリーの主要著作の1つである『生物学の問題(\emph{Les problèmes de la biologie})」(1892)を翻訳していることである.フランスにおける生命起源論争はステファンヌ・ティラール(Stépahne Tirard)によれば,物質から生命が生まれるという説は19世紀末になってドイツのダーウィニストの1人だったエルンスト・ヘッケル(Ernst Haeckel)やイギリスのハクスリーらによって展開されフランスにも翻訳を通じて紹介された.そして,ハクスリーを翻訳していたのが,ヴァリニィだったのである.ヘッケルもまた,とりわけ自然哲学(Naturphilosophie)を展開したシェリングやヘーゲルらドイツ観念論の影響が大きく\footnote{Bernhard, Kleeberg, \emph{Theophysis~: Ernst Haeckels Philosophie des Naturganzen}, Böhlau Verlag, Köln Weimar, 2005.},クーザンを通してドイツ観念論(フィヒテ,ヘーゲル)の自我論をパヴロフスキーは知っており(PT15),間接的な影響がなかったとは断言できない.そこで,ドイツにおける自然発生説の議論を簡単に見ておく.

ドイツの自然発生説は,ドイツ観念論がその背景にあり,とりわけフリードリヒ・シェリングの著作はドイツの科学者たちの思想的背景を形成していた.シェリングは,1797年に『自然哲学の理念(\emph{Ideen zu einer Philosophie der Natur})』,1799年には『自然哲学の体系の企図への序論(\emph{Erster Entwurf eines Systems der Naturphilosophie})』が発表され,ドイツ科学界の哲学背景に大きな影響を与えた.とりわけ,宇宙を1つの有機的な体制と捉える世界観によって,生物と無生物が連続していると科学者たちの多くが考えるようになった.例えば,パヴロフスキーが『四次元郷』でも言及しているヘルムホルツは,「進化論推進者であり,生物と無生物物質が連続し,同一の法則により支配される一元論的な世界観を強力に唱えた\footnote{ヘンリー・ハリス『物質から生命へ --- 自然発生説論争 』長野敬・太田英彦訳,95頁.}」.ヘッケルは,ティラールによれば,「生命の出現の概念を一元論哲学に一致させて,無機物と有機物の結びつきが存在しているという着想を彼は得ていた(Accordant sa conception de l'apparition du vivant avec sa philosophie moniste, il conçoit qu'il existe une unité de la matière inorganique et organique)\footnote{Tirard, \emph{op. cit. }, p. 116.}」.ヘッケル自身も,「有機的生命はこの最初のモネラ〔原始海洋にタンパクの質の塊のこと.ヘッケルは単細胞無核の生物体であるとしてその存在を想定した.〕,そして遺伝という固有の機能とともに始まる(Avec cette première monère commence la vie organique et sa fonction propre, l'hérédité)\footnote{Ernst Haeckel, \emph{Le Monisme Lien entre la Religion et la Science, Profession de Foi d'un Naturaliste}, Pari, Schleicher frères, 1897, p. 11.}」と述べている.ここからわかるようにヘッケルは,物質から生命が誕生すると考えていたうえに,生命の固有性を遺伝に見ていたことがわかる.では,パヴロフスキーの参照していた可能性が極めて高いハクスリーの自然発生説はどのようなものだったのだろうか.

ハクスリーは自然発生説論者を自称したことはないものの,『生物学の問題』に所収されている論考「生命の物理学的な基礎(Les bases physiques de la vie)」の中で,原形質(protoplasmique)についての考察を通じて,それが生命の「形式的な基礎(la base formelle)」であると考えた.この原形質と生命の起源をめぐる論考は,生命の出現が化学的な合成にあるということを問題にしていた点で同世代の生命起源論と一線を画していた.例えば,同世代の生命起源論は,すでに触れたスペンサーも関わっている.スペンサーは,自然発生を否定していた.しかし,『生物学の原理』で「原初の世界では,今日の実験室と同じように,有機的物体の下位の種類のものが,適合した条件下で互いに作用しあい,有機的物体のより高位の種類のものを生み出し,そして原形質を用意するに至ったのだ(Dans le monde primitif, comme dans le laboratoire d'aujourd'hui, les types inférieurs de substances organiques, en agissant les uns sur les autres sous des conditions appropriées, ont produit par évolution les types supérieurs de substances organiques, aboutissant à un protoplasme organisable )\footnote{Herbert Spencer, \emph{Principes de biologie}, Paris, Félix Alcan, 1893, p. 585.} 」と結論づけている.これは,ハクスリーに比べると「有機体の下位の種類のもの」が何か明らかではないように,具体的に原形質について議論はされていない.パヴロフスキーがヴァリニィの記事から引用することで化学的な統合が生命の起源に関わっていることを示してる点は,『四次元郷』では直接名前の挙げられているスペンサーよりも,こうしたハクスリーの議論が背景にあるのである.

ところで,パヴロフスキーはステファン・ルデュクの自然発生説に言及していた.ルデュクはアカデミーから学説を否定されていたが,ハクスリーらの議論はアカデミーからどのように扱われていたのかを確認しておこう.

19世紀半ば,ルイ・パストゥールがフェリックス・アルシメード・プーシェ(Félix Archimède Pouchet)の自然発生説を証明したとする一連の実験を論難して以来,科学アカデミーで自然発生説は科学的な学説として認められていなかった.コンリーは自然発生とダーウィニズムの関係を『ダーウィニズムと自然発生説(\emph{La darwinisme et les générations spontanées})』で論じたダリウス=C・ロッシ(Darius-C. Rossi)がプーシェにダーウィンの特に遺伝に関する考えが自然発生を証明するに足る十分な根拠になると手紙を送っているように,論駁されたプーシェはダーウィニズム的進化論が自然発生説を裏付けるはずだと考えて研究をしていた.ヘンリー・ハリスはこの論争に言及している科学史研究\footnote{以下の文献を参照のこと.Gerald. L. Geison, \emph{The private science of Louis pasteur}, Princeton University Press, 1995. Bruno Latour, \emph{Les Microbes~: guerre et paix, suivi de irréductions}, Paris, A.M. Métailié, 1984. Georges Pennetier, \emph{Un débat scientifique, Pouchet et Pasteur (1858-1868)}, Paris, J. Girieud, 1907.}を簡単にまとめたうえで,それらの評価が「関係者の発表方法の巧拙,あるいは学士院会員の公平性またはその欠如に注意を向けていることが多\footnote{ハリス,\emph{op. cit. }, 149頁.}」いと述べ,プーシェの実験がパストゥールに比べると非常に杜撰であったことがハリス自身の科学者としての技術的な視点から明示されている.自然発生説に関する議論が極めて思弁的であったためにそれを証明する手段が限られているうえに,当時の実験の精度では自然発生が起きていることを証明するという証拠を否定することの方が簡単であった.フランスのアカデミックな言説において否定されていた自然発生説は,ヴァリニィなどによって大衆化された科学の文脈の中でよってパヴロフスキーの物語に大きな影響を与えているのである.

パヴロフスキーが生物のモチーフをどのように描いているかに注目することで,進化論と遺伝が『四次元郷』の1つのテーマをなしていることが明らかとなった.すでに「テツノミ」が「機械の反乱」へ繋がることが『四次元郷』の中で言及されていることを取り上げたが,以下では,「機械の反乱」の分析を通じて,遺伝という概念が『四次元郷』の中で4次元と関わっていることを明らかにしたい.

\section{遺伝的類似}

1912年版27章1923年版29章「機械の反乱(La révolté des machines)」は第一科学時代の閏3年(le 3 intercalaire)にある工場で起きた機械の暴走についての出来事である.以下で,簡単にあらすじを述べる.現場主任H・G・28が電気設備が急停止しているのに気づき,調べさせると,設備の運転が停止しているものの,電気自体は流れていることが判明した.さらには,化合物の塩が銅板を伝って部屋の扉の前に蓄積したり,外側からの力がかかっていないのにもかかわらず主要部分が壊れ,制御装置が捻じ曲がっているといった現象が起きていた.あるエンジニアはこの出来事について命を持った金属という「何らかの奇妙な生命(de quelle vie étrange)」(173/164)が原因ではないかと考えた.工場の中ではこの異変が続き,金属の分子レベルの変異によって鉄鋼が銅に変わり,機械の動作に異常が生じた.そこで,工場をヨードホルムの蒸気で満たして,クロロホルムを浸した詰め物で塞ぎ,沈静化を図った.翌年の閏4年,不注意から電圧をかけすぎたところ,機械が捻じ曲がり,動き始め,球体を形成し,工場の扉へと向かっていった.工場の隣には,悪化した器官を取り替えるために,移植用の臓器が保管されている倉庫があった.球体となった機械は電気を帯びながらその倉庫に向かっていき,臓器は散乱した.暴走する機械を止めるため,人々は球体を人工的に凍らせて艀に乗せ,冷たい海に沈めることで解決した.

この章で重要なのは,機械の暴走の原因が金属にも生命が宿るということをエンジニアの考えを明らかにするときに示している点である.1912年版では次にように説明されている.
\begin{quote}
しかし,動植物の生命に類似した正真正銘の生命を物質に認めつつあることさえ決してなく,そして,不安を抱えながら,新しくも悩ましい発見がこの主題についてなされることがこれからもないのかどうか自問した.実際,地球の形成以来,生命を構成したものはなく,天空から私たちにやってきたこともないのを認めるべきだ.最初は,大地はただのガス状の塊であり,大地の物質が混交していた.それが動植物が生まれた原初的物質であり,私たちは知っている生命はミネラルの中にすでに存在したと十分に考えさせる.最近では,完全なものとなった機械についてなされた興味深い観察によってその確証はますます強固なものとされた.金属,とりわけ細工されたもの,それらを作るために使われ,強化されて,化学物質の数を増大させたものは,新しい真なる有機体の一種となり,思いがけないところまでそうした現象を起こすことができるようになった.電流の終わりのない伝達やヘルツ波の衝撃がさらに,もっと興味深い質的な超現代的な金属を与えた.ある場合には,正真正銘,機械にも随意の病気,そして,かつて労働者階級を殲滅したものと同じ害悪な何かが観察された.
\end{quote}
\begin{quote}
Cependant, on n'avait jamais été jusqu'à attribuer à la matière une vie véritable analogue à la vie des plantes et des animaux, et l'on se demandait avec angoisse si de nouvelles et inquiétantes découvertes n'allaient pas être faites à ce sujet. Il fallait bien reconnaître, en effet, que depuis la formation du globe, rien de ce qui constituait la vie ne pouvait nous venir du ciel. Au début, la terre n'était qu'une masse gazeuse, puis de la terre matière en fusion~; c'est de cette matière primitive que sont sortis les plantes et les animaux, et cela donne à penser suffisamment que la vie telle que nous la connaissons préexistait dans les minéraux. Ces constatations faciles avaient été renforcées, dans les derniers temps, par de curieuses observations faites sur des machines perfectionnées. Les métaux, particulièrement travaillés, que l'on employait pour leur construction, renforcés, doublés de nombreuses matières chimiques, étaient devenus des sortes d'organismes véritablement nouveaux, capables d'engendrer des phénomènes jusque-là imprévus. La perpétuelle transmission de courants électriques et le choc d'ondes hertziennes avaient pourvu ces métaux ultra-modernes de qualités plus curieuses encore. On avait même observé, dans certains cas, de véritables maladies volontaires se produisant dans les machines, quelque chose comme des vices, identiques à ceux qui décimaient jadis la classe ouvrière. (1912, 174-5)
\end{quote}
金属は生命を持ち,それは生命が金属と同じく物質に由来しており,そのために物質にも生命が宿るのである.この考えは,先に見たように,ヴァリニィの背景となっているハクスリーの考えに由来していると考えられる.生命がどのような化合物から成立しているかを付け加えた1923年版では,パヴロフスキーはこの一節を一部書き加えて,以下のように表現している.
\begin{quote}
最も原始的な細胞はすでに1個の複雑な組織である.バクテリオファージの中に,細胞のさらに下の,微生物の本当の意味での寄生,つまり生きているがさらに原始的な存在があると考えられているのは,その影響が微生物の遺伝的特徴を変更する能力があるからだ\footnote{1900年代にはデオキシリボ核酸(いわゆるDNA)が未発見であり,遺伝要素を何が伝達しているのか知られていなかったので,微生物がその役割を担っていると考えられていた.}.しかし,あたかも生命はもっぱら炭素,酸素,水素,硫黄,リン,貴金属といったある物質の物理化学的な属性に由来しているとして考えられているとしても,基本的原子の構成に他ならないものの中にまで,常にさらに遠くの起源や,惑星の運動の変化がエネルギーを生み出したり吸収したりするができる真なる無限小の宇宙,そして,中心核の周囲を回っている物質の電子の数の違いのほかに物質のうちに別の違いを認識しない摩訶不思議な錬金術師たちを追求しないほうがよい.生命は,すでに運動や行為の中で何かする力を持ってはいないが,水や風の渦によって運ばれていく不活の物質でないと,言うことができるのだろうか.もしも太陽系全体が原子の世界の荘厳な模倣に過ぎないのであれば,私たちの精神にとって,詩人が私たちに自然から与える描写の心に染み入る魅力を生み出すこと,それが,何世紀にも渡って,雲,海,森の運動,そして揺れ動く多様な私たちの思考の運動を結びつける曖昧な類似であるのは明らかではないのだろうか.
\end{quote}
\begin{quote}
La cellule la plus primitive est déjà un édifice complex. Au-dessus d'elle on a cru voir dans le bactériophage, véritable parasite du microbe, un être plus primitif encore mais vivant, puisque son influence suffit à modifier les caractères héréditaires des microbes. Mais si l'on considère la vie comme émanant uniquement des propriétés physico-chimiques de certains corps~: carbone, oxygène, hydrogène, soufre, phosphore et métaux catalyseurs, ne doit-on pas en rechercher les origines toujours plus loin, jusque dans la constitution même de l'atome élémentaire, ce véritable univers infiniment petit, dont les modifications de mouvements planétaires suffisent à créer ou absorber de l'énergie, et qui, merveilleux alchimiste, ne connaît d'autres différences entre les corps que celles du nombre de ses électrons gravitant autour d'un noyau central. La vie, mais n'est-elle pas déjà en puissance dans les mouvements, dans les gestes, pourrait-on dire, de la matière inerte entraînée par les remous de l'eau ou du vent? Et si l'on peut penser que tout le système solaire n'est que une imitation grandiose du monde atomique, n'est-il pas évident que ce qui fait, pour notre esprit, le charme pénétrant des descriptions que les poètes nous donnent de la nature, c'est l'obscure parenté qui unit, au travers des siècles, les mouvement des nuages, des mers ou des forêts et ceux de notre pensée ondoyante et diverse? (1923, 164-5)
\end{quote}

パヴロフスキーは最後の反語表現で,物質と生命が連続的な存在であることを示唆している.これを自然発生説に由来する生物学の知識を得ていたと考えられるパヴロフスキーが記述していること自体は驚きはないが,真に注目すべきは生命を持つ限り産業植物を例とした遺伝的異常も見られるという1912年版の考えから一歩進んで,「雲,海,森の運動,そして揺れ動く多様な私たちの思考の運動を結びつける曖昧な類似」とあるように,生命の起源である物質と思考の類似(parenté)があると1923年版で述べているのは注目すべきことである.

\emph{parenté}という語は,「もう一方の人々が子孫にあたるような人々の関係(Rapport entre personnes descendant les unes des autres)\footnote{\emph{Le Nouveau Petit Robert de la langue française}, 2008, Électronique.より.}」とあるように,血縁の類縁性を示す語である.これは,遺伝を1つのテーマとしている『四次元郷』において,重要な表現である.なぜなら,\emph{parenté}の示す血縁は,『四次元郷』では遺伝の文脈に置くことができるからである.遺伝現象の結果としての類似が,思考と物質の類似であるとここでは述べられている.これを思考と物質が\bou{遺伝的類似}の関係にあると言い換えてみよう.前章で見たように,パヴロフスキーは知性を宇宙論的に基礎付けており,3次元から4次元には精神によって到達することができた.そして,パヴロフスキーは別の箇所で思考と精神の関係について「精神だけが4次元に到達することができて,思考や行為は物質的で,つかのまの,現実には存在しない現れによる翻訳である(l'esprit seul peut atteindre dans la quatrième dimension et dont les pensées et les gestes se traduisent par des apparences matérielle, fugitives et irréelles)」(1923, 112)と述べている.思考が物質的なものであるということは,遺伝的類似の網目の中で,生命と物質の連続性という自然発生説から導かれる.3次元における物質・生命・思考の関係は遺伝的類似なのだ.また,この遺伝的類似は思考以外にも拡大していく.

1923年版24章の「記憶の拡張(L’Agrandissement des souvenir)」では,脳の研究が進んだことで,脳内には個人的な記憶だけでなくて,先祖の記憶も潜在していることが発見された(1923, 147).その後,手術を施して歴史の記憶を巡る人を「下意識の旅人(voyageurs du subconscient)」と呼ぶようになった.この旅は流行したのだが,その中で事件が起きる.その事件の被害者は「旅人」の一人のナトリウム(Sodium)という科学者だった.この科学者の先祖はカリベルト1世で,彼は6世紀のメロヴィング朝の王だった.この王は2人の侍女と内縁関係を結んでいたと伝えられている.その王の下意識に入り込んでナトリウムはまた次女への感情を共有していまい,二度目の旅をしたところ,その2人の侍女と離れることに絶望を覚えて自殺してしまった.先祖の感情にとらわれてしまうナトリウムの姿は本章で明らかにした遺伝のテーマを意識した表現に直せば,感情の先祖返りとも言える.「記憶の拡張」の章が示しているように,パヴロフスキーが背景としている進化論の言説には,奇形の発現が先祖返りの一種であり,それこそ進化の確証と言えるのではないかというダーウィンの考えもまた表れていると言える.このように,遺伝的な関係をモチーフにして,パヴロフスキーは意識と記憶の問題を取り扱っており,記憶という精神の働きに関わる領域を科学的に分析する「心理学は私たちに世界の\bou{物理的}現実に手を届かせるのみである.すなわち,創造と生とに(la \emph{psychologie} nous permet seule d'atteindre la réalité \emph{physique} du monde~: la création et la vie)」(1923, 252)とパヴロフスキーは述べている.創造と生が広がる3次元的世界を把握するために心理学が求められるというこの評価は,遺伝的類似によって思考や記憶が物質と関係していることで初めて意味をなす.パヴロフスキーの4次元概念の参照元であったベルクソンが『物質と記憶』(1896)や『精神のエネルギー』(1919)で重要な業績として心理学者ピエール・ジャネの業績に対して肯定的に述べていることも心理学への関心の意味を説明している\footnote{Henri Bergson, \emph{L'énergie spirituelle}, éd. Frédric Worms, Paris, PUF, 2009, p. 113, p. 115, p. 122. Henri Bergson, \emph{Matière et mémoire}, éd. Frédric Worms, Paris, PUF, 2008, p. 8, p. 113, pp. 195-6. }.なぜなら,前章で見たように,パヴロフスキーが,人間の精神は4次元に至ることができると考えているように,思考や記憶といった心理学的な対象の分析もまた,4次元に至るための手段たりうるのである.

以上のように,遺伝的類似は,いわば,進化論における種の連続性とその変化を,物質と思考にまで延長している.これは私が第4章で示した4次元に由来する3次元の物質の現象と同じ図式を持っており,さらに,精神が物質に作用しているということをふまえると,物質から思考までの連続性を持っていることと,3次元と4次元の連続性はパラレルであると考えられる.このことから,以下のように結論づけられる.パヴロフスキーの4次元は進化論における遺伝の理論と非ユークリッド幾何学の2つを独自の解釈によって組み合わせることで成立しているのである.

\chapter{結論}
私は,『四次元郷』について主に3つの観点から考察した.初めに,出版やジャーナリズムを中心とした歴史的観点から『四次元郷』が連載小説という形式をとった理由と,連載によって一種の科学記事のような側面を持っていたことを指摘した.それと同時に,科学小説の成立の中で特異な位置にあったことを示した.2つ目の観点は『四次元郷』の4次元が他の作品と比べてどのような特徴があるかを調べ,ヒントンらとは違ってそこに時間を認めることはなく,運動も存在しないことを確かめた.そして,ベルクソンが4次元によって規定されている3次元の世界を解明する論者として扱われていることから,「未来しか存在しない」というテーゼを取り出すことで,現代のベルクソニアンが示す芸術論を参考に,その応用可能性を指摘した.3つ目に,『四次元郷』で3次元は果たして4次元に対してどのような世界を形成しているのかを考察するために,物質と遺伝の2つテーマからエピソードを読み解き,3次元と4次元を結びつけている観念的な偶然性と遺伝的類似という概念にそれぞれまとめた. 観念的な偶然性とは,4次元にあるただ1つの原子の模像である人間が社会的な関係に依存しない形で統一される時に働く偶然である.3次元の観点からすると,それがいつ起きるか分からない点で全くの偶然と言える.次に,遺伝的類似は,自然発生説に由来した物質と生命の連続性が,思考や記憶にまで延長していることを意味する概念であった.よって遺伝的類似は,物質から思考への連続性を示している.

この2つの概念が興味深いのは,パヴロフスキーの支離滅裂とも思えるエピソードが,現実の世界で生じている現象すべてを4次元という経験的でない理念的な領域にある原理によってあたかも基礎づけているようであるからだ.ところで,その領域と3次元を生きる人間の関係は『四次元郷』の第1章から示されている.
\begin{quote}
精神は事物の普遍性と一体になっているにほかならない.その考えは現実的で,反発はありえない.\bou{静かなる魂}は世界の雑音をもはや気にも留めない.
\end{quote}
\begin{quote}
L'esprit ne fait plus que'un avec l'universalité des choses~; ses idée sont toutes positives, sans réaction possible. \emph{L'âme silencieuse} ne s'inquiète plus des bruits du monde.(8/59)
\end{quote}
「精神が事物の普遍性と一体になっている」というのは,遺伝的類似で示したように,物質と思考の連続性のことである.そして,「静かなる魂」とは4次元に到達した精神のことを意味していると同時に,「世界の雑音」,すなわち3次元の現象から切り離されている場所こそ4次元であるということを示唆しているのだ.3次元と4次元が連続でありながらも切断されている状態は,第4章で私が取り上げた非ユークリッド幾何学的対象が実在しているのか否かの問題に発展したことを想起させる.ところで,非ユークリッド幾何学の歴史を紹介する中で触れたリーマンが生み出した多様体という概念は,パヴロフスキーの4次元が現実に関わりつつも経験的でないという存在の身分を持っているのと非常に似通っている.そして,多様体の示した存在の身分の問題は,数学史の転換点となった.加藤文元のリーマンの多様体の解釈に依拠して\footnote{特に下記を参照のこと.加藤,前掲書,第7章.},具体的にリマーンの多様体とパヴロフスキーの4次元の関連性を見てみたい.

加藤の見立てでは,リーマンの多様体が登場した後の数学とそれ以前の数学の大きな違いは,経験的な対象を扱っているか,現実には存在していないものを対象として扱っているか,という点にある.リーマンの多様体を代表とした非ユークリッド幾何学的な図形は,現実には存在しないにもかかわらず計量することができる幾何学的な対象である.具体的な例として有名なクラインの壺を取り上げてみよう.クラインの壺を私たちの世界で図示する場合,壺から伸びている管が自分自身の中に入り混む時に交差してしまうため,交差地点が点線で示されていることがよくある.しかし,クラインの壺は4次元において滑らかな曲線であり,自身と交差することは本来ありえない.ちょうど,パヴロフスキーのインド風の箱が3次元では結ばれているのに,4次元でそれが解けてしまうのと同じである.では,クラインの壺の本当の姿について私たちは何も知ることができないのかと言えばそうではない.多次元空間の幾何学的対象に挑む数学者たちは,数式の操作によって部分的に対象の存在を知ることができ,それを繰り返すことによって自分が格闘している対象の全体像が把握されるようになる.加藤はこうした対象が存在している領域を「叡知的存在領野」と名付けている.私は序章でベクトルのn次元空間を取り上げたが,集合の要素の組み合わせによって無限の次元が展開されるこの空間もまた,叡知的存在領野に存在している.そして,叡知的存在領野に存在し,私たちの世界から独立して存在していながらにして,様々な分析によってその姿を知ることができるのである\footnote{2012年8月30日に望月新一が発表した宇宙際タイヒミュラー理論(Inter-universal Teichmüller Theory)もまた,この叡知的存在領野を様々なパースペクティブの積み重ねによって検証したものである.乗法と加法の関連性という極めて抽象度の高い課題に対して,宇宙際タイヒミュラーという空間を新たに構築することで検証する極めて高度な数学的手法は,私たちにこの叡知的存在領野の驚異的な側面を教えている.}.リーマン以降,現代数学は叡知的存在領野の抽象的な概念を分析していくことで,対象を具体化していくための様々な手法が開発されるようになった.加藤は三宅岳史の見解\footnote{三宅岳史,「リーマンと心理学,そして哲学」,『現代思想』,44巻,6号,青土社,2016年,161-175頁.}をふまえ,抽象的な対象の具体化のプロセス自体は経験的な観察なのであり,それは再び経験論に立ち戻ることなのだと述べている.そして,「ベルクソンによる「純粋持続」(中略)などのように,これら叡知的存在領野と人間との間に生じた相互依存と確執という新しい問題系を手に入れたように思われる\footnote{加藤,前掲書,174頁.}」というのだ.私は,パヴロフスキーの4次元が経験的でない理念的な存在であると述べたが,ベルクソンの経験論を知っていたパヴロフスキーが自然に似たような議論を構築していたとも考えられる.

パヴロフスキーの4次元はベルクソンの影響を受けつつ,持続については3次元的なものだとして却け,過去・現在・未来の本質をめぐる議論もすれ違っていた.しかし,本論で見たように,物質のエネルギーとしての側面とそれを動かす観念的な偶然性や,物質から生命,そして思考の連続性を示している遺伝的類似はパヴロフスキーの4次元を具体化していくものであり,叡知的存在領野にある抽象的な対象を様々な方法で具体化しているのと同じようなプロセスを辿っている.まさに加藤が例示しているベルクソンの純粋持続のような意味での経験論である.この経験論の視座において物語を書いていたパヴロフスキーが単なるプラトニストではなかったことは次の言葉からもうかがえる.

\begin{quote}
4次元の感覚は人間に先んじて働いていて,それは3次元の世界で\bou{未来の感覚}と呼ばれているものである(中略)
\end{quote}
\begin{quote}
  Car le sens de la quatrième dimension marche en avant de l'homme, il est ce que l'on appellerait \emph{le sens du futur} dans un monde à trois dimensions (...)(1923, 243)
\end{quote}
叡知的存在は理念として理解するものではなくて,感覚するものであるというこの主張は,いまや非ユークリッド幾何学のパラダイムが生み出した「真の経験論」(ベルクソン)として読むことができるのを示している.

本論全体についての結論は以上として,今後の展望を簡単に述べたい.本論第3章で連載小説の観点からジャーナリズムや,それがもたらした科学の大衆化が『四次元郷』の背景となっていることを指摘した.しかし,この論点では作品の文体と記事の文体の比較による具体的な検証が求められる.また,文体以外にも,センセーショナリズムによって,事件の報道と虚構の物語が区別できなくなるような自体が生じることについて,ナラトロジーの観点から分析するべきだろう.例えば,ジュネットは『フィクションとディクション』に所収された「虚構的物語言説,事実的物語言説\footnote{ジュラール・ジェネット,「虚構的物語言説,事実的物語言説」,『フィクションとディクション』,和泉涼一・尾河直哉訳,水声社,2004年,55-75頁.}」で,具体例としてルポルタージュや新聞調査の派生ジャンルとしてノン・フィクション・ノヴェルを取り上げ,虚構的物語言説と事実的物語言説の相互作用が働いていると指摘している.この相互作用は19世紀末の三面記事とロマン主義文学の関係にも当てはめることができるだろうし,同じように科学記事と科学小説~---~あるいはアカデミーの報告と科学記事にも~---~に当てはめることができるだろう.こうした分析手法は『四次元郷』だけでなくパヴロフスキーの他の作品にも応用できると考えられる.それはガストン・ド・パヴロフスキーの作品群に新しい解釈をもたらすかもしれない.

\begin{comment}
「幼生」と原子を言い換えている1912年版第31章「悪霊祓い(La Conjuration des Larves)」1923年版第33章もその傍証となる.この章は「機械の反乱」と似た構成を持っている.
幼生はgerme.遺伝的類似と連続性.
幼生の隠喩,世界が1つの卵であること.
多様性の容器,4次元の秘密を知るための3次元の緻密な分析.4次元によって保証された3次元の豊かさが示されている.
3次元の豊かさはとりわけ芸術に隠されている.パヴロフスキーは遺伝的類似を示唆する1923年版において,le succès éternel des fictions littéraires qui symbolisent, sans y prendre garde, la Vie et l'univers; de là le sens, plus profond qu'on ne le croit, de cette grande comédie imaginée que l'on appelle la Comédie de la Vie 251と述べられる.宇宙の生を象徴として描くことのできる文学の虚構こそ,
1923,250で,人間の人格が愛を生み出している説明がある.愛は人格ごとの特徴を強調するものである.それは,les différences dont vit la Pensée uniqueなのである.唯一の思考を生きる差異
遺伝的類似はひとつながりの存在の階梯を想像することを用意するが,それに知性的存在,抽象的思考を含めることはできない.パヴロフスキーが真に独創的だったのは,科学を一旦は認識論的に還元し,私たちの共通の意識をもっているということを前提に,知性を宇宙論的に基礎づけることで,実は科学的な存在として私たちを見なしうるとしたところにある.すなわち,進化論的遺伝的宇宙は一旦は科学的思考にみなす=3次元化することで,私たちの知性を含めたあらゆる側面を遺伝的な類似に並べることを可能にする.そして,そうした類似の中に生じる様々な人格の差異を,これは存在者の差異と言い換えもできるだろうが,ただ一つの思考という多様性の中に回収するのである.
\end{comment}

\begin{center}
  {\Large 参考文献表}\\
\end{center}

\noindent {\small {\textgt 注記\\}}
{\small 出版社が不明なものに関しては,{\textbf Gallica}に従って,[s. n.]と表記した.}
\\
\\
{\large 外国語文献}\\
Abbott, Edwin Abbott, \emph{Flatland~: A Romance of Many Dimensions by Square}, London, Seeley \& Co. , 1884.\\
Barel-Moisan, Clauire, «~écrire pour instruire~», \emph{La Civilisation du Jounal, Histoire Culturelle et Littérature de la Présse Française au XIX\textsuperscript{e} Siècle}, éd. Dominique Kalifa et. al. , Paris, Nouveau édition, 2011, pp. 752-65.\\
Baudin, Henri, \emph{La science-fiction~: un univers en expansion}, Paris, Bordas, 1971.\\
Bergson, Henri, «~Le réel et le possible~», \emph{La pensée et le mouvant}, Paris, PUF, 1955.\\
--- , \emph{Matière et mémoire}, éd. Frédric Worms, Paris, PUF, 2008.\\
--- , \emph{L'énergie spirituelle}, éd. Frédric Worms, Paris, PUF, 2009\\
--- , \emph{Matière et Mémoire}, éd. Frédéric Worms, Paris,PUF, 2010\\
Bernard, Claude, \emph{La Science Epérimentale}, Paris, J. -B. Bailliere, 1878.\\
--- , \emph{Leçons de Physiologie Opératoire}, Paris, M. Duval, p. XIV..\\
--- , \emph{Leçons sur les phénomènes de la vie communs aux animaux et aux végétaus}, t. 1, Paris, J. -B. Baillière, 1879.\\
Berthelo, Marcelin, «~Remarques sur l'existence réelle d'une matière monoatomique, à la suite d'une communication de M. Villarceau~», \emph{Académie des Sciences. Compte rendus hebdomadaires des séances}, n. 82, 1876, pp. 1129-1130.\\
Blanqui, Auguste, \emph{L'éternité par les astres},G. Baillière, Paris, 1872.\\
Boissy, Gabriel, «~Gaston de Pawlowski premier rédacteur en chef de «~Comœdia~» est mort prèmaturément hier~», \emph{Comœdia}, 3 Fév 1933, pp. 1-2.
Bridenne, Jean-Jacques, \emph{La Littérature Français d'Iimagination Scientifique}, Paris, Gustave Arthur Dassonville, 1950.\\
Brotchie, Alastair, \emph{Alfred Jarry~: ein pataphysisches Leben}, Bern, Piet Meyer, 2014\\
Carton, Yves, \emph{Darwinien convaincu: médecin, chercheur et journaliste, 1855-1934}, Paris, Hermann, 2008.\\
Clair, Jean, \emph{Marcel Duchamp, ou, Le grand fictif~: essai de mythanalyse du Grand verre}, Paris, Galilée, 1975.\\
--- , «~Introduction~», \emph{Voyage au pays de la quatrième dimension}, Gaston de Pawlowski, Images modernes, 2004, pp. 11-27\\
Compère, Daniel, «~Fait divers et vulgarisation scientifique~», \emph{romantisme}, n. 97, Paris, Armand Colind, 1997, pp. 69-76.\\
Conry, Yvette, \emph{Introduction du Darwinisme en France au XIX\textsuperscript{e}siècle}, Paris, J. Vrin, 1974.\\
Curval, Philippe, «~Surréalisme et Science-Fiction~», \emph{Europe}, v. 79, n. 870, oct. 2001, pp. 32-50.\\
Dumasy-Queffélec, Lise, «~Le feuilleton~», \emph{La Civilisation du Jounal, Histoire Culturelle et Littérature de la Présse Française au XIX\textsuperscript{e} Siècle}, éd. Dominique Kalifa et. al. , Paris, Nouveau édition, 2011, pp. 925-936.\\
During, Élie, «Le souvenir du présent et la fausse reconnaissance», \emph{L'énergie Spirituelle}, dir. Frédéric Worms, Paris, PUF, pp. 307-13.\\
Flammarion, Camille, «~L'Eternité par les astres par A. Blanqui~», \emph{L'Opinion Nationale}, 25 mars 1872, p. 3.\\
Gomel, Elana, \emph{Narrative Space and Time~: representing impossible topologies in literature}, New York, Routledge, 2014.\\
Hannequin, Arthur, \emph{Essai critique sur l'hypothèse des atomes}, Paris, Allan, 1899.\\
Henderson, Linda Dalrymple, \emph{The fourth dimension and non-Euclidean geometry in modern art}, Massachusetts, MIT Press, 2013.\\
Herbert Spencer, \emph{The Principles of Biology, The Works of Herbert Spencer}, v. II, Osnabrück, Otto Zelleri, 1966.\\
Herp, Jacques Van, \emph{Panorama de la Science Fiction. Les Thème, Les Genres, Les écoles, Les Problèmes}, Verviers, Gérard \& Co, 1973.\\
Jouffret, Esprit Pascal, \emph{Traité de géométrie à quatre dimension}, Paris, Gauthier-Villars, 1903.\\
Kleeberg, Bernhard \emph{Theophysis~: Ernst Haeckels Philosophie des Naturganzen}, Böhlau Verlag, Köln Weimar, 2005.\\
Le bon, Gustave, \emph{L'évolution des Forces}, Flammarion, Paris, 1907.\\
Loison, Laurent, «~Le projet du néolamarckisme français (1880-1910)~», \emph{Revue d'histoire des sciences}, t. 65, janv. 2012, pp. 61-79.\\
Lucas, Prosper, \emph{Traité philosophique et physiologique de l'hérédité naturelle dans les états de santé et de maladie du système nerveux}, t. 1 et 2, Paris, J. -B. Baillière, 1847-50.\\
Marcandier-Colard, Christine, \emph{Crimes de sang et scènes capitales : essai sur l'esthétique romantique de la violence}, Paris, PUF, 1998.\\
Marrati, Paola, «~Le nouveau en train de se faire~», \emph{Revue internationale de philosophie}, n. 241, mars 2007, pp. 267-8.\\
McLean, Steven, \emph{The Early Fiction of H.G. Wells: Fantasies of Science}, London, Palgrave Macmillan, 2009.\\
Mollier, Jean-Yves, \emph{L'argent et les lettres, histoire du capitalisme d'édition, 1880-1920}, Fayard, Paris, 1988.\\
--- , «~écrivaint-édituer~: un face-à-face déroutant~», \emph{Travaux de littérature}, v. XV, t. 2, Paris, L'Adirel, 2002, pp. 17-39. \\
Morange Michel, «~Darwin dans L'histoire de la Pensée~», \emph{Transversalités}, n. 114, fév. 2010, pp. 111-7.\\
Naudin, Charles, «~Considérations philosophiques sur l'espèce et la variété~», \emph{Revue horticole}, v. 4, n. 1, 1852, pp. 102-9.\\
Pawlowski, Gaston de, \emph{Sociologie Nationale. Une définition de l'état}, Paris, V. Giard \& E. Brière, 1897.\\
--- , \emph{Philosophie du Travail}, V. Giard \& E. Brière, 1901.\\
--- , \emph{Voyage au pays de la quatrième dimension}, Paris, Bibliothèque-Charpentier, 1912. (https:\slash\slash archive.org\slash details\slash voyageaupaysdela00pawl)\\
--- , «~La Semaine Littéraire~», \emph{Comœdia}, 11 janvier 1914, p. 3.\\
--- , «~Où allons-nous?~», \emph{Les Annales politiques et littéraires~: revue populaire paraissant le dimanche}, dir. Adolphe Brisson, 6 Déc 1925, Paris, [s. n.], pp. 581-2.\\
--- , \emph{Voyage au pays de la quatrième dimension}, Paris, Denoël, 1971.\\
Pierre Kropotkine, \emph{L'Entr'aide, un facteur d'évolution}, Paris, Hachette, 1906.\\
Poincaré, Henri, \emph{La Science et l'Hypothèse}, Paris, Ernst Flammarion, 1902.\\
Proust, Marcel, \emph{Correspondance de Marcel Proust}, éd. Philip Kolb, t. 3, Paris, Plon, 1976.\\
Príncipe, João, «~La physique laplacienne dans la seconde moitié du XIXe siècle: Joseph Boussinesq – la pratique et la réflexion autour de l’atomisme en France vers 1875~», \emph{Kairos Journal of Philosophy and Science}, vol. 13, 2015, pp. 179-212.\\
Régnier, Pilippe, « Place, fonctions et formes de l'ecriture utopique chez Fourier », \emph{Pamphlet, utopie, manifeste, XIX\textsuperscript{e}-XX\textsuperscript{e} siècles}, textes réunis par Lise Dumasy et Chantal Massol, L'Harmattan, 2001, pp. 385-401.\\
Stableford, Brian, "Introduction", \emph{Journey to the Land of the Fourth Dimension},  London, Black Coat, 2011, Electronic.\\
Suzamel (Blanqui), «~Le Père Gratry. Science et foi. (3\textsuperscript{e} article)~», \emph{Candide. Journal à Cinq centimes}, 1\textsuperscript{ème} année, n. 8, 27 mai 1865, p. 1.\\
Taine, Hippolyte, \emph{De l'Intelligence}, Paris, Hachette, 1870.\\
Tarde, Gabriel de, \emph{Fragment d'histoire future}, Paris, V. Giard \& E. Brière, 1896.\\
Thénty, Marie-Ève, \emph{La Littérature au quotidien. Poétiques Journalstique au XX\textsuperscript{e} siècle}, Paris, Seuil, 2007.\\
Tirard, Stéphane, «~L’histoire du commencement de la vie à la fin du XIX\textsuperscript{e} siècle~», \emph{Cahiers François Viète}, Série I, n. 9-10, 2005, pp. 105-18.\\
--- , «~Origin of Life and Definition of Life, from Buffon to Oparin~», \emph{Origins of Life and Evolution of Biospheres}, n. 40, 2010, pp. 215-20.\\
Troy, Nancy J. , "Staging Haute Couture in Early 20th-Century France", \emph{Theatre Journal}, v. 53, n. 1, 2001, pp. 1-32.\\
Varigny, Henry de, \emph{Journal Débat}, 8 fév 1923, p. 4.\\%記事名!
Versins, Pierre, \emph{Encyclopédie de l'utopie des voyages extraordinaires et de la science fiction}, Lausanne, L'Age D'Homme, 1972.\\
Vincent, Berdoulay et Olivier, Soubeyran, «~Lamarck, Darwin et Vidal~: aux fondements naturalistes de la géographie~», \emph{Annales de Géographie} , t. 100, n. 561-562, 1991, pp. 617-634.\\
Walbecq, Eric, «~Gaston de Pawlowski~», \emph{Le Visage Vert}, n. 4, Paris, Joëlle Losfeld, 1998, pp. 148-52.\\
Walter, Benjamin, \emph{Gesammelte Schriften}, v. I, éd. Rolf Tiedemann, Frankfurt, Suhrkamp, 1991.\\
\emph{Comœdia}, 3. Fév. 1933, pp. 1-2.\\
\emph{Histoire générale de la presse française}, dir. Claude Béllanger, et al. ,t. 3, Paris, Presses universitaires de France\\
\emph{Le Nouveau Petit Robert de la langue française}, 2008, Électronique.\\
\foreignlanguage{russian}{Ибаньес, Рауль, \emph{Четвертое измерение~: Является ли наш мир тенью другой Вселенной?~}, Москва, Де Агостини}, 2014.\\
\\
{\large 日本語文献}\\
石橋正孝『〈驚異の旅〉または出版をめぐる冒険:ジュール・ヴェルヌとピエール=ジュール・エッツェル』,左右社,2013年.\\
カーオ,ヘリガ『20世紀物理学史:理論・実験・社会 上』,有賀暢廸・稲葉肇他訳,名古屋大学出版会,2015年.\\
加藤文元『リーマンの数学と思想』,共立出版,2017年.\\
クロウ,マイケル・J『地球外生命論争1750-1900』鼓澄治・山本啓二・吉田修訳,工作者,2001年.\\
コンペール,ダニエル『大衆小説』,宮川朗子訳,国文社,2014年.\\
シヴァリエ,ルイ『三面記事の栄光と悲惨 近代フランスの犯罪・文学・ジャーナリズム』,小倉孝誠・岑村傑訳,白水社,2005年.\\
ジェネット,ジュラール『フィクションとディクション』,和泉涼一・尾河直哉訳,水声社,2004年.\\
スーヴィン,ダルコ『SFの変容』,大橋洋一訳,国文社,1991年.\\
鈴木雅雄「星々は夢を見ない --- オーギュスト・ブランキに関する覚え書き」,早稲田大学大学院文学研究科紀要,第2分冊,53号,2007年,3-16頁.\\
デュリング,エリー「レトロ未来」,新村一宏訳,早稲田表象・メディア論学会,表象・メディア研究,第5号,2015年,1-29頁.\\
中沢新一「四次元の花嫁」,『東方的』,講談社,2012年,48-120頁.\\
ブランキ,オーギュスト『天体による永遠』浜本正文訳,雁思社,1985年.\\
ブロック,W・H『化学の歴史 II』,大野誠他訳,朝倉書店,2006年,284頁.\\
ベルクソン,アンリ『物質と記憶』,田島節夫訳,白水社,1965年,5頁.\\
ベンヤミン,ヴァルター『パサージュ論I』今村仁・三島憲一他訳,1993年.\\
ボドゥ,ジャック『SF文学』,新島進訳,白水社,2011年.\\
森川亮,「量子論の歴史 --- 未知なる放射線,その発見ラッシュの裏面史」,生駒経済論叢,第13巻,第2号,2015年,283-300頁.\\
三宅岳史,「リーマンと心理学,そして哲学」,『現代思想』,44巻,6号,青土社,2016年,161-175頁.\\
\\
{\large カンファレンス}\\
Patrizia D’Andrea, «~Voyage au pays du silence~: rêve, philosophie, utopie entre symbolisme et anticipation (Han Ryner, Gaston de Pawlowski)~», Mobilités dans les récits de fantastique \& de science­fiction (XIX~-~XXIe siècles)~: quête \& enquête(s), l'IUT Sénart-Fontainebleau, Université Paris Est Créteil, Jeudi 20 Nov.\\
Sophie Lucet, «~Gaston de Pawlowski et le journalisme ou l’art de l’échappée~», Comœdia (1907-1937). un continent inexploré dans l'histoire du théâtre, Bibliothèque Seebacher, Université Paris-Diderot, 21 juin 2014. \\

\end{document}


