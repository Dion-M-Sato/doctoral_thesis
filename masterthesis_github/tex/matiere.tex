\chapter{物質とエネルギー}
\section{物質について}

先行研究の終わりで示したように,本章では『四次元郷』で描かれる物質に関する描写を分析することによってパヴロフスキーが3次元の世界をどのように描いていたのかを分析する子.そのことによって,相対的にパヴロフスキーの4次元の特徴を明らかにすることができる.それでは,パヴロフスキーがどのように3次元について語っているか見てみよう.

彼は,第1章の始めで「他にもっとふさわしい言い方がないので私たちが4次元と呼んでいる量によっては測りしれない違いを,入れ物とその中身の間の,観念と物質の間の,芸術と科学の間の違いを,3次元で構築された数字や語では説明することができない(Et de cette différence non mesurable par des quantités, que faute de mieux nous appelons quatrième dimension, de cette différence entre le contenant et le contenu, entre l'idée et la matière, entre l'art et la science, ni les chiffres, ni les mots construits à trois dimensions ne peuvent rendre compte. )\footnote{1923年版では«quatrième dimension»は斜体.}」(2/2-3).パヴロフスキーによれば4次元は本来,3次元では表現できない翻訳不可能な世界とされていると述べている.その中でのこうした説明は,『四次元郷』における4次元の範例となっている.すなわち,私たちが生きている3次元の世界は,4次元が「入れ物」だとすると,「中身」にあたる.その入れ物は,「観念」と「芸術」とも言い換えることができる.そして,「観念」の中身には「物質」が,「芸術」の中身には「科学」が入っている.

これは『四次元郷』で語られる人類史と大きくかかわっている.2つの科学の時代を通じて,科学による物質の操作の技術が発展していき,最後に人類は4次元の世界に至る.4次元に至ることで知られるようになったのは,物質が観念という入れ物の中身であるので「物質は観念の指示に従ってその部分を変化させられ(la matière se modifiait d'après les indications de l'Idée)」(307/242)てきたことだった.この時,観念の入れ物にある物質は精神の働きかけをうける.ここで示されているのは,物質に関するもう1つの考えである.すなわち,観念は物質に直接働きかけるのではなくて,人間の精神を媒介にするのである.入れ物とその中身の関係は直接的ではなくて,その媒介が存在するのである.

精神と物質が関わるというこの考えは,1912年版第8章1923年版第8章「時間の原子の変異(La transmutation des atomes de temps)」で詳しく述べられている.「時間の原子の変異」は,未来をどのようにして知り,それをどのようにして書いたのかという時間の移動とその記述の方法が描かれている.つまり,4次元での移動とその経験をどのようにして3次元の言葉で語ることができたのかという4次元と3次元の関係が詳しく述べられている.その関係は主に,3次元と4次元の差異が主に運動の観点から説明されている.パヴロフスキーによれば,3次元における運動は,時間と場所の移動を伴うものであるが,4次元の世界はすべてが連続的につながっていて,連続的(continu)であるのでこうした移動は起きない.なぜなら,物質を構成しているとされている原子の「変異」が4次元における移動を表現しているからである.原子論において3次元における移動は原子が他の原子の位置と入れ替わっていくと説明できる.一方で,4次元においてはすべてが連続状態にあるので,「変異」しかしない.つまり,隣り合った原子の性質が変化することによって3次元における移動が表現されるのである\footnote{「隣り合った原子の間の質の変化によって移動が生じる«~Un déplacement se fait donc par un échange de qualités entre atomes voisins~» (46/79)}  .パヴロフスキーはこの説明をするために,海水面を進む船を例に持ち出す.3次元の観点からすれば,船の原子が水の原子を押して出して元の原子があった場所に収まることの繰り返しによって船が進んでいくように見える.一方で,これを4次元の観点からみると,船の原子は水の原子に変異していると表現できる.移動といった3次元で生じるあらゆる物質の運動は,4次元の観点からすると,原子の変異の連続の過程として説明することができる.ただし,原子は仮説にすぎないという点をパヴロフスキーは強調し,「原子は現実には存在していない(les atomes n'existent pas en réalité. 」(1912-48)と明言する.そして,物質が原子から構成されていると解釈する必要があるのは,「物質の特性や性質から物質を切り離す精神(l'esprit qui isole la matière avec tous ses attributs, avec toutes ses qualités.) 」(48/80) の作用によって4次元へと至ることができるからである. つまり,4次元に至ることができるために持ち出される原子論は,『四次元郷』の物語の帰結である人類の4次元への移行するうえで非常に重要だと考えられる.ところが,パヴロフスキーにとって原子論は仮説でしかない.精神が物質に介入するという先ほどの一節は,原子論は仮説であるものの,精神を媒介として観念と物質が関係を結んでいる場合には,原子は存在しているという奇妙な論理が立てられているのだ.これはどういうことなのだろうか.

原子論を仮説とみなしながらも,その存在を認めるという奇妙な思考法の萌芽は彼の博士論文である『労働の哲学』に見られる.社会と個人の関係を個人の主観的意識に主眼を向けることで隷属的な労働からの解放がいかにして可能なのかを考察したこの著作で,原子論が重要な概念となっている理由を,ヨーロッパ諸国の歴史に求めることができる.国民国家の出現し始めた近代において,神による人間の救済が個人との直接的な関係によるものであると信じられるにあたって個人の価値の称揚が促され,国家を構成する個人は原子のようであるという考えが広まっていった.17世紀にはガッサンディーがエピクロス派の復権と原子的宇宙論を語り,ライプニッツはモナドロジーの理論を構築することで個人と原子をアナロジカルにとらえた.こうした歴史認識は一定の説得力を持っている\footnote{中谷猛・足立幸男編『概説西洋政治思想史』,ミネルヴァ書房,1994年,82-86頁.}.さらに,パヴロフスキーが,個人によって構成される社会という図式が原子によって構成される物質と相似しているのに注目しているのはこの歴史的文脈の他にも哲学的な文脈がある\footnote{本論では触れないが,パヴロフスキーの原子論の需要と冒険譚には大きな関わりがある.彼は1924年に「偉大なる革命 シラノ復活!(Un grand révolté. Ressuscitons Cyrano!)」という文を執筆している(Gaston de Pawlowski, «Un grand révolté. Ressuscitons Cyrano!», \emph{Cyrano}, Paris, S.I. , 1924.).そして,シラノ・ド・ベルジュラックが扱っていたのはガッサンディ派の原子論であった.ガッサンディからライプニッツ,そしてベルクソンに至るまでの原子論の概略は下記を参照のこと.Marie Cariou, \emph{L'atomisme. trois essais : Gassendi, Leibniz, Bergson et Lucrèce}, Paris, A. Montaigne, 1978.}.当時,科学的にその実在が疑われていた原子は,世界に対する認識の問題としてとらえられていた.そして,私たちが存在している世界についての認識を左右しているのは科学を代表とする唯物論と形而上学を代表とする観念論の2つで,認識においてそれらは拮抗していると考えられていた.この拮抗は,『労働の哲学』の第4章「科学の限界」では,「私たちを占めている主体において,つまりは,社会の科学と個人の自由(l'étendue et la durée ou dans le sujet qui nous occupe : la science sociale et la liberté individuelle)」(PT184)の問題であると簡潔に表現されている.また,原子論は仮説であるものの,原子の存在が人格(personnalité)の存在様式を反映しているのではないか,というアルチュール・アヌカン(Arthur Hannequin)の学説がここでは同時に紹介される.パヴロフスキーが原子論でアヌカンの学説に立脚して人格と原子の相似形を語ることは,『四次元郷』における物質を考えるうえで極めて重要である.作品の中では,ブランキの宇宙を形成する原子の無限の反復といった考えが示されている.一方でフランスでは熱の運動をめぐる議論の中で,熱を伝えているものの正体として原子が再び注目されるようになる.しかし,ラプラースやアンペールらの力学における研究成果への評価が揺るぎなかった伝統的な科学者たちは原子の存在を基本的を認めようとはしなかった.それは以下のような科学アカデミーの記録にみることができる.

アウグスト・クント(August Kundt)とエミル・ヴァルブルク(Emil Warburg)は水銀蒸気の特定の熱量について,イギリスのクラジウスなどが新しく提唱していた気体運動理論に基づいた計算を行ったところ,計測結果と理論的数値が一致したという報告をした.それについてアカデミー会員のイヴォン・ヴィラルソー(Yvon Villarceau)が実験結果の重要性を指摘したところ,当時高名だったマルスラン・ベルトロ(Marcellin Berthelot)が「不可分である一方で広がったり連続したりしている原子という考え自体,\bou{また},\bou{固まりが与えられている一方で物質の点として還元される原子という考えも},それ自身において矛盾しているように思われる(La notion même d'atome indivisible, et cependant étendu et continu, \emph{aussi bien que celle d'un atome doué de masse et cependant réduit à un point matériel, }semble contradictoire en soi)\footnote{Marcelin Berthelot, «~Remarques sur l'existence réelle d'une matière monoatomique, à la suite d'une communication de M. Villarceau~», \emph{Académie des Sciences. Compte rendus hebdomadaires des séances}, t. 82, 1876, p. 1129. }」と述べた.ヴィラルソーはこの報告を受けて,ベルトロの観察不可能なものに対する実在を認めない実証主義的な態度を批判している\footnote{下記を参照のこと.João Príncipe,«~La physique laplacienne dans la seconde moitié du XIX\textsuperscript{e} siècle: Joseph Boussinesq---~la pratique et la réflexion autour de l’atomisme en France vers 1875~»,\emph{Kairos Journal of Philosophy and Science}, vol. 13, 2015, pp. 183-189.}.ベルトロを代表とするアカデミーの年長者たちの考えの影響は非常に大きく,アカデミーでの記事の後,1886年に文部大臣となったベルトロは,原子論は仮説に過ぎないので学校教育の場では教えるべきではないという法令を作成した\footnote{W・H・ブロック『化学の歴史 II』,大野誠他訳,朝倉書店,2006年,284頁.}.こうして,フランスでは原子論を科学的な概念として認めないような風潮が作られていった一方で,そのほかの国々では放射能現象や光電気をめぐる実験データの定性的な説明を可能とするジョセフ・ジョン・トムソンの原子モデルが支持を集めるなど,とりわけ化学者らを中心に原子の実在を前提とした学説が推し進められ,アーネスト・ラザフォードやニールス・ボーアなどが1900年代に実験を進めていき,最終的に電子の存在が次第に実証的に確認された\footnote{以下を参照のこと.ヘリガ・カーオ『20世紀物理学史:理論・実験・社会 上』,有賀暢廸・稲葉肇他訳,名古屋大学出版会,2015年,第4章.}.科学教育史家ニコル・ユラン(Nicole Hulin)の言葉を借りれば,1870年代から始まったフランスにおける「化学の活気(la vitalité de la chimie)\footnote{Nicole Hulin, «~Les doctorats dans les disciplines scientifiques au XIX\textsuperscript{e} siècle~», \emph{Revue d'histoire des sciences}, v. 43, n. 4, 1990, p. 425.} 」はこうして陰りを見せることとなった.

以上をふまえると,パヴロフスキーがアヌカンの『原子仮説についての批判的試論(\emph{Essai critique sur l'hypothèse des atomes})\footnote{Arthur Hannequin, \emph{Essai critique sur l'hypothèse des atomes}, Paris, Allan, 1899, p. 20.}』(1895)を参照していたのは(PT168),フランス特有の事情が原因だったと考えられる.ベルトロによる1886年の法令の制定された時にパヴロフスキーは12歳であり,アヌカンの著作に触れていた20代前半にフランスにおいて原子論を科学的に実証しうると考えることの方が困難だった.『労働の哲学』を執筆する過程で原子論を検討したパヴロフスキーにとって,ルクレティウスやエピキュロスといった古代の哲学からライプニッツのモナドロジーに至る存在論的な概念が原子だった\footnote{パヴロフスキーが1923年の時点でラザフォードを知っていたうえで原子の実在を認めなかったことがさらにこのことを確証している.(1923, 126)}.これらの時代背景に加えて,アヌカンはカントの批判哲学に基づいて原子論は仮説的存在でありながらも人間の認識の構成的原理となっているという学説を展開し,ある種の科学哲学を提案していた.パヴロフスキーはこのアヌカンの学説に依拠しており,仮説的にしか存在してないはずの原子がやはり存在しているという論理はここからそこに淵源があると考えられる.そこで,アヌカンの学説について以下で詳しく見てみよう.

アヌカンによれば,原子論は古代からなされている議論で科学者の間でも議論になっているのにもかかわらずその実在については確証が得られていないが,原子論が繰り返し提示されるのは,カントが『純粋理性批判』で述べているような意味で,人間の認識の構成的原理に要請されているからである.パヴロフスキーが『労働の哲学』で引用しているアヌカンは「原子というこの数学的な概念は,場所と時間において,ある相対的な存在だけを持っており,私たちを,おそらく,空間そして時間の上位で,絶えず生み出され自身を完成させていく存在の統一へと私たちを導き,決定論的な活動と創造するものの影をその持続に投じ,実現され,動けなくなり,すでに過ぎ去り,すでに死者として,結果の影をその延長に投じている.さらに,活動と,連続と原子はただの反射した姿でしかない真の統一体へと人が高められるのに従って,連続と連続とともにある原子は,モナドと精神に向かって,消え失せる(l'atome, ce concept mathématique, qui n'avait dans l'Espace et le Temps, qu'une existence relative nous, conduira peut-être, au-dessus de l'Espace et au-dessus du Temps, à l'unité d'un être qui sans cesse se fait et s'achève soi-même, en projetant dans la durée l'ombre de son action déterminante et créatrice, et dans l'étendue l'ombre des résultats réalisés, fixé, déjà passés et comme déjà morts. Ainsi s'évanouissent le continu et, avec lui, l'atome, à mesure qu'on s'élève vers les activités et les unités véritables, dont ils ne sont que le reflet, vers les monades et les esprits. )\footnote{Hannequin, \emph{op. cit. }, p. 21. }」と述べている.パヴロフスキーが注目しているのは,原子とは数学的な概念でしかなく実在してはいないという主張である.アヌカンは私たちが認識を構成するうえで必然的に要求されるのが原子論という仮説だと考えている\footnote{Hannequin,\emph{op. cit.} ,  p. 12.  アヌカンはカントの研究を1890年代の前半に行っている.カントは認識のあり方を「統制的」と「構成的」に分けているが,アヌカンは原子を認識そのものを形作る構成的原理として考えている.}.これを受けてパヴロフスキーは「時間の原子の変異」で,「精神は原子が自身のイメージでしかなく,原子から4次元の完全でただ1つの世界が作り上げられていて,それは,複数の鏡の中であるかのように,3次元の不完全な世界は,様々な様相の下でそのただ1つの原子が無限に反射しているという感覚の幻想なのである(L'esprit conçoit l'atome à son image, il en fait donc un monde complet et unique à quatre dimensions et c'est une illusion des sens qui reflète à l'infini comme dans des glaces multiples cet atome unique sous les aspects divers du monde incomplet à trois dimensions.)」(48/80)と言い換えている.原子とは精神自身の姿であり,その原子はただ1つで世界はその反射によって構成されている.これは世界は精神を通じてしか認識できないという考えがあってこそ初めて述べることができる.物語の冒頭に掲げられたこの世界観を最終的に人類は科学技術の発展によって手に入れるのだが,そのためには,精神による物質の操作を3次元において行う必要がある.つまり,物質を構成する原子自体を操作できることが求められている.もちろん,科学の時代において人類は原子の操作を最終的にものとするが,実はここで重要となるのが物質に干渉することで生じるエネルギーなのである.1912年第31章1923年第33章「悪魔祓い(La Conjuration des larves)」でのパヴロフスキーはエネルギーと物質がついにあることを示している.その具体的箇所を検討する前に,この章のあらすじを見ておこう.

人類は科学の時代にエネルギーを物質を分離することによって取り出そうとし失敗してしまった.そこで,物質の分離ではなくて物質を構成しているとされる原子を操作すれば,自由に物質やエネルギーを生み出すことができると考えた.そこで注目されたのは,すでに人類がその力や可能性を知っていて,組み合わせが容易な基本的原子だった.基本的原子は「幼生\footnote{ここで「幼生」と訳出した単語\emph{larve}はラテン語の\emph{larva},すなわち亡霊や仮面を意味する単語に由来している.「悪魔祓い」の章は,後者の意味で翻訳したが,作中では\emph{larve}を培養する描写があり,パヴロフスキーは地口を用いている.}」を呼ばれるようになった.科学者たちは,「幼生」を大中央研究所で培養してその数を増やそうとしたのだが,「幼生」は幽霊のように壁をすり抜けて研究所の外に出てしまう.その結果,記念碑などが生物のように動きはじめるなどといった様々な混乱が生じた.騒動が収まった後,人類は物質にも生命があるのではないか,という考えを抱くことになる.以上が章のあらすじである.

このエピソードで重要なのはエネルギーと物質が対になっているという考えがパヴロフスキーによって示されるところである.以下にその箇所を示す.
\begin{quote}
この基本的原子は,あらゆる単純な物質とあらゆる既知のエネルギーの父なのである.科学時代の続きの中で幼生と原子は名付けられ,最終的には原子を解放し,【エーテルの|星雲の】単純な粒子だけがあった世界の始まりに存在したような原始的な状態において合成することで原子を再構成するようになった.
\end{quote}
\begin{quote}
  Cet atome élémentaire, père de tous les corps simples et de toutes les énergies connues\{;|---\} cette \emph{larve}, comme on le surnomma dans la suite[=les chercheurs de la période scientifique]\{,|---\} on finit par le dégager, par le reconstituer par synthèse dans son état primitif, tel qu'il existe au début des mondes, lorsqu'il n'est encore qu'une simple particule de \{l'éther|la nébuleuse\}. (200/178-9)
\end{quote}
基本的原子を再構成することで人類は自由に物質を構成することができるようになったが,基本的原子は「エネルギーの父」でもあるということが示しているのは,物質とエネルギーは別の形でありながらも同じ根源的なものによって構成されているということである.では,『四次元郷』においてエネルギーはどのように描かれてきたのだろうか.

\section{エネルギーについて}

1912年版第20章(1923年版第19章)「引き裂かれた犬(Le Chien Dissocié)」は,パヴロフスキーがエネルギーをどのように考えていたかを知るうえで非常に重要な章である.あらすじを簡単に見ていこう.

科学の時代の重要な物語として「火星商業開発社会(Société d'Exploitation Commerciale de la Planète Mars)」(124/125)と後世の人々が呼ぶことになる事件があった.この章では,その顛末が語られている.火星人との交流を目指す人類が様々な試みを行っていく中で,ある日火星人からのメッセージを受信することに成功する.同じ頃,物質を分離することで得られるエネルギーによって商業的利益を得ようとしていたので,火星人にその方法を尋ねると,通信者の食べようとしてた仔牛の肉がいきなり焦げてしまうという火星人側からの実演がなされる.ところが,その後も物質の分離が続き,ついに通信所の守衛の犬にもその被害が及ぶ.狂ったように吠え出した犬は守衛のもとを走り去り,川辺で引き裂かれた状態で発見される.これが物語のあらましである.

まず注目したいのは火星人との交流が,黒色光(lumière noire)という当時存在すると考えられていた放射線の1種を通じてなされていたことである.この通信は極めて安定的であった.パヴロフスキーはこうした黒色光やエネルギーに関する知識をギュスターヴ・ル・ボンから得ていたことが以下の記述から窺える.

\begin{quote}
日毎に火星人との関係が発展していき,重要な質問が私たちの隣人になされた.それは,物質の分離によって安価にエネルギーを手に入れる方法についてである.現実に,長い間,ギュスターヴ・ル・ボン博士の予言的な仕事とラジウムの発見以来,この問題は地上のあらゆる学者の心を占めていた.事実,よく知られていたのは,物質は,かつて不活であり,それにあらかじめ与えてしまったエネルギーを回復することはできないということであったが,全く反対に,エネルギーの巨大な貯蔵庫であった.それに加えて,ル・ボン博士によれば,うまく分離されるかもしれないのである.
\end{quote}
\begin{quote}
Les relations se développant chaque jour davantage, d'importantes questions furent posées à nos voisins sur la façon dont on pouvait obtenir l'énergie à bon marché par la dissociation de la matière. Depuis longtemps en effet, depuis les travaux prophétiques du docteur Gustave Le Bon et la découverte du radium, cette question préoccupait vivement sur terre tous les savants. On comprenait bien, en effet, que la matière, jadis considérée comme inerte et ne pouvant restituer que l'énergie qu'on lui avait d'abord fournie, était au contraire un colossal réservoir d'énergie. C'est ainsi, d'après le docteur Le Bon, que si l'on arrivait à dissocier.
\end{quote}

ギュスターヴ・ル・ボンは,社会学者として『群集心理(\emph{La psychologie des foules})」(1895)を著す一方で,『物質の進化(\emph{L'évolution de la matière})』(1905)や『力の進化(\emph{L'évolution des force})』(1907)など科学研究をまとめた著作も執筆していた.ステーブルフォードが指摘していたように,パヴロフスキーはル・ボンの影響を強く受けていたことが知られている.パヴロフスキーのエネルギー観はル・ボンの著作に基づいていたと考えられる.それについて以下で見ていこう.

『物質の進化』では,物質は不安定なものであるとされる.物質は放射線を放射することでエーテルへと変化するため,安定的な存在ではない.そして,エーテルは物質の源であり,物質はエーテルから生まれる.さらに,物質からエーテルへの変化の過程で放出される放射線がある.それが,ル・ボンがその存在を確証したと考えていた黒色光の正体であった.『力の進化』第2巻第4部「黒色光」によれば,レントゲンのX線に関する研究を知ってすぐに,「私[=ル・ボン]は物質を通過することができる特別な放射線を黒色光と命名したのは,全く目に見えない光のように時折振る舞う特性を考慮してのことだった(Je les [=des radiations particulières, capables de traverser les corps] désignai sous le nom de Lumière noire en raison de leurs propriétés d'agir quelquefois comme la lumière tout en ètant invisibles.)」.そして,その性質は3つある.まず,「陰極線のグループの放射性粒子である(Paricules radio-actives de la famille des rayons cathodiques.).次に,「波長が長大な放射線(Radiations de grande longueur d'onde)」.最後に,「不可視のリン光に起因する放射線(Radiations dues à la phosphorescence invisible.)\footnote{Gustave Le bon, \emph{L'évolution des Forces}, Flammarion, Paris, 1907, p. 277.}」.また,『力の進化』では燃焼を例に物質がエネルギーを貯蔵し,また原子が分離することで生じるエネルギーについての説明がある\footnote{\emph{Ibid. }, p. 196.}.「引き裂かれた犬」では,物質を分離することでエネルギーを取り出す人類の姿が描かれているが,ル・ボンのこの記述が参照されているのは明らかである.

現在,ル・ボンが発見したと考えた黒色光はもちろん存在していない.彼が述べているような「不可視のリン光」なるものは存在しない\footnote{リン光とは,芳香族化合物などが光を吸収した場合に,エネルギーの高いとそれを放出する時の発光現象なのであって,「不可視のリン光」など存在しないし,ましてそれが放射線であることもありえない.}.しかし,量子論史研究者の森川亮の述べているように,「ル・ボンのアイデア(彼自身はそれが単なるアイデアなのではなく本当に存在すると確信していたのだが)は,いわば,時代の変わり目にあって,それまでの物質観,さらに広く述べれば世界観が変貌しつつあることを傍証する」ようであり,「物質の非物質性,言い換えれば物質と非物質の連続性とでも述べるべき物質観であった\footnote{森川亮,「量子論の歴史 --- 未知なる放射線,その発見ラッシュの裏面史」,生駒経済論叢,第13巻,第2号,2015年11月,298頁.}」.「物質と非物質の連続性」は,パヴロフスキーにおいては観念と物質の連続性と言い換えられる.物質が黒色光を放ちながらエーテルに還るようにして,3次元の物質は精神を介して4次元の観念へと還るのである.3次元における物質の変化は,精神の働きかけによって生じ,パヴロフスキーはそれを最終的に「脱物質化(dématérialisation)」(314)と呼んでいる.つまり,精神は物質からその特性を引き離し,そこからエネルギーが生じる.それによって初めて物質は4次元における本来の姿を取り戻すのである.物質とエネルギーが対になっている「悪魔祓い」の章での記述は,精神の作用を介した物質とエネルギーの関係を示している.ただし,エネルギーという言葉は『四次元郷』において経済活動と結びついた語彙でしかなく,精神の作用と関わっているような直接的な記述は見られない.エネルギーと精神が結びつくためには,さらに上位の宇宙の法則を人類は理解する必要があった.1912年版第21章1923年版第23章「万有浮揚力(Lévitation Universelle)」ではその上位の法則が明らかにされる.まずはあらすじを見てみよう.

火星人との接触での失敗の後,物質の分離によるエネルギーの抽出に成功することができた.しかし,それはより上位の法則を知らないために世界的な危機を招いてしまう.物質を分離することでエネルギーを得ていく一方で地球を軽くしてしまい,地球自体に働く力もまた弱くなってしまったのだ.ここで弱まる力というのは遠心力(force centrifuge)である.人類がそのことに気付いた頃に地球に接近していたのが,1910年に地球を通過する際に人々を驚かせたハレー彗星だった.そこで,人類はハレー彗星の持っているエネルギーを奪うことで失われたエネルギーを補填しようとした.その試みは成功し,人類はそのエネルギーを転用して地球の回転速度まで操れるようになった.これがのあらすじである.

この章で明らかになるのは,章題にもある通りの万有浮揚力である.この「浮揚」はオカルト用語で用いられる幽体によって物体が浮き上がる現象や,超伝導磁石によって金属が落下せず空中に固定されているような状態を指す言葉である.万有引力(Gravitation Universelle)の地口で名付けられたと考えられる万有浮揚力は,万有引力を補完するものだと述べられている.そして,「それは,結局,【物質の力の完全な説明という,世界の根源についての|引力と斥力,連合と分離という拮抗する2つの力,つまり,世界,いわゆる物質の出現と消失を左右する反対の2つのエネルギーという】決定的な啓示だった(Ce fut, en somme, la révélation définitive des \{origines des mondes, l'explication complète des forces matérielles ;|deux forces antagonistes d'atttraction et de répulsion, d'association et de dissociation, des deux énergies contraires dont dépendent l'apparition et la disparition des mondes, c'est-à-dire de la matiére\})」(132/144).この「決定的な啓示」は,まとめると,万有引力は全てを引き付ける力,万有浮揚力は全てを引き離す力としてそれぞれ働き,2つの組み合わせによって宇宙は成り立っている.「引き裂かれた犬」の章では,分離によるエネルギーが重要であったように,ここでも万有浮揚力という引き離す力が重要となる.「万有浮遊力」の章では,万有浮遊力は現実では「遠心力(force centrifuge)」として様々な場所で働いていると考えられている.これは慣性の法則で働く遠心力とは違い,パヴロフスキーは物質が持っているエネルギーとして捉えている.パヴロフスキーはこの遠心力の重要性を,ラプラースが唱えた星雲説におけるあるミッシングリンクを埋めることができると考え,強調している.なぜ,「遠心力」は重要であり,星雲説という宇宙の起源に関する議論と関わっているのだろうか.

星雲説は,宇宙の塵が凝集していくことで,星雲になり,次第に恒星や惑星を形成するというものである.これが私たちの宇宙の始まりであったと星雲説は教えている.パヴロフスキーは,ではなぜ全ての塵が凝集して1つだけの塊とならないか説明できない,と星雲説に反論している.パヴロフスキーによれば,物質が持っている遠心力によって違いが万有浮揚し合うことによって1つの塊になることは決してないのだという.パヴロフスキーがこうした仮説を持ち出すのは,物質のエネルギーとしての側面を見落としている科学の時代の人類の姿を強調するためである.あらすじであったように,人類が物質を分離することでエネルギーを得ていくと同時に「遠心力」が弱まっていったのは,物質が持っている「遠心力」を失っていたからなのだ.

ところで,「万有浮揚力」の章は物質のもつエネルギーの正体を教え,その時に生じた危機を回避する出来事を物語っているが,危機の回避が彗星によってもたらされるというのはいささか唐突な顛末である.1910年のハレー彗星が連載時に記憶に新しかったために時事的な話題を挿入したとも考えられるが,戦争を挟んだ1923年版においてもその表現を変更していないので,時事的な逸話であるという説明には説得力があまりない.すると,何か彗星でなければならない理由があると考えられる. その理由は,彗星について言及する箇所に隠されている.

\begin{quote}
最も完成された空気ポンプによって作られる相対的な真空は彗星の実体よりもなおずっと密度が高い
\end{quote}
\begin{quote}
le vide relatif produit par la machine pneumatique la plus parfaite est encore beaucoup plus dense que la substance cométaire(1912, 134)
\end{quote}

1912年版のこのフレーズは,ある著作を一部改変したものである.原文では以下のようになっている.

\begin{quote}
空気ポンプによって作られた完全な真空は彗星の実体よりもなおずっと密度が高い
\end{quote}
\begin{quote}
  le vide le plus parfait d'une machine pneumatique est encore beaucoup plus dense que la substance cométaire\footnote{Auguste Blanqui, \emph{L'éternité par les astres}, G. Baillière, Paris, 1872, p. 18.}
\end{quote}

その著作とは19世紀の革命家オーギュスト・ブランキによって執筆された『天体による永遠』(1872)である.『天体による永遠』は,パヴロフスキーが「遠心力」を説明した時に批判の対象として挙げていたラプラスの星雲説の批判から始まる.ラプラスが彗星を太陽と同等のものとみなしているのに対して,ブランキは地球を通過しても影響を与えることはない空虚な存在であり,惑星や恒星とは別の存在であると断言する.彗星は全くの未知の物体なのである.これが意味するのは,彗星は全くの未知であるためにあらゆる秩序から逃れ去るということだ.星雲説で,ラプラスが必然性をもった宇宙の秩序という必然的な宇宙観を提示したとするのであれば,ブランキは彗星が決して秩序の中には回収されない偶然に満ちた宇宙観を提示した.この考えの背景にあるのは,1864年にハギンズが太陽と他の恒星は同じ元素で構成されていることの発見である.ブランキはこの観測事実から,宇宙がもしも有限の元素によって構成されているのであれば,元素の組み合わせも有限に違いないのであり,宇宙のどこかで同じ元素の組み合わせが反復されているに違いないと考えた.すなわち,地球で起きるあらゆることもまたどこかで反復されているのに違いないと考えた.しかし,発表された当時,カミーユ・フラマリオンが論駁していたように,元素の数がたとえ1つしかなかったとしてもそこから作られるものがたかだか有限個しかないということは論理的な帰結として導くことはできない\footnote{Camille Flammarion, «~L'Eternité par les astres par A. Blanqui~», \emph{L'Opinion Nationale}, 25 mars 1872, p. 3.}.鈴木雅雄の的確な要約を加えておけば,「要素の有限性とそこから作られるものの有限性とはまったく別の問題である\footnote{鈴木雅雄「星々は夢を見ない --- オーギュスト・ブランキに関する覚え書き」,早稲田大学大学院文学研究科紀要,第2分冊,53号,2007年,6頁.}」.しかし,ブランキにとって重要だったのは,この無限の反復は,少しずつ差異を含んでいるということだった.なぜなら,未知の彗星の介入によって少しずつ反復される歴史の結果が異なるからである.ナポレオン戦争おけるいくつかの戦いの結果が変わっている地球の話をブランキは例に挙げている\footnote{オーギュスト・ブランキ『天体による永遠』浜本正文訳,雁思社,1985年,95-6頁.}.こうしたブランキの考えは,当時のフランスの人文学者の間での宇宙論においても極めて特殊なものであった.マイケル・J・クロウが語っているように,フランスにおいては基本的に宇宙には生命体がおり,異なる世界が繰り広げられているという多世界論に関する言説が中心的だったからだ\footnote{マイケル・J・クロウ『地球外生命論争1750-1900』鼓澄治・山本啓二・吉田修訳,工作者,2001年}.では,この論争が起きていた頃に生まれたパヴロフスキーはブランキをどのように評価していたのだろうか.『四次元郷』にはそれがわかる一節がある.

\begin{quote}
ただ一人の人間が,かつての唯物論において,自分の意見を持つ勇気をもち,限界までそれを追求した.その男はブランキだった.トーロー要塞に収容されていた時,孤立と監獄での内省の中で,『天体による永遠』と題された興味深い冊子を書き上げた.厳密な論理はあらゆる同時代人に衝撃を与えたはずだったろう.
\end{quote}
\begin{quote}
Un seul homme, dans le matérialisme ancien, eut le courage de son opinion et la poursuivit jusqu'à ses extrêmes limites ; cet homme fut Blanqui. Dans la solitude et le recueillement de son cachot, lorsqu'il fut enfermé au fort du Taureau, il écrivit une curieuse brochure intitulée, l'\emph{Eternité par les Astres}, dont la logique rigoureuse aurait dû frapper tous les contemporains.
\end{quote}

ブランキの『天体による永遠』はパヴロフスキーにとって重要であったことがここから窺える.では,パヴロフスキーはブランキのどの論理を取り上げているのだろうか.

ブランキの宇宙観の基礎をなしていたのは基本的な元素が有限しかなく,それが反復することで宇宙が形成されているというものであった.パヴロフスキーは世界を構成する物質の基礎である原子をそれに重ね,世界が原子の反復によって多様な形を織りなしていると考えていた,と推論できる.ところで,この推論は正しい.さらに,パヴロフスキーはフラマリオンが批判していた極端な例を採用してすらいる.原子はただ1つしかないのだ.パヴロフスキーがブランキの影響を受けているとしたら,このような極端な結論は導かれなないはずである.これは何を意味しているのであろうか.

\section{観念的な偶然性}

パヴロフスキーの宇宙観について考えるためには,宇宙の歴史という時間に注目する必要がある.本章では,物質をめぐってパヴロフスキーの3次元について考察してきたが,時間もまた3次元を構成する重要な要素である.よく知られているように,19世紀は自然が永遠のものから歴史的な時間軸へと移行することになる転換期であった.パヴロフスキーがブランキの宇宙観をどのように引き受けたのかを知るために,この転換期からまずは見ていきたい.

永遠で循環するものだと考えられていた自然にも歴史があることが説明されるようになったのは,星雲説が嚆矢となった.例えば,ラプラースと同時代に古生物学者だったラマルクは,形態変化という,生物の種の形態が漸進的に変化していく説を唱えていたが,大局的には生命は円環的(cyclique)な時間を循環していると考えていた.それが19世紀の半ばを境にして,進化論の登場によって一方向への矢のように(sagittal)進んでいく時間が自然に導入されるようになる\footnote{Stéphane Tirard, «~L’histoire du commencement de la vie à la fin du XIX\textsuperscript{e} siècle~», \emph{Cahiers François Viète}, dir. Gabriel Gohau et Stéphane Tirard, n. 9, pp. 106-9. }.このように自然は19世紀に歴史化されていく過程にあったと考えられる.自然観に変容を与えた様々な科学の様々な発展の中で,ブランキが注目したのは分光学を代表する天文学的知識であり,これは19世紀後半のフランスにおいては居住世界の複数性というテーマにつながっていた.これを代表する論者がカミーユ・フラマリオンやルイ・フィギエらであり,カトリックの一部論者たちも生命体が他の世界に存在していることを認める議論をしていた.この中でも反教権派のブランキにとって主要な論敵は神学者のアルフォンス・グラトリーだった.ブランキはグラトリーに対して極めて批判的であった.ブランキのグラトリーに対する批判内容は,ブランキの宇宙観を浮かび上がらせている.

グラトリーが天体の運行においては目的地が存在していると考えていたのに対して,ブランキは「私としては,この惑星がいつどのようにして,どこかへ\bou{到着する}気になったのか,一向にわからない(Pour moi, j'ignore où, quand et comment, notre planète se propose d'\emph{arriver})」\footnote{Suzamel (Blanqui), «~Le Père Gratry. Science et foi. (3e article)~», \emph{Candide. Journal à Cinq centimes}, 1ème année, n. 8, 27 mai 1865, p. 1.}と述べており,『天体による永遠』の偶発的な進展を見せる宇宙論のモデルが定時されている\footnote{鈴木雅雄,前掲書,15頁.}.こうした宇宙論はブランキの革命のイデオロギーをも支えていた.宇宙でさえ偶然によってその運命が左右される.それと同時に,元素の組み合わせに限りがあるためにあらゆる場所で私たち自身の歴史が反復されている.もしかすると,私たちが生きている世界では革命が成功するかもしれない.こうして,革命は宇宙論的に基礎付けられ正当化される.こうした正当化のプロセスは,ベンヤミンによれば,「希望のない諦観(resignation sans espoir)」であり,「世紀は,技術的な新しい潜在性に対して新たな社会秩序をもって応ずることができ(Le siècle n'a pas su répondre aux nouvelles virtualités techniques par un ordre social nouveau)」ず\footnote{ヴァルター・ベンヤミン『パサージュ論I』今村仁・三島憲一他訳,1993年,57頁.Walter Benjamin, Gesammelte Schriften, v. I, éd. Rolf Tiedemann, Frankfurt, Suhrkamp, 1991, p. 76.},革命によって来るべき未来の社会像を持ち出していない.鈴木はベンヤミンのこの記述を受けて,「未来について積極的に語れると思いこんでいるものたちはみな「狂人」,つまりは白昼夢を見ながら夢見ていることを知らないものたちであろう\footnote{鈴木雅雄,前掲書,11頁.}」と指摘している.19世紀のこうした「狂人」たちは,鈴木の見立てに従うと次のような革命思想の四象限図~\ref{fig:fourtypes}を描くことになる.\myfig[height=18.2cm, width=12.8cm]{fourtypes}{革命思想の四象限}{fourtypes}第一象限では,未来の社会はこうなるべきであるという必然性に由来するサン=シモンを代表とするユートピア思想である.そうしたユートピア思想を批判したマルクスを代表とする共産主義は現在的な問題から導かれる必然的な世界の展開に基づいている(第四象限).一方で,これら必然性の革命思想に対して,フーリエは異なった視点を持っていた.フーリエはファランジュと呼ばれる来たるべき未来の共同体を構想した.フィリップ・レニエ(Philippe Régnier)によれば,こうした共同体の構想は,未来の社会から現在を見た時に,いつの地点からでも偶然的理想的な社会が出現するという思想であるという
\footnote{以下の解釈に基づく.Pilippe Régnier, « Place, fonctions et formes de l'ecriture utopique chez Fourier », \emph{Pamphlet, utopie, manifeste, XIX\textsuperscript{e}-XX\textsuperscript{e} siècles}, textes réunis par Lise Dumasy et Chantal Massol, L'Harmattan, 2001, pp. 385-401.}.この説を踏まえれば,フーリエは未来の社会から現在を見て,偶然性に依拠した理論を立てていると言える(第二象限).そして,これらのいずれも当てはまらないのが,現在への偶然の介入によって理想的な社会が生まれうるという革命思想を持っているブランキなのである(第三象限).

ところで,これらの思想には,いずれも特徴的なある種の科学的進歩に対する肯定感がある.実際,ブランキ以外は未来の社会の理想像を持っている.このように,科学の進歩が理想社会の実現をしうるという考えは19世紀後半に広まったものである.セルジュ・レーマン(Serge Lehman)が指摘しているように,科学が実現する未来の肯定的なヴィジョンが文学的な想像力として表現されるのは,19世紀後半のフランスにおいては極めて例外的なことであった\footnote{Serge Lehman, «~La Physique des metaphores~», \emph{Europe}, octobre, v.79, n. 870, 2001, p. 49. 32-50.}.

パヴロフスキーが評価したブランキとその時代の思想を偶然と必然の宇宙観,そして未来と現在のどちらの地点から来るべき社会,あるいは単なる未来を予想しているかという点から以上のように分類した.では,パヴロフスキーはこの四象限のどこに入るのであろうか.パヴロフスキーが『四次元郷』で描いている未来はリヴァイアサンによる支配,科学者たちの支配,4次元への到達である.しかし,パヴロフスキーは博士論文『労働の哲学』の中で,そもそも科学の進歩そのものに限界を感じていたのだった.

その時代においてパヴロフスキーは社会と個人を労働の観点から形而上学的に考察する自身のプログラムにおいて科学に限界を見ていた.科学は確かに人間の労働を軽減してきたが,未だに余暇を持つことができない悲惨な労働者を生み出してきた.科学がこの問題を解決できるかできないかの悲観的ないし楽観的な見方はいずれにせよ,「鏡の一種によって未来に過去のイメージを投影しており,そうした味方はは私たちに常に根本的に悲観的と思われるのは,これらの見方が,未来について,私たちがすでに離れたいと欲しているより劣った状況を強固にさせ,統制させるのみであるからで(s’établissent toujours sur l’étude du passé, c’est a tort qu’elles[= théories basées] prétendent avec ce passé constituer le devenir qui, par définition même, en demeure différent.)」(PT19),「より劣った状況」に陥っている社会問題を解決するためには行為の分析といった道徳に関する考察を扱うほかない.この時,形而上学が再び有用となるのである.しかし,パヴロフスキーによれば,形而上学には修正すべき問題があるという.パヴロフスキーはベーコン的帰納法とコントの社会学的演繹法の2つが形而上学で支配的な思考法であり,どちらの立場も「私たちが自分たちの経験の円環から少しでも出るはずがない(nous ne devons pas sortir du petit cercle de notre expérience)」(PT8)し,「行為において内的な必然性などはない」(PT8)コント的な立場が観察による客観的な経験主義が行為の分析の妨げなっていることを示している(PT10).そこで,ベルクソンが『直接与えられたもの』で述べているように,持続と連続性に基づく主観性による形而上学によって行為を論じることが可能であると考えた.

しかし,客観的な経験主義に基づく科学をパヴロフスキーは却けていたわけではない.そもそも,『四次元郷』での様々なエピソードは,科学技術の進展なしにはありえない.すると,パヴロフスキーは主観性に基づく形而上学を用いつつ,科学と共存し,4次元においてそうした対立が解消されるという奇妙な論理を立てている.よって,パヴロフスキーは,先の革命の四象限のどこに位置するのか不明確なままである.では,ブランキらの宇宙観と社会論にどのようにしてパヴロフスキーを比較すればよいのだろうか.

革命の四現象との比較のためには,パヴロフスキーが革命思想や社会問題に対してどのように考察していたのかを具体的に知る必要がある.それは,『労働の哲学』で取り上げられている.パヴロフスキーは労働問題の解決を目指して道徳と行為が労働においてどのように現れるかを研究していた.その結果,機械化による労働の削減を道徳によってコントールしつつ進めていくことで,全ての人類が余暇を手に入れることができると述べている.そして,『四次元郷』でも,4次元に到達する人類が「魂の静けさ(l'âme silencieuse)」を手に入れて現実の煩わしさから解放さえるということが第1章から述べられている.これらは共通して人類の未来を語っているのだが,驚くべきことに,ブランキと同じく,具体的な来るべき理想的社会像は全く描かれていない.しかし,パヴロフスキーをブランキと同じ象限に収めると,ある問題が生じる.まず,パヴロフスキーは,この世界に複数の同じような世界があるとは全く考えていない.次に,ベルクソンとの影響関係を調べた際に引用したように,未来しか存在しないというテーゼを立てているため,4次元に到達した途端に,3次元のあらゆる偶然によって生じる出来事は,少なくとも偶然とはみなせなくなってしまうのである.以上のことから,私は,次のように考えている.パヴロフスキーはブランキの影響を強く受けいるので,先行世代の革命思想とブランキの関係の図式に収めることができると考えていたが,そもそも4次元という宇宙観による大きな違いを乗り越えることはできないのではないか.そして,4次元が3次元の入れ物であるという本章で最初に検討したことに従えば,入れ物とその中身として語られていた,観念と物質を新しい四象限の軸の基準とすべきではないか.

問題となっていた未来と現在の縦軸を観念と物質に書き換えるのであれば,革命思想の四象限の人々は全て物質の側に配置される.なぜなら,社会を思考の出発的にするということは,3次元的な物質の世界を対象としているからである.一方で,パヴロフスキーは科学の進展の結果としての4次元の到達を描いているため,観念における必然性の宇宙を描いていると考えられる.しかし,これは2つの理由から誤っている.

第一の理由は,原子という,3次元においては観念的であるものの運動を必然性によって説明しているのがライプニッツである,とパヴロフスキーは考えているからである.それと同時に,パヴロフスキーによれば,「ライプニッツのモナドは,\bou{ただ一つの多様性}の中で,そして,言葉と思考といったものを一致させている予めある予定調和の奇跡の中で無残にも頓挫した(la monade de Leibnitz échouait lamentablement dans la \emph{multiplicité de l'uniuqe} et dans le miracle de l'harmonie préétablie qui faisait coïncider par exemple la parole et la pensée)」(1923, 247)のであり,観念的世界の必然性は4次元においては頓挫してしまっているというのである.この理由から,4次元の四象限は次のように描かれる~図~\ref{fig:pawfourtypes}.\myfig[height=18.2cm, width=12.8cm]{pawfourtypes}{4次元の四象限}{pawfourtypes}

図を示したところで,もう1つの理由を示す.パヴロフスキーは『四次元郷』の中でほとんど偶然という言葉を用いないが,3次元における物質と偶然の関係について,ブランキを高く評価した直後で偶然について述べている.パヴロフスキーは,ブランキの著作名を暗示しつつ,ブランキの宇宙論が示すように,全てが過去や未来に繰り返されるのであれば,意志や進歩といったものは全て意味がないということになり,それは科学が基盤としている唯物論的思考そのものの限界を示していると考える(270/219).パヴロフスキーはこの限界は,人類が4次元に到達することで乗り越えることができると考えている.この乗り越えは,ライプニッツの予定調和ではない偶然と,物質ではない観念を繋いでいる.そのことを詳しく示しているのが,1912年第44章1923年第45章の「観念による不死(L'immotalité par les idée)」だ.この章では,肉体を持ちながらにして精神によって不死に至ることができるということがテーマになっている.そして,この精神の持っている力の1つが偶然に関わっている.パヴロフスキーはこのその力と偶然について以下のように述べている.

\begin{quote}
それこそ,古代の宗教や哲学が,いつもさらなる高みにある地上で連続的に進んでいく魂の漸進的な浄化によって象徴化したものなのだ. 人間の研究は,この観点において,貴重な情報を与えた.3次元空間を自由にできるだけの身体が,生きている間と死んだ瞬間での,いわゆる部分と全体の鋳直しの連続であるような,分離の運命に結びつけられているのに対して,人間の精神はすでに4次元に到達していたし,そのために不死にも接近していた.つまり,精神は同じ瞬間に過ぎ去った,あるいは来るべき現象をそこから思い浮かべることができる.精神は抽象によって,物質的な偶然性を超えて,高みに登り,何らかの,事物の普遍的で変化しない実体をそのうちに持つことができる.
\end{quote}
\begin{quote}
C'est ce que les religions et les philosophies anciennes symbolisaient fort justement par l'épuration progressive de l'âme passant successivement dans des sphères toujours plus élevées. L'étude de l'homme donnait, à ce point de vue, de précieux enseignements. Tandis que le corps, ne disposant que de l'espace à trois dimensions, est voué fatalement à la désagrégation, c'est-àdire à une suite de refontes partielles ou totales, durant sa vie ou au moment de sa mort, l'esprit humain atteint déjà la quatrième dimension et se rapproche par là de l'immortalité ; il peut envisager, dans le même instant, des phénomènes passés ou à venir ; il peut s'élever, par l'abstraction, au-dessus des contingences matérielles, et participer, en quelque sorte, de la substance universelle et immuable des choses. (271/220) 
\end{quote}

精神は4次元へと向かっていくのだが,その際に「物質的な偶然性を超え」るのだという.偶然性を乗り越えるという記述は対立概念である必然性の肯定にも取りうるが,その一方で,精神が4次元に至ることが必然的な宇宙の宿命なのであるという記述は存在しない.では,偶然を超えるということは何を意味しているのだろうか.

ブランキは宇宙論に偶然を持ち込むことで,必然性の革命思想と袂を分かっていた.従って,パヴロフスキーがブランキについて言及しながら偶然を乗り越えるというのは,ブランキの偶然を乗り越えるということであると考えられる.ブランキは物質の偶然的な変化を3次元の宇宙の本質に見出し,一方でパヴロフスキーは4次元の観念においてそれを示そうとした.それは,4次元の四象限に従えば,\bou{観念的な偶然性}と言うことができる.この観念的な偶然性について,パヴロフスキーは,1912年版第3章1923年版第3章の「数え切れないほどの乗合馬車(La Diligence Innombrable)」で示している.

「数え切れないほどの乗合馬車」は4次元による移動とはどのようなものなのかを示している章である.パヴロフスキーはまず次のような伝説を紹介する.なんと,アジアやアラブの世界では信じらないほど遠くの距離でも電報なしでコミュニケーションができるというのだ.それは空間の抽象化(abstraction),すなわち2点間の距離を取り払ってしまうということができることを示しているとパヴロフスキーは指摘する.こうしたことが私たちに不可能であるのは,かつての乗合馬車による輸送が車にとって代わられたように,ひたすら量的な速度が追い求められている時代に生きているからである.ところで,精神は4次元に至ることができるので実は私たちは普段から4次元に通じているのだが,3次元の物質的な身体に囚われているためにそれを理解できない.パヴロフスキーが4次元による空間の抽象化に気づくことができなかったのは,かつての乗合馬車のおかげであった.ある日,田舎に滞在している時に心が望むと,「4次元の中を動き回る私の意志の気まぐれに従って(suivant le caprice de ma volonté agissant dans l'espace à quatre dimensions)」(20/65)必ず乗合馬車がやってきてそれに乗ることができた.パヴロフスキーはこの日の体験について,「その現象は私にとって合理的な説明のつかない自然発生的に起きたのだった(le phénomène produisit pour moi spontanément sans explication raisonnable)」(20/65)と述べている.精神が4次元に繋がっているために,物質的な偶然性という3次元の出来事ではなくて,精神の作用による観念的な偶然が起きているのである.

ブランキとパヴロフスキーの関わりは,パヴロフスキーが結局,新しい社会を示すことはできなかった理由を説明していると思われる.ベンヤミンも指摘していたように,ブランキもまた,来るべき社会の理想像など抱いていなかったからである.科学技術の発展の先で人類が見るものは「\bou{私たちの知的生が宇宙の生でさえあり,最も高度な宇宙の表出である}(\emph{notre vie intellectuelle est la vie même de l'univers et son expression la plus haute})(1923,249).私たちは宇宙論的に基礎付けられ,それは私たちの知性と一致するのである.こうした宇宙論はブランキのそれと非常に似通っている.しかし,決定的に違うのは,ブランキは有限の元素の組み合わせが反復されるとしたのであって,決して原子をただ1つであるとしなかったのに対して,パヴロフスキーはただ1つの原子によって3次元的宇宙が織り成されていると考えた点である.パヴロフスキーは以下のように,原子はただ1つしかないと考えている.

\begin{quote}
  精神は原子が精神自身のイメージであることを知っており,精神は4次元の特有で完全な原子から作られていて,そしてそれは,3次元の不完全な世界の多様な様態のもとで,合わせ鏡のようにただ1つの原子が無限に像を映しているというある感覚の幻想なのである.
\end{quote}
\begin{quote}
  L'esprit conçoit l'atome à son image, il en fait donc un monde complet et unique à quatre dimensions et c'est une illusion des sens qui reflète à l'infini comme dans des glaces multiples cet atome unique sous les aspects divers du monde incomplet à trois dimensions.(48/80)
\end{quote}

私たちの世界はただ1つの原子によって構成されており,それは本当の意味では存在していない.これをパヴロフスキーは最終的に「統一体はただ1つだった(\emph{l'unité était unique})」(1912, 315)と言い表している.この表現は,原子によって構成されていると考えられていた異なる様々な物質の集合はたった1つの原子によって成り立っていることを簡潔に表現しており,これは物質とエネルギーをめぐるテーマと物語が大きく関わっていることを示してもいる.

3次元における物質は原子によって構成されている.これは連合(association)と分離(dissociation)の関係にある.すなわち,リヴァイアサンの時代に1つの巨大な連合体であった人類が科学の時代の移行に際して,分離したという構造とパラレルである.人類が4次元に到達する大いなる観念論の復興の時代において,人類は再び4次元において連合するが,このとき人々の精神の働きであるところの知性は宇宙論的に基礎付けられているのであり,物質・原子・人間の集団が相似した図式のうちにある.原子の分離と集合によって物質は変化し,人類は同じ作用によってリヴァイアサンを生み出し,そこから脱出して個性を回復したパヴロフスキーは未来の社会像を打ち出すことなく,観念的な偶然性によって結ばれる人々の姿をブランキの偶然性を超えて描いている.

\section{文学と科学}

鈴木雅雄はブランキの特徴として,科学的真実から自らの文学的想像力を切り離していると述べている\footnote{鈴木,前掲書,14頁.}.つまり,ブランキは科学的な事実に依拠しながらも,決定的に科学とはずれた結論を引き出しているということである.科学の進歩と文学や思想がその歩調を合わせていた時代に,ブランキが科学的事実に依拠しながらもそれを世界を語るための真理とみなさなかったことは確かに他と比して注目に値するだろう.

ところで,この科学的真実と文学的想像力の関係は,パヴロフスキーの場合,揺らいでいる.それは,「文学的なてらいはなしで(sans la prétention littéraire)」(3/56)という作品の冒頭の表現によく表れている.「批判的吟味」においても,パヴロフスキーは自分は「作品(travail)」を書いたのであり,その結果出版されたものはあくまで「本(livre)」であって,それを「小説(roman)」であると規定することは決してない.あくまでも「時間の探求(exploration du temps)」(1923, 9)なのだ.芸術作品には3次元と4次元の接触が現れているという考えを抱いているのにもかかわらず,自身の4次元を題材にした作品はそういった身分にはふさわしくないかのような主張は,『四次元郷』という明らかな文学的想像力を前にした読者に奇妙な印象を与える.また,科学の進展がもたらした人類の災禍であった第一次世界大戦の前からすでに,パヴロフスキーはこの現実の世界における進歩のもたらす福音を前世代に比べていささかも信じていない点も奇妙である.リヴァイアサンという国家有機体説の最悪のヴィジョンを見せ,その後個人の価値が復権するものの,再び人類は失敗し,最後は4次元がすべてを解決するが,そこには人間が理想的に暮らす社会などない.魂の解放と福音というキリスト教的終末論の1つのバージョンにも思われる展開だが,結局のところ,パヴロフスキーは来るべき社会は存在しないうえ,個人の意志によって人々が結びつくことができないと考えていたのだ.ブランキの偶然性を超えて人々は4次元で統一されることになる.しかし,それは全てが1つの原子に由来しているという真理によって可能となっていて,その事実に至るのは人類の科学の進歩の結果にほからない.科学は真理を語らないことを何度も示すことによってパヴロフスキーが4次元の本質を語るときに,逆説的に科学なしでは4次元について何も語れないという論理の捻れを示している.文学的想像力を科学的真実から引き離せば引き離すほど,その繋がりを強調してしまっているのだ.

パヴロフスキーにおける文学的想像力と科学的真実の関係は以上のように,ブランキと異なっている.パヴロフスキーは科学に反対するための想像力を科学の進歩を描き続けることでしか真実を示すことができないという困難な立場を取っている.これこそ,彼がユーモアという概念を『四次元郷』の中で何度も示している態度と関係していると考えられるが,それについては別の機会に改めて論じる.
