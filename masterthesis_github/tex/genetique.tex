\chapter{遺伝と4次元}
\section{テツノミ}
物質を構成する原子のうち,すでに知られている力を有しているものは「幼生」と呼ばれていたのは前章で見たとおりである.そこでは取り上げなかったが,『四次元郷』は4次元やそれに関わるモチーフよりも「幼生」といった語が示すように,生物のモチーフが頻出している.具体的に,本章で取り上げるもののみを挙げていくと,1912年版第26章1923年版第27章「テツノミ(ferropuceron)」では鉄でできたノミ,1912年版第23章「生を捕らえて(La Capture De La Vie)」1923年版第30章「産業植物(Les Plantes Industrielles)」での工場生産のために機械化された植物,1912年版第33章1923年版第37章「巨大バクテリア(Les Bactéries Géantes)」では,章題の通り巨大なバクテリアが登場する.『四次元郷』で重要なテーマとなっているこれらの生物は,前章で焦点を当てた物質とも密接な関わりを持っている.パヴロフスキーは科学の時代に物質から生命に至るという自然発生説を取り上げ,それは先の章で見たいように,全てはただ1つの原子からできているという4次元の原理によって保証されている.しかし,今までの研究では4次元と生物のモチーフが関係しているという指摘はなされてこなかった.パヴロフスキーは4次元と生物のモチーフをどのように関係付けているのか,先に挙げた生物のモチーフのある章を取り上げて検討していこう.

「l2人の不死老人(des Douze Vieillards immortels)」の1人である「水素(Hydrogène)」は4次元を通じて2000年代に起きた昔話を語り手に聞かせる.その話によると,この時代には人造鳥に人が乗って狩猟などをしていた.ある時,パイロット「671-98」が乗り込んでいたハヤブサを模倣した人造鳥が高度3000mから墜落し,671-98は病院で36時間かけて子牛,猿,犬の臓器を移植され,命をつなぎとめた.その後,人造鳥墜落事故の調査行ったものの,機械に異常が見られなかった.調査を一旦終わらせて飛行テストをすると,右脚に相当する部位を横側に自動的に動かそうとしてか平衡を失って墜落してしまうという挙動が繰り返されたのでハンガーで精密検査を行うこととなった.その結果,なんと鉄でできたノミが人造鳥の表面から発見された.鉄でできたノミは「テツノミ(ferropuceron)」と呼ばれ,人造鳥のある特徴のために事故を引き起こしてしまったということが判明した.その特性とは,人造鳥に与えられていた,渡り鳥が磁気に従って飛行するなどといった鳥の性質や,加えて,感覚(sensibilité)の機能も与えられていたために,テツノミに由来する痒みを感じて,前脚で掻こうとしてしまったことが事故の原因だったのである.環境に適応してヤスリ粉を栄養とするノミが生まれたことは,のちに起きることとなる「機械の反乱」という工場の機械が生命のように暴走することになる事件を前日譚であるとパヴロフスキーは別の章への関係があることを示唆している.

テツノミの出現に際して,「かつてのノミが,どうやってやすり粉を栄養にし,飛行機の翼の上で生き,その環境に適応しながら変化したのかをできる限り説明しようと努力がなされた(On s'ingénia donc, du mieux que l'on put, à expliquer comment d'anciens pucerons, s'alimentant avec de la limaille de fer et vivant sur les ailes des aéroplanes, avaient pu se transformer en s'adaptant au milieu.)」(171/159-60)と述べている.また,テツノミは,ノミが環境に適応した結果生まれたと考えられて,それは擬態(mimétisme)の一例であるというある種の自然選択の理論が人類によって展開されたという(171/160).しかし,パヴロフスキーはそれに対して,「この時代,未だに進化論者の馬鹿げた教義が考えに染み込んでいて,自然発生は馬鹿げたことの1つだと思われていた(on était encore, à cette époque, entièrement imbu des absurdes doctrines évolutionnistes, et la génération spontanée paraissait une simple absurdité.)」(171/159)と述べられている.つまり,進化論だけでテツノミの出現を理解しようとした人類が批判されているのである.パヴロフスキーの自然発生説の支持は,前章で取り上げた章である「悪魔祓い」で,あらゆる物質にも生命が宿っているという考えが示されていることから科学の時代に入って人類は自然発生説を完全に肯定するようになる.以上のように,パヴロフスキーは人類が4次元に到達する過程で自然発生説を採用していることから,自然発生説は『四次元郷』において非常に重要な思想であると言える.パヴロフスキー自身が自然発生説について語っているところは,『四次元郷』の「批判的吟味」に見つけることができる.そこでは,ステファン・ルデュク(Stéphane Leduc)の浸透(osomotique)という現象に依拠することで打ち立てられた自然発生説が取り上げられている(1923, 40).ルデュクの浸透の実験の多くは問題が多いのものであったためにアカデミーから彼の実験結果は否定されることになる.パヴロフスキーはルデュクの仮説のうち,物質から自然に生命が生まれるというアイディアの重要性を4次元と3次元の接触によって生じる芸術を引き合いに出して,「芸術の最も優れた表れとは,私たちに物質の最も優れた集合とともに現れる(les premières manifestations de l'art nous apparaissent avec les premiers groupements de la matière)」(1923, 40)と述べている.パヴロフスキーが自然発生説を芸術論の根拠にもしていたことがここからわかる.

そもそも自然発生説とは,1つの種が長い期間を経てその特徴を変化させていく進化論とアリストテレスが『動物誌』の中で種子に依らない植物の発生を説明する理論がその嚆矢だった.自然発生説は生命起源論争と合流し,19世紀のフランスの科学の諸分野では,その正当性をめぐって意見が二分された. パヴロフスキーはルデュクの他にも生命起源論争や,テツノミで否定はされていた進化論に関心があったことが『四次元郷』の多くの場所から読み取ることができる.具体例を挙げれば,『四次元郷』の中でダーウィンに2度言及しており,1度目は『種の起源』と『人間の由来』の作者で自然選択説を提唱した学者として(94/115),2度目はモウセンゴケ(drosera rotondifolia)の報告者としてである(183/169).のちに触れるように,パヴロフスキーは『労働の哲学』や『国家社会学』の中で有機体国家説を取り上げる際にハーバート・スペンサー(Herbert Spencer)へ言及することが多く,必ずしもダーウィンの進化論に直接依拠して物語を書いているとは限定できない.しかし,下記で述べるように,例えばパヴロフスキーがセンモウゴケの報告者としてダーウィンを挙げて,センモウゴケが登場する章は植物のサイボーグ化が遺伝現象によって失敗するという科学的管理が生命の抵抗力に及ばないというエピソードであり,当時のフランスの進化論の議論の一部がが園芸や農業における人工交配と深く結びついていたことから,スペンサー的な社会進化論よりも,ダーウィンの進化論をめぐるフランスにおける議論にパヴロフスキーは影響を受けていた.そこで,ダーウィンの進化論がフランスでどのように受容されたのか,そして生命起源論争として自然発生説はどのように取り上げられたのかを最初に見ていき,次に,パヴロフスキーが誰のテキストからその影響を受けてどのような考えを示しているのかを見ていこう.

\section{フランスにおける進化論の導入とパヴロフスキー}

フランスでは,解剖学や生理学を背景とした遺伝をめぐる言説,人類学・古生物学・生物学に影響を与えたダーウィニズムをめぐる言説が,19世紀後半に反目し合いながら共存していた.パヴロフスキーが言及する哲学者や思想家もその例に漏れない.例えば,ドイツの唯物論の議論を俯瞰したポール・ジャネ\footnote{Paul Janet, \emph{Matérialisme Contemporain en Allemangne}, Paris, G. Baillière, 1864},観念力(idée-force)という用語を生み出したルフレッド・フイエ\footnote{Alfred Fouillée, «~Le plaisir et la douleur du point de vue de la sélection naturelle~», \emph{Revue des Deux Mondes}, 1er avril 1886, pp. 658-682.},『知られざるもの\footnote{Camille Flammarion, \emph{L'Inconnu}, Paris, E. Flammarion, 1900.}』で人間の可視光帯域や可聴帯域の外側を感覚することで未知の世界へ至るという議論を展開しているカミーユ・フラマリオン\footnote{Camille Flammarion, \emph{Le monde avant la création de l'homme}, Paris,  C. Marpon et E. Flammarion, 1886.}もダーウィニズムの影響が色濃く見られている\footnote{この具体例はすべてイヴェット・コンリーの下記の著作を参考にした.Yvette Conry, \emph{Introduction du Darwinisme en France au XIX\textsuperscript{e}siècle}, Paris, J. Vrin, 1974.}.また先行研究で言及したエレーナ・ゴメルも,4次元での時間旅行のアイディアを最初に小説にしたウェルズも,4次元が3次元の存在の進化だと考えていたと指摘している\footnote{Gomel, \emph{op. cit. }, p. 155.}.また,スティーヴン・マクリーン(Steven McLean)のような論者は,『タイムマシン』を「ポスト・ダーウィン的な宇宙における不運な出来事(Misadventures in a Post-Darwinian Universe)\footnote{Steven McLean, \emph{The Early Fiction of H.G. Wells: Fantasies of Science}, London, Palgrave Macmillan, 2009, p. 6.}」という枠組みに位置付けている.さらには,ウェルズが生物学者T・H・ハクスリー(T. H. Huxley)とともに,生物学者になるための勉強していたことから,進化論に通じていたと考えられているし,マクリーンが『タイムマシン』で描かれている退化した人類の姿は,動物学者のレイ・ランカスター(Ray Lankester)のダーウィニズムの影響下にあった退化論が関係している指摘もあり\footnote{\emph{Ibid.} , p. 25-6.},当時の4次元を扱った小説は進化論と深い関係を結んでいた.
しかし,パヴロフスキーのいたフランスでは,イヴェット・コンリー(Yvette Conry)が明らかにしたように,進化論は,その提唱者ダーウィンの名を冠したダーウィニズムの名前で普及し,ダーウィン自身とフランスのアカデミアンたちの専門的な論争,それを伝える大衆雑誌や新聞の言説,それに反応する知識人たちといったようにダーウィニズム論争の言説全体を複雑なものにしている.とりわけ,フランスではラマルクが19世紀の始めに提唱した生物変移(transformism)の仮説が1871年に終わった普仏戦争以降のナショナリティックな雰囲気の中で自国の進化論的モデルとして再評価され,ネオラマルキストを生むこととなった.ネオラマルキストは生物変移説をより科学的なものにしようとして,唯物論的宇宙における機械論的因果論という基礎的理論を求めた\footnote{Laurent Loison, «~Le projet du néolamarckisme français (1880-1910)~», \emph{Revue d'histoire des sciences}, t. 65, janv. 2012, pp. 61-79.}.一方で,ダーウィン的な進化論を独自に展開させたネオダーウィニトは,進化が先天的に要因にあるものだ考え,目的論的進化観を展開した\footnote{Paola Marrati, «~Le nouveau en train de se faire~», \emph{Revue internationale de philosophie}, n. 241, mars 2007, pp. 267-8.}.パヴロフスキーもまた,『労働の哲学』や『四次元郷』で目的論(finalime)や機械論(machinisme)を,個人の自由や意志について論じるためによく取り上げている.個人の行動が目的論的であれば,あらかじめ決められたプロセス以上のことをなしえないために自由はない.一方で,機械論の立場をとる場合は,個人を包摂する社会との関係によって自動的に行動が決まってしまうので,やはり自由はない.では,個人の自由を論じるためにはどうすれば良いのか,というのがパヴロフスキーの問いの中心であった.こうした立論自体が,ネオラマルキスムとネオダーウィニスムという進化論をめぐる2つの立場を背景としているものだったのだ. 

以上のように,哲学的な議論に影響を与えた進化論だが,『四次元郷』に登場するテツノミや人造鳥,さらには巨大なバクテリア,産業植物といった未来の生物を解釈するためには,さらに広いスコープで進化論の議論を振り返る必要がある.「変異の発見,遺伝の科学,有機体の粒子的仕組みは,生態学的な実験と衝突したのは,ダーウィニズムから真理,多産性,必然性を導くためだった.(Une détection des mutations, une science de l'hérédité, un schéma particulaire de l'organisation interféreront avec une expérimentaion écologique pour induie la vérité, la fécondité et la nécessité du darwinisme)\footnote{Conry, \emph{op. cit. }, p. 424.}」とコンリーが述べているように,当時のフランスにおける進化論は生物学に止まらない広範囲で取り上げられた.それはダーウィンが『人間の由来(\emph{The Descent of Man, and selection in Relation to Sex})』(1871)を発表したことで,人類学における進化論の検討がなされたことからもよくわかる.加えて,コンリーは,当時のフランスでは,人類学は医学の一部と見なされていたので,進化論の議論が病跡学と結びついたと指摘している\footnote{\emph{Ibid.} , p. 81-2.}.そして,医学の中でも精神疾患を取り扱う精神医学(psychiatrie)では,奇形学(tératologie)と呼ばれていた遺伝疾患の事例を先祖返り的(atavique)な現象とみなされ,ダーウィンはそれを後退とみなし,人間と動物の連続性を示す事例であると考えていたために,進化論が注目されることがあった\footnote{\emph{Ibid. }, p. 82.}.『四次元郷』1912年版では,人類がリヴァイアサンから解放される契機がドイツにおける類人猿(grands singes)の実験であること(1912, 76)や,「テツノミ」で飛行士の手術で動物の臓器を人間に移植する話が取り上げられるのは,動物が人間の先祖であり,類縁性を持っているという議論を背景として描かれていると考えられる.また,人間の精神を動物に移植するストーリーが登場する1912年版第31章「自然の力の彼方へ(Au delà des forces naturelles)」1923年版第34章「自然の形態の彼方へ(Au delà des formes naturelles)」では,動物たちが人間のような精神的疾患を引き起こす様子が描かれるといったように,フランスにおけるダーウィニズムの導入によって引き起こされた議論がパヴロフスキーの知的背景を形作っていたことは想像に難くないのである.

進化論のフランスにおける影響に話を戻すと,こうした人間と動物の類縁性をめぐる議論は,遺伝(hérédité)と深く関わることとなる.19世紀フランスでは遺伝の議論には雑種状態(hybridité)という問題系があった.メンデルに先んじて人工的な配合から遺伝について考えていた人物として知られているシャルル・ノダン(Charles Naudin)は雑種状態に関して代表的な論者だった.1852年,ノダンは園芸雑誌にて「種と多様性についての哲学的考察」\footnote{Charles Naudin, «~Considérations philosophiques sur l'espèce et la variété~», \emph{Revue horticole}, v. 4, n. 1, 1852, p. 104.})を公表した.この論文で彼は,種の形成が自然の場合も人工の場合も同じであるということを指摘している.しかし,人工交配などの個体の選択が自然界で起きていることはノダンは考えなかった.ノダンは後年,自然選択の反証を示しているとされる,アレクシス・ジョルダン(Alexis Jordan)の固定主義(fixisme)の立場を採用して,進化論における自然選択には反対している.その一方で,ダーウィンはノダンの研究が自分の仮説を裏付けるものであると考えたことが知られている\footnote{Conry, \emph{op. cit. }, pp.110-11}.それというのも,遺伝を決定する因子に関してノダンとダーウィンは,意見を異にしていたが,あらゆる生物の種は同じ方法で進化するという概念は共通していた.ただし,ノダンにとって「進化は,原理的な種の分割という過程であり,「形態の増殖の方法」の1つであると定義される(l'évolution, dont le processus consiste en une subdivision des types primordiaux, se définit comme un «~procédé de multiplication des formes~»\footnote{\emph{Ibid. }, p. 117.}」.Conryはそれを「増殖とは,ダーウィニズムにおける条件であり,Naudinにとって進化そのものだった (La multiplication, qui est condition dans le darnisme, devient pour Naudin l'évolution même)\footnote{\emph{Ibid. }, p. 118}」と評しているように,Naudinは進化の本質は分裂と増殖にあると考えていた.パヴロフスキーが進化論における形態の変化に興味があったことは,1912年版第33章1923年版第37章「巨大バクテリア(Les Bactéries Géantes)」から知ることができる.

この章では,大中央研究所から19世紀から20世紀にかけての古いバクテリアが流出し,感染病が人々の間で広がってしまう.ある日本人科学者はバクテリアを目に見えるように巨大化させることに成功し,体内に侵入できないようにするという方法によってこの事件は終息した.このエピソードで日本人科学者はバクテリアの「巨大化」という秋からに不条理な解決手段をとっている.一見して,パヴロフスキーの時代の医学の知識の限界であるようにも思われるが,その可能性は低い.パヴロフスキーが『四次元郷』を執筆していた1900年代には,バクテリアないし細菌への対処は現代に比べれば難しいとはいえ,全くの未知というわけではなかった.実際,作中で復活した病は「【チフス,コレラ,ペスト,黄熱病,狂犬病,破傷風|梅毒疹,黄熱病,ヘルペス脳炎,狂犬病,破傷風】(\{le typhus, le choléra, le pester, la fiévre jaune, la rage, le tetanos | la syphilèpre, la fièvre jaune, l'herpès encéphalique, la rage, le tétanos\}」(1912-213/1923-193)であり,いずれもワクチン接種によって治癒可能である.なので,科学者によるバクテリア自体の「巨大化」というのは非常に不条理な処置のように思われる.このエピソードの不条理さはパヴロフスキーの単なるユーモアとして片付けることはできない意味を持っている.なぜなら,日本人科学者の処置は形態を変化させることによって事件を解決しているからである.日本人科学者のとった手段である「巨大化」は,当時の医学用語でもあったからだ.

「巨大化」はフランス語で\emph{gigantisme}とされており,これは19世紀の頃から,医学用語では「巨人症」という遺伝疾患の1つであり,植物学でも植物体の巨大化現象を示す際の用語でとして用いられている.奇形学の症例の1つとしてどちらの分野でも扱われていた\footnote{医学では以下を参照のこと.Charles Bouchard, \emph{Traité de pathologie générale}, t. 1, Masson, 1895, p. 317..植物学では以下を参照のこと.Auguste Bellynck, \emph{Cours élémentaire de botanique}, G. Mayolez, 1876, p. 158.}.よって,日本人科学者のとった手段である「巨大化」は遺伝の文脈に読み替えると,「巨人症」として解釈することを可能にしているし,精神医学や遺伝学に関心があれば,これは\emph{gigantisme}の二重性にかけた地口であると言える.現代的に言い換えると,日本人科学者の行なっているのは遺伝子の組み替えによって変異を作り出すことなのだ.巨大なバクテリアをめぐるエピソードは形態の操作であり,遺伝学的な文脈が背景にあった.そこで,進化における遺伝現象がフランスでどのように受容されたか見ていこう.

当時,ダーウィンとノダンは雑種状態と遺伝2つの間で対立している点があるかのように論じられていたが,種の親子は類似する一方で変化し続け続けてくことは避けられないということを背景にしているため,理論の基礎的な部分で共通している点が多い\footnote{Conry, \emph{op. cit. }, pp. 129-30.}.ただし,ノダンが形態の哲学から出発しているのに対して,ダーウィンがパンゲネシス粒子の伝播効果として遺伝が生じると考えている点が多く違っている.結果的に19世紀後半のフランスでダーウィンのパンゲネシス説が遺伝の説明で採用されなかったのは,代謝の研究から生化学(biochimique)の理論や原子の力のメカニズムに関する研究が発展していたために,形態学のような形の類似性に基礎をおいた研究はあまり顧みられなくなっていた.しかし,遺伝の考え自体は19世紀後半のフランスにおいて非常に強い影響力を持っていた\footnote{\emph{Ibid. }, pp. 327-8. また,以下の資料で詳細を知ることができる.Prosper Lucas, \emph{Traité philosophique et physiologique de l'hérédité naturelle dans les états de santé et de maladie du système nerveux}, t. 1 et 2, Paris, J. -B. Baillière, 1847-50.}.パヴロフスキーの次のエピソードも,生化学と遺伝が背景にあるよと考えられる.
それは,1912年版第23章「生を捕える(La Capture de la Vie)」1923年版第30章「産業植物(Les Plantes Industrielles)」である.以下で簡単に物語のあらすじを見ていこう.

「機械の反乱」があった後,物質を操作するのが人間だけではなく,植物もまたその能力を有しているということに気づいた.植物はそもそも巨大な工場や複雑な仕掛けもなくそこから多様な要素からなる物質を取り出せるのではないかと人類は考えた.穀物の種子はある土地に蒔くことで,根と茎を伸ばし,豊かな色素を抽出でき,芳香を放ち,人間に滋養のある果実をつける.さらに,植物は,化学的にも興味深い性質を持っている.土壌が炭酸成分を含んでいてもそこから炭素を固定することができたし,不活性(inerte)な単体ないし化合物を与えても,有機物を生み出すことができた.「つまり,植物だけで,見えない過程による,おそらくは困惑するような単純な方法で,植物は単体を他の単体への信じられないような変異を実現しているのだった(A elles[=~plantes] seules, en un mot, au moyen de procédés invisibles et sans doute d'une déroutante simplicité, les plantes réalisaient ces invraisemblables transmutations d'un corps simple en un autre)」(180/168) .そこで,人類は新しい植物を作ることで自由に物質を生産しようと考えた.人間の求める化合物を生み出す新しい植物は生産者が育成をしやすいように動物のように栄養摂取するようにした.また,その植物の根と茎は土壌と機械の間でのみ生育するようにもした.その改良した植物のおかげで化学物質の生産が数年間飛躍的に向上した.人々は更なる生産を求めたために,植物のための工場が次々と建てられていった.しかし,「産業化された植物は生殖の喜びを奪われて,刺激のある状態が永遠に保たれ,下等動物のやり方で,気味が悪く陰険でむごたらしくなった(Les plantes industrialisées, privées des joies de la reproduction, maintenues dans un perpétuel état d'excitation, devinrent méchantes, sournoises, cruelles, à la manière des animaux inférieurs.)」(182-83/169).その結果,産業植物はモウセンゴケのような触手を伸ばすようになり,犬や猫,兎といった小型動物を捕食するようになってしまった.さらに時が経つと,ついに産業植物は根から養分を吸収しなくなり,人間の脳のような機能をもち,感覚器官としての役割を持った.その後,根の先端は小さな感受性菌(champignon sensible)へと変化して,大地を自由に移動できるになった.この頃になると,産業植物は嫌悪を催す(infâme)物質を放出するようになっていた.人々はこれを遺伝における劣化現象の1つである変質(dégénérescence)だと考えた.産業植物はすでに自然にあるような姿をしていなかった.人々は工場の壁に沿って生えている産業植物を慰めようと,櫛でといたり,植物が繋がっている機械を彩色したりした.しかし,効果は見られなかったため,産業植物の花がむしられ,それらは死滅した.

この産業植物のエピソードは農業ないし園芸における人工交配を背景にしていると考えれる.そして,さらに興味深いことに,一般的に,農業で人工交配が行われる理由は生産高の増加や害虫への耐性を上げるためになされるのに対して,このエピソードはどの植物でも行なっている生体機能自体を,人間が求めている化合物を取り出すために改造する点にある.つまり,「産業植物」での植物は,進化論において言い換えると,その種全体として扱われているのである.それぞれの種の緩やかな変化が進化論という概念が発明される契機となっていたことを踏まえると,このエピソードは次のように総括することができる.産業植物は根や茎が機械と繋がれたサイボーグ状態にあり,本来の生体機能を超えた能力を植物に与えたが,結局はそれは変質という遺伝現象が原因で死ぬことになってしまうのである.変質は,19世紀後半,精神医学と進化論が遺伝という現象を鎹にしてその関係が論じられていた.以下では,進化論の中に変質が登場する文脈を見ていこう.

%そして,分裂と増殖が必要としているのは,エネルギーであると考えた. Piquemalが指摘しているように(Canguilhem G., Lapassade G., Piquemal J., Ulmann J. (1962). Du développement à l’évolution au XIX ème siècle, PUF, p.21),Naudinはキュヴィエから借りたtourbillon vitalによるエネルギーによる形態変化のイメージ125注釈6を持っており,また,遺伝と先祖返りを同一視していたと考えられる(128).スペンサーが1862年に発表し,1871年にフランス語に訳されたFirst Principalから,Naudinは形態とは運動のリズムであるということを述べている(ちなみに,本では126,1863年に初版であるかのような記述がなされていて誤解を招くが,発表されたのは1862年.The Synthetic Philosophyの1冊として刊行された).

ダーウィンの進化論は,「科学的経験主義(empirisme scientifique)」の名の下で精神医学における変質の理論に組み込まれた.精神医学者たちは,病を分類できても治すことができない病気が変質であると考えていたので,そうした傾向が促された.結果的に,精神疾患患者のデータを調べることで,家族内に精神疾患がいる場合の発症率や患者の死亡率が高いことから,自然選択(selection naturelle)が疾患の発症において働いているのではないかと考えられた\footnote{変質説を唱えたベネディクト・モレル(Bénédict Morel)は,優生学的な記述もしており,結婚の制限などが推奨されるなどとしていた.Conry, \emph{op. cit. }, p. 328.}.このように,精神医学での進化論はフランス独自の遺伝の概念に適合する形で受容された.実験による証明を重視する実証主義科学の陣営からは,そもそも実験で確かめようのない進化論は否定された.よって,先ほど取り上げた産業植物のエピソードが重要なのは,生化学的な技術の応用が遺伝現象によって失敗する物語であると解釈できるからである.物語の中では,進化論の理論の一部を構成していた遺伝が実証主義科学を代表とする生化学に勝利しているのである.そこで,生化学陣営のからの進化論の批判を取り上げつつ,パヴロフスキーがそれに対して『四次元郷』でどのように応じているのか検討していこう.

当時の代表的な生化学者だったクロード・ベルナール(Claude Bernard)は,進化論に対して否定的な意見を寄せている.1879年には「ダーウィニズムは,生命機構が他のものからあるものを生じさせるという進化を備えることができることを認めつつ,何も説明していないし,私たちにとって全く理解できないままにとどまっているその最初の力について比較的には何も述べていない(la darwinisme, en admettant que les \emph{mécanismes} vitaux peuvent avoir une évolution qui les fasse tout procéder les uns des atures, n'explique rien et ne dit rien relativement à cette force première qui reste tout aussi incompréhensible pour nous)\footnote{Claude Bernard, \emph{Leçons de Physiologie Opératoire}, Paris, M. Duval, p. XIV.}」と述べている.また,他の著作では,「もし生体が進化論の法則に従うなら,再生と癒合の能力は,その法則がより盛んに現出する生体におけるどんなことであれ,生体にとって独占的でもないことなる(Si les corps vivants ne sont pas seuls soumis à la loi d’évolution, la faculté de se régénérer, de se cicatriser, ne leur est pas non plus exclusive, quoique ce soit sur eux qu’elle se manifeste plus activement)\footnote{Claude Bernard, \emph{La Science Epérimentale}, Paris, J. -B. Bailliere, 1878, p. 172}」とあるように,進化論は科学にとってあまり有益な見解を引き出せないと考えていた.さらに,『動植物に共通する生命現象についての講義\footnote{Claude Bernard, \emph{Leçons sur les phénomènes de la vie communs aux animaux et aux végétaus}, t. 1, Paris, J. -B. Baillière, 1879.}』の中で,「実のところ,遺伝は実験の条件として考えられている(A la vérité, on peut considérer l'hérédité comme une condition expérimentale)\footnote{\emph{Ibid. }, p. 342.} 」と述べ,ベルナールにとって遺伝は生命の根本原理というよりも,実験のさいの条件に過ぎなかった.すなわち,遺伝はベルナールにとって実在するものとしては考えられなかった.

生化学と進化論の対立,遺伝の実在の是非を考えると,ベルナールに依拠した論をパヴロフスキーがそのまま展開することはありえないはずだ.では,パヴロフスキーはベルナールの何に注目していたのだろうか.『四次元郷』で引用されているベルナールの言葉は次のようなものである.
\begin{quote}
生命の固有性は,現実的には,生きている細胞の中にのみにある.その他の全て[=~細胞以外に生命を構成しているもの]は適切に配置されて機械的である
\end{quote}
\begin{quote}
  Les propriétés vitales, avait-il dit, ne sont en réalité que dans les cellules vivantes~; tout le reste est arrangement et mécanique(100/111)\footnote{Claude Bernard, \emph{La Science epérimentale}, p. 167.}
\end{quote}
ベルナールが引用されている箇所では,リヴァイアサンが「社会的な有機体の発展(un développement de l'organisme social)」(100/111)の結果として生じ,支配された個人が細胞の状態にあるということについて説明している.ベルナールが指摘する生命の固有性が細胞にあるということがこの文脈の中で引かれている.このことから,パヴロフスキーは,ここで社会有機体説を踏まえつつ,社会にとっての個人を生命にとっての細胞であると述べていると考えられる.これは,スペンサーの社会と有機体のアナロジーを踏まえた,パヴロフスキーの次の引用によく表れている.
\begin{quote}
スペンサーは,同じようなものであれ[=~有機体であれ]有機体でないものであれ,全体が部分のために生きており,人間の体でもそうであるように,部分が全体のために現れてくるということを理解させた.
\end{quote}
\begin{quote}
Spencer avait fait entendre que dans un pareil et non organisme, le tout vivait pour les parties, point, comme dans le corps humain, les parties pour le tout.(100/111)
\end{quote}
ここでパヴロフスキーは『生物学の原理』でスペンサーが生物は群体からは個体を切り離して存在することは不可能であると論じているのを踏まえていると考えられる\footnote{Herbert Spencer, \emph{The Principles of Biology, The Works of Herbert Spencer}, v. II, Osnabrück, Otto Zelleri, 1966, p. 244.}.これは,社会を構成する個人はどのような存在なのか,というパヴロフスキーのテーマと関係している.前章では,原子と個人のアナロジカルな関係に焦点を当ててこのテーマを取り扱ったが,本章で見てきたように,パヴロフスキーはもう1つの別のアナロジーを用いていたのは,社会の発展が進化論に重ねられるという19世紀に強い影響力を持っていたアナロジーである.

フランスでは,進化論の流行と時期を同じくして,元来コントの用語であった社会学が独自の発展をし,デュルケームなどの近代的な社会学の嚆矢となる思想家がいたように,社会と個人がどのように関係を取り結んでいるかという議論に,進化論が反映させられていた.例えば,1906年にはピョートル・クロポトキンが執筆した『相互扶助論\footnote{Pierre Kropotkine, \emph{L'Entr'aide, un facteur d'évolution}, Paris, Hachette, 1906.}』の仏訳が刊行され,植物・動物・人間の社会性こそが生存に不可欠であること,そして,個体の能力獲得は生存するために絶対に必要なわけではないことが述べられている.こうした考えはとりわけ機械論的世界観を重視したネオラマルキストが好んで取り上げ,社会と個人が自然選択の中でいかなる変化をするのかという点が重要となった\footnote{Berdoulay Vincent et Soubeyran Olivier, «~Lamarck, Darwin et Vidal~: aux fondements naturalistes de la géographie~», \emph{Annales de Géographie} , t. 100, n. 561-562, 1991, pp. 617-634.}.パヴロフスキーは『労働の哲学』(PT20)や『国家社会学』(SN12)で社会有機体説に基づいた議論を展開している.同じことが,『四次元郷』においてリヴァイサンという怪物の名によって語られている.すなわち,『四次元郷』の最初の支配者であるリヴァイアサンは国家の進化の結果だったのである.

パヴロフスキーが進化論の遺伝と国家有機体説を扱っていたことは以上に述べた通りである.本章の冒頭で述べたように,パヴロフスキーは進化論とは別に,生命起源論争に関心を持ち,自然発生説を支持しているのであった.無機物から有機物への変化というテーマは,進化論でも扱われていた.ダーウィンは生命の起源について明確な言及を避けていたものの,ダーウィンの書簡や暗示的な言及箇所から,ダーウィンが無生物から生物へ移行していたことを認めていると考えられている\footnote{以下の文献を参照のこと.Michel Morange, «~Darwin dans L'histoire de la Pensée~», \emph{Transversalités}, n. 114, fév. 2010, p. 112. あるいは以下の文献を参照のこと.Stéphane Tirard, «~Origin of Life and Definition of Life, from Buffon to Oparin~», \emph{Origins of Life and Evolution of Biospheres}, n. 40, 2010, p. 217.}.しかし,パヴロフスキーは書簡を参照することはもちろんできないし,遺伝や生命の起源に関してダーウィンへ言及がなされているわけではないので,ダーウィンから自然発生説の着想をえたとは考えられない.これを踏まえて『四次元郷』を読み直すと,物質から生命が生まれるエピソードが語られる「機械の反乱(La Révolte des machines)」という章において,パヴロフスキーが自然発生説をどのように理解していたかを知ることができる.「機械の反乱」を読み直してみると,生命がどのような物質で構成されているのかを述べる際に1923年版では具体的なそれらの具体的な名前が挙げられており,「生命はもっぱら炭素,酸素,水素,硫黄,リン,貴金属といったある物質の物理化学的な属性に由来しているとして(la vie comme émanant uniquement des propriétés physico-chimiques de certains corps~: carbone, oxygène, hydrogène, soufre, phosphore et métaux catalyseurs)」(1923, 164)とある.この1節は,『ジュルナル・デバ(\emph{Journal Débat})』でサイエンスライターとして活躍したアンリ・クロスニエ・ド・ヴァリニィ(Henry Crosnier de Varigny)が執筆した1923年2月8日付のジュルナル・デバの科学記事「生命はどこから始まったのか? 脾臓抽出と結核(Où commence le vivant? L'extrait de rate et la tuberculose)」のある1節から引用されていると考えられる.その1節とは,「頑固にも,機械論者であるジャン・ナジョット氏は生命は「もっぱら,炭素,酸素,水素,硫黄,リン,とりわけ貴金属といった他の何らかものといったある物質の物理化学的な属性に基づいている(Inflexiblement mécaniste, M. J. Nageotte considère la vie comme reposant «~uniquement~» sur les propriétés physicochimiques de certqins corps~: carbone, oxygène, azote, hydrogéne, soufre, phosphore et quelques autres~: des métaux catalyseurs en particulier)
\footnote{Henry de Varigny, \emph{Journal Débat}, 8 fév 1923, p. 4.}」である.これらの文面はほとんど同一である.また,1923年版でこの1節がある段落では,細胞に寄生するウィルスであるバクテリオファージ(bactériphage)が最も原始的な細胞(la cellule la plus primitive)として紹介されている.ヴァリニィは記事で2つのトピックを紹介しており,その最初が単純な化合物で構成されているバクテリオファージに注目して,最小限の構成要素から生命は生まれえるかどうかを実験を紹介し,実験の示す意義は哲学的な議論に任せるというものだった\footnote{もう一方が,脾臓抽出による結核治療の話題である.}.この点からいっても,パヴロフスキーがこの記事を読むことで小説を加筆していたことは間違いないと断定できる.この事実から,2つのことが推論できる.まず,1923年版が加筆されていた時期である.少なくとも,1923年2月8日の時点までは加筆されていたことがこのことから明らかとなった.次に,パヴロフスキーが定期的にこの連載を読んでいたと考えられ,ヴァリニィの記事の影響を強く受けていたと考えられることである.とりわけ後者の可能性から初めて私たちはパヴロフスキーがどのように自然発生説に触れていたのかを理解することができる.

アンリ・ド・ヴァリニィがいかなる人物であったかはイヴ・カルトン(Yves Carton)による研究に詳しい\footnote{Yves Carton, \emph{Darwinien convaincu: médecin, chercheur et journaliste, 1855-1934}, Hermann, 2008.}.ハワイ領事だったシャルル・クロスニエ・ド・ヴァリニィ(Charles Crosnier de Varigny)の息子に生まれ,科学書の翻訳や大衆向けの科学記事を多く執筆した.その活動は主に1893年から記者となった『ジュルナル・デバ』であった.ある時期からダーウィンに関する著作を多く翻訳するようになり,『チャールズ・ダーウィン(\emph{Charles Darwin})』(1889)という概説書も執筆するほどのダーウィニストだった.ダーウィンに関係する論者のうち,とりわけ注目したいのは,T・H・ハクスリーの主要著作の1つである『生物学の問題(\emph{Les problèmes de la biologie})」(1892)を翻訳していることである.フランスにおける生命起源論争はステファンヌ・ティラール(Stépahne Tirard)によれば,物質から生命が生まれるという説は19世紀末になってドイツのダーウィニストの1人だったエルンスト・ヘッケル(Ernst Haeckel)やイギリスのハクスリーらによって展開されフランスにも翻訳を通じて紹介された.そして,ハクスリーを翻訳していたのが,ヴァリニィだったのである.ヘッケルもまた,とりわけ自然哲学(Naturphilosophie)を展開したシェリングやヘーゲルらドイツ観念論の影響が大きく\footnote{Bernhard, Kleeberg, \emph{Theophysis~: Ernst Haeckels Philosophie des Naturganzen}, Böhlau Verlag, Köln Weimar, 2005.},クーザンを通してドイツ観念論(フィヒテ,ヘーゲル)の自我論をパヴロフスキーは知っており(PT15),間接的な影響がなかったとは断言できない.そこで,ドイツにおける自然発生説の議論を簡単に見ておく.

ドイツの自然発生説は,ドイツ観念論がその背景にあり,とりわけフリードリヒ・シェリングの著作はドイツの科学者たちの思想的背景を形成していた.シェリングは,1797年に『自然哲学の理念(\emph{Ideen zu einer Philosophie der Natur})』,1799年には『自然哲学の体系の企図への序論(\emph{Erster Entwurf eines Systems der Naturphilosophie})』が発表され,ドイツ科学界の哲学背景に大きな影響を与えた.とりわけ,宇宙を1つの有機的な体制と捉える世界観によって,生物と無生物が連続していると科学者たちの多くが考えるようになった.例えば,パヴロフスキーが『四次元郷』でも言及しているヘルムホルツは,「進化論推進者であり,生物と無生物物質が連続し,同一の法則により支配される一元論的な世界観を強力に唱えた\footnote{ヘンリー・ハリス『物質から生命へ --- 自然発生説論争 』長野敬・太田英彦訳,95頁.}」.ヘッケルは,ティラールによれば,「生命の出現の概念を一元論哲学に一致させて,無機物と有機物の結びつきが存在しているという着想を彼は得ていた(Accordant sa conception de l'apparition du vivant avec sa philosophie moniste, il conçoit qu'il existe une unité de la matière inorganique et organique)\footnote{Tirard, \emph{op. cit. }, p. 116.}」.ヘッケル自身も,「有機的生命はこの最初のモネラ〔原始海洋にタンパクの質の塊のこと.ヘッケルは単細胞無核の生物体であるとしてその存在を想定した.〕,そして遺伝という固有の機能とともに始まる(Avec cette première monère commence la vie organique et sa fonction propre, l'hérédité)\footnote{Ernst Haeckel, \emph{Le Monisme Lien entre la Religion et la Science, Profession de Foi d'un Naturaliste}, Pari, Schleicher frères, 1897, p. 11.}」と述べている.ここからわかるようにヘッケルは,物質から生命が誕生すると考えていたうえに,生命の固有性を遺伝に見ていたことがわかる.では,パヴロフスキーの参照していた可能性が極めて高いハクスリーの自然発生説はどのようなものだったのだろうか.

ハクスリーは自然発生説論者を自称したことはないものの,『生物学の問題』に所収されている論考「生命の物理学的な基礎(Les bases physiques de la vie)」の中で,原形質(protoplasmique)についての考察を通じて,それが生命の「形式的な基礎(la base formelle)」であると考えた.この原形質と生命の起源をめぐる論考は,生命の出現が化学的な合成にあるということを問題にしていた点で同世代の生命起源論と一線を画していた.例えば,同世代の生命起源論は,すでに触れたスペンサーも関わっている.スペンサーは,自然発生を否定していた.しかし,『生物学の原理』で「原初の世界では,今日の実験室と同じように,有機的物体の下位の種類のものが,適合した条件下で互いに作用しあい,有機的物体のより高位の種類のものを生み出し,そして原形質を用意するに至ったのだ(Dans le monde primitif, comme dans le laboratoire d'aujourd'hui, les types inférieurs de substances organiques, en agissant les uns sur les autres sous des conditions appropriées, ont produit par évolution les types supérieurs de substances organiques, aboutissant à un protoplasme organisable )\footnote{Herbert Spencer, \emph{Principes de biologie}, Paris, Félix Alcan, 1893, p. 585.} 」と結論づけている.これは,ハクスリーに比べると「有機体の下位の種類のもの」が何か明らかではないように,具体的に原形質について議論はされていない.パヴロフスキーがヴァリニィの記事から引用することで化学的な統合が生命の起源に関わっていることを示してる点は,『四次元郷』では直接名前の挙げられているスペンサーよりも,こうしたハクスリーの議論が背景にあるのである.

ところで,パヴロフスキーはステファン・ルデュクの自然発生説に言及していた.ルデュクはアカデミーから学説を否定されていたが,ハクスリーらの議論はアカデミーからどのように扱われていたのかを確認しておこう.

19世紀半ば,ルイ・パストゥールがフェリックス・アルシメード・プーシェ(Félix Archimède Pouchet)の自然発生説を証明したとする一連の実験を論難して以来,科学アカデミーで自然発生説は科学的な学説として認められていなかった.コンリーは自然発生とダーウィニズムの関係を『ダーウィニズムと自然発生説(\emph{La darwinisme et les générations spontanées})』で論じたダリウス=C・ロッシ(Darius-C. Rossi)がプーシェにダーウィンの特に遺伝に関する考えが自然発生を証明するに足る十分な根拠になると手紙を送っているように,論駁されたプーシェはダーウィニズム的進化論が自然発生説を裏付けるはずだと考えて研究をしていた.ヘンリー・ハリスはこの論争に言及している科学史研究\footnote{以下の文献を参照のこと.Gerald. L. Geison, \emph{The private science of Louis pasteur}, Princeton University Press, 1995. Bruno Latour, \emph{Les Microbes~: guerre et paix, suivi de irréductions}, Paris, A.M. Métailié, 1984. Georges Pennetier, \emph{Un débat scientifique, Pouchet et Pasteur (1858-1868)}, Paris, J. Girieud, 1907.}を簡単にまとめたうえで,それらの評価が「関係者の発表方法の巧拙,あるいは学士院会員の公平性またはその欠如に注意を向けていることが多\footnote{ハリス,\emph{op. cit. }, 149頁.}」いと述べ,プーシェの実験がパストゥールに比べると非常に杜撰であったことがハリス自身の科学者としての技術的な視点から明示されている.自然発生説に関する議論が極めて思弁的であったためにそれを証明する手段が限られているうえに,当時の実験の精度では自然発生が起きていることを証明するという証拠を否定することの方が簡単であった.フランスのアカデミックな言説において否定されていた自然発生説は,ヴァリニィなどによって大衆化された科学の文脈の中でよってパヴロフスキーの物語に大きな影響を与えているのである.

パヴロフスキーが生物のモチーフをどのように描いているかに注目することで,進化論と遺伝が『四次元郷』の1つのテーマをなしていることが明らかとなった.すでに「テツノミ」が「機械の反乱」へ繋がることが『四次元郷』の中で言及されていることを取り上げたが,以下では,「機械の反乱」の分析を通じて,遺伝という概念が『四次元郷』の中で4次元と関わっていることを明らかにしたい.

\section{遺伝的類似}

1912年版27章1923年版29章「機械の反乱(La révolté des machines)」は第一科学時代の閏3年(le 3 intercalaire)にある工場で起きた機械の暴走についての出来事である.以下で,簡単にあらすじを述べる.現場主任H・G・28が電気設備が急停止しているのに気づき,調べさせると,設備の運転が停止しているものの,電気自体は流れていることが判明した.さらには,化合物の塩が銅板を伝って部屋の扉の前に蓄積したり,外側からの力がかかっていないのにもかかわらず主要部分が壊れ,制御装置が捻じ曲がっているといった現象が起きていた.あるエンジニアはこの出来事について命を持った金属という「何らかの奇妙な生命(de quelle vie étrange)」(173/164)が原因ではないかと考えた.工場の中ではこの異変が続き,金属の分子レベルの変異によって鉄鋼が銅に変わり,機械の動作に異常が生じた.そこで,工場をヨードホルムの蒸気で満たして,クロロホルムを浸した詰め物で塞ぎ,沈静化を図った.翌年の閏4年,不注意から電圧をかけすぎたところ,機械が捻じ曲がり,動き始め,球体を形成し,工場の扉へと向かっていった.工場の隣には,悪化した器官を取り替えるために,移植用の臓器が保管されている倉庫があった.球体となった機械は電気を帯びながらその倉庫に向かっていき,臓器は散乱した.暴走する機械を止めるため,人々は球体を人工的に凍らせて艀に乗せ,冷たい海に沈めることで解決した.

この章で重要なのは,機械の暴走の原因が金属にも生命が宿るということをエンジニアの考えを明らかにするときに示している点である.1912年版では次にように説明されている.
\begin{quote}
しかし,動植物の生命に類似した正真正銘の生命を物質に認めつつあることさえ決してなく,そして,不安を抱えながら,新しくも悩ましい発見がこの主題についてなされることがこれからもないのかどうか自問した.実際,地球の形成以来,生命を構成したものはなく,天空から私たちにやってきたこともないのを認めるべきだ.最初は,大地はただのガス状の塊であり,大地の物質が混交していた.それが動植物が生まれた原初的物質であり,私たちは知っている生命はミネラルの中にすでに存在したと十分に考えさせる.最近では,完全なものとなった機械についてなされた興味深い観察によってその確証はますます強固なものとされた.金属,とりわけ細工されたもの,それらを作るために使われ,強化されて,化学物質の数を増大させたものは,新しい真なる有機体の一種となり,思いがけないところまでそうした現象を起こすことができるようになった.電流の終わりのない伝達やヘルツ波の衝撃がさらに,もっと興味深い質的な超現代的な金属を与えた.ある場合には,正真正銘,機械にも随意の病気,そして,かつて労働者階級を殲滅したものと同じ害悪な何かが観察された.
\end{quote}
\begin{quote}
Cependant, on n'avait jamais été jusqu'à attribuer à la matière une vie véritable analogue à la vie des plantes et des animaux, et l'on se demandait avec angoisse si de nouvelles et inquiétantes découvertes n'allaient pas être faites à ce sujet. Il fallait bien reconnaître, en effet, que depuis la formation du globe, rien de ce qui constituait la vie ne pouvait nous venir du ciel. Au début, la terre n'était qu'une masse gazeuse, puis de la terre matière en fusion~; c'est de cette matière primitive que sont sortis les plantes et les animaux, et cela donne à penser suffisamment que la vie telle que nous la connaissons préexistait dans les minéraux. Ces constatations faciles avaient été renforcées, dans les derniers temps, par de curieuses observations faites sur des machines perfectionnées. Les métaux, particulièrement travaillés, que l'on employait pour leur construction, renforcés, doublés de nombreuses matières chimiques, étaient devenus des sortes d'organismes véritablement nouveaux, capables d'engendrer des phénomènes jusque-là imprévus. La perpétuelle transmission de courants électriques et le choc d'ondes hertziennes avaient pourvu ces métaux ultra-modernes de qualités plus curieuses encore. On avait même observé, dans certains cas, de véritables maladies volontaires se produisant dans les machines, quelque chose comme des vices, identiques à ceux qui décimaient jadis la classe ouvrière. (1912, 174-5)
\end{quote}
金属は生命を持ち,それは生命が金属と同じく物質に由来しており,そのために物質にも生命が宿るのである.この考えは,先に見たように,ヴァリニィの背景となっているハクスリーの考えに由来していると考えられる.生命がどのような化合物から成立しているかを付け加えた1923年版では,パヴロフスキーはこの一節を一部書き加えて,以下のように表現している.
\begin{quote}
最も原始的な細胞はすでに1個の複雑な組織である.バクテリオファージの中に,細胞のさらに下の,微生物の本当の意味での寄生,つまり生きているがさらに原始的な存在があると考えられているのは,その影響が微生物の遺伝的特徴を変更する能力があるからだ\footnote{1900年代にはデオキシリボ核酸(いわゆるDNA)が未発見であり,遺伝要素を何が伝達しているのか知られていなかったので,微生物がその役割を担っていると考えられていた.}.しかし,あたかも生命はもっぱら炭素,酸素,水素,硫黄,リン,貴金属といったある物質の物理化学的な属性に由来しているとして考えられているとしても,基本的原子の構成に他ならないものの中にまで,常にさらに遠くの起源や,惑星の運動の変化がエネルギーを生み出したり吸収したりするができる真なる無限小の宇宙,そして,中心核の周囲を回っている物質の電子の数の違いのほかに物質のうちに別の違いを認識しない摩訶不思議な錬金術師たちを追求しないほうがよい.生命は,すでに運動や行為の中で何かする力を持ってはいないが,水や風の渦によって運ばれていく不活の物質でないと,言うことができるのだろうか.もしも太陽系全体が原子の世界の荘厳な模倣に過ぎないのであれば,私たちの精神にとって,詩人が私たちに自然から与える描写の心に染み入る魅力を生み出すこと,それが,何世紀にも渡って,雲,海,森の運動,そして揺れ動く多様な私たちの思考の運動を結びつける曖昧な類似であるのは明らかではないのだろうか.
\end{quote}
\begin{quote}
La cellule la plus primitive est déjà un édifice complex. Au-dessus d'elle on a cru voir dans le bactériophage, véritable parasite du microbe, un être plus primitif encore mais vivant, puisque son influence suffit à modifier les caractères héréditaires des microbes. Mais si l'on considère la vie comme émanant uniquement des propriétés physico-chimiques de certains corps~: carbone, oxygène, hydrogène, soufre, phosphore et métaux catalyseurs, ne doit-on pas en rechercher les origines toujours plus loin, jusque dans la constitution même de l'atome élémentaire, ce véritable univers infiniment petit, dont les modifications de mouvements planétaires suffisent à créer ou absorber de l'énergie, et qui, merveilleux alchimiste, ne connaît d'autres différences entre les corps que celles du nombre de ses électrons gravitant autour d'un noyau central. La vie, mais n'est-elle pas déjà en puissance dans les mouvements, dans les gestes, pourrait-on dire, de la matière inerte entraînée par les remous de l'eau ou du vent? Et si l'on peut penser que tout le système solaire n'est que une imitation grandiose du monde atomique, n'est-il pas évident que ce qui fait, pour notre esprit, le charme pénétrant des descriptions que les poètes nous donnent de la nature, c'est l'obscure parenté qui unit, au travers des siècles, les mouvement des nuages, des mers ou des forêts et ceux de notre pensée ondoyante et diverse? (1923, 164-5)
\end{quote}

パヴロフスキーは最後の反語表現で,物質と生命が連続的な存在であることを示唆している.これを自然発生説に由来する生物学の知識を得ていたと考えられるパヴロフスキーが記述していること自体は驚きはないが,真に注目すべきは生命を持つ限り産業植物を例とした遺伝的異常も見られるという1912年版の考えから一歩進んで,「雲,海,森の運動,そして揺れ動く多様な私たちの思考の運動を結びつける曖昧な類似」とあるように,生命の起源である物質と思考の類似(parenté)があると1923年版で述べているのは注目すべきことである.

\emph{parenté}という語は,「もう一方の人々が子孫にあたるような人々の関係(Rapport entre personnes descendant les unes des autres)\footnote{\emph{Le Nouveau Petit Robert de la langue française}, 2008, Électronique.より.}」とあるように,血縁の類縁性を示す語である.これは,遺伝を1つのテーマとしている『四次元郷』において,重要な表現である.なぜなら,\emph{parenté}の示す血縁は,『四次元郷』では遺伝の文脈に置くことができるからである.遺伝現象の結果としての類似が,思考と物質の類似であるとここでは述べられている.これを思考と物質が\bou{遺伝的類似}の関係にあると言い換えてみよう.前章で見たように,パヴロフスキーは知性を宇宙論的に基礎付けており,3次元から4次元には精神によって到達することができた.そして,パヴロフスキーは別の箇所で思考と精神の関係について「精神だけが4次元に到達することができて,思考や行為は物質的で,つかのまの,現実には存在しない現れによる翻訳である(l'esprit seul peut atteindre dans la quatrième dimension et dont les pensées et les gestes se traduisent par des apparences matérielle, fugitives et irréelles)」(1923, 112)と述べている.思考が物質的なものであるということは,遺伝的類似の網目の中で,生命と物質の連続性という自然発生説から導かれる.3次元における物質・生命・思考の関係は遺伝的類似なのだ.また,この遺伝的類似は思考以外にも拡大していく.

1923年版24章の「記憶の拡張(L’Agrandissement des souvenir)」では,脳の研究が進んだことで,脳内には個人的な記憶だけでなくて,先祖の記憶も潜在していることが発見された(1923, 147).その後,手術を施して歴史の記憶を巡る人を「下意識の旅人(voyageurs du subconscient)」と呼ぶようになった.この旅は流行したのだが,その中で事件が起きる.その事件の被害者は「旅人」の一人のナトリウム(Sodium)という科学者だった.この科学者の先祖はカリベルト1世で,彼は6世紀のメロヴィング朝の王だった.この王は2人の侍女と内縁関係を結んでいたと伝えられている.その王の下意識に入り込んでナトリウムはまた次女への感情を共有していまい,二度目の旅をしたところ,その2人の侍女と離れることに絶望を覚えて自殺してしまった.先祖の感情にとらわれてしまうナトリウムの姿は本章で明らかにした遺伝のテーマを意識した表現に直せば,感情の先祖返りとも言える.「記憶の拡張」の章が示しているように,パヴロフスキーが背景としている進化論の言説には,奇形の発現が先祖返りの一種であり,それこそ進化の確証と言えるのではないかというダーウィンの考えもまた表れていると言える.このように,遺伝的な関係をモチーフにして,パヴロフスキーは意識と記憶の問題を取り扱っており,記憶という精神の働きに関わる領域を科学的に分析する「心理学は私たちに世界の\bou{物理的}現実に手を届かせるのみである.すなわち,創造と生とに(la \emph{psychologie} nous permet seule d'atteindre la réalité \emph{physique} du monde~: la création et la vie)」(1923, 252)とパヴロフスキーは述べている.創造と生が広がる3次元的世界を把握するために心理学が求められるというこの評価は,遺伝的類似によって思考や記憶が物質と関係していることで初めて意味をなす.パヴロフスキーの4次元概念の参照元であったベルクソンが『物質と記憶』(1896)や『精神のエネルギー』(1919)で重要な業績として心理学者ピエール・ジャネの業績に対して肯定的に述べていることも心理学への関心の意味を説明している\footnote{Henri Bergson, \emph{L'énergie spirituelle}, éd. Frédric Worms, Paris, PUF, 2009, p. 113, p. 115, p. 122. Henri Bergson, \emph{Matière et mémoire}, éd. Frédric Worms, Paris, PUF, 2008, p. 8, p. 113, pp. 195-6. }.なぜなら,前章で見たように,パヴロフスキーが,人間の精神は4次元に至ることができると考えているように,思考や記憶といった心理学的な対象の分析もまた,4次元に至るための手段たりうるのである.

以上のように,遺伝的類似は,いわば,進化論における種の連続性とその変化を,物質と思考にまで延長している.これは私が第4章で示した4次元に由来する3次元の物質の現象と同じ図式を持っており,さらに,精神が物質に作用しているということをふまえると,物質から思考までの連続性を持っていることと,3次元と4次元の連続性はパラレルであると考えられる.このことから,以下のように結論づけられる.パヴロフスキーの4次元は進化論における遺伝の理論と非ユークリッド幾何学の2つを独自の解釈によって組み合わせることで成立しているのである.
