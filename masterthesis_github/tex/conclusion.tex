\chapter{結論}
私は,『四次元郷』について主に3つの観点から考察した.初めに,出版やジャーナリズムを中心とした歴史的観点から『四次元郷』が連載小説という形式をとった理由と,連載によって一種の科学記事のような側面を持っていたことを指摘した.それと同時に,科学小説の成立の中で特異な位置にあったことを示した.2つ目の観点は『四次元郷』の4次元が他の作品と比べてどのような特徴があるかを調べ,ヒントンらとは違ってそこに時間を認めることはなく,運動も存在しないことを確かめた.そして,ベルクソンが4次元によって規定されている3次元の世界を解明する論者として扱われていることから,「未来しか存在しない」というテーゼを取り出すことで,現代のベルクソニアンが示す芸術論を参考に,その応用可能性を指摘した.3つ目に,『四次元郷』で3次元は果たして4次元に対してどのような世界を形成しているのかを考察するために,物質と遺伝の2つテーマからエピソードを読み解き,3次元と4次元を結びつけている観念的な偶然性と遺伝的類似という概念にそれぞれまとめた. 観念的な偶然性とは,4次元にあるただ1つの原子の模像である人間が社会的な関係に依存しない形で統一される時に働く偶然である.3次元の観点からすると,それがいつ起きるか分からない点で全くの偶然と言える.次に,遺伝的類似は,自然発生説に由来した物質と生命の連続性が,思考や記憶にまで延長していることを意味する概念であった.よって遺伝的類似は,物質から思考への連続性を示している.

この2つの概念が興味深いのは,パヴロフスキーの支離滅裂とも思えるエピソードが,現実の世界で生じている現象すべてを4次元という経験的でない理念的な領域にある原理によってあたかも基礎づけているようであるからだ.ところで,その領域と3次元を生きる人間の関係は『四次元郷』の第1章から示されている.
\begin{quote}
精神は事物の普遍性と一体になっているにほかならない.その考えは現実的で,反発はありえない.\bou{静かなる魂}は世界の雑音をもはや気にも留めない.
\end{quote}
\begin{quote}
L'esprit ne fait plus que'un avec l'universalité des choses~; ses idée sont toutes positives, sans réaction possible. \emph{L'âme silencieuse} ne s'inquiète plus des bruits du monde.(8/59)
\end{quote}
「精神が事物の普遍性と一体になっている」というのは,遺伝的類似で示したように,物質と思考の連続性のことである.そして,「静かなる魂」とは4次元に到達した精神のことを意味していると同時に,「世界の雑音」,すなわち3次元の現象から切り離されている場所こそ4次元であるということを示唆しているのだ.3次元と4次元が連続でありながらも切断されている状態は,第4章で私が取り上げた非ユークリッド幾何学的対象が実在しているのか否かの問題に発展したことを想起させる.ところで,非ユークリッド幾何学の歴史を紹介する中で触れたリーマンが生み出した多様体という概念は,パヴロフスキーの4次元が現実に関わりつつも経験的でないという存在の身分を持っているのと非常に似通っている.そして,多様体の示した存在の身分の問題は,数学史の転換点となった.加藤文元のリーマンの多様体の解釈に依拠して\footnote{特に下記を参照のこと.加藤,前掲書,第7章.},具体的にリマーンの多様体とパヴロフスキーの4次元の関連性を見てみたい.

加藤の見立てでは,リーマンの多様体が登場した後の数学とそれ以前の数学の大きな違いは,経験的な対象を扱っているか,現実には存在していないものを対象として扱っているか,という点にある.リーマンの多様体を代表とした非ユークリッド幾何学的な図形は,現実には存在しないにもかかわらず計量することができる幾何学的な対象である.具体的な例として有名なクラインの壺を取り上げてみよう.クラインの壺を私たちの世界で図示する場合,壺から伸びている管が自分自身の中に入り混む時に交差してしまうため,交差地点が点線で示されていることがよくある.しかし,クラインの壺は4次元において滑らかな曲線であり,自身と交差することは本来ありえない.ちょうど,パヴロフスキーのインド風の箱が3次元では結ばれているのに,4次元でそれが解けてしまうのと同じである.では,クラインの壺の本当の姿について私たちは何も知ることができないのかと言えばそうではない.多次元空間の幾何学的対象に挑む数学者たちは,数式の操作によって部分的に対象の存在を知ることができ,それを繰り返すことによって自分が格闘している対象の全体像が把握されるようになる.加藤はこうした対象が存在している領域を「叡知的存在領野」と名付けている.私は序章でベクトルのn次元空間を取り上げたが,集合の要素の組み合わせによって無限の次元が展開されるこの空間もまた,叡知的存在領野に存在している.そして,叡知的存在領野に存在し,私たちの世界から独立して存在していながらにして,様々な分析によってその姿を知ることができるのである\footnote{2012年8月30日に望月新一が発表した宇宙際タイヒミュラー理論(Inter-universal Teichmüller Theory)もまた,この叡知的存在領野を様々なパースペクティブの積み重ねによって検証したものである.乗法と加法の関連性という極めて抽象度の高い課題に対して,宇宙際タイヒミュラーという空間を新たに構築することで検証する極めて高度な数学的手法は,私たちにこの叡知的存在領野の驚異的な側面を教えている.}.リーマン以降,現代数学は叡知的存在領野の抽象的な概念を分析していくことで,対象を具体化していくための様々な手法が開発されるようになった.加藤は三宅岳史の見解\footnote{三宅岳史,「リーマンと心理学,そして哲学」,『現代思想』,44巻,6号,青土社,2016年,161-175頁.}をふまえ,抽象的な対象の具体化のプロセス自体は経験的な観察なのであり,それは再び経験論に立ち戻ることなのだと述べている.そして,「ベルクソンによる「純粋持続」(中略)などのように,これら叡知的存在領野と人間との間に生じた相互依存と確執という新しい問題系を手に入れたように思われる\footnote{加藤,前掲書,174頁.}」というのだ.私は,パヴロフスキーの4次元が経験的でない理念的な存在であると述べたが,ベルクソンの経験論を知っていたパヴロフスキーが自然に似たような議論を構築していたとも考えられる.

パヴロフスキーの4次元はベルクソンの影響を受けつつ,持続については3次元的なものだとして却け,過去・現在・未来の本質をめぐる議論もすれ違っていた.しかし,本論で見たように,物質のエネルギーとしての側面とそれを動かす観念的な偶然性や,物質から生命,そして思考の連続性を示している遺伝的類似はパヴロフスキーの4次元を具体化していくものであり,叡知的存在領野にある抽象的な対象を様々な方法で具体化しているのと同じようなプロセスを辿っている.まさに加藤が例示しているベルクソンの純粋持続のような意味での経験論である.この経験論の視座において物語を書いていたパヴロフスキーが単なるプラトニストではなかったことは次の言葉からもうかがえる.

\begin{quote}
4次元の感覚は人間に先んじて働いていて,それは3次元の世界で\bou{未来の感覚}と呼ばれているものである(中略)
\end{quote}
\begin{quote}
  Car le sens de la quatrième dimension marche en avant de l'homme, il est ce que l'on appellerait \emph{le sens du futur} dans un monde à trois dimensions (...)(1923, 243)
\end{quote}
叡知的存在は理念として理解するものではなくて,感覚するものであるというこの主張は,いまや非ユークリッド幾何学のパラダイムが生み出した「真の経験論」(ベルクソン)として読むことができるのを示している.

本論全体についての結論は以上として,今後の展望を簡単に述べたい.本論第3章で連載小説の観点からジャーナリズムや,それがもたらした科学の大衆化が『四次元郷』の背景となっていることを指摘した.しかし,この論点では作品の文体と記事の文体の比較による具体的な検証が求められる.また,文体以外にも,センセーショナリズムによって,事件の報道と虚構の物語が区別できなくなるような自体が生じることについて,ナラトロジーの観点から分析するべきだろう.例えば,ジュネットは『フィクションとディクション』に所収された「虚構的物語言説,事実的物語言説\footnote{ジュラール・ジェネット,「虚構的物語言説,事実的物語言説」,『フィクションとディクション』,和泉涼一・尾河直哉訳,水声社,2004年,55-75頁.}」で,具体例としてルポルタージュや新聞調査の派生ジャンルとしてノン・フィクション・ノヴェルを取り上げ,虚構的物語言説と事実的物語言説の相互作用が働いていると指摘している.この相互作用は19世紀末の三面記事とロマン主義文学の関係にも当てはめることができるだろうし,同じように科学記事と科学小説~---~あるいはアカデミーの報告と科学記事にも~---~に当てはめることができるだろう.こうした分析手法は『四次元郷』だけでなくパヴロフスキーの他の作品にも応用できると考えられる.それはガストン・ド・パヴロフスキーの作品群に新しい解釈をもたらすかもしれない.

\begin{comment}
「幼生」と原子を言い換えている1912年版第31章「悪霊祓い(La Conjuration des Larves)」1923年版第33章もその傍証となる.この章は「機械の反乱」と似た構成を持っている.
幼生はgerme.遺伝的類似と連続性.
幼生の隠喩,世界が1つの卵であること.
多様性の容器,4次元の秘密を知るための3次元の緻密な分析.4次元によって保証された3次元の豊かさが示されている.
3次元の豊かさはとりわけ芸術に隠されている.パヴロフスキーは遺伝的類似を示唆する1923年版において,le succès éternel des fictions littéraires qui symbolisent, sans y prendre garde, la Vie et l'univers; de là le sens, plus profond qu'on ne le croit, de cette grande comédie imaginée que l'on appelle la Comédie de la Vie 251と述べられる.宇宙の生を象徴として描くことのできる文学の虚構こそ,
1923,250で,人間の人格が愛を生み出している説明がある.愛は人格ごとの特徴を強調するものである.それは,les différences dont vit la Pensée uniqueなのである.唯一の思考を生きる差異
遺伝的類似はひとつながりの存在の階梯を想像することを用意するが,それに知性的存在,抽象的思考を含めることはできない.パヴロフスキーが真に独創的だったのは,科学を一旦は認識論的に還元し,私たちの共通の意識をもっているということを前提に,知性を宇宙論的に基礎づけることで,実は科学的な存在として私たちを見なしうるとしたところにある.すなわち,進化論的遺伝的宇宙は一旦は科学的思考にみなす=3次元化することで,私たちの知性を含めたあらゆる側面を遺伝的な類似に並べることを可能にする.そして,そうした類似の中に生じる様々な人格の差異を,これは存在者の差異と言い換えもできるだろうが,ただ一つの思考という多様性の中に回収するのである.
\end{comment}
