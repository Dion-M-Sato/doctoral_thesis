\chapter{序文}
ガストン・ド・パヴロフスキー.フランス人,ジャーナリスト,小説家,自転車愛好家.彼は1912年に『四次元郷への旅(\emph{Voyage au pays de la quatrième dimension})\footnote{日本では,一般的に『四次元国への旅』と訳されることが多い.パヴロフスキーは国家論に関する著作を残していることから,「国」を意味するのであれば\emph{État}という語を用いると考えられる.さらに,パヴロフスキーは作品の冒頭でその題名における「\bou{4次元}とは,それ自身が,総合的な状態に他ならない(\emph{quatrième dimension} n'est, elle-même, que la manifestation d'un état synthétique)」(8/59)と述べているように,それは私たちがイメージするような国のことでないことが示されている.このことから,郷土や黄金郷といったようなどこかの場所を示す\emph{pays}だと考えるのが適切である.以上の理由から,「四次元郷」という訳語を用いた.}』(以下,『四次元郷』)を世に送り出し,世紀末から知識人たちの間で流行していた4次元思想をフランスで普及する第一人者となり,芸術家たちに4次元を題材にした作品を制作させるほどの影響力を持っていた.その後,アインシュタインの一般相対性理論が登場し,ミンコフスキ空間では4次元が幾何学的に計算可能な対象に過ぎないことが明らかになってなお,パヴロフスキーは自身の4次元の概念を捨てることはなかった.アインシュタインの理論が人口に膾炙するようになる頃に彼は心臓発作で亡くなり,そのうちに人々は4次元の崇高さを完全に忘れてしまった.

パヴロフスキーは研究の主題となることがほとんどないので,日本でもあまり知られていない.20世紀の最後の年に英訳された他に,私家版をおくとして,翻訳は管見の限り確認されていない.それでも日本でこの名前を何度か言及した人物は存在して,その最初の例として中沢新一を挙げることができるかもしれない\footnote{中沢新一「四次元の花嫁」,『東方的』,講談社,2012年,48-120頁.}.彼のエッセイは,南方熊楠の書庫の探索から始まり,高次元空間に神秘主義的な期待を寄せた人々の思想が紹介され,最後に,現代の宇宙物理学から提示された超弦理論に触れて,この宇宙には高次元が現実に存在している可能性があることを示している.

確かに,現実に高次元空間があるかどうかという宇宙論は興味深い話題であり,いずれ天体観測のデータから宇宙の曲率を割り出して,私たちの現実がいくつの次元で構成されているのか知ることになるだろう.しかし,そもそも高次元空間の理解なしに多くの学問領域や私たちの生活はもはや立ちゆかなくなっていることを理解しておく必要がある.例えば,線形代数という数学の理論を取り上げてみよう.線形代数は統計学から計算機科学まであらゆる領域で基礎となっている一連の数学理論である.線形代数はベクトルを代数的に扱うことができて,ベクトル空間という幾何学的対象を計量することができる.ベクトル空間は実数や複素数の集合を扱うため,ベクトルが含んでいる集合の要素の数だけ次元が高くなる.4個の要素をとれば4次元となり,1000個の要素でとれば1000次元となる.ベクトル空間は,要素の数だけ無限に増えていくn次元空間なのだ.このn次元空間を扱う線形代数は,21世紀になって再び流行語となった人工知能の基礎となっている機械学習でも前提となっている.人間にとって認識できない高次元空間は数学的な概念道具としてなくなてはならない存在だ.

ところで,パヴロフスキーの『四次元郷』はこうした高次元空間を利用する未来を予想していたわけでは全くない.ハーバード・ジョージ・ウェルズ(Herbert George Wells)の『タイム・マシン(\emph{Time Machine})』(1895)を代表として,様々な論者が4次元に熱狂していたこの時代に,彼もまた4次元に魅せられ,時間と空間をめぐる理論を構築していくなかでこの小説を書いた.しかし,彼が残したこの小説は,4次元とは関わりのないような,科学技術に対する風刺のきいたグロテスクなコントや,哲学的なエッセイによって構成されている.そのため,先に紹介した中沢は,『四次元郷』を人類の愛が4次元への鍵となっているといったように,ごく一部分の要素だけを切り取って物語を解釈することはできない.『四次元郷』は複雑なテーマを有した物語なのだ.

中沢の他にも,これまで多くの論点が与えられてきた『四次元郷』について,本論はそれらをまとめた上で,新しい論点を提示する初めてのモノグラフとなるだろう.本論は,この著作を構成するエピソードの中に隠されたイメージを繋ぎ合わせることで,19世紀フランスにおける科学の大衆化によって人口に膾炙した4次元と遺伝というテーマが伏在する『四次元郷』の姿を明らかにする.
