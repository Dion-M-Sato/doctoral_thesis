\documentclass[uplatex,a4j,11pt]{jsarticle}
\renewcommand{\abstractname}{要旨}
%\renewcommand{\baselinestretch}{1.1}
\usepackage[vdivide={3cm,,3cm}]{geometry}
\usepackage{fixltx2e}
\usepackage{setspace}
\usepackage{wrapfig}
\usepackage{setspace}
\usepackage{indent}
\usepackage{fancybox}
\usepackage{textcomp} % T1のシンボルを使う
\usepackage[T2A,T1]{fontenc}%フォントでT2AとT1を使う
\usepackage[utf8]{inputenc}% ファイルがUTF8であること
\usepackage{zi4}%等幅フォントをInconsolataで
\usepackage[multi,deluxe,uplatex,jis2004]{otf}%繁簡ハングル多ウェイトOpenTypeなし
\usepackage[prefernoncjk]{pxcjkcat} % なるべく「半角」扱いで.
\cjkcategory{sym18}{cjk} % sym18 (U+25A0 - U+25FF Geometric Shapes) を和文文字あつかい

\usepackage{pxjahyper}
\usepackage{plext} %傍点をふる
\usepackage{pxrubrica} %ルビをつける
\usepackage{comment} %コメント環境の用意
\setcounter{tocdepth}{3}
\renewcommand{\thefootnote}{\arabic{footnote}}
\usepackage[dvipdfmx]{graphicx}
\title{\empty}
\author{\empty}
\date{\empty}

\begin{document}
\maketitle
\begin{center}
  {\Large 論文要旨}\\
  \vspace{30mm}
  {\large ガストン・ド・パヴロフスキー『四次元郷への旅』について}\\
\end{center}
\vspace{30mm}
\begin{flushright}
  言語情報科学専攻\\
  31-166004\\
  佐藤 正尚\\
\end{flushright}
\vspace{40mm}

本論はガストン・ド・パヴロフスキーの『四次元郷への旅(\emph{Voyage au pays de la quatrième dimension})』(以下,『四次元郷』)についてのモノグラフである.私は先行研究を体系的に分類し,独自の論点として,出版史と19世紀フランスの科学の大衆化の関係から作品の成立背景を分析し,4次元を取り扱ったほかの思想家や文学者との差異を指摘した.さらに,作品の中で特権的なテーマとなっている「物質(matière)」と「遺伝(hérédité)」の2つに注目し,それをパヴロフスキーの他の著作やフランスの科学史をふまえて,物質を構成している原子が3次元を観念的な偶然性によって規定しているということと,物質と生命,さらに思考には遺伝的類似による連続性があるということの2つを新しい解釈として示した.結論では,『四次元郷』のそれぞれのエピソードが,直接には経験することができない理念的な存在である4次元を,「観念的な偶然性」と「遺伝的類似」という2つのパースペクティブによって具体化していると捉え直すことによって,非ユークリッド幾何学の広がりの中で経験論を鋳直して「真の経験論」を示したベルクソンと同時代的な関係性があることを私は指摘した.この結論によって,パヴロフスキーがいくつかの研究で指摘されているように,単なるプラトニスト(あるいはネオプラトニスト)としては捉えきれないことが明らかとなった.

以下では,各章の概要を述べる.

第1章では,19世紀末の非ユークリッド幾何学の文化的流行を取り上げていた日本語文献として中沢新一の「四次元の花嫁」が『四次元郷』に言及していることを足がかりに現代社会における高次元空間の利用について触れる.

第2章では,ガストン・ド・パヴロフスキーの来歴と『四次元郷』の書誌情報,ならびにあらすじ,先行研究を取り上げる.パヴロフスキーの来歴では,1874年の誕生から1933年の逝去までを,大学生時代のジャーナリストしての活動,戦間期の活動,戦後の活動の3つの期間に分けて追った.『四次元郷』の書誌情報に関しては,先行研究の調査を参照し,連載時期や書籍として出版される前に予定されていたタイトルをまとめた.先行研究では,パヴロフスキーに関する言説を作り上げている文献をSF研究とモダンアート研究の2種類に大別し,解釈の傾向を示した.また,SF研究において,『四次元郷』で未来史を描くパヴロフスキーが歴史をどのように理解しているかの解釈で齟齬が生まれていることを指摘した.

第3章では,出版史とジャーナリズムを最初に取り上げて,『四次元郷』が新聞で連載されたのにはどのような背景があったのかを示しつつ,三面記事とロマン主義文学の相互関係が科学記事と科学小説の関係にも表れていることを指摘した.また,19世紀後半のフランスの科学小説についてSF研究を参照しつつ,『四次元郷』がそれまでのジャンルから逸脱していることを示した.

第4章では,『四次元郷』における4次元がどのように同時代の小説家や思想家と違っていたのかを示した.パヴロフスキーの4次元は,4次元という言葉を広めたH・G・ウェルズの『タイム・マシン』や4次元の考察から独自の哲学を打ち立てたC・H・ヒントンと,運動と時間の観点から大きく異なっている.また,パヴロフスキーの4次元はアンリ・ベルクソンの影響もあることを指摘し,ベルクソン研究者の芸術論との関係性を考察した.

第5章では,『四次元郷』の中で頻繁に用いられる物質というキーワードを手がかりに注目することで,物質を構成している原子と物質から生じるエネルギーという概念が鍵になっていることを示す.また,エネルギーが鍵概念であることを示しているエピソードの奇妙な顛末に注目することで,オーギュスト・ブランキとの関係性を浮かび上がらせた.そして,鈴木雅雄のブランキ論に依拠することで,パヴロフスキーにとって3次元は「観念的な偶然性」によって4次元と関係していることを示した.

第6章では,1912年版と1923年版の異同に注目しながら,進化論と遺伝という潜在するテーマがあることを示し,自然発生説に由来する物質と生命の連続性に加えて,思考もまた物質と連続的に関係しているということを「遺伝的類似」という概念によってまとめた.また,遺伝的類似によってパヴロフスキーが進化論と非ユークリッド幾何学を独自に解釈して組み合わせていることを指摘した.

第7章では,「観念的な偶然性」と「遺伝的類似」が4次元という経験不可能で抽象的な空間を具体的に表現するために各エピソードの中に示されているという考えを示すことで,パヴロフスキーはリーマン以降の数学的展開やベルクソンと同じように,新しい経験論を『四次元郷』において実践していたのではないのかということを示した.
\end{document}
