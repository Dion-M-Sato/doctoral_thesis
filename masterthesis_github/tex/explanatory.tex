\begin{center}
  {\Large 凡例\\}
\end{center}
\vspace{2cm}
\begin{itemize}
  \item 本文中引用での原文イタリック体または和文傍点はすべて原著ママ.
  \item 本文中にとくに断りのなく(数字/数字)が挿入されている場合,斜線左側には\emph{Voyage au pays de la quatrième dimension}の1912年版の引用頁数を,左側には1923年版の再版であるDenoël版の引用頁数を示している.例えば,(8/59)とあれば,1912年版8頁と同一の文章はDenoël版の59頁にあることを示している.
  \item 1912年版と193年版の章題は常に併記する.ただし,版ごとに章題の異同がある場合,それも併記する.例えば,1912年版第31章「自然の力の彼方へ(Au delà des forces naturelles)」1923年版第34章「自然の形態の彼方へ(Au delà des formes naturelles)」,表記した場合は,1912年版においては第31章である「自然の力の彼方へ」は,1923年版において第34章に変更され,章題が「自然の形態の彼方へ」となっていることを示している.
  \item 本文中にとくに断りなく(1912, 10)や(1923, 10)と挿入されている場合,それぞれその版にのみある文章の頁数を示す.例の場合,前者は1912年版10頁,後者は1923年版10頁からの引用を示す.
  \item 引用箇所で和文で【 | 】という記号がある時,縦棒左側は1912年版を右側は1923年版を示す.また,フランス語の原文で\{ | \}とある場合,縦棒左側は1912年版を右側は1923年版を示す.
  \item 引用した原文中に[= ]という記号がある時,原文の指示語の補足である.ただし,和文中に同様の記号が用いられている場合のみ,それは文脈の補足であり,必ずしも原文に対応する文章があることを意味しない.
  \item \emph{Philosophie du Travail}からの引用は例えば(PT10)と略記する.この場合,10頁からの引用を示す.
  \item \emph{Sociologie Nationale. Une définition de l'état}からの引用は例えば(SN10)と略記する.この場合,10頁からの引用を示す.
  \item (中略)ないし(...)と示した場合,それは引用箇所の一部を省略していることを示している.
  \item 『』の鉤括弧は書名を示し,『』(数字)とある場合,原著年の刊行年を()内の数字で示す.例えば,『科学と仮説』(1912)とあれば,『科学と仮説』の原著は1912年に刊行されたことを意味する.
\end{itemize}
