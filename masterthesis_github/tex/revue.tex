\chapter{出版史とジャンルの観点からみる『四次元郷への旅』}
\section{出版史における『四次元郷への旅』}
1980年代以降,ジャン=イヴ・モリエ(Jean-Yves Mollier)の研究以降\footnote{Jean-Yves Mollier, \emph{L'argent et les lettres, histoire du capitalisme d'édition, 1880-1920}, Fayard, Paris, 1988.},19世紀出版史は文学研究においてとみにその重要性を増している.その理由は,出版社ないし編集者による要請は作品のスタイルや内容にも大きな影響を与えていることが明らかになってきたからである\footnote{編集者と作家の関係ついては以下を参照のこと.Jean-Yves Mollier, «~écrivaint-édituer~: un face-à-face déroutant~», \emph{Travaux de littérature}, v. XV, t. 2, Paris, L'Adirel, 2002, pp. 17-39. 石橋正孝『〈驚異の旅〉または出版をめぐる冒険:ジュール・ヴェルヌとピエール=ジュール・エッツェル』,左右社,2013年,第1章.}.ガストン・ド・パヴロフスキーの作品を扱う場合も,それは例外ではない.なぜなら,彼は編集者でなおかつ作家であり,19世紀以降のフランスの出版文化を背景に活動していたからだ.パヴロフスキーの活動において出版文化の影響が特に表れているのは連載小説という形式とそれがもたらす内容である.そこで,新聞や雑誌で見られた連載という形式の誕生からまずは見ていきたい.

小説が雑誌に連載されることが連載小説であるとするなら,18世紀の時点から小説が分載されることは珍しくはなかった\footnote{ダニエル・コンペール『大衆小説』,宮川朗子訳,国文社,2014年,35頁.}.しかし,19世紀以降のフランスの出版市場の拡大はそれまでの連載小説の性質を大きく変えることとなった.出版市場の拡大は,行商や貸本屋が増えていったことや,カトリック系の教会の派閥による教化運動の一貫としての印刷物の普及,科学技術の発達に伴う技術書の需要の高まり,特に1933年のギゾー法によって各県に1つの小学校が設置されたことによる識字率の向上といった複合的な原因によってもたらされた.

この当時,小説は,詩や演劇と消費者が異なり,主に下層中産階級や職人・奉公人といった人々の娯楽の1つとなっていた.小説は,主に貸本屋や行商によって消費者に供給されていた.この受容の形態が大きく変化するのは,「連載小説(roman-feuilleton)」の普及によるものであった.新聞の最下欄にある連載欄(feuilleton)は科学,文芸,小説などが掲載されている.このシステムを最初に生み出したのは1800年代の『ジュルナル・デ・デバ(\emph{Journal des Débats})』だった\footnote{Lise Dumasy-Queffélec, «~Le feuilleton~», \emph{La Civilisation du Jounal, Histoire Culturelle et Littérature de la Présse Française au XIX\textsuperscript{e} Siècle}, éd. Dominique Kalifa et. al. , Paris, Nouveau édition, 2011, p. 925.}.ナポレオン帝政期に登場した連載欄は,当初政治的話題以外に触れる文化通信欄として用いられていた.七月王政期(1830-1848)に入ると,『両世界評論(\emph{Reuve des deux mondes})』や『ルヴュ・ド・パリ(\emph{Revue de Paris})』に定期的な発表に合わせて小説が分載されるようになり,若い世代の小説家たちをとり込んでいった\footnote{Marie-ève Thérenty, \emph{Mosaïques. être écrivain entre presse et roman (1829-1836)}, Paris, Honoré Champion, 2003, Paris, Honoré Champion, 2003.}.

1836年に商業広告を掲載することで値段を引き下げることに成功した『プレス(\emph{Press})』は,連載欄に小説を掲載することで売り上げ部数を伸ばしていった.しかし,現在の新聞でイメージされる連載小説は日毎に掲載されるというのが連載小説の形式と思われているが,それとは異なり,『プレス』では曜日ごとに掲載されるジャンルが異なり,月曜日は連載記事はなかった.以下に,どのような連載がなされていたのかを示しておこう.日曜日はアレクサンドル・デュマの歴史の名場面集,火曜日はフェデリック・スリエの短編の悲劇小説(feuilleton dramatique),水曜日はランベル(Lambert)博士の科学アカデミーに関する記事,木曜日は書籍・音楽・演劇の紹介,金曜日は農業や産業に関する記事,土曜日は海外の珍しい習俗などを伝える記事が載っていた.これらの連載のうち,日曜日と火曜日の小説の連載が新聞の売り上げに繋がり,多くの小説家がそちらの方に流れることとなった\footnote{Dumasy-Queffélec, \emph{op. cit. }, p. 926.}.こうして生まれたのが「連載小説」という形式だった\footnote{ただし,これを現代の連載小説の起源であるとは言い難いことも付記しておきたい.デュマジー=クフェレク(Dumasy-Queffélec)の研究によると,当時の代表的な小説家ウージェーヌ・シュー(Eugène Sue)が1837年から『アルチュール(\emph{Arthur})』の連載を『プレス』で始めた時,それは雑報欄(Variété)で始められており,連載が終了した1839年に連載欄だったことから中短編(courtes nouvelles)中心の連載欄と長編中心の雑報欄という使い分けが少なくとも1839年まではあったのではないかと考えられるという.以下を参照のこと.\emph{Ibid. }, p. 925.}.

連載小説が黄金時代を迎えたのはウージェーヌ・シュー『パリの秘密(\emph{Les Mystères de Paris})』が『ジュルナル・デ・デバ』で1842年6月19日から1843年10月15日まで連載されて時である.この成功をきっかけにして,1845年までには全ての日刊紙が連載小説の形式を受け入れた\footnote{コンペール,前掲書,39頁.}.連載小説の流行によって,多くの大衆小説家が生まれて,オノレ・ド・バルザックやジョルジュ・サンドが活躍し始めたのもこの頃だった.連載小説のブームは逐次刊行物の出版の誕生などに繋がり,文学そのものにも大きな影響を与えていくこととなる.

連載小説が同時代のロマン主義文学に与えた最も大きな影響とは,三面記事(fait divers)で話題となるような犯罪を取り上げる傾向が強まったことである.その具体例としてよく知られているのは,19世紀最も有名な犯罪者のフランソワ・ヴィドック(François Vidocq)が犯罪者から改心して正規の刑事にまで上り詰めて,『回想録』(1829)を執筆し,先の『パリの秘密』やロマン主義文学の参照点ともなった.このように,連載小説の流行と三面記事の流行は密接に関係していた.この流行は1863年,『プチ・ジュルナル(\emph{Petit Journal})』の刊行によって頂点を極める.『プチ・ジュルナル』はそれまでの新聞の3分の1程度の値段で販売され,犯罪,事故,痴情のもつれといったまさに三面記事的な話題を中心に取り上げていたこともあり広く読まれた\footnote{具体的な数字を述べておこう.初年38,000部程度だったこの日刊紙は,翌年には150,000部と約3倍に膨れ上がり,1867年には250,000部に到達する.パヴロフスキーの『コメディア』が刊行していた1910年にはその数は835,000部になっており,刊行されていた83種の日刊紙の総発行部数の1.5割程度が『プチ・ジュルナル』1紙によるものだった.下記を参照のこと.ルイ・シヴァリエ『三面記事の栄光と悲惨 近代フランスの犯罪・文学・ジャーナリズム』,小倉孝誠・岑村傑訳,白水社,2005年,199頁.\emph{Histoire générale de la presse française}, \emph{op. cit.} , p. 296.}.『プチ・ジュルナル』にも小説が連載され,ポンソン・デュ・テラーユの『謎の遺産(\emph{L'Héritage mystérieuse})』(1857)の連載の反響は大きく,「ロカンボル」という主人公が活躍するシリーズは多くの支持を得た.テラーユのロカンボルは大衆向けのシリーズ小説で1つの定型となる,蘇る主人公の造形の初出とも言われている\footnote{コンペール,前掲書,52-57頁.}.

19世紀後半を通じて犯罪を題材にとる小説は,連載小説という形式が一般的になっていくにつれてその傾向が強まっていくことは他にも多くの研究が示しているが\footnote{一例として以下挙げる.Christine Marcandier-Colard, \emph{Crimes de sang et scènes capitales : essai sur l'esthétique romantique de la violence}, Paris, PUF, 1998. },ロマン主義以外の文学作品にあったもう1つの局面を同じような構造において示すことができる.すなわち,科学の大衆化に伴ってその数が増えていくに従ってその数が増えていった科学記事と科学雑誌,そして現在はSFと呼ばれている一連の文学作品はちょうど三面記事とロマン主義と同じ枠組みにおいて語ることができると考えられる.こうした論点は今までほとんど示されてこなかった.

最初に,科学の専門的な内容が新聞でどのように取り上げられるようになったのかの歴史を見ていこう.1836年に登場した『プレス』の翌年,科学アカデミーの会議がジャーナリストに対して公開され,新聞でもその専門的な内容を解説するような新しいタイプの記事が生まれることとなった\footnote{下記の記述は注釈がない限り,以下に依拠している.Clauire Barel-Moisan, «~écrire pour instruire~», \emph{La Civilisation du Jounal, Histoire Culturelle et Littérature de la Présse Française au XIX\textsuperscript{e} Siècle}, éd. Dominique Kalifa et. al. , Paris, Nouveau édition, 2011, pp. 752-65.}.それが,連載科学記事(feuilleton scientifique)となっていった.『プレス』ではランベル博士による科学アカデミーの記事が週ごとに連載されていたことはすでに見た通りだが,この連載科学記事は19世紀後半の科学の大衆化を促した.その最初の契機を作ったのが,科学アカデミーでの会議の内容を題材にし,読者の関心を煽るような記事を書いたヴィクトル・ムニエ(Victor Meunier)などの存在である.彼は1853年3月22日の『プレス』にて,同紙の連載科学記事をいくつか取り上げ,その引用を組み合わせて自分の意見を加えて,「ハチの知性(Sur l'inteligence des abeilles)」や「月の住民(Les habitants)」という題名の記事を書き上げた.こうした大衆向けの科学記事は1880年代に頂点を極めて,その後どの新聞にも何らかの形で科学記事が連載されるのは,一般的なこととなっていく.その理由として,第二帝政以降,国家の産業化が促進され,科学技術に関係する媒体を支援していたので科学教育のための雑誌が多く求められていたことが挙げられる.さらに,普仏戦争のフランスの敗北によって愛国心から教育意識が高まったため,ドイツの科学力に対抗するための科学精神を養うことを謳ったに『ラ・ナチュール\emph{La Nature}』(1873),国際関係を意識して地理を取り上げた『ジュルナル・デ・ヴォイヤージュ\emph{Journal des voyages}』(1877)などが刊行される.この頃,新聞・雑誌メディア全般でジャーナリズム的言説のあり方が変化し,結果的に,ルポルタージュ・三面記事・インタビューの3つの領域で形成されるようになっていく\footnote{Marie-Yve Thérenty, \emph{La Littérature au quotidien. Poétique journalistiques au XIX\textsuperscript{e} siècle}, Paris, Seuil, 2007.}.その領域編成は連載科学記事の内容をも変化させた.異常気象や地震といった耳目を引くニュースが選ばれるようになり,『ジュ・セ・トゥ(\emph{Je sais tout})』が1905年に創刊され,同紙では,センセーショナルな内容の科学記事が増えていった.その一方で,1900年代に入る頃には,1860年代から続く科学精神を支えてきた実証主義派の巨魁たちが相次いで亡くなり,大衆に科学を広めようとするモチベーションが出版業界から去っていくにつれて,専門的な科学雑誌は次第に姿を消していった.
そうした中でも 1880年代から90年代を通じて人気を獲得していった科学雑誌は,レクリエーションを教育手段に用いた.『イリュストラシオン(\emph{L'Illustration})』はその中でも大きな成功を収めたが,その時に用いたフレーズが「楽しい(amusant)」であった.彼が編成したシリーズである「楽しい科学(La science amusante)」は1889年から1893年まで続き,それをまとめた書籍は40,000部を売り上げ,7ヶ国語に翻訳された.

この娯楽を重視した方針の変化によって,経済的に大きな成功をしたのが,小説による科学教育の実戦だった\footnote{本論では出版史の研究から科学小説による科学の大衆化について確認しているが,SF研究では以前よりこの教育的性格が指摘されている.Jean-Jacques Bridenne, \emph{La Littérature Français d'Iimagination Scientifique}, Paris, Gustave Arthur Dassonville, 1950, chapitre 1.}.ボーラン書店の経営者であり編集者でありエッツェルは『教育娯楽雑誌(\emph{magasin d'éducation et de récréation})』を1864年に刊行し,ジュール・ヴェルヌが「驚異の旅(voyage extraordinaire)」シリーズを連載すると,大成功を収めた.ヴェルヌのようなスタイルの科学小説(roman scientifique)は19世紀を通じて多く書かれたが,作者たちは教育的側面を意識して執筆していた.こうした小説のようにある特定の主題を意識して書かれているいわゆる問題小説(le roman thèse)と呼称されているが,問題小説の研究者シュザンヌ・シュレマン(Susan Suleiman)の指摘によれば,曖昧さを排して,記述が細部にわたる冗長さがあり,語の一義性に配慮しているという点を問題小説の特徴としてあげており\footnote{Susan Rubin Suleiman, \emph{Le Roman à Thèse ou l'Autorité Fictive}, Paris, PUF, 1983. },当時の科学小説がそうした側面を待っている.ヴェルヌの影響は大きく,たいていの科学小説は「冒険(aventure)」という言葉が題名につけられていた.ジャン・ド・ラ・イール(Jean de La Hire)が執筆した『火花散る歯車(\emph{La Roue fulguratnte})』は,当初「冒険の科学小説(roman scientifique d'aventures)」という題名だったことや,ファイヤール社が冒険叢書を出版したことなどを挙げられるだろうが,枚挙にいとまがない.大衆文学史家のジャック・ボドゥは,こうした科学小説におけるヴェルヌの影響の大きさとは別に,ウェルズの影響がもう1つの傾向を生み出していると指摘している\footnote{ジャック・ボドゥ『SF文学』,新島進訳,白水社,2011年,62-3頁.}.ウェルズが初めて著した科学小説は『タイム・マシン』(1895)で,1898年12月と1899年1月に分けて大衆向けではない文学性の高い雑誌『メルキュール・ド・フランス\emph{Mercure de France}』に翻訳され,アルフレッド・ジャリからプルーストまで幅広く読まれた.『タイム・マシン』は未来予想の小説の典型をこの時点で早くも生み出していたが,自然主義運動に参加していたJ・H・ロニー兄は,全く逆に,はるか遠い昔の時代を描いた『火の戦争\emph{La Guerre du Feu}』(1909)などを書き,先史時代の物語というジャンルを作り上げた.ボドゥは,これらいずれもヴェルヌを代表とする大衆小説とは違った受容がなされていたと考えている.

ここまで,科学小説のほとんどは新聞か雑誌に連載されたものがまとめられることで出版されており,三面記事とロマン主義の相互関係のように,科学小説はメディアのニーズすなわち教育を意識した内容がそのほとんどを占めていくことを示してきた.しかし,フランスの出版史における三面記事の影響を完全に免れたわけではない.コンペールは,19世紀後半に著名だったサイエンスライターで編集者であるルイ・フィギュエ(Louis Figuier)が1887年に創刊した『ラ・シアンス・イリュストレ(\emph{La Science Illustré})』の科学記事に自然現象や事故をドラマチックに報じたり,科学的観点からの分析ではなくて別の科学記事を引用し意見を加えていく手法が見られることを具体的に検討して「科学的知識を\bou{揺らがせて}いたものかのような(comme ce qui vient \emph{étonner} le savoir scientifique)\footnote{Daniel Compère, «~Fait divers et vulgarisation scientifique~», \emph{romantisme}, n. 97, Paris, Armand Colind, 1997, p. 76.}」状況になっていたものの,それは同時に科学記事のスタイルを変化させていただけで,科学知識をないがしろにしていたということでは決してないと指摘している\footnote{\emph{Ibid}.}.SF研究では,一般的に1870年代は普仏戦争や植民地戦争の激化の時代に合わせた戦争を題材にした近未来小説が増え,政治を題材にした小説がSFを要因として増えた時期だとされている点も,こうした三面記事と科学記事の関係を例示していると言える.ダルコ・スーヴィンは「物語は,超兵器の力に頼るだけで,(略)心理的側面の真実を想像できない無能さだけをぶざまにさらけだしている\footnote{ダルコ・スーヴィン『SFの変容』,大橋洋一訳,国文社,1991年,264頁.}」と評しているが,少なくとも出版史の観点からすると,それは科学アカデミーの会議が公に開かれて以来始まった科学記事の歴史の中で,三面記事化していった科学小説の姿であると言える.しかし,科学小説は科学専門誌が1900年代を超えるとほとんど生き残れなかったのと同様に,1920年代を頂点にして,それ以後は顧みられなくなっていく\footnote{ジャック・ボドゥ,前掲書,66頁.}.

\section{ジャンルを逸脱する『四次元郷への旅』}

私たちは,フランスにおける出版史の,とりわけ三面記事との文学の関係を見てきた.三面記事はロマン主義の多くの文学作品に影響を与え,教育を意識して書かれた科学記事もまたその影響を受け,同時代の合わせ鏡のように三面記事的な科学小説も増えていったのである
.『四次元郷』もまたその例外ではなかった.

『四次元郷』が執筆された時代を振り返っておくと,同作品は1900年代に連載が始められ,最終的に1912年に出版された.すでに,科学記事が連載の形で新聞に載るのは一般的だった.また,1860年代以降のヴェルヌの活躍により,1870年代生まれのパヴロフスキーらにとってヴェルヌを代表とする科学小説を読むことも極めて一般的なことだったと考えられる.すると,自身が編集する『コメディア』が演劇・文芸が中心の日刊紙でありながら,パヴロフスキーが投稿する科学的な知識を盛り込んだ小説が掲載されることや,のちに検討するように,パヴロフスキーが『四次元郷』に科学記事から再引用しつつ自分の文脈と繋げていく三面記事的な手法を用いていることは,自然なことだったのだ.しかし,同時代の科学小説に比べて,『四次元郷』はどのジャンルにも属し難い異なる特徴を持っていた.

『四次元郷』は未来を旅することで将来の社会を知るという点で,いわゆる未来空想物語(le roman d'anticipation)やウェルズの『タイム・マシン』的な筋書きを持っている.しかし,一読すればわかるように,ジャンルやあらすじの形式だけに注目することはこの小説の理解を困難なものとする.それを如実に示しているのは,ヴァン=エルプによる同作品の位置付けである.ヴァン=エルプは\emph{Panorama de la science fiction}の第2章「時空間の回廊で(Dans les corridors de l'espace-temps)」で,時空間の移動を題材にした小説を7種類に分類している\footnote{以下の通り.時間の中の旅(Le voyage dans le temps) ,パラドックスの時間(Le temps des paradoxes),操作された時間(Le temps manipulé),歴史改変(Les uchronies),ポリティカルフィクション(Politique-fiction),並行宇宙(Les univers parallèles),4次元(Quatrième dimension).}.そのほとんどは時間の移動に注目した分類項目やそこから発展したフィクションの形態に基づく分類である.最後に「4次元」という分類を設けている.そこではパヴロフスキーがまず紹介され,その淵源にアボット・アボットの『フラットランド』(1884)のような幾何学を題材にした小説群の系譜があるとされ,1930年代以降の系譜が辿られる.しかし,『四次元郷』で幾何学的な題材が扱われるのはほとんど冒頭の章だけで,あとは原子論的な宇宙観が提示され,リヴァイアサンの支配と解放が描かれる.『フラットランド』は2次元人と1次元人の対決を描いており,ヴァン=エルプが示す後継作品も異なる次元どうしの接触がテーマとなっていることからすると,『四次元郷』はそうしたジャンルには全く当てはまらない.こうした見解は他の論者も共有しており,ジャック・ボドゥは『SF文学』の「四次元およびその他の次元」というジャンルの作品として,20世紀のSF史上最も有名な「四次元」を名に冠したパヴロフスキーの同作品を挙げていない\footnote{ジャック・ボドゥ,前掲書,120-1頁. }.

異次元空間を舞台にしている作品ではないとしても,やはりウェルズや同時代の異次元ブームとの関係性を論じている研究者がいる.非ユークリッド幾何学の文化的流行が芸術に与えた影響を多角的に論じたヘンダーソンはパヴロフスキーをウェルズの「フランスの門弟(French disciple)\footnote{Henderson, \emph{op. cit. }, p. 134.}」と見做している.パヴォロフスキーはウェルズのような社会的批判をしながら\footnote{パヴロフスキーは第一次世界大戦後に手を加えた1923年版の『四次元郷』で,「その時代の科学の圧制に対抗する抗議(une protestation révolté contre tyrannie scientifique du moment)」(1923, 10)の試みであると述べているように,同時代の科学技術に対して楽観的な意見を持ち合わせていなかった.この態度は,科学の進展に対する興味について,オーギュスト・ブランキやシャルル・クロと比すべきものがあるというのが本論における私の主張でもある.},高次元空間の哲学を展開した数学者チャールズ・ハワード・ヒントン(Charles Howard Hinton)から続く文脈との「特有のブレンド(unique blend)\footnote{\emph{Ibid. }, p. 151.}」であると述べている.ところで,この「特有のブレンド」という言葉こそ『四次元郷』を特徴づけるものなのだ.確かに,SF研究では後続の作品と合わせて先駆的作品を決定していくために,注目する点は物語で採用されている舞台や展開,あるいは登場する機械などに絞られている.しかし,『四次元郷』で展開される世界観や,断片的なエピソードだけの報告といった三面記事的構成,非ユークリッド幾何学や原子論などの科学的知識の紹介といった科学記事的な文体の特徴は後続の作品で取り上げられることはなかった.また,同時に,それはヴェルヌに影響を受けた冒険譚とも違うし,未来社会の予想図が統一的に描かれていない.本書では触れなかったが,アルベール・ロビタ(Albert Robita)は,『20世紀(\emph{Le Vingtième Siècle})』(1883)に代表される未来の社会の様子を断片的に描く作品を残しているが,ヴェルヌらが常に同時代の科学知識から予見されるうる出来事を中心に描いていると評されるのと同様に\footnote{ジャン・ガッテニョ『SF小説』,小林茂訳,1971年,14頁.},ロビタの描く世界は物理的な現実に根ざしている.その一方で,パヴロフスキーは「心霊主義の小説家」とボダンに言われるように,魂が実在する世界を描き,現実の世界を突き放している.『四次元郷』という作品は非ユークリッド幾何学といった数学的知見や,化学や生物学,時には精神医学の知見が活かされているのにもかかわらず,この現実を描く意志を欠いてしまっているのだ.それは,カイヨワの言葉をもじるのであれば,まさに「SFから妖精へ」と言える幻想小説への回帰なのかもしれない.しかし,パヴロフスキーは「四次元は私たちの世界の理解を完全なものにする(Elle[=~la vision de la quatrième dimension] complète notre compréhension du monde)」(1/55)と述べているように,あくまでもこの世界の真実について論じようとしていたのであり,幻想小説と単純に分類することはできない.

以上のように『四次元郷』は出版史の観点からはその位置付けを明確にできる.しかし,SF研究からは曖昧な定義しかできない.さらに,文化史的な観点を援用することで小説と哲学の両方から作品の特徴を定めることが可能である一方,その試みは同時代に比べて評価が定めにくいことが明らかになった.そこで,次章では,4次元を一般に広めたパヴロフスキーの活動に注目することで,『四次元郷』がフランスの4次元の流行にどのように関わったのたかを知ることで,文化史的な位置づけを補足したい.

