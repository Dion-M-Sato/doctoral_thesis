\chapter{4次元とベルクソン}
\section{4次元について}
\subsection{フランスにおける4次元の大衆化}
第3章では,『四次元郷』を出版史的な観点からその姿を位置付けることができるということを示した.実際は,パヴロフスキーが追悼記事で4次元の大衆化の立役者であったと評せられているように,『四次元郷』は4次元に関する科学記事のような役割を持っていた.以下では,第3章の最後で示した,ヘンダーソンが提示したパヴロフスキーの評価,すなわちウェルズとヒントンの「特有のブレンド」が具体的に何を意味するのかを,4次元の文化史を示すことで確認したい.

4次元の文化史的な起源を線・平面・立体の3つの次元に加えて第4の次元がある,という考えがあったことだとすれば,それは18世紀に求められるだろう.1754年にはダランベールが『百科全書』の\emph{Dimension}の項目の中で,もう1つの次元としての第4の次元は時間であることを指摘していることが知られている\footnote{次がその1節にあたる.「私は3次元以上は認識できないとはっきりと述べた.その一方で,私の知人の才気ある男は,持続は4次元だみなしうると信じている(J’ai dit plus haut qu’il n’étoit pas possible de concevoir plus de trois dimensions. Un homme d’esprit de ma connoissance croit qu’on pourroit cependant regarder la durée comme une quatrieme dimension)」(引用箇所は以下を参照のこと.\emph{Encyclopédie ou Dictionnaire raisonné des sciences, des arts et des métiers}, v. IV, 1754, pp. 1009-10. また,ENCCREが電子版で公開しているウェブページは次の通り.http:\slash\slash enccre.academie-sciences.fr\slash encyclopedie\slash article\slash v4-2546-0\slash )持続(durée)とは,ある運動などがある一定期間続く意味での時間のことである.}.結論から言えば,この18世紀的な4次元の定義が一般に流布するようになったのは,1895年のウェルズの『タイム・マシン』が登場してからのことだと考えられる.しかし,数学の研究における4次元は,ギリシャ以来のユークリッド幾何学の公理を超える新しい数学的概念として発明されたものであり,時間とは全く関係のないものであった.そこで,数学史における4次元の歴史を以下で概観する.

19世紀になると,数学者たちは,ユークリッド幾何学での平行線の公理や内角の和が180度になる公理などに従わない空間の存在を考えていた.いわゆる非ユークリッド幾何学である.この新しい幾何学の研究者として,カール・ガウス(Carl Gauss),ニコライ・ロバチェフスキー(Nikolai Lobachevsky), ヤノス・ボヤイ(Jànos Bolyai),エウジェニオ・ベルトラミ(Eugenio Beltrami),ベルンハルト・リーマン(Bernhard Riemman),ヘルマン・フォン・ヘルムホルツ(Hermann von Helmholtz),フェリックス・クライン(Félix Klein)が挙げられる.このうち,パヴロフスキーはロバチェフスキー,ベルトラミ,リーマンの名前を挙げている(11/60).比較的早い時期からガウス,ロバチェフスキー,ボヤイらは独自に非ユークリッド幾何学の可能性を模索していった.特に,ベルトラミの擬球が有名である~図~\ref{fig:pseudo}\myfig[height=6.4cm, width=9.1cm]{pseudo}{ベルトラミの擬球}{pseudo}\footnote{CC BY-SA 3.0で公開されている以下の画像を用いた.https:\slash\slash commons.wikimedia.org\slash wiki\slash File\%3APseudoSphere.jpg}.3人の成果以降,ベルトラミやヘルムホルツは,さらに発展的な研究をした.それはリーマンの研究が基礎となっていた.

リーマンが1854年にゲッティンゲン大学での教授資格審査(Habilitation)で発表した「幾何学の基礎にある仮説について(über die Hypothesen, welche der Geometrie zu Grunde liegen)」が1868年にゲッティンゲンの王立科学学会紀要にて出版され\footnote{\emph{Abhandlungen der Königlichen Gesellschaft der Wissenschaften zu Göttingen}, t. 13, 1868, pp. 133-150.},ギョーム=ジュール・オウエル(Guillaume-Jules Hoüel)が著名な数学誌『純粋・応用数学年報(\emph{Annali di matematica pura ed applicata})』にフランス語で訳出した後であった\footnote{Georg Friedrich Bernhard Riemann, «~Sur les hypothèses sur lesquelles est fondée la géométrie~», \emph{Annali di matematica pura ed applicata}, t. III, n. 2, trad. Jules Hoül, Heidelberg, Springer, 1870, pp. 309-26.}.リーマンがそこで主張していたのは,幾何学的対象となっているユークリッド幾何学は実は経験に基づいたものであり,そもそも幾何学の対象となる空間は全て仮説にほかならないので,一切の経験的対象から独立し,それ自体で自立した幾何学的な対象が考察しうるという主張だった.リーマンはさらに,その幾何学的対象は計測できるのだと考えた.これは,当時においては画期的な発想だった.ユークリッド幾何学における図形は経験的な対象で可視的な存在であり,外部との関係で位置を特定することができるからこそ計量ができるが,それ自体で自立している幾何学的対象を計測する方法をその時代の人々は知らなかった.リーマンはその問題を解決するために,それ自身によって存在する幾何学的対象を多様体(Mannigfaltigkeitslehre)と定義した.ユークリッド幾何学の図形の計量は他の図形との比較によって可能となっていたが,多様体では,ものさしとなる線素\footnote{ベクトル解析を行う場合の基本単位.空間の距離を意味する.}の無限小の運動可能性を仮定することで,条件によって変化し続けても計量できると考えた.この考えはユークリッド幾何学で扱われていた対象がいかなる運動においても変形しない対象(いわゆる剛体)であったのに対して,対象の変形(可変曲率)をも計測できるようになったことによって,あらゆる曲面を計測するための道を開いたのだった\footnote{以上のリーマンの解釈は以下を参照のこと.加藤文元『リーマンの数学と思想』,共立出版,2017年,第4章.}.しかし,リーマンの多様体の概念が論文の形になった後でも,一部の専門家にしか知られていなかった.一連の非ユークリッド幾何学が一般に知られていくようになるのは,数学に造詣のあるサイエンスライターや数学者たちによる一般に向けての解説によるものだった.フランスではその仕事の役割を担ったのは,とりわけアンリ・ポアンカレ(Henri Poincaré)であった.

では,ポアンカレが担った非ユークリッド幾何学の普及は,ポアンカレ以前のフランスではどのように進んだのかをその歴史的文脈から見ていこう.19世紀フランスで非ユークリッド幾何学を数学の専門家以外が扱った例として有名なものは哲学者イポリット・テーヌ(Hippolyte Taine)だろう.彼は1870年の著書『知性について(\emph{De l'Intelligence})』において,「読者は自然の中には存在しない幾何学的対象を知っている(le lecteur sait que les objets géométriques n'existent pas dans la nature)\footnote{Hippolyte Taine, \emph{De l'Intelligence}, Paris, Hachette, 1870, p. 57.}」例として,曲面や球体を挙げている.また,19世紀後半を通じてフランスでは,実証主義に由来した経験主義的な雰囲気が支配的だったためか,テーヌは非ユークリッド幾何学は現実で経験することはできないが,計算で扱うことができるものであるので感覚によって捉える必要がないと結論づけている\footnote{Taine, \emph{op. cit. }, pp. 58-9.}.こうした議論に関係し,ヘルムホルツの高次元空間に関する考察で,雑誌『哲学(\emph{Revue Philosophique})』や『形而上学と道徳(\emph{Revue  de Métaphysique et de Morale})』は1880年代から1890年代を通じて定期的にその翻訳が掲載された.ヘルムホルツはユークリッド幾何学の信奉者たちがカントの幾何学的時空間は直観によって把握されているものであり,非ユークリッド幾何学で想定される高次元空間はその直観では正当化できないとしていたのに対して,高次元空間もまた直観によって把握できるうえ,それ自体独立して表象もできると考えていた.その後,高次元空間の表象については,エスプリ・パスカル・ジュフレ(Esprit Pascal Jouffret)によって通俗化されたイメージがもたらされた.彼は1903年に『4次元幾何学序説(\emph{Traité de géométrie à quatre dimension})\footnote{Esprit Pascal Jouffret, \emph{Traité de géométrie à quatre dimension}, Paris, Gauthier-Villars, 1903.}』を発表して,3次元において4次元がどのように表現しうるかというを,エドウィン・アボット・アボット『フラットランド(\emph{Flatland})\footnote{Edwin Abbott Abbott, \emph{Flatland~: A Romance of Many Dimensions by Square}, London, Seeley \& Co. , 1884.}』の2次元人と3次元人の接触の場面を引き合いに出している\footnote{Jouffret, \emph{op. cit. }, p. 187.}.『フラットランド』はフランス語に翻訳されていなかったものの,非ユークリッド幾何学の発明より後に書かれるようになった異なる次元の空間どうしの接触というテーマはすでに英語圏ではよく知られていた.ジュフレの著作によってこのテーマはフランスに持ち込まれ,何人かのキュビストやマルセル・デュシャンなどに影響を与えたと考えられている\footnote{Henderson, \emph{op. cit. }, Chpater 2 and 3.}.

ジュフレーによる通俗化された高次元空間だったが,ポアンカレはヘルムホルツが行なっていた議論を再び取り上げることで,イメージとしてではなくて人間の近くにおいて高次元空間は把握できるのかどうかを解説した.ポアンカレはヘルムホルツの高次元空間をめぐる議論を大衆向けの著作『科学と仮説\footnote{Henri Poincaré, \emph{La Science et l'Hypothèse}, Paris, Ernst Flammarion, 1902.}』(1902)の中で,高次元空間の表象可能性は,数学の対象である幾何学空間におけるものなのか,それとも知覚によって把握される表象空間(espace représentatif)におけるものなのかを区別しなければならないと主張した.幾何学空間は連続(continu),無限(inifini),3次元空間を持っている,等質である(homogène)\footnote{空間上のあらゆる点はどの点についても点どうしは同じであるということ.あらゆる点の性質が同じであることを意味する.},等方的である(isotrope)\footnote{ある同一の点を通る空間上の全ての線についても線どうしは同じであるということ.あらゆる直線の性質が同じであることを意味する.},という5点によって定義される\footnote{Poincaré, \emph{op. cit.} , p. 69.}.一方で,表象空間は視覚的・触覚的・運動的(motrice)の3つによって定義される.この3点は,等質的でも等方的でもないのは自明であるので,「誰一人として3次元があるということさえ言うことはできない(on ne peut même pas dire qu'il ait trois dimension)\footnote{\emph{Ibid.} , p. 74.}」ということになる.ポアンカレはこれらの帰結として,空間をどのように扱うかはその時の「規約(convention)」に従うものであるとした.ヘンダーソンは,ヘルムホルツが1つの空間の中で高次元空間の実在を是非を思考したのに対して,ポアンカレは規約(convention)という概念を導入することで複数の空間があることを認めて,これらの問題を解決したが,それによっていわゆる「4次元世界の可能性を開いたままにしている(leaves open the possibility of a four-dimensional world)\footnote{Henderson, \emph{op. cit. }, p. 137.}」と指摘している.ポアンカレはその後の著作\footnote{ポアンカレは1904年に『科学の価値』を,1908年に『科学と方法』を出版した.Henri Poincaré, \emph{La Valeur de la science}, Paris, Ernest Flammarion, 1904. Henri Poincaré, \emph{Science et méthode}, Paris, Ernest Flammarion, 1908. }でも数学と物理学で議論となっていることやその理論的な背景を大衆に向けて発表しつづけ,それに伴って非ユークリッド幾何学の存在が広く知られていくこととなった.

『四次元郷』が出版された1912年にポアンカレは逝去した.1908年から連載されている『四次元郷』が取り扱っているテーマが読まれる下地を作ったのはポアンカレだったが,パヴロフスキーは『四次元郷』の中でポアンカレについて次のように述べている.
\begin{quote}
球面の表面を移動しているかもしれない平面の人間たちは,常に180度より大きい三角形の内角の和における幾何学を自然と考えているだろう.同じように,立体がない世界では,私たちの幾何学はわずかばかり苦しめられるのが明らかとなるだろう.アンリ・ポアンカレは優れた洞察力のある一節でこのテーマについて書いている.
\end{quote}
\begin{quote}
Des êtres plats, qui se déplaceraient sur une surface sphérique, concevraient tout naturellement une géométrie dans laquelle la somme des angles d'un triangle serait toujours supérieure à deux droits. De même aussi dans un monde dépourvu de solides, notre géométrie pourrait éprouver quelque peine à se faire jour. H. Poincaré a écrit sur ce sujet des pages fort clairvoyantes.(28/70)
\end{quote}
ここで述べられているポアンカレの一節とは切断(découper)の手法についてである.切断は高次元の対象を部分的に低次元に置き換えて考える手法である.例えば,3次元の立体は切断するとそこに2次元の平面が現れ,その平面もまた切断することで1次元の線が現れる.高次元の対象をそのまま扱うのは困難なので,例えば4次元の対象を切断してその断面を調べることで元の対象の全体を復元することができる.パヴロフスキーは,切断によって「私たちのユークリッド的な科学は欠けたものとなり消え失せる(notre science euclidienne fait défaut et s'évanouit)」(28/70)と考えている.このことから,パヴロフスキーは独自の論点を提出する.すなわち,切断は高次元空間を低次元空間に翻訳するという行為であり,それは科学の用いている慣用的な(conventionnel)言語が,物理的な現象を基礎付けている「質的世界(le monode des qualités)」を扱うことができないことを示している.この「質的世界」とはパヴロフスキーによれば4次元のことである.パヴロフスキーは『科学と仮説』や『科学の価値』で知られていたポアンカレの著作を引き合いに出すことによって自らをフランスの4次元の文化史的文脈に置きつつ,独自の4次元観を示したのだった.

\begin{comment}
Nous pouvons découper des volumes au moyen de surfaces. Nous pouvons découper des surfaces au moyen de lignes, nous pouvons déterminer des lignes au moyen de points. Mais, lorsqu'il s'agit pour nous de définir le point, notre science euclidienne fait défaut et s'évanouit. Lorsqu'il nous faut rendre compte du continu physique, notre impuissance est extrême. Nous comprenons bien que la science n'est autre chose qu'un langage conventionnel qui nous permet de cataloguer et de classifier certaines fractions de phénomènes que nous détachons artificiellement l'une de l'autre, d'après leurs qualités, mais nous sentons bien que cette science, de même que le langage, est incapable de traduire cette continuité qui appartient au monde des qualités et que l'on ne saurait définir par des chiffres.
4次元小説の代表例
ただし,これはEncyclopédie de l'utopie des voyages extraordinaires et de la science fictionのDimensionの項目に依拠している.

1844, Edwin. A. Abbott, Flatland, une aventure à plusieurs dimensions
1886, C. H. Hinton, A Plane World, Scientific Romances
1893, Ambrose Bierce, Charles Ashmore’s Trail, Can such things be?
1895, H. G. Wells, Un étrange phénomène
1896, H. G. Wells, L’histoire de Plattner
1921, Austin Hall, The Blind Spot
1922, Gabriel de Lautrec, Dan le monde voisin…, La vengeance du portrait ovale
1923, Claude Farrere, LA-BAS, Où?
1925, Jean Ray, Les étranges études du docteur Paukenschlager, Contes du whisky
1932, Homer Eon Flint, The Spot of life
1944, Léon Groc, La planète de cristal
1944, René Barjavel, Le Voyageur imprudent
1957, David Ducan, La raiser d’Occam
1969, Harlen Ellison, A Boy and his Dog
\end{comment}
\subsection{ヒントンの超空間哲学とパヴロフスキー}

パヴロフスキーがポアンカレの著作を踏まえたことで自らを4次元の文化史に位置付けていることはすでに示した通りであるが,それだけでは『四次元郷』の4次元の特徴は明らかにはならない.なぜなら,前章で述べたように,『四次元郷』における4次元はパヴロフスキーの独自の哲学を背景にしているために,他の4次元ないし異次元を扱った小説とは一線を画しているからである.ヘンダーソンは,それに対して,ウェルズとヒントンの「特有のブレンド」としてパヴロフスキーを4次元の文化史の中で位置付けようとした.そこで,4次元空間の哲学を展開したヒントンの哲学を概観することで「特有のブレンド」の意味を具体的に検討していきたい.

リーマンの多様体の影響が徐々に広まっていく中,1870年代になると,リーマンとは別のアプローチでn次元幾何学の一般化が試みられるようになっていく.1880年,アーヴィング・ストリングハム(Irving Stringham)が現在は\emph{Polytope}と呼ばれる図形のうち,4次元の場合の図形である\emph{Polychoron}(\emph{Polyhedroid})を図示した\footnote{Washington Irving Stringham, "Regular Figuers in n-Dimension Space", \emph{American Journal of Mechanics}, III, 1880, pp. 1-12.}.これは各次元の最小構成の図形を次元が増えるごとに構成し直すというアイディアに基づいている.例えば,2次元空間で線や面の数が最小の構成は三角形である.3次元空間では立体である.このことから,4次元空間では5つの四面体から構成された図形が最小の構成であると言える.ストリングハムの\emph{Polychoron}はそうして構成される図形のパターンを描いたものである.このように高次元を図形で表現した数学者としてチャールズ・ハワード・ヒントン(Charles Howard Hinton)の名前を挙げることができる.ヘンダーソンはヒントンを高次元空間に関する哲学を生み出した「真の超空間哲学者(true hyperspace philosopher)\footnote{Henderson, \emph{op. cit.} , p. 127.}」であると述べている.超空間哲学の核にあるのは,人間の空間感覚(space sence)の拡張によって3次元を超えた空間を知覚できるようになるという信念だった.1880年代にヒントンは精力的にこの思想を練り上げていき,ついに『思考の新時代(\emph{A New Era of Thought})』(1888)を,1904年には『四次元(\emph{The Fourth Dimension})』を刊行した.この2つの著作で展開されているのは,経験に基づいたユークリッド幾何学を直観として考えているカントを批判しているような,リーマンを代表とする非ユークリッド幾何学的な考えとは逆に,カントが空間を直観によって把握できるとすることは正しく,その直観を4次元にまで適応させればよいという経験主義的な理論であった.そして,その直観を手に入れるために考案された道具が\emph{tesseract}だった.その姿は『四次元』の口絵で非常によく知られている~図~\ref{fig:tes}\footnote{下記口絵を参照のこと.Charles Howard Hinton, \emph{The Fourth Dimension}, London, S. Sonnenschein, 1904.}.
\myfig[height=18.2cm, width=12.8cm]{tes}{\emph{tesseract}}{tes}
\emph{tesseract}が重要なのは,訓練方法によってヒントンが4次元の特徴をどのように示しているか明確になるからである.\emph{tesseract}による訓練の意義は,2次元と3次元の関係に置き換えると理解しやすい.もしも2次元に住んでいる人がいたとすれば,3次元の球体がそこをとおりぬける時,円が徐々に広がっていき,しばらくするとそれが閉じていくように見える.すわち,3次元の一部分しか知覚しえないのである.そうであるならば,4次元の図形が3次元を通り過ぎるときも同様でないかとヒントンは考え,4次元の図形の一部分はカラキューブのバリエーションとして表現されうると考えた.この時,ヒントンは4次元を運動において捉えようとしており,運動の持続する時間の中でしか捉えられないと考えていた.彼によれば,「あらゆる4次元を視覚化する試みは無駄である.それは3次元の中の時間的経験と関係しているに違いないからである(All attemps to visualize a fourth dimension are furtile. It must be connected with a time experience in three space)\footnote{Hinton, \emph{op. cit. }, p. 207.}」.また,ヒントンの超空間哲学のもう1つの特徴は自らをプラトンのイデアやカントの物自体と結びつけていることにある.『四次元』の第4章と第5章は4次元空間の歴史に当てられていて,プラトンのイデアの定義はより高い次元のアナロジーとして考えられるし,カントの物自体はヒントンにとって啓示されうる3次元を超えたものだった.そして,超空間を私たちが知覚しうるのは,3次元は4次元に延長しうるものであるからである.

ヘンダーソンは以上のようにヒントンの超空間哲学をまとめているが,確かに,パヴロフスキーの描く4次元と似ているように思われる.『四次元郷』では4次元の存在を確信するのは,あらすじで見たように,身辺で生じた異常な現象である.それは知覚によって4次元と接しているということである.また,本論では第6章で詳しく述べるように,パヴロフスキーは3次元は4次元という入れ物の中身であるかのような表現をしているので,3次元は4次元の延長であるかのように描かれている.しかし,次の一節はヒントンとも僅かに異なる.
\begin{quote}
3つの次元に加えられた第4の尺度のような,というよりもむしろ宇宙を理解するプラトン主義的方法のようであり,プラトン主義的方法のためにアリストテレスと仲違いする必要もなしに,永遠かつ不動の4次元の様相において物事を理解し,行為の質よりほかに到達しはしないための,量における運動からの解放を可能とする脱走方法のような,4次元.
\end{quote}
\begin{quote}
la quatrième dimension comme une quatrième mesure ajoutée aux trois autres, mais plutôt comme une façon platonicienne d'entendre l'univers, sans qu'il soit besoin pour cela de se brouiller avec Aristote, comme une méthode d'évasion permettant de comprendre les choses sous leur aspect éternel et immuable et de se libérer du mouvement en quantité pour ne plus atteindre que la seule qualité des faits.
\end{quote}
ここでプラトン主義に親和的である4次元の存在はやはりヒントンと関係していると言えるが,一方で,パヴロフスキーの4次元には「量における運動」は存在しないので時間もない.よって,「永遠かつ不動」なのである.ヘンダーソンが『四次元郷』を「特有のブレンド」と称している意味は,この差異がヒントンとパヴロフスキーを分けているということなのだ.

\section{ベルクソンとパヴロフスキーの4次元}
4次元の文化史的な観点からは,ヒントンとパヴロフスキーの相違点が明らかになった.ヘンダーソンはヒントンとの比較で時間と運動に注目したが,ヒントンとは別の人物とも比較している.それは4次元についての小説や哲学的テーゼを打ち出しているわけではないが,パヴロフスキーや彼と同時代の思想家・芸術家にとって重要な参照点だった.その人物はアンリ・ベルクソンという.

パヴロフスキーにとってアンリ・ベルクソンが大きな参照点であったことは,彼の博士論文『労働の哲学』の執筆時に『意識に直接与えられたものについての試論(\emph{Essai sur les données immédiates de la conscience})』(以下,『直接与えられたもの』)の議論を検討していることからもよくわかる(PT166).以下で,その議論について見てみよう.パヴロフスキーは『労働の哲学』で,社会理論(théories sociales)を構築するために科学を用いることはできないと述べている.というのも,社会とは個人で構成されているものであるから,社会理論とは,個人の行為を理論的に説明するものであり,個人の行為とは,意志と自由の問題であり,科学の対象とはならない.なぜなら,科学は私たちの外側にある世界について数量的に扱っているのに対して,私たちが生きている状態である質を扱うことができないからである.この根拠として用いられているのが『直接与えられたもの』であった.

パヴロフスキーは1923年の自著改題である「批判的吟味」や,1912年版の冒頭でも4次元について多くの言葉を費やしているが,それの多くは,質(qualité)と量(quantité),あるいは持続(durée)と同時性(simultané)という2つの言葉を鍵概念にして語られている.例えば,パヴロフスキーは精神が日々の複雑な活動を把握する方法は連続しているものをそれぞれ理解するのではなくて同時的に理解するほかないと考えている.パヴロフスキーは次のようにまとめている.
\begin{quote}
4次元についての全体的な理解の知性的なこれらのきらめき,私たちは時間の中のある持続を自然とそのきらめきだとしてしまうし,あまりにもきらめきがほのかなので私たちはそれを少なくとも何らかのつかの間の持続だと思っている.しかし,その持続は存在さえしない.というのは,4次元世界の中で持続がありえないようだし,したがって,要するに大理石の像の異ったあらゆる部分のような同時的な活動における必然的な連続などありえないだろうからだ.
\end{quote}
\begin{quote}
Ces lueurs intellectuelles de compréhension totale à quatre dimensions, nous leur attribuons nécessairement une durée dans le temps et, si fugitives qu'elles soient, nous leur supposons tout au moins une durée de quelques secondes. Or, cette durée n'existe même pas, car il ne saurait y 'avoir de durée dans le monde à quatre dimensions et par conséquent aucune succession nécessaire dans des actes qui sont, en somme, simultanés comme toutes les parties distinctes d'une statue de marbre.(37/74) 
\end{quote}
パヴロフスキーが大理石の像を持ち出しているのは,4次元には運動も時間も存在していないことを示すためだと考えられる.静止した運動しないものの全体を把握する方法は部分を取り上げることではなくて全体を同時的に把握するしかないのである.

ベルクソンとパヴロフスキーの差異は奇妙なねじれを含んでいる.カント以来の数学的な等質時空間モデルを否定して持続という概念を生み出し,それは根本的に質と量の混同に由来しているベルクソンと同じように,数学的な世界の限界と質への志向を主張しているのにもかかわらず,パヴロフスキーは持続を認めていないのである.ヘンダーソンもこの点に注目しており,「パヴロフスキーは,一方で,ベルクソンの優先順位を覆して,存在の真の同時性の啓示を4次元の美点の1つだと考えた(Pawlowski, on the other hand, reversed the preferences of Bergson and considered as one of the virtues of the fourth dimension its revalation of the true simultaneity of existence, in contrast to the appearance of succession in three dimensions.)\footnote{Henderson, \emph{op. cit. }, p. 204.}」とまとめている.それでは,この違いはどこに由来しているのだろうか.

この違いを説明するのは,パヴロフスキーとベルクソンの過去と未来への態度の差である.ベルクソンにとってこれらの時間の様相の問題は主体と記憶の問題に他ならなかった.1896年の著作『物質と記憶』が1911年に第7版の刊行に際して寄せた序文で,ベルクソンは「この書物は精神の実在と物質の実在を肯定し,両者の関係を特定の例,すなわち記憶の例によって規定しようとする(Ce livre affirme la réalité de l'esprit, la réalité de la matière, et essaie de déterminer le rapport de l'un à l'autre sur un example précis, celui de la matière)\footnote{アンリ・ベルクソン『物質と記憶』,田島節夫訳,白水社,1965年,5頁.Henri Bergson, \emph{Matière et Mémoire}, dir. Frédéric Worms, Paris,PUF, 2010, p. 1.}」と書き出しているように,記憶を考察することで実在論を組み立てることがベルクソンのプロジェクトの1つだった.

このプロジェクトでは,『直接与えられたもの』のテーマであった質としての時間すなわち持続によって物質はイマージュの総体であると説明される.ところで,記憶は「精神と物質の交錯点(le point d'intersection entre l'esprit et la matière)\footnote{同書,9頁.\emph{Ibid. }, p. 5.}」であり,さまざまな心理状態における記憶や記憶力が分析の対象となる.ところで,記憶は一般的に過去の出来事であり,記憶力によって現在において想起されるが,「記憶力の本質はけっして現在から過去への遡行にあるのではなく,反対に過去から現在への前進にあるのだ(La vérité est que la mémoire ne consiste pas du tout dans une régression du présent au passé, mais au contraire dans un progrès du passé au présent)\footnote{同書,266頁.\emph{Ibid. }, p. 269.}」から,持続とは現在へと流れていくものである.

この考えを踏まえてベルクソンの未来と過去についての考えを簡潔にまとめる.『物質と記憶』では生成される現在というテーマが中心であり,記憶の諸相の分析によって,過去は現在へ前進していくものであると考えられていた.『精神のエネルギー(\emph{L'énergie Spirituelle})』(1919)に収めらている1908年初出の「現在の回想と誤った認知(Le Souvenir du présent et la fausse reconnaissance)」の章で述べられているように,過去には決して現在にはならない純粋過去というものの存在を認めている.すなわち,流れゆく持続とは別に現在と共存する過去があると述べている\footnote{以下を参照のこと.Élie During, «Le souvenir du présent et la fausse reconnaissance», \emph{L'énergie Spirituelle}, dir. Frédéric Worms, Paris, PUF, pp. 307-13.}.一方で,人は一般的に未来の記憶を持っていることはないので,未来についてはこのようには語られていない.精神が記憶力として作用する場合,それは「未来をめざしての過去と現在の総合(synthèse du passé et du présent en vue de l'avenir)\footnote{ベルクソン,前掲書,246頁.Bergson, \emph{op. cit. }, p. 248.}」と表現されたりしているように,未来は持続の流れていく先のことを示している.よって,ベルクソンにとって未来は持続の流れていく方向のことを示している.実際に,ベルクソンが未来は独立して存在しているかのように語ることを錯覚だと示唆する逸話も存在する\footnote{Henri Bergson, «~Le réel et le possible~», \emph{La pensée et le mouvant}, Paris, PUF, 1955, p. 110.}.

ベルクソンの過去と未来の態度は以上に見たとおりである.次にパヴロフスキーについて見てみよう.パヴロフスキーが『物質と記憶』を『四次元郷』やそれ以前の著作で引用していることはないものの,明らかに参照していると思われる.例えば,プルーストが『失われた時を求めて スワン家の方へ』を出版した1914年に,パヴロフスキーが書いた同作についての書評では,プルーストがベルクソン的な方法によって執筆しているという指摘をしている.ベルクソン的な方法をとることによって,「著者は,私たちの印象が精神に生じるにつれて,脳内で相次いで起きている\bou{常に現実的な},無意志的想起,印象,感覚を書き記すことだけができる(Il[=~l'auteur] ne peut que noter, au fur et à mesure qu'elles[=nos impressions] se présentent à son esprit, les réminiscences, les impressions, les sensations \emph{toujours actuelles} qui se succèdent dans son cerveau. )\footnote{Gaston de Pawlowski, «~La Semaine Littéraire~», \emph{Comœdia}, 11 janvier, 1914, p. 3.}」と述べている.記憶の現実的な感覚が記されているという表現は,『物質と記憶』で「観念が言語的イマージュという特殊なイマージュの中で肉体を獲得するにいたる(l'idée arrive à prendre corps dans cette image particulière qui est l'image verbale)\footnote{ベルクソン,前掲書,148頁.Bergson, \emph{op. cit. }, p. 145.}」,あるいは「純粋記憶は,現実化するにつれて,対応するすべての感覚を身体の中に生ずる傾向をもつ(Le souvenir pur, à mesure qu'il s'actualise, tend à provoquer dans le corps toutes les sensations correspondantes)\footnote{同書,149頁.\emph{Ibid. }, p. 146.}」という一節を思わせるし,パヴロフスキーが「脳内で」と限定していることは記憶の現実化について論じている第2章の副題「記憶力と脳」が暗示されていると考えられる.では,ベルクソンの議論をパヴロフスキーが知っていたとして,パヴロフスキーは過去と未来についてどのように語っているのだろうか.

パヴロフスキーの過去と未来に対する態度は,極めて明確なテーゼにまとめることができる.それは,\bou{過去は存在しない},\bou{未来しか存在しない}というものである.以下を見てみよう.
\begin{quote}
過去の中の旅には関しては,この物語の途中で見つかることがないのに驚くこともないだろう,というのもその手の旅は不可能なのだから.四次元郷ではこの一瞬に未来だけが存在している.過去はもはや存在しない,それは全体的に現在の中に含まれているからであり,私たちに起きているあらゆることを知るための力ある意志とともに私たちの記憶を内側に喚起するのには十分である.
\end{quote}
\begin{quote}
Quant aux voyages dans le passé, on ne s'étonnera pas de n'en point trouver au cours de ce récit, car ces sortes de voyages sont impossibles. L'avenir seul existe en ce moment dans le pays de la quatrième dimension. Le passé n'existe plus, puisqu'il est entièrement contenu dans le présent, et il suffit d'évoquer intérieurement nos souvenirs avec une volonté puissante pour connaître tout ce qui s'est passé jusqu'à nous.(56/84)
\end{quote}
パヴロフスキーが過去は現在に含まれているために存在していないという考えは文脈は大きく異なるものの,ベルクソンの過去についての考えと近い部分がある.過去から現在への持続の流れは,過去が現在に含まれているという見方を可能にするものであるからだ.しかし,現在と共存する過去が存在することもベルクソンが認めているので,この点においてやはり決定的に異なっている.とりわけ,「未来だけが存在している」というフレーズは,生成し,展開していく現在というベルクソンのモデルと大きな隔たりがある.しかし,これらの差異はパヴロフスキーが4次元を対象に述べているのだから必然的に生じるものであると言える.

パウロフスキーにとってベルクソンの議論は3次元でしか成立しないのだ.パヴロフスキーが「過去はもはや存在しない,それは全体的に現在の中に含まれている」と述べているのは3次元であり,ベルクソンのとりわけ『物質と記憶』に基づいていることをうかがわせるのに対して,未来だけが存在する4次元というテーゼこそ,パヴロフスキーの独自性なのである.すでに本論ではヘンダーソンによるヒントンとパヴロフスキーの相違点の確認した.そこでは,単に高次元空間に時間と運動を認めるか否かが議論されていた.4次元は時間と運動がない一方で,未来だけが存在しているというヴィジョンこそパヴロフスキーの4次元が特異であることがベルクソンとの比較によって示すことができるのである.

ところで,近年,ベルクソン研究者のエリー・デュリングによって提示された「レトロ未来」という概念がある.これはベルクソンの「現在の回想と誤った認知」の研究から導かれた芸術論の1つである.現在と共存するが現在とは独立した過去の存在があるのであれば,現在と独立した未来も存在するのではないか,というのがレトロ未来の着想源である.私は,この概念はパヴロフスキーの4次元における未来だけが存在しているというテーゼと極めて近いものがあると考えている.というのも,パヴロフスキーは4次元が3次元に発現する契機に,芸術作品をあげているからだ.例えば,「着手された芸術作品,永遠の〈観念〉と物質の間の特有の接触点(l'œuvre d'art entreprise, la création personnelle qui est l'unique point de contact entre l'Idée éternelle)」(1912, 320)と述べている箇所では,4次元が「永遠の〈観念〉」と言い換えられて,まさに作られようとしている芸術作品は4次元が3次元に表現される契機であると語られている.レトロ未来では,現在から見た未来は存在せず,過去からみた未来しか存在しないというテーゼが掲げられている\footnote{エリー・デュリング「レトロ未来」,新村一宏訳,早稲田表象・メディア論学会,表象・メディア研究,第5号,2015年,12-3頁.}.つまり,未来が存在するのであれば,ベルクソン的にはそれは現在のただ中であり,そこから導かれるのは,現在と共存する,確かに存在している潜在的な未来なのである. パヴロフスキーの4次元は完成された状態のものなので潜在的なものなど存在していない.しかし,芸術作品の存在が示しているように,3次元では常に,未来のみが存在する4次元がその姿を間接的に見せているのである.パヴロフスキーの4次元の議論はこのようにレトロ未来というベルクソニズムのパターンに当てはめることによってまた1つ別の独自性を持っていることが明らかになった.すなわち,未来だけが存在しているのではなく,4次元においてはあらゆる未来が存在しており,それは芸術作品において現実化されるのである.パヴロフスキーにおいて潜在性を4次元の3次元的な表現と見做しうるかについてはまた別の機会取り上げる.



\begin{comment}
  
パヴロフスキーの4次元に関する理解が極めて特殊なのは,過去は存在せず未来だけが存在しているというその考えにある.

パヴロフスキーはクラインの壺のことを引用している.2次元多様体で,ユークリッド空間に埋め込むために4次元で曲率0とする5次元が必要となる..また,結び目だけに注目しているのも興味深い.しかもdurerと言う言葉を使っているので完全に自分の解けてしまったリボンの文脈に繋げてしまっている.
\end{comment}


\begin{comment}

杉山本で補足しておく.


竹内訳『意識に直接与えられているものについての試論』2010年
p. 140
「さらに言えば,原子の存在そのものが,何よりも疑わしいのである.原子に次から次に新しい属性が付与されてきた歴史を思い起こせば,原子というのは,どうやら実在する物ではなく,数々の力学的説明の物質化された残滓ではないかと思われるほどだ.」   

p. 203
「しかし,未来の予備的形成にはもう一つの別の形,意識が直接的にそのイメージを提供してくれるがゆえに,人間精神にはさらにいっそう身近な形がある.われわれはさまざまな意識状態のなかを通り過ぎてゆく.われわれはさまざまな意識状態のなかを通り過ぎてゆく.そして,後に続く状態がそれに先立つ状態のなかに含まれていなくとも,とにかく漠然とではあっても,そのような内在観念をわれわれが表象していたことは間違いない.この観念が現実化するものとして自覚されていたわけではないが,ありうるかもしれない可能性として感じられてはいたのだ.いずれにしろ,観念と行動のあいだには,ほとんど感じられない程度の中間項が入り込んでいて,その中間項の全体が,われわれにとってはある種の独特の形をとって現れる.それをわれわれは努力感情と呼んでいる.観念から努力へ,努力から行動への進展はきわめて連続的なので,どこで観念が終わり,どこで努力が終わったのか,どこで行動が始まったのか,それを言うことはできない.そこで人はこう考える.ある意味では,この場合も,未来が現在のなかであらかじめ形作られている,と言ってもよいのではないか,と.しかし同時に,ここで予備的に形成される未来はきわめて不完全なものである,と付け加えておく必要はあるだろう.」 
原因と結果は類比的な関係に過ぎない
因果律に関する2つの仮説
1 すべての現象は内的持続と同じように持続する=>現在から未来への移行は目的への努力,となる. 「自然現象にまで偶然性の概念を適用」(207)
2 持続を意識状態に固有のものとみなす=>現在のうちに未来が数理的な状態で存在している.「物理現象の必然的決定性を外的事物の非持続性によるものとする.」(207)
これら2つがベルクソンの批判する考え


P. 210
「例えば,科学者たちが,ファラデーのように,延長をもつ原子を力学的な点に置き換えても,彼らは力点とか力線を数理的に扱うのであって,活力とか努力とみなされる力そのものを問題にすることはない.したがって,外的な因果関係は純粋数理的なものであって,心的な力とそこから発現してくる行為とのあいだにある関係とは,どのような類似性もありえないということはすでに了解事項となっている.」

言語という枠組みにおいては自由を語ることができない,という考えは「3次元においては真理を語ることができない」というパヴロフスキーのテーゼと似ている.というか,内的持続の連続性こそ4次元のことではないか.
つまり,そもそも4次元はau-dessusではなくて,au-dessousではないか.

P. 219
「外的延長と内的持続,この相異なる二つの要素を,外的事物の研究を深化させるために,科学は切り離す.先に示したと思うが,持続からは同時性だけが,運動からは動くものの位置つまりは不動性だけが選びとられるのである.この切断は,科学においては,断乎として,しかも空間優位の形で遂行されるのである.
両者の分離はなお遂行されるべきであろうが,内的現象を研究する場合には,持続を優先させなければならない.内的現象といっても,おそらくその完了形でもなく,論証的知性がそれを説明するために,等質空間のなかにおいて分離し,展開した後の内的現象でもない.そうではなく,形成途上に」220「ある内的現象,相互浸透によって自由な一個の人格の持続的発展を構成するものとしての内的現象である.その原初の純粋状態に戻された内的持続は,まったき質として存在する多様性であり,相互に融合しあう内的諸要素の絶対的な相互異質性なのである.
そう考えてみれば,この当然なすべき両概念の分離を怠ったがゆえに,ある者は自由を否定することになり,またある者は自由を定義することになり,まさに定義することによって,意図に反して,これもまた自由を否定することになったのである.」
ベルクソンの話が科学の限界になる理由がまったくわからん・・・・

P. 221
「そうなると,異なったふたつの自我が存在することになる.一方の自我は,本来の自我が外界に投影された影絵のようなもの,その空間的表象であり,こう言ってよければ,社会的表象である.本来の自我に到達するためには,反省的思索をさらに深めることによって,われわれの内的諸状態を,絶えず生成途上にある,生きた存在として捉えなければならない.」


新訳 ベルクソン全集      月報1                                                                                                                                                                                                                                                                                               2       010年10月
加賀野井秀一 「解題」
「いわく,私たちは絶えず思考を言語によって表現し,しばしば空間的に考える.おかけで言語は,観念相互の間に,物質間に見られ(6)るような区別や不連続を設定する.こうした設定は実生活にも学問にも必要ではあるが,ひょっとすると,ある種の哲学的難問は,空間を占めていない諸現象を空間的に捉えることから来るのではないか.つまり,非延長的なものを延長に,質を量に,不当に翻訳したがために生じているのではないか.(7)」

本田裕志 ベルクソン哲学における空間・延長・物質 (一部引用記号を省略)
P. 9
『試論』における純粋持続に関する特性の1つ
「(5)意識への直接所与性 数学的計算の対象とならない純粋持続は
注意深い意識によって直接的に知覚され到達されるという仕方でのみ認識・把捉される.言いかえれば,それは私たちの直接的意識に現前するがままの持続,生きられる持続である.」

p. 16
『試論』における空間と延長の特徴
「(2)空間と延長の同義性 『試論』の多くの箇所では「広がりのある(extensif)」ものは計測可能であること,延長(étendue)は分割可能であり,延長をつうじて見られたものは量となること,延長を持つものは直接に数で表されること,などが語られ,また「持続=継起=非延長的なもの=質」と「延長(広がり)=同時性=量」とを対比した言い方もしばしば見出される.これらの箇所を見るかぎりでは,同書においてベルクソンは「延長」「広がり」を「空間」と区別せず,等質的環境としての空間そのものの外延的性質を言い表すのに用いている,と解される.
(3)等質的空間の実在性 次元を異にする二つの実在を私たちは知っている.その一つは異質的な,感覚的質の実在,すなわち実在的持続である純粋持続であり,いま一つは等質的な実在,すなわち空間である.空間は,意識への直接所与性によって実在性の確認される純粋持続=心的諸状態とは別次元に属するとはいえ,同様に堅固な実在である.」

p. 17 『試論』における物質的対象
「16物質的対象と空間性 『試論』において物質的対象(objet matériel)・物体(corps)・外的対象(objet extérieur)・17外的事物(chose extérieur)などと呼ばれるものは,空間性をその本質としている.私たちは物質的対象について語るとき,その位置を空間中に見出し,空間中に分散している諸物体の性質が知覚されるのは,空間とともにである.」
「無持続性 外的対象としての物質は,流れた時間のいかなるしるしも形跡も留めず,持続するとは言えない.私たちの外部すなわち外的世界には,空間と,現在すなわち同時性(simultanéité)のみが見出され,持続・継起は存在しない.」

p. 45 同書=『試論』
ベルクソンが感覚などの心的事象の実在性とは次元を異にする空間の実在性について同書で語る場合,それはあくまで感覚に付け加わる観念ないし直観としての空間についてのみ言われているのであって,意識の外なる空間の実在性が主張されているのではなく,したがって彼の言う空間の実在性は,カント流の用語によって」46「言えば,超越論的実在性ではなく経験的実在性にすぎない,と解することもできるように思われる.このように考えれば,空間中に広がる外的・物質的世界と持続する非延長的な心的事象という二実在の区別は,後者は意識の内にあるのに対して前者は意識の外なる実在である,という区別ではなくて,意識の内なる実在同士の区別であることになり,実在性ということの意味は2次元的ではなくて,意識への直接所与性ということに一元化されることになる.」 





ゼノンのパラドックスに関する論争
J. Milet, Bergson et la calcul infinitésimal, PUF, 1974, pp. 44-50. 
F. Heidesieck, Henri Bergson et la nation d'espace, Le cercle du livre, 1957, pp. 19-27. 

杉山本
72エヴェランに対するベルクソンの反論 
「「したがって我々は,現代における一人の思想家の繊細で深遠な分析の後においても,二つの運動体の遭遇が含意するのが,実在的運動と想像的運動との相違,即自的空間と無限分割可能な空間との相違,そして具体的時間と抽象的時間との相違であるという,そうした見解に同意しなければならないとは考えないのである」(DI 84-85/76 翻訳111-112).ここで言及される「思想家」エヴェランの考察は,実際には,「具体的時間/抽象的時間」といったベルクソン的な概念構成というよりも,時間や空間の分割可能性をめぐってのより伝統的なものであった.」二つの運動体とはアキレスと亀のこと


 P. 72エヴェランの議論について
「彼の議論のトポスはあくまでアリストテレス的なものであって,具体的な時間や空間,実在的運動は,何ら実無限を含意せず,無限分割を容れないものであるが,そうした単に潜在的なものにとどまる無限を実在としてしまうところに,ゼノンの逆説が生じるというのである.数学的無限とは実在なのか虚構なのかという問題は,当時においては「有限主義(finitisme)」と「無限主義(infinitisme)」との対立という形を採っていたのだが,エヴェランはそこにおいて明確な有限主義者として論を立てている.数学においてはともかく,実在する時間・空間は無限に分割されていないし,分割され得ない.それらは離散的な単位から成るのであり,そこから彼はゼノンの逆接を覆そうとするわけである.」
エヴェラン−アリストテレス的解決
時空は現勢的に無限の点から成っていないので,無限の数え上げという作業はそもそもない.有限主義.
極限値の非運動論的解釈(微分積分)
アキレウスと亀の間にある数列は有限値において収束することが数学から示せるので,追いつけないのは見かけの問題に過ぎない.



該当付近の翻訳
112
「具体的に存在する空間の分割可能性に一定の制限を設ける必要などどこにもない.二つの運動体が同時点で占める位置,これは確かに空間のなかにあるが,そのことと,空間内に位置付けることのできない運動,外的延長ではなく内的持続であり,質であって量ではない運動とを,明確に区別しておきさえすれば,空間はどこまでも分割できるものとして何の不都合もない.ある運動の速さを計測するのは,後にも述べるように,単に一つの同時性を確認しているのに過ぎない.(略)だから,ある時点においてアキレウスと亀の位置を同時に決定したり,この両者がある地点Xにおいて出会うこと,この出会いそれ自体が一つの同時性なのであるが,それをア・プリオリに予測したりするときには,数学はその役割の範囲にとどまっていると言ってよい.しかし,この二つの同時性の中間で起きていることを再構成できると主張するとすれば,数学はその分限(杉山訳 役割)を踏み越えることになる.あるいは少なくとも,そうしようとしても,相変わらず新規の同時性を考慮しなければならず,そのようにして無限に増大する同時性を前にして,不動のものをいくら集めてもそれから運動を作り出すことはできず,空間で時間を作り出すことはできないことを,数学は思い知ることになるだろう.要するに,内的持続のなかに等質なものが」113「あるとしても,それは持続しないもの,つまり空間でしかなく,そこに多数の同時性が並んでいるのだが,それと同じように,運動のなかに等質的要素があるとすれば,それは運動とは最も疎遠な要素,つまり運動体の軌跡空間であろうし,その軌跡空間とはすなわち不動のものなのである.
ところで,まさしくこれと同じ理由によって,科学は時間や運動というものを扱うときに,まずもってそれからその本質的な要素,つまりその質的な要素を──時間からは内的持続を,運動からは動性を,抹消せざるをえないことになる.それをわかりやすく説明するために,天文学と力学において,時間,運動,速度などがどのように扱われているかを検討してみたい.」
116
「力学は,時間に関しては同時性しか,運動に関しては不動性しか取り扱っていない,ということがこれで根拠づけられたと言って良いであろう.
このことは,力学が必然的に方程式の操作に基づいていること,代数的方程式は常に完了した自体を表現するものであることを,考えれば予見できた結果である.それに対して,われわれの意識に直接現前する持続や運動の本質は,それらが絶えず生成途上にあるという点にこそある.」
117
「空間はただ一つの等質的存在であり,空間内に置かれた事物は個別の多数体であり,これら個別の多数体なるものは空間内に展開されることで認識される,ということである.同様に,空間内には,意識が理解している意味での,持続も継起すらも存在しない,ということである.外的世界において継起すると言われる状態は,その一つ一つが単独で存在するものであって,それらの多数性は,それらを保持し,次いでそれらを相互に外的な関係において並置することができる意識にとってのみ存在しうるものである.意識がそれらを保持するのことができるのは,これら外的世界の様々な意識的事象を生み出し,これらの意識的事象が相互浸透し,それと認識されることなく有機的に統合され,この有機的統合の連帯によって」118「過去を現在に結びつけているからである」.
=>つまり,ゼノンのパラドックスは無限分割に由来するものでない.そもそも,空間中の位置移動としての運動に関してばかりでなく,幼年・成年・青年・老年といったように,生成変化一般に成り立つ(『創造的進化』).杉山73-74


杉山本
P. 141
「ビラン以降のいわゆる「スピリチュアリスム」が世界の客観的存在を主張する際には,主にビランが持ち出され,私の側の意志的努力に対する抵抗こそが非我の存在を告げるといった立論がむしろ普通だったのである(クーザン,ジャネ).」

同ページ
「観念論に対する彼の批判は,基本的に,観念論には説明できないことが多すぎる,という形を採る.何と言っても観念論にとって不利なのは,「科学が存在する(MM22/177-178)」という事実,つまり科学と呼ばれる活動に事実上の「成功(réussite)」(MM23/179)が見られるという事実である.別の言い方をするなら,現象には一定の秩序があり,「科学」という営みはそれを顕在化し,かつ精緻に突き詰めていくことに成功しつつあるのであって,観念論はその秩序を説明できないというのである.」
P. 143
「彼に言わせれば,観念論は,秩序だった外界を,あるいはより正確に言えば外界の「外在性」の意味であるところの「秩序」の先住を,どこかで前提としないで済ませられるものではない.」

パヴロフスキーがイマージュという場合,やはりベルクソン『物質と記憶』で述べられているイマージュの集合体としての世界という考えをうけているのだろうか.

p. 182
「すでに「記号(symbole)」に対する批判的観点は明らかである.『試論』が連合主義的な心理記述を「記号的表象(représentation symbolique)の価値しか持たない」(DI29/113)といった言い方で批判していることはよく知られていよう.「記号」とは,「直接与件」に代えて別のものをその「等価物(équivalent)」とする規約によって成立しながら,その恣意性によって実在を覆い隠し,私たちを実在から遠ざけるものなのである.」

p. 205
「しかしベルクソンは,「知性性」も「物質性」も,それぞれ発生を有していると述べている.知性は永遠の昔から今の人間の有するような知性として存在していたわけでもなければ,物質もまた今日人間知性が捉えるような姿を以前からずっと保持してきたわけではない──ベルクソンはそう言おうとしているのではないか.
不可能な解釈ではないが,相当に高いコストがかかると思う.高過ぎる,と私たちは見積もる.なぜなら,以上のように考えると,まず適応の「相互」性が,実際のところ理解しがたいものになる.頻繁なキャッチボールのやり取りを通じて,大きく見れば知性と物質は互いに互いを合わせてきた(単に知性が試行錯誤を繰り返して物質の実相に近づいてきた,というのではない.物質の方もまるで交渉の対話者のごとく知性に合わせて意見を次第に拵えてきた,というのである),と述べているのは,困難を小さくすれば見逃してもらえようといった詐術でしかあるまい.」
p. 208-9
「しかしベルクソンは物質を生命的傾向の欠如,「中断・逆転」として定義する.人間知性の機能を,生命や持続とその創造性を排除する諸表象に基づいた思考であると見定める.それによって,両者の一致と再会は,反対に,「全く自然に」生じるものと言い得るようになる.それを裏切る予見不可能性,創造性がここでは定義上」「排除されているのだから.」
«banqueroute de la science»
F.  Brunetière, «Après  une visite au Vatican», Revue des deux mondes, janvier 1895. 
科学の成功を認めなかった論者.



竹内訳『物質と記憶』
要約と結論
P. 329
「現実には粒子群に分割されている延長世界を一方に置き,空間内に投影されてはいるが,それ自身としては非外延的な諸感覚を伴う意識を他方に置いて見れば,このような物質とこのような意識とのあいだには,すなわち身体と精神とのあいだには,どんな共通点も見出せないのは自明のことであろう.しかし,この知覚と物質世界との対立というものは,自らの習慣や法則に沿うようにものごとを分解し,再構成する人間悟性の人為的構成物なのであって,無媒介の直観に直接与えられているものではない.直接的直観に与えられているのは,非外延的な諸感覚ではない.」
物質の分割された延長世界と精神の純粋な非延長世界が対応しているのは,その中間にある「外延的なもの」による.

P. 334
ベルクソンは生命体のことを,「生命ある物質」(matière vivante, PUF280)と述べている.ただし,当時の進化論を色濃く反映している.グーグルブックスの年代別検索でもっと深掘りする必要があるか.
「生命ある物質の進展=進化は,生命諸機能の分化に存するのであって,外的刺激を誘導し,行動を組織することを可能にする神経組織の形成と,それに続く段階的複雑化の歩みが,その分化によって開かれているのである.」
\end{comment}



