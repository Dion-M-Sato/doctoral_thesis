\begin{center}
  {\Large 参考文献表}\\
\end{center}

\noindent {\small {\textgt 注記\\}}
{\small 出版社が不明なものに関しては,{\textbf Gallica}に従って,[s. n.]と表記した.}
\\
\\
{\large 外国語文献}\\
Abbott, Edwin Abbott, \emph{Flatland~: A Romance of Many Dimensions by Square}, London, Seeley \& Co. , 1884.\\
Barel-Moisan, Clauire, «~écrire pour instruire~», \emph{La Civilisation du Jounal, Histoire Culturelle et Littérature de la Présse Française au XIX\textsuperscript{e} Siècle}, éd. Dominique Kalifa et. al. , Paris, Nouveau édition, 2011, pp. 752-65.\\
Baudin, Henri, \emph{La science-fiction~: un univers en expansion}, Paris, Bordas, 1971.\\
Bergson, Henri, «~Le réel et le possible~», \emph{La pensée et le mouvant}, Paris, PUF, 1955.\\
--- , \emph{Matière et mémoire}, éd. Frédric Worms, Paris, PUF, 2008.\\
--- , \emph{L'énergie spirituelle}, éd. Frédric Worms, Paris, PUF, 2009\\
--- , \emph{Matière et Mémoire}, éd. Frédéric Worms, Paris,PUF, 2010\\
Bernard, Claude, \emph{La Science Epérimentale}, Paris, J. -B. Bailliere, 1878.\\
--- , \emph{Leçons de Physiologie Opératoire}, Paris, M. Duval, p. XIV..\\
--- , \emph{Leçons sur les phénomènes de la vie communs aux animaux et aux végétaus}, t. 1, Paris, J. -B. Baillière, 1879.\\
Berthelo, Marcelin, «~Remarques sur l'existence réelle d'une matière monoatomique, à la suite d'une communication de M. Villarceau~», \emph{Académie des Sciences. Compte rendus hebdomadaires des séances}, n. 82, 1876, pp. 1129-1130.\\
Blanqui, Auguste, \emph{L'éternité par les astres},G. Baillière, Paris, 1872.\\
Boissy, Gabriel, «~Gaston de Pawlowski premier rédacteur en chef de «~Comœdia~» est mort prèmaturément hier~», \emph{Comœdia}, 3 Fév 1933, pp. 1-2.
Bridenne, Jean-Jacques, \emph{La Littérature Français d'Iimagination Scientifique}, Paris, Gustave Arthur Dassonville, 1950.\\
Brotchie, Alastair, \emph{Alfred Jarry~: ein pataphysisches Leben}, Bern, Piet Meyer, 2014\\
Carton, Yves, \emph{Darwinien convaincu: médecin, chercheur et journaliste, 1855-1934}, Paris, Hermann, 2008.\\
Clair, Jean, \emph{Marcel Duchamp, ou, Le grand fictif~: essai de mythanalyse du Grand verre}, Paris, Galilée, 1975.\\
--- , «~Introduction~», \emph{Voyage au pays de la quatrième dimension}, Gaston de Pawlowski, Images modernes, 2004, pp. 11-27\\
Compère, Daniel, «~Fait divers et vulgarisation scientifique~», \emph{romantisme}, n. 97, Paris, Armand Colind, 1997, pp. 69-76.\\
Conry, Yvette, \emph{Introduction du Darwinisme en France au XIX\textsuperscript{e}siècle}, Paris, J. Vrin, 1974.\\
Curval, Philippe, «~Surréalisme et Science-Fiction~», \emph{Europe}, v. 79, n. 870, oct. 2001, pp. 32-50.\\
Dumasy-Queffélec, Lise, «~Le feuilleton~», \emph{La Civilisation du Jounal, Histoire Culturelle et Littérature de la Présse Française au XIX\textsuperscript{e} Siècle}, éd. Dominique Kalifa et. al. , Paris, Nouveau édition, 2011, pp. 925-936.\\
During, Élie, «Le souvenir du présent et la fausse reconnaissance», \emph{L'énergie Spirituelle}, dir. Frédéric Worms, Paris, PUF, pp. 307-13.\\
Flammarion, Camille, «~L'Eternité par les astres par A. Blanqui~», \emph{L'Opinion Nationale}, 25 mars 1872, p. 3.\\
Gomel, Elana, \emph{Narrative Space and Time~: representing impossible topologies in literature}, New York, Routledge, 2014.\\
Hannequin, Arthur, \emph{Essai critique sur l'hypothèse des atomes}, Paris, Allan, 1899.\\
Henderson, Linda Dalrymple, \emph{The fourth dimension and non-Euclidean geometry in modern art}, Massachusetts, MIT Press, 2013.\\
Herbert Spencer, \emph{The Principles of Biology, The Works of Herbert Spencer}, v. II, Osnabrück, Otto Zelleri, 1966.\\
Herp, Jacques Van, \emph{Panorama de la Science Fiction. Les Thème, Les Genres, Les écoles, Les Problèmes}, Verviers, Gérard \& Co, 1973.\\
Jouffret, Esprit Pascal, \emph{Traité de géométrie à quatre dimension}, Paris, Gauthier-Villars, 1903.\\
Kleeberg, Bernhard \emph{Theophysis~: Ernst Haeckels Philosophie des Naturganzen}, Böhlau Verlag, Köln Weimar, 2005.\\
Le bon, Gustave, \emph{L'évolution des Forces}, Flammarion, Paris, 1907.\\
Loison, Laurent, «~Le projet du néolamarckisme français (1880-1910)~», \emph{Revue d'histoire des sciences}, t. 65, janv. 2012, pp. 61-79.\\
Lucas, Prosper, \emph{Traité philosophique et physiologique de l'hérédité naturelle dans les états de santé et de maladie du système nerveux}, t. 1 et 2, Paris, J. -B. Baillière, 1847-50.\\
Marcandier-Colard, Christine, \emph{Crimes de sang et scènes capitales : essai sur l'esthétique romantique de la violence}, Paris, PUF, 1998.\\
Marrati, Paola, «~Le nouveau en train de se faire~», \emph{Revue internationale de philosophie}, n. 241, mars 2007, pp. 267-8.\\
McLean, Steven, \emph{The Early Fiction of H.G. Wells: Fantasies of Science}, London, Palgrave Macmillan, 2009.\\
Mollier, Jean-Yves, \emph{L'argent et les lettres, histoire du capitalisme d'édition, 1880-1920}, Fayard, Paris, 1988.\\
--- , «~écrivaint-édituer~: un face-à-face déroutant~», \emph{Travaux de littérature}, v. XV, t. 2, Paris, L'Adirel, 2002, pp. 17-39. \\
Morange Michel, «~Darwin dans L'histoire de la Pensée~», \emph{Transversalités}, n. 114, fév. 2010, pp. 111-7.\\
Naudin, Charles, «~Considérations philosophiques sur l'espèce et la variété~», \emph{Revue horticole}, v. 4, n. 1, 1852, pp. 102-9.\\
Pawlowski, Gaston de, \emph{Sociologie Nationale. Une définition de l'état}, Paris, V. Giard \& E. Brière, 1897.\\
--- , \emph{Philosophie du Travail}, V. Giard \& E. Brière, 1901.\\
--- , \emph{Voyage au pays de la quatrième dimension}, Paris, Bibliothèque-Charpentier, 1912. (https:\slash\slash archive.org\slash details\slash voyageaupaysdela00pawl)\\
--- , «~La Semaine Littéraire~», \emph{Comœdia}, 11 janvier 1914, p. 3.\\
--- , «~Où allons-nous?~», \emph{Les Annales politiques et littéraires~: revue populaire paraissant le dimanche}, dir. Adolphe Brisson, 6 Déc 1925, Paris, [s. n.], pp. 581-2.\\
--- , \emph{Voyage au pays de la quatrième dimension}, Paris, Denoël, 1971.\\
Pierre Kropotkine, \emph{L'Entr'aide, un facteur d'évolution}, Paris, Hachette, 1906.\\
Poincaré, Henri, \emph{La Science et l'Hypothèse}, Paris, Ernst Flammarion, 1902.\\
Proust, Marcel, \emph{Correspondance de Marcel Proust}, éd. Philip Kolb, t. 3, Paris, Plon, 1976.\\
Príncipe, João, «~La physique laplacienne dans la seconde moitié du XIXe siècle: Joseph Boussinesq – la pratique et la réflexion autour de l’atomisme en France vers 1875~», \emph{Kairos Journal of Philosophy and Science}, vol. 13, 2015, pp. 179-212.\\
Régnier, Pilippe, « Place, fonctions et formes de l'ecriture utopique chez Fourier », \emph{Pamphlet, utopie, manifeste, XIX\textsuperscript{e}-XX\textsuperscript{e} siècles}, textes réunis par Lise Dumasy et Chantal Massol, L'Harmattan, 2001, pp. 385-401.\\
Stableford, Brian, "Introduction", \emph{Journey to the Land of the Fourth Dimension},  London, Black Coat, 2011, Electronic.\\
Suzamel (Blanqui), «~Le Père Gratry. Science et foi. (3\textsuperscript{e} article)~», \emph{Candide. Journal à Cinq centimes}, 1\textsuperscript{ème} année, n. 8, 27 mai 1865, p. 1.\\
Taine, Hippolyte, \emph{De l'Intelligence}, Paris, Hachette, 1870.\\
Tarde, Gabriel de, \emph{Fragment d'histoire future}, Paris, V. Giard \& E. Brière, 1896.\\
Thénty, Marie-Ève, \emph{La Littérature au quotidien. Poétiques Journalstique au XX\textsuperscript{e} siècle}, Paris, Seuil, 2007.\\
Tirard, Stéphane, «~L’histoire du commencement de la vie à la fin du XIX\textsuperscript{e} siècle~», \emph{Cahiers François Viète}, Série I, n. 9-10, 2005, pp. 105-18.\\
--- , «~Origin of Life and Definition of Life, from Buffon to Oparin~», \emph{Origins of Life and Evolution of Biospheres}, n. 40, 2010, pp. 215-20.\\
Troy, Nancy J. , "Staging Haute Couture in Early 20th-Century France", \emph{Theatre Journal}, v. 53, n. 1, 2001, pp. 1-32.\\
Varigny, Henry de, \emph{Journal Débat}, 8 fév 1923, p. 4.\\%記事名!
Versins, Pierre, \emph{Encyclopédie de l'utopie des voyages extraordinaires et de la science fiction}, Lausanne, L'Age D'Homme, 1972.\\
Vincent, Berdoulay et Olivier, Soubeyran, «~Lamarck, Darwin et Vidal~: aux fondements naturalistes de la géographie~», \emph{Annales de Géographie} , t. 100, n. 561-562, 1991, pp. 617-634.\\
Walbecq, Eric, «~Gaston de Pawlowski~», \emph{Le Visage Vert}, n. 4, Paris, Joëlle Losfeld, 1998, pp. 148-52.\\
Walter, Benjamin, \emph{Gesammelte Schriften}, v. I, éd. Rolf Tiedemann, Frankfurt, Suhrkamp, 1991.\\
\emph{Comœdia}, 3. Fév. 1933, pp. 1-2.\\
\emph{Histoire générale de la presse française}, dir. Claude Béllanger, et al. ,t. 3, Paris, Presses universitaires de France\\
\emph{Le Nouveau Petit Robert de la langue française}, 2008, Électronique.\\
\foreignlanguage{russian}{Ибаньес, Рауль, \emph{Четвертое измерение~: Является ли наш мир тенью другой Вселенной?~}, Москва, Де Агостини}, 2014.\\
\\
{\large 日本語文献}\\
石橋正孝『〈驚異の旅〉または出版をめぐる冒険:ジュール・ヴェルヌとピエール=ジュール・エッツェル』,左右社,2013年.\\
カーオ,ヘリガ『20世紀物理学史:理論・実験・社会 上』,有賀暢廸・稲葉肇他訳,名古屋大学出版会,2015年.\\
加藤文元『リーマンの数学と思想』,共立出版,2017年.\\
クロウ,マイケル・J『地球外生命論争1750-1900』鼓澄治・山本啓二・吉田修訳,工作者,2001年.\\
コンペール,ダニエル『大衆小説』,宮川朗子訳,国文社,2014年.\\
シヴァリエ,ルイ『三面記事の栄光と悲惨 近代フランスの犯罪・文学・ジャーナリズム』,小倉孝誠・岑村傑訳,白水社,2005年.\\
ジェネット,ジュラール『フィクションとディクション』,和泉涼一・尾河直哉訳,水声社,2004年.\\
スーヴィン,ダルコ『SFの変容』,大橋洋一訳,国文社,1991年.\\
鈴木雅雄「星々は夢を見ない --- オーギュスト・ブランキに関する覚え書き」,早稲田大学大学院文学研究科紀要,第2分冊,53号,2007年,3-16頁.\\
デュリング,エリー「レトロ未来」,新村一宏訳,早稲田表象・メディア論学会,表象・メディア研究,第5号,2015年,1-29頁.\\
中沢新一「四次元の花嫁」,『東方的』,講談社,2012年,48-120頁.\\
ブランキ,オーギュスト『天体による永遠』浜本正文訳,雁思社,1985年.\\
ブロック,W・H『化学の歴史 II』,大野誠他訳,朝倉書店,2006年,284頁.\\
ベルクソン,アンリ『物質と記憶』,田島節夫訳,白水社,1965年,5頁.\\
ベンヤミン,ヴァルター『パサージュ論I』今村仁・三島憲一他訳,1993年.\\
ボドゥ,ジャック『SF文学』,新島進訳,白水社,2011年.\\
森川亮,「量子論の歴史 --- 未知なる放射線,その発見ラッシュの裏面史」,生駒経済論叢,第13巻,第2号,2015年,283-300頁.\\
三宅岳史,「リーマンと心理学,そして哲学」,『現代思想』,44巻,6号,青土社,2016年,161-175頁.\\
\\
{\large カンファレンス}\\
Patrizia D’Andrea, «~Voyage au pays du silence~: rêve, philosophie, utopie entre symbolisme et anticipation (Han Ryner, Gaston de Pawlowski)~», Mobilités dans les récits de fantastique \& de science­fiction (XIX~-~XXIe siècles)~: quête \& enquête(s), l'IUT Sénart-Fontainebleau, Université Paris Est Créteil, Jeudi 20 Nov.\\
Sophie Lucet, «~Gaston de Pawlowski et le journalisme ou l’art de l’échappée~», Comœdia (1907-1937). un continent inexploré dans l'histoire du théâtre, Bibliothèque Seebacher, Université Paris-Diderot, 21 juin 2014. \\
